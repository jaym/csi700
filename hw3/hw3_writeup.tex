\documentclass{jhwhw}
\usepackage{listings}
\usepackage[T1]{fontenc}
\usepackage{xcolor}
\usepackage{color}
\usepackage{moreverb}
\usepackage{paralist}
\usepackage{multicol}

\definecolor{MyDarkGreen}{rgb}{0.0,0.4,0.0}
\lstloadlanguages{Matlab}%

\definecolor{solarized@base03}{HTML}{002B36}
\definecolor{solarized@base02}{HTML}{073642}
\definecolor{solarized@base01}{HTML}{586e75}
\definecolor{solarized@base00}{HTML}{657b83}
\definecolor{solarized@base0}{HTML}{839496}
\definecolor{solarized@base1}{HTML}{93a1a1}
\definecolor{solarized@base2}{HTML}{EEE8D5}
\definecolor{solarized@base3}{HTML}{FDF6E3}
\definecolor{solarized@yellow}{HTML}{B58900}
\definecolor{solarized@orange}{HTML}{CB4B16}
\definecolor{solarized@red}{HTML}{DC322F}
\definecolor{solarized@magenta}{HTML}{D33682}
\definecolor{solarized@violet}{HTML}{6C71C4}
\definecolor{solarized@blue}{HTML}{268BD2}
\definecolor{solarized@cyan}{HTML}{2AA198}
\definecolor{solarized@green}{HTML}{859900}

\lstset{
  language=Matlab,
  columns=fixed,
  tabsize=2,
  extendedchars=true,
  breaklines=true,
  frame=single,
  numbers=left,
  numbersep=5pt,
  rulesepcolor=\color{solarized@base03},
  numberstyle=\tiny\color{solarized@base01},
  basicstyle=\footnotesize\ttfamily,
  keywordstyle=\color{solarized@green},
  stringstyle=\color{solarized@cyan}\ttfamily,
  identifierstyle=\color{solarized@blue},
  commentstyle=\color{solarized@base01},
  emphstyle=\color{solarized@red}
}

\relpenalty=9999
\binoppenalty=9999

\title{Homework Set 3}%replace X with the appropriate number
\author{Jay Mundrawala}

\begin{document}
%%%%%%%%%%%% PROBLEM #1 %%%%%%%%%%%%%%%%%%%%%%%%%
\problem{Scientific Computing 3.2, Page 149}
Suppose you are fitting a straight line to the three data points
(0,1), (1,2), (3,3).
\begin{enumerate}
  \item Set up the overdetermined lineary system for the least squares problem.
  \item Set up the corresponding normal equations.
  \item Compute the least squares solution by Cholesky factorization.
\end{enumerate}
\solution
\part
\[y = mx + b \]

\[
\begin{bmatrix}
  1 & 0 \\
  1 & 1 \\
  1 & 3
\end{bmatrix}
\begin{bmatrix}
  b \\
  m
\end{bmatrix}
=
\begin{bmatrix}
  1 \\
  2 \\
  3
\end{bmatrix}
\]

\part
\[
A^TAx = A^Tb
\]

\[
\begin{bmatrix}
  1 & 1 & 1 \\
  0 & 1 & 3
\end{bmatrix}
\begin{bmatrix}
  1 & 0 \\
  1 & 1 \\
  1 & 3
\end{bmatrix}
\begin{bmatrix}
  b \\
  m
\end{bmatrix}
=
\begin{bmatrix}
  1 & 1 & 1 \\
  0 & 1 & 3
\end{bmatrix}
\begin{bmatrix}
  1 \\
  2 \\
  3
\end{bmatrix}
\]

\[
\begin{bmatrix}
  3 & 4 \\
  4 & 10
\end{bmatrix}
\begin{bmatrix}
  b \\
  m
\end{bmatrix}
=
\begin{bmatrix}
  6 \\
  11 
\end{bmatrix}
\]

\part
\[
\begin{bmatrix}
  a_{11} & a_{21} \\
  a_{21} & a_{22}
\end{bmatrix}
=
\begin{bmatrix}
  l_{11} & 0 \\
  l_{21} & l_{22}
\end{bmatrix}
\begin{bmatrix}
  l_{11} & l_{21} \\
  0      & l_{22}
\end{bmatrix}
\]

\begin{eqnarray*}
l_{11} &=& \sqrt{a_{11}} \\
l_{21} &=& a_{21}/l_{11} \\
l_{22} &=& \sqrt{a_{22} - l_{21}^2}
\end{eqnarray*}

\[
L =
\begin{bmatrix}
  \sqrt{3}           & 0 \\
  \frac{4}{\sqrt{3}} & \sqrt{10 - \frac{16}{3}}
\end{bmatrix}
=
\begin{bmatrix}
  \sqrt{3}           & 0 \\
  \frac{4}{\sqrt{3}} & \sqrt{\frac{14}{3}}
\end{bmatrix}
\]

\begin{eqnarray*}
LL^Tx &=& b \\
x &=& (L^T)^{-1}(L^{-1})b \\
x &=& U^{-1}(L^{-1})b
\end{eqnarray*}

\[
Lw=b
\]

\[
\begin{bmatrix}
  w_1 \\
  w_2
\end{bmatrix}
=
\begin{bmatrix}
  6 \\
  11
\end{bmatrix}
\]
\begin{eqnarray*}
  w_1 &=& \frac{6}{\sqrt{3}} \\
  w_2 &=& \frac{11 - 8}{\sqrt{\frac{14}{3}}} = \frac{3\sqrt{3}}{\sqrt{14}}
\end{eqnarray*}

\[
Ux = w
\]
\[
\begin{bmatrix}
  \sqrt{3}           & \frac{4}{\sqrt{3}} \\[6pt]
   0 & \sqrt{\frac{14}{3}}
\end{bmatrix}
\begin{bmatrix}
  b \\
  m
\end{bmatrix}
=
\begin{bmatrix}
  \frac{6}{\sqrt{3}} \\[6pt]
  \frac{3\sqrt{3}}{\sqrt{14}}
\end{bmatrix}
\]

\[
  m =  \frac{9}{14}
\]

\begin{eqnarray*}
  \sqrt{3}b &=& \frac{6}{\sqrt{3}} - \frac{4}{\sqrt{3}}\frac{9}{14} \\
  b &=& 2 - \frac{12}{14} \\ 
  &=& \frac{8}{7} 
\end{eqnarray*}

\[
x_{ls} = 
\begin{bmatrix}
  \frac{8}{7} \\[6pt]
  \frac{9}{14}
\end{bmatrix}
\]

%%%%%%%%%%%% PROBLEM #2 %%%%%%%%%%%%%%%%%%%%%%%%%
\problem{Scientific Computing 3.5, Page 149}
Let $x$ be the solution to the linear least squares problem \(Ax\cong b\), where
\[
A =
\begin{bmatrix}
  1 & 0 \\
  1 & 1 \\
  1 & 2 \\
  1 & 3
\end{bmatrix}
\]
Let \(r = b - Ax\) be the corresponding residual vector. Which of the following 
three vectors is a possible value for r? Why?
\begin{center}
\begin{inparaenum}
\item 
  \(
  \begin{bmatrix}
	1 \\
	1 \\
	1 \\
	1
  \end{bmatrix}
  \)
\item 
  \(
  \begin{bmatrix}
	-1 \\
	-1 \\
	 1 \\
	 1
  \end{bmatrix}
  \)
\item
\(
  \begin{bmatrix}
	-1 \\
	 1 \\
	 1 \\
	-1
  \end{bmatrix}
  \)
\end{inparaenum}
\end{center}
\solution
To have a linear least squares solution, \(A^Tr = 0\)
\[
A^Tr_a = 
\begin{bmatrix}
  4 \\
  6
\end{bmatrix}
\]

\[
A^Tr_b = 
\begin{bmatrix}
  0 \\
  4
\end{bmatrix}
\]

\[
A^Tr_c = 
\begin{bmatrix}
  0 \\
  0
\end{bmatrix}
\]

Since \(A^Tr\) is \(\textbf{\textit{0}}\) only for vector c, c is the least squares
solution for A.

%%%%%%%%%%%% PROBLEM #3 %%%%%%%%%%%%%%%%%%%%%%%%%
\problem{Scientific Computing 3.17, Page 149}
Determine the Householder transformation that annihilates all but the first entry of the
vector \(
\begin{bmatrix}
  1 &
  1 &
  1 &
  1
\end{bmatrix}^T
\). Specifically, if 
\[
\left(I - 2\frac{vv^T}{v^Tv}\right)
\begin{bmatrix}
  1 \\
  1 \\
  1 \\
  1
\end{bmatrix}
=
\begin{bmatrix}
  \alpha \\
  0 \\
  0 \\
  0
\end{bmatrix}
\]
what are the values of $\alpha$ and $v$.
\solution
\begin{eqnarray*}
  v &=& a - \alpha e_1 \\
  \alpha &=& \pm \| a \|_2 = -2
\end{eqnarray*}

\[
v = 
\begin{bmatrix}
  1 \\
  1 \\
  1 \\
  1
\end{bmatrix}
+
2
\begin{bmatrix}
  1 \\
  0 \\
  0 \\
  0
\end{bmatrix}
=
\begin{bmatrix}
  3 \\
  1 \\
  1 \\
  1
\end{bmatrix}
\]
Check:
\[
Ha = 
\begin{bmatrix}
  1 \\
  1 \\
  1 \\
  1
\end{bmatrix}
-
\frac{2*6}{12}
\begin{bmatrix}
  3 \\
  1 \\
  1 \\
  1
\end{bmatrix}
=
\begin{bmatrix}
 -2 \\
  0 \\
  0 \\
  0
\end{bmatrix}
\]
%%%%%%%%%%%% PROBLEM #4 %%%%%%%%%%%%%%%%%%%%%%%%%
\problem{Scientific Computing 3.23, Page 151}
In Section 3.2.1 we objserved that the cross-product matrix $A^TA$ is exactly
singulare in floating-point arithmetic if
\[
A = 
\begin{bmatrix}
  1 & 1 \\
  \epsilon & 0 \\
  0 & \epsilon
\end{bmatrix}
\]
where $\epsilon$ is a positive number smaller than $\sqrt{\epsilon_{mach}}$
in a given floating-point system. Show that if $A=QR$ is the factorization 
for this matrix, then R is not singulare, even in floating-point arithmetic.
\solution
\[
A = QR
\]

\[
\underbrace{\begin{bmatrix}
  . & . \\
  . & . \\
  . & .
\end{bmatrix}}_A
=
\underbrace{\begin{bmatrix}
  . & . \\
  . & . \\
  . & .
\end{bmatrix}}_Q
\underbrace{\begin{bmatrix}
  . & . \\
  . & .
\end{bmatrix}}_R
\]

\[
\begin{bmatrix}
  a_1 & a_2
\end{bmatrix}
=
\begin{bmatrix}
  \hat{q_1} & \hat{q_2}
\end{bmatrix}
\begin{bmatrix}
  r_{11} & r_{12} \\
  0      & r_{22} \\
\end{bmatrix}
\]
where
\begin{eqnarray*}
  a_1 &=& q_1r_{11} \\
  a_2 &=& \hat{q_1}r_{12} + \hat{q_2}r_{22}
\end{eqnarray*}

\begin{eqnarray*}
  q_1 &=&
  \begin{bmatrix}
	1 &
	\epsilon &
	0
  \end{bmatrix}_T \\
\hat{q_1} &=&
\frac{\begin{bmatrix}
  1 &
  \epsilon &
  0
\end{bmatrix}^T}{\|q_1\|}
\cong
\begin{bmatrix}
  1 &
  \epsilon &
  0
\end{bmatrix}^T \\
  q_2 &=&
  \begin{bmatrix}
	1 \\
	0  \\
	\epsilon
  \end{bmatrix}
  -
  \begin{bmatrix}
	1 \\
	\epsilon \\
	0
  \end{bmatrix}
  =
  \begin{bmatrix}
	0 \\
	-\epsilon \\
	\epsilon
  \end{bmatrix} \\
\hat{q_2} &=&
\begin{bmatrix}
  0 \\[6pt]
  \frac{-1}{\sqrt{2}} \\[6pt]
  \frac{1}{\sqrt{2}} 
\end{bmatrix}
\end{eqnarray*}
\begin{eqnarray*}
  r_{11} &=& 1 \\
  r_{12} &=& 1 \\
  r_{22} &=& \sqrt{2}\epsilon
\end{eqnarray*}
\[
det(R) = r_{11}r_{22} - 0r_{12} = \sqrt{2}\epsilon
\]
%%%%%%%%%%%% PROBLEM #5 %%%%%%%%%%%%%%%%%%%%%%%%%
\problem{Scientific Computing 3.2, Page 153}
A common problem in surverying is to determine the altitudes of a series of points with
respect to some reerence point. The measurements are subject to error, so more observations
are taken than are strictly necessary to determine the altitudes, and the resulting 
overdetermined system is solved in the least squares sense to smooth out errors. Suppose
that there are four points whose altitudes $x_1$, $x_2$, $x_3$, $x_4$ are to be determined.
In addition to direct measurements of each $x_i$ with respect to the reference point,
measurements are also taken of each point with respect to all of the others. The resulting 
measurements are:

\begin{multicols}{5}
  \begin{inparablank}
  \item \(x_1=2.95\)
  \item \(x_2=1.74\)
  \item \(x_3=-1.45\)
  \item \(x_4=1.32\)
  \item \(x_1 - x_2 = 1.23\)
  \item \(x_1 - x_3 = 4.45\)
  \item \(x_1 - x_4 = 1.61\)
  \item \(x_2 - x_3 = 3.21\)
  \item \(x_2 - x_4 = 0.45\)
  \item \(x_3 - x_4 = -2.75\)
  \end{inparablank}
\end{multicols}

Set up the corresponding least squares system \(Ax \cong b\) and use a library routine,
or one of your own design, to solve it for the best values of the altitudes. How do your
computed values compare with the direct measurements?

\lstinputlisting{./hw3_05/tex.m}

%%%%%%%%%%%% PROBLEM #6 %%%%%%%%%%%%%%%%%%%%%%%%%
\problem{Scientific Computing 3.4, Page 153}
\begin{enumerate}
  \item Solve the following least squares problem using any method you like:
	\[
	\begin{bmatrix}
	  0.16 & 0.10 \\
	  0.17 & 0.11 \\
	  2.02 & 1.29
	\end{bmatrix}
	\begin{bmatrix}
	  x_1 \\
	  x_2 
	\end{bmatrix}
	\cong
	\begin{bmatrix}
	  0.26 \\
	  0.28 \\
	  3.31
	\end{bmatrix}
	\]
  \item Now solve the same least squares probelm again, but this time use the
	slightly perturbed right-hand side
	\[
	b =
	\begin{bmatrix}
	  0.27 \\
	  0.25 \\
	  3.33
	\end{bmatrix}
	\]
  \item Compare the results from parts \textit{a} and \textit{b}. Can you explain the
	difference?
\end{enumerate}
\solution
\part
\lstinputlisting{./hw3_06/tex_a.m}
\part
\lstinputlisting{./hw3_06/tex_b.m}
\part
The condition number for $A$ is on the order of $10^5$, so we can expect small perturbations to cause
large changes. In this case, the system in \textit{(a)} is solved exactly, i.e. the residual is the 
zero vector. This is not the case for in \texit{(b)}, where $x$ is actually the least squares 
approximate solution to the system.

\end{document}
