\documentclass{jhwhw}
\usepackage{listings}
\usepackage[T1]{fontenc}
\usepackage{xcolor}
\usepackage{color}
\usepackage{moreverb}
\usepackage{paralist}
\usepackage{multicol}
\usepackage{tikz,pgfplots}
\usepackage{gnuplot-lua-tikz}
\usepackage{adjustbox}
\usepackage{float}

\definecolor{MyDarkGreen}{rgb}{0.0,0.4,0.0}
\lstloadlanguages{Matlab}%

\newfloat{Code}{H}{myc}

\definecolor{solarized@base03}{HTML}{002B36}
\definecolor{solarized@base02}{HTML}{073642}
\definecolor{solarized@base01}{HTML}{586e75}
\definecolor{solarized@base00}{HTML}{657b83}
\definecolor{solarized@base0}{HTML}{839496}
\definecolor{solarized@base1}{HTML}{93a1a1}
\definecolor{solarized@base2}{HTML}{EEE8D5}
\definecolor{solarized@base3}{HTML}{FDF6E3}
\definecolor{solarized@yellow}{HTML}{B58900}
\definecolor{solarized@orange}{HTML}{CB4B16}
\definecolor{solarized@red}{HTML}{DC322F}
\definecolor{solarized@magenta}{HTML}{D33682}
\definecolor{solarized@violet}{HTML}{6C71C4}
\definecolor{solarized@blue}{HTML}{268BD2}
\definecolor{solarized@cyan}{HTML}{2AA198}
\definecolor{solarized@green}{HTML}{859900}

\lstset{
  language=Matlab,
  columns=fixed,
  tabsize=2,
  extendedchars=true,
  breaklines=true,
  frame=single,
  numbers=left,
  numbersep=5pt,
  rulesepcolor=\color{solarized@base03},
  numberstyle=\tiny\color{solarized@base01},
  basicstyle=\footnotesize\ttfamily,
  keywordstyle=\color{solarized@green},
  stringstyle=\color{solarized@cyan}\ttfamily,
  identifierstyle=\color{solarized@blue},
  commentstyle=\color{solarized@base01},
  emphstyle=\color{solarized@red},
  showstringspaces=false,
  float=htpb
}

\relpenalty=9999
\binoppenalty=9999

\title{Homework Set 4}%replace X with the appropriate number
\author{Jay Mundrawala}

\begin{document}
%%%%%%%%%%%% PROBLEM #1 %%%%%%%%%%%%%%%%%%%%%%%%%
\problem{}
Let A denote the $5\times 5$ Hilbert matrix. Using the guess
$\left(1,1,1,1,1\right)^T$, how many power iteration steps are 
needed to get the dominant eigenvalue correct to twelve
significant digits? How many steps are necessary if you use 
Rayleigh quotient iteration instead?
\solution
\lstinputlisting{./hw4_01/hw4_01.m}
\lstinputlisting{./hw4_01/power_iteration.m}
\lstinputlisting{./hw4_01/rayleigh_iteration.m}

\begin{figure}
  \label{fig:hw4_01}
  	\centering
	\scalebox{.5}{
		\newlength\figureheight
		\newlength\figurewidth
		\setlength\figureheight{12cm}
		\setlength\figurewidth{\linewidth} 
		\input{./hw4_01/hw4_01_plot.tikz}
	}
  \caption{Problem 1 - Iterations vs Correct Digits}
\end{figure}

Figure 1 shows that the eigenvalue has 12 correct 
significant after 4 iterations using the Rayleight quotient iteration,
while it takes 15 iterations using the power iteration technique.


\problem{Scientific Computing, Computer Problem 4.2, page 211}
\begin{enumerate}
  \item Implement power iteration to compute the dominant value 
	and corresponding eigen vector of the matrix
	\[
	A =
	\begin{bmatrix}
	  2 & 3 & 2 \\
	  10 & 3 & 4 \\
	  3 & 6 & 1
	\end{bmatrix}
	\]
  \item Using any of the methods for deflation given in Section 4.5.4,
	deflate out the eigen value found in part \textit{a} and apply 
	power iteration again to compute the second largest eigenvalue 
	of the same matrix.
  \item Use a general real eigensystem library routine to compute 
	all of the eigenvalues and eigenvectors of the matrix, and compare
	the results with those obtained in parts \textit{a} and \textit{b}.
\end{enumerate}
\solution
\part
\lstinputlisting{./hw4_02/power_iteration.m}
\part
\lstinputlisting{./hw4_02/deflation.m}
\lstinputlisting{./hw4_02/householder.m}
\part
\lstinputlisting{./hw4_02/hw4_02.m}

The results show that the eigenvalue computation comes close to the 
value computed by the library routine for all 3 eigenvectors, and 
the first 2 eigenvalues. However the final eigenvalue is completely
wrong. Its not clear why as it seems to work for some matricies
and not others.

\problem{Scientific Computing, Computer Problem 4.3, page 211}
\begin{enumerate}
  \item Implement inverse iteration with a shift to compute the eigenvalue 
	nearest to 2, and the corresponding eigenvector, of matrix
	\[
	A =
	\begin{bmatrix}
	  6 & 2 & 1 \\
	  2 & 3 & 1 \\
	  1 & 1 & 1 \\
	\end{bmatrix}
	\]
  \item Use a real symmetric eigensystem library routine to compute 
	all the eigenvalues and eigenvectors and compare the results with
	those obtained in part \textit{a}.
\end{enumerate}
\solution 
\part
\lstinputlisting{./hw4_03/inverse_iteration.m}
\part
\lstinputlisting{./hw4_03/hw4_03.m}
\begin{figure}[H]
  \label{fig:hw4_02}
  	\centering
	\scalebox{.7}{
		\setlength\figureheight{12cm}
		\setlength\figurewidth{\linewidth} 
		\input{./hw4_03/hw4_03_plot.tikz}
	}
  \caption{Problem 3 - Iterations vs Correct Digits}
\end{figure}

\end{document}

