\documentclass{jhwhw}
\usepackage{listings}
\usepackage[T1]{fontenc}
%\usepackage{xcolor}
\usepackage{color}
\usepackage{moreverb}
\usepackage{paralist}
\usepackage{multicol}
\usepackage{tikz,pgfplots}
\usepackage{gnuplot-lua-tikz}
\usepackage{adjustbox}
\usepackage{float}
\usepackage{cleveref}

\definecolor{MyDarkGreen}{rgb}{0.0,0.4,0.0}
\lstloadlanguages{Matlab}%

\newfloat{Code}{H}{myc}

\definecolor{solarized@base03}{HTML}{002B36}
\definecolor{solarized@base02}{HTML}{073642}
\definecolor{solarized@base01}{HTML}{586e75} 
\definecolor{solarized@base00}{HTML}{657b83}
\definecolor{solarized@base0}{HTML}{839496}
\definecolor{solarized@base1}{HTML}{93a1a1}
\definecolor{solarized@base2}{HTML}{EEE8D5}
\definecolor{solarized@base3}{HTML}{FDF6E3}
\definecolor{solarized@yellow}{HTML}{B58900}
\definecolor{solarized@orange}{HTML}{CB4B16}
\definecolor{solarized@red}{HTML}{DC322F}
\definecolor{solarized@magenta}{HTML}{D33682}
\definecolor{solarized@violet}{HTML}{6C71C4}
\definecolor{solarized@blue}{HTML}{268BD2}
\definecolor{solarized@cyan}{HTML}{2AA198}
\definecolor{solarized@green}{HTML}{859900}

\lstset{
  language=Matlab,
  columns=fixed,
  tabsize=2,
  extendedchars=true,
  breaklines=true,
  frame=single,
  numbers=left,
  numbersep=5pt,
  rulesepcolor=\color{solarized@base03},
  numberstyle=\tiny\color{solarized@base01},
  basicstyle=\footnotesize\ttfamily,
  keywordstyle=\color{solarized@green},
  stringstyle=\color{solarized@cyan}\ttfamily,
  identifierstyle=\color{solarized@blue},
  commentstyle=\color{solarized@base01},
  emphstyle=\color{solarized@red},
  showstringspaces=false,
  float=htpb
}

\relpenalty=9999
\binoppenalty=9999

\title{Homework Set 5}%replace X with the appropriate number
\author{Jay Mundrawala}

\begin{document}
%%%%%%%%%%%% PROBLEM #1 %%%%%%%%%%%%%%%%%%%%%%%%%
\problem{Exercise 5.6 in Scientific Computing, page 249}
Suppose we wish to develop an iterative method to compute the square root of a given positive number $y$, i.e., to 
solve the nonlinear equation $f(x) = x^2 -y = 0$ given the value of $y$. Each of the functions $g_1$ and $g_2$
listed next gives a fixed-point problem that is equivalent to the equation $f(x) = 0$. For each of these functions,
determine whether the corresponding fixed-point iteration scheme $x_{k+1} = g_i(x_k)$ is locally convergent to 
$\sqrt{y}$ if $y = 3$. Explain your reasoning in each case.
\begin{enumerate}
	\item $g_1(x) = y + x - x^2$.
	\item $g_2(x) = 1 + x - x^2/y$.
	\item What is the fixed-point iteration function given by Newton's method for this particular problem?
\end{enumerate}
\solution
\part
\begin{eqnarray*}
	g_1(x) &=& 3 + x - x^2 \\
	g_1^\prime (x) &=& -2x + 1
\end{eqnarray*}
The solution is known to lie between $(1,2)$. $g_1^\prime(x)$ takes the values $(-1,-3)$ for $x\epsilon(1,2)$.
This means that the iteration does not locally converge.
\part
\begin{eqnarray*}
	g_1(x) &=& 1 + x - \frac{x^2}{3} \\
	g_1^\prime (x) &=& 1 - \frac{2}{3}x
\end{eqnarray*}
The solution is known to lie between $(1,2)$. $g_2^\prime(x)$ takes the values $(-0.3,0.3)$ for $x\epsilon(1,2)$.
This means that the iteration locally converges.
\part
%\begin{eqnarray*}
	%x_{k+1} &=& x_k - \frac{f(x_k)}{f^\prime(x_k)} \\
	%f(x) - x^2 - y &=& 0 \\
	%f^\prime(x) &=& 2x \\
	%x_{k+1} &=& g(x_k) \\
%\end{eqnarray*}
\begin{eqnarray*}
	g(x_k) &=& x_k - \frac{x_k^2 - y}{2x_k} \\
           &=& x_k - \frac{1}{2}x_k + \frac{y}{2x_k} \\
	       &=& \frac{1}{2}x_k + \frac{y}{2x_k} \\
	       &=& \frac{1}{2}\left(x_k + \frac{y}{x_k}\right) 
\end{eqnarray*}

%%%%%%%%%%%% PROBLEM #2 %%%%%%%%%%%%%%%%%%%%%%%%%
\problem{Exercise 5.1 in Scientific Computing, page 250}
\begin{enumerate}
	\item How many zeroes does the function
		\(
			f(x) = sin(10x) - x
		\)
		have?
	\item Use a library routine or one of your own design to find all of the zeros of this function.
\end{enumerate}
\solution
\part
\begin{figure}[H]
  \label{fighw502}
  \centering
  \scalebox{.5}{
  \newlength\figureheight
  \newlength\figurewidth
  \setlength\figureheight{12cm}
  \setlength\figurewidth{\linewidth} 
  % This file was created by matlab2tikz v0.2.3.
% Copyright (c) 2008--2012, Nico Schlömer <nico.schloemer@gmail.com>
% All rights reserved.
% 
% The latest updates can be retrieved from
%   http://www.mathworks.com/matlabcentral/fileexchange/22022-matlab2tikz
% where you can also make suggestions and rate matlab2tikz.
% 
% 
% 
\begin{tikzpicture}

\begin{axis}[%
view={0}{90},
width=\figurewidth,
height=\figureheight,
scale only axis,
xmin=-2, xmax=2,
xmajorgrids,
ymin=-3, ymax=3,
ymajorgrids,
legend style={fill=none,draw=none,nodes=left,legend plot pos=right}]
\addplot [
color=blue,
solid,
forget plot
]
coordinates{
 (-2,1.08705474927237)(-1.99986049107143,1.08748543949197)(-1.99972098214286,1.08791790544322)(-1.99958147321429,1.08835214601291)(-1.99944196428571,1.08878816008437)(-1.99930245535714,1.08922594653747)(-1.99916294642857,1.08966550424862)(-1.9990234375,1.09010683209083)(-1.99888392857143,1.09054992893361)(-1.99874441964286,1.09099479364305)(-1.99860491071429,1.0914414250818)(-1.99846540178571,1.09188982210908)(-1.99832589285714,1.09233998358066)(-1.99818638392857,1.09279190834888)(-1.998046875,1.09324559526264)(-1.99790736607143,1.09370104316743)(-1.99776785714286,1.09415825090529)(-1.99762834821429,1.09461721731485)(-1.99748883928571,1.09507794123132)(-1.99734933035714,1.09554042148648)(-1.99720982142857,1.09600465690868)(-1.9970703125,1.09647064632289)(-1.99693080357143,1.09693838855063)(-1.99679129464286,1.09740788241003)(-1.99665178571429,1.0978791267158)(-1.99651227678571,1.09835212027924)(-1.99637276785714,1.09882686190827)(-1.99623325892857,1.09930335040737)(-1.99609375,1.09978158457766)(-1.99595424107143,1.10026156321684)(-1.99581473214286,1.1007432851192)(-1.99567522321429,1.10122674907567)(-1.99553571428571,1.10171195387377)(-1.99539620535714,1.10219889829763)(-1.99525669642857,1.10268758112801)(-1.9951171875,1.10317800114228)(-1.99497767857143,1.10367015711441)(-1.99483816964286,1.10416404781502)(-1.99469866071429,1.10465967201134)(-1.99455915178571,1.10515702846722)(-1.99441964285714,1.10565611594316)(-1.99428013392857,1.10615693319625)(-1.994140625,1.10665947898027)(-1.99400111607143,1.10716375204558)(-1.99386160714286,1.10766975113923)(-1.99372209821429,1.10817747500486)(-1.99358258928571,1.10868692238279)(-1.99344308035714,1.10919809200997)(-1.99330357142857,1.10971098262001)(-1.9931640625,1.11022559294315)(-1.99302455357143,1.1107419217063)(-1.99288504464286,1.11125996763302)(-1.99274553571429,1.11177972944353)(-1.99260602678571,1.11230120585471)(-1.99246651785714,1.11282439558009)(-1.99232700892857,1.11334929732989)(-1.9921875,1.11387590981098)(-1.99204799107143,1.11440423172691)(-1.99190848214286,1.11493426177789)(-1.99176897321429,1.11546599866082)(-1.99162946428571,1.11599944106927)(-1.99148995535714,1.11653458769349)(-1.99135044642857,1.11707143722042)(-1.9912109375,1.11760998833367)(-1.99107142857143,1.11815023971357)(-1.99093191964286,1.1186921900371)(-1.99079241071429,1.11923583797797)(-1.99065290178571,1.11978118220656)(-1.99051339285714,1.12032822138996)(-1.99037388392857,1.12087695419196)(-1.990234375,1.12142737927305)(-1.99009486607143,1.12197949529044)(-1.98995535714286,1.12253330089802)(-1.98981584821429,1.12308879474643)(-1.98967633928571,1.123645975483)(-1.98953683035714,1.12420484175177)(-1.98939732142857,1.12476539219351)(-1.9892578125,1.12532762544573)(-1.98911830357143,1.12589154014263)(-1.98897879464286,1.12645713491517)(-1.98883928571429,1.12702440839102)(-1.98869977678571,1.12759335919458)(-1.98856026785714,1.128163985947)(-1.98842075892857,1.12873628726616)(-1.98828125,1.12931026176668)(-1.98814174107143,1.12988590805993)(-1.98800223214286,1.13046322475402)(-1.98786272321429,1.13104221045382)(-1.98772321428571,1.13162286376093)(-1.98758370535714,1.13220518327372)(-1.98744419642857,1.13278916758732)(-1.9873046875,1.13337481529361)(-1.98716517857143,1.13396212498124)(-1.98702566964286,1.13455109523561)(-1.98688616071429,1.13514172463891)(-1.98674665178571,1.13573401177009)(-1.98660714285714,1.13632795520488)(-1.98646763392857,1.13692355351577)(-1.986328125,1.13752080527205)(-1.98618861607143,1.13811970903978)(-1.98604910714286,1.13872026338179)(-1.98590959821429,1.13932246685774)(-1.98577008928571,1.13992631802404)(-1.98563058035714,1.14053181543391)(-1.98549107142857,1.14113895763736)(-1.9853515625,1.14174774318121)(-1.98521205357143,1.14235817060907)(-1.98507254464286,1.14297023846136)(-1.98493303571429,1.14358394527531)(-1.98479352678571,1.14419928958495)(-1.98465401785714,1.14481626992112)(-1.98451450892857,1.1454348848115)(-1.984375,1.14605513278057)(-1.98423549107143,1.14667701234963)(-1.98409598214286,1.14730052203682)(-1.98395647321429,1.14792566035708)(-1.98381696428571,1.14855242582221)(-1.98367745535714,1.14918081694083)(-1.98353794642857,1.14981083221839)(-1.9833984375,1.15044247015719)(-1.98325892857143,1.15107572925636)(-1.98311941964286,1.15171060801189)(-1.98297991071429,1.15234710491661)(-1.98284040178571,1.15298521846019)(-1.98270089285714,1.15362494712918)(-1.98256138392857,1.15426628940696)(-1.982421875,1.15490924377377)(-1.98228236607143,1.15555380870674)(-1.98214285714286,1.15619998267984)(-1.98200334821429,1.15684776416392)(-1.98186383928571,1.15749715162669)(-1.98172433035714,1.15814814353275)(-1.98158482142857,1.15880073834356)(-1.9814453125,1.15945493451747)(-1.98130580357143,1.16011073050972)(-1.98116629464286,1.16076812477242)(-1.98102678571429,1.16142711575459)(-1.98088727678571,1.16208770190212)(-1.98074776785714,1.16274988165781)(-1.98060825892857,1.16341365346136)(-1.98046875,1.16407901574935)(-1.98032924107143,1.1647459669553)(-1.98018973214286,1.1654145055096)(-1.98005022321429,1.16608462983958)(-1.97991071428571,1.16675633836946)(-1.97977120535714,1.16742962952041)(-1.97963169642857,1.16810450171047)(-1.9794921875,1.16878095335466)(-1.97935267857143,1.16945898286488)(-1.97921316964286,1.17013858864998)(-1.97907366071429,1.17081976911573)(-1.97893415178571,1.17150252266487)(-1.97879464285714,1.17218684769702)(-1.97865513392857,1.17287274260879)(-1.978515625,1.17356020579372)(-1.97837611607143,1.1742492356423)(-1.97823660714286,1.17493983054194)(-1.97809709821429,1.17563198887706)(-1.97795758928571,1.176325709029)(-1.97781808035714,1.17702098937607)(-1.97767857142857,1.17771782829352)(-1.9775390625,1.17841622415362)(-1.97739955357143,1.17911617532556)(-1.97726004464286,1.17981768017551)(-1.97712053571429,1.18052073706666)(-1.97698102678571,1.18122534435911)(-1.97684151785714,1.18193150041001)(-1.97670200892857,1.18263920357344)(-1.9765625,1.18334845220051)(-1.97642299107143,1.18405924463929)(-1.97628348214286,1.18477157923488)(-1.97614397321429,1.18548545432934)(-1.97600446428571,1.18620086826176)(-1.97586495535714,1.18691781936823)(-1.97572544642857,1.18763630598184)(-1.9755859375,1.18835632643269)(-1.97544642857143,1.18907787904791)(-1.97530691964286,1.18980096215164)(-1.97516741071429,1.19052557406503)(-1.97502790178571,1.19125171310626)(-1.97488839285714,1.19197937759056)(-1.97474888392857,1.19270856583016)(-1.974609375,1.19343927613435)(-1.97446986607143,1.19417150680943)(-1.97433035714286,1.19490525615876)(-1.97419084821429,1.19564052248274)(-1.97405133928571,1.19637730407883)(-1.97391183035714,1.19711559924152)(-1.97377232142857,1.19785540626235)(-1.9736328125,1.19859672342995)(-1.97349330357143,1.19933954902998)(-1.97335379464286,1.20008388134518)(-1.97321428571429,1.20082971865536)(-1.97307477678571,1.20157705923738)(-1.97293526785714,1.20232590136519)(-1.97279575892857,1.20307624330983)(-1.97265625,1.20382808333939)(-1.97251674107143,1.20458141971907)(-1.97237723214286,1.20533625071114)(-1.97223772321429,1.20609257457498)(-1.97209821428571,1.20685038956704)(-1.97195870535714,1.2076096939409)(-1.97181919642857,1.20837048594721)(-1.9716796875,1.20913276383375)(-1.97154017857143,1.20989652584539)(-1.97140066964286,1.2106617702241)(-1.97126116071429,1.21142849520901)(-1.97112165178571,1.21219669903633)(-1.97098214285714,1.2129663799394)(-1.97084263392857,1.21373753614868)(-1.970703125,1.21451016589179)(-1.97056361607143,1.21528426739343)(-1.97042410714286,1.21605983887549)(-1.97028459821429,1.21683687855695)(-1.97014508928571,1.21761538465397)(-1.97000558035714,1.21839535537984)(-1.96986607142857,1.21917678894499)(-1.9697265625,1.21995968355703)(-1.96958705357143,1.2207440374207)(-1.96944754464286,1.22152984873791)(-1.96930803571429,1.22231711570774)(-1.96916852678571,1.22310583652643)(-1.96902901785714,1.22389600938738)(-1.96888950892857,1.22468763248118)(-1.96875,1.2254807039956)(-1.96861049107143,1.22627522211557)(-1.96847098214286,1.22707118502322)(-1.96833147321429,1.22786859089788)(-1.96819196428571,1.22866743791605)(-1.96805245535714,1.22946772425142)(-1.96791294642857,1.23026944807491)(-1.9677734375,1.23107260755461)(-1.96763392857143,1.23187720085583)(-1.96749441964286,1.23268322614109)(-1.96735491071429,1.23349068157013)(-1.96721540178571,1.23429956529989)(-1.96707589285714,1.23510987548454)(-1.96693638392857,1.23592161027547)(-1.966796875,1.2367347678213)(-1.96665736607143,1.23754934626789)(-1.96651785714286,1.2383653437583)(-1.96637834821429,1.23918275843288)(-1.96623883928571,1.24000158842918)(-1.96609933035714,1.24082183188202)(-1.96595982142857,1.24164348692344)(-1.9658203125,1.24246655168278)(-1.96568080357143,1.24329102428658)(-1.96554129464286,1.24411690285868)(-1.96540178571429,1.24494418552018)(-1.96526227678571,1.24577287038943)(-1.96512276785714,1.24660295558206)(-1.96498325892857,1.24743443921097)(-1.96484375,1.24826731938635)(-1.96470424107143,1.24910159421566)(-1.96456473214286,1.24993726180366)(-1.96442522321429,1.25077432025238)(-1.96428571428571,1.25161276766115)(-1.96414620535714,1.2524526021266)(-1.96400669642857,1.25329382174267)(-1.9638671875,1.25413642460059)(-1.96372767857143,1.25498040878889)(-1.96358816964286,1.25582577239344)(-1.96344866071429,1.25667251349739)(-1.96330915178571,1.25752063018124)(-1.96316964285714,1.25837012052281)(-1.96303013392857,1.25922098259721)(-1.962890625,1.26007321447693)(-1.96275111607143,1.26092681423175)(-1.96261160714286,1.26178177992883)(-1.96247209821429,1.26263810963264)(-1.96233258928571,1.263495801405)(-1.96219308035714,1.2643548533051)(-1.96205357142857,1.26521526338945)(-1.9619140625,1.26607702971195)(-1.96177455357143,1.26694015032383)(-1.96163504464286,1.2678046232737)(-1.96149553571429,1.26867044660754)(-1.96135602678571,1.2695376183687)(-1.96121651785714,1.27040613659791)(-1.96107700892857,1.27127599933326)(-1.9609375,1.27214720461024)(-1.96079799107143,1.27301975046173)(-1.96065848214286,1.27389363491798)(-1.96051897321429,1.27476885600667)(-1.96037946428571,1.27564541175284)(-1.96023995535714,1.27652330017896)(-1.96010044642857,1.27740251930489)(-1.9599609375,1.27828306714791)(-1.95982142857143,1.2791649417227)(-1.95968191964286,1.28004814104139)(-1.95954241071429,1.28093266311349)(-1.95940290178571,1.28181850594596)(-1.95926339285714,1.28270566754319)(-1.95912388392857,1.28359414590699)(-1.958984375,1.28448393903662)(-1.95884486607143,1.28537504492878)(-1.95870535714286,1.28626746157761)(-1.95856584821429,1.28716118697469)(-1.95842633928571,1.28805621910908)(-1.95828683035714,1.28895255596727)(-1.95814732142857,1.28985019553321)(-1.9580078125,1.29074913578835)(-1.95786830357143,1.29164937471156)(-1.95772879464286,1.29255091027922)(-1.95758928571429,1.29345374046517)(-1.95744977678571,1.29435786324073)(-1.95731026785714,1.29526327657471)(-1.95717075892857,1.29616997843339)(-1.95703125,1.29707796678059)(-1.95689174107143,1.29798723957756)(-1.95675223214286,1.2988977947831)(-1.95661272321429,1.2998096303535)(-1.95647321428571,1.30072274424255)(-1.95633370535714,1.30163713440156)(-1.95619419642857,1.30255279877935)(-1.9560546875,1.30346973532226)(-1.95591517857143,1.30438794197418)(-1.95577566964286,1.30530741667648)(-1.95563616071429,1.3062281573681)(-1.95549665178571,1.3071501619855)(-1.95535714285714,1.30807342846269)(-1.95521763392857,1.30899795473121)(-1.955078125,1.30992373872016)(-1.95493861607143,1.31085077835619)(-1.95479910714286,1.3117790715635)(-1.95465959821429,1.31270861626386)(-1.95452008928571,1.3136394103766)(-1.95438058035714,1.31457145181862)(-1.95424107142857,1.31550473850437)(-1.9541015625,1.31643926834592)(-1.95396205357143,1.31737503925288)(-1.95382254464286,1.31831204913246)(-1.95368303571429,1.31925029588948)(-1.95354352678571,1.32018977742631)(-1.95340401785714,1.32113049164296)(-1.95326450892857,1.322072436437)(-1.953125,1.32301560970363)(-1.95298549107143,1.32396000933566)(-1.95284598214286,1.3249056332235)(-1.95270647321429,1.3258524792552)(-1.95256696428571,1.3268005453164)(-1.95242745535714,1.32774982929038)(-1.95228794642857,1.32870032905806)(-1.9521484375,1.32965204249798)(-1.95200892857143,1.33060496748633)(-1.95186941964286,1.33155910189692)(-1.95172991071429,1.33251444360124)(-1.95159040178571,1.33347099046839)(-1.95145089285714,1.33442874036516)(-1.95131138392857,1.33538769115597)(-1.951171875,1.33634784070294)(-1.95103236607143,1.33730918686581)(-1.95089285714286,1.33827172750202)(-1.95075334821429,1.33923546046669)(-1.95061383928571,1.34020038361261)(-1.95047433035714,1.34116649479023)(-1.95033482142857,1.34213379184774)(-1.9501953125,1.34310227263097)(-1.95005580357143,1.34407193498348)(-1.94991629464286,1.34504277674652)(-1.94977678571429,1.34601479575903)(-1.94963727678571,1.34698798985769)(-1.94949776785714,1.34796235687687)(-1.94935825892857,1.34893789464865)(-1.94921875,1.34991460100286)(-1.94907924107143,1.35089247376704)(-1.94893973214286,1.35187151076644)(-1.94880022321429,1.35285170982408)(-1.94866071428571,1.3538330687607)(-1.94852120535714,1.35481558539478)(-1.94838169642857,1.35579925754255)(-1.9482421875,1.356784083018)(-1.94810267857143,1.35777005963286)(-1.94796316964286,1.35875718519662)(-1.94782366071429,1.35974545751656)(-1.94768415178571,1.3607348743977)(-1.94754464285714,1.36172543364284)(-1.94740513392857,1.36271713305255)(-1.947265625,1.36370997042521)(-1.94712611607143,1.36470394355694)(-1.94698660714286,1.36569905024169)(-1.94684709821429,1.36669528827118)(-1.94670758928571,1.36769265543494)(-1.94656808035714,1.36869114952029)(-1.94642857142857,1.36969076831237)(-1.9462890625,1.37069150959413)(-1.94614955357143,1.37169337114633)(-1.94601004464286,1.37269635074753)(-1.94587053571429,1.37370044617417)(-1.94573102678571,1.37470565520046)(-1.94559151785714,1.37571197559846)(-1.94545200892857,1.37671940513809)(-1.9453125,1.37772794158708)(-1.94517299107143,1.37873758271103)(-1.94503348214286,1.37974832627337)(-1.94489397321429,1.3807601700354)(-1.94475446428571,1.38177311175628)(-1.94461495535714,1.38278714919301)(-1.94447544642857,1.38380228010047)(-1.9443359375,1.38481850223143)(-1.94419642857143,1.38583581333652)(-1.94405691964286,1.38685421116424)(-1.94391741071429,1.38787369346099)(-1.94377790178571,1.38889425797106)(-1.94363839285714,1.38991590243663)(-1.94349888392857,1.39093862459778)(-1.943359375,1.39196242219248)(-1.94321986607143,1.39298729295662)(-1.94308035714286,1.394013234624)(-1.94294084821429,1.39504024492634)(-1.94280133928571,1.39606832159327)(-1.94266183035714,1.39709746235234)(-1.94252232142857,1.39812766492905)(-1.9423828125,1.39915892704681)(-1.94224330357143,1.400191246427)(-1.94210379464286,1.40122462078889)(-1.94196428571429,1.40225904784976)(-1.94182477678571,1.40329452532479)(-1.94168526785714,1.40433105092713)(-1.94154575892857,1.40536862236791)(-1.94140625,1.4064072373562)(-1.94126674107143,1.40744689359905)(-1.94112723214286,1.40848758880149)(-1.94098772321429,1.40952932066651)(-1.94084821428571,1.41057208689509)(-1.94070870535714,1.41161588518621)(-1.94056919642857,1.41266071323681)(-1.9404296875,1.41370656874187)(-1.94029017857143,1.41475344939434)(-1.94015066964286,1.41580135288517)(-1.94001116071429,1.41685027690333)(-1.93987165178571,1.41790021913582)(-1.93973214285714,1.41895117726764)(-1.93959263392857,1.42000314898179)(-1.939453125,1.42105613195936)(-1.93931361607143,1.4221101238794)(-1.93917410714286,1.42316512241905)(-1.93903459821429,1.42422112525348)(-1.93889508928571,1.42527813005587)(-1.93875558035714,1.4263361344975)(-1.93861607142857,1.42739513624766)(-1.9384765625,1.42845513297374)(-1.93833705357143,1.42951612234116)(-1.93819754464286,1.43057810201343)(-1.93805803571429,1.43164106965212)(-1.93791852678571,1.43270502291689)(-1.93777901785714,1.43376995946546)(-1.93763950892857,1.43483587695366)(-1.9375,1.43590277303539)(-1.93736049107143,1.43697064536266)(-1.93722098214286,1.43803949158557)(-1.93708147321429,1.43910930935235)(-1.93694196428571,1.4401800963093)(-1.93680245535714,1.44125185010086)(-1.93666294642857,1.44232456836957)(-1.9365234375,1.44339824875612)(-1.93638392857143,1.44447288889931)(-1.93624441964286,1.44554848643606)(-1.93610491071429,1.44662503900145)(-1.93596540178571,1.44770254422869)(-1.93582589285714,1.44878099974914)(-1.93568638392857,1.4498604031923)(-1.935546875,1.45094075218584)(-1.93540736607143,1.45202204435558)(-1.93526785714286,1.45310427732551)(-1.93512834821429,1.45418744871778)(-1.93498883928571,1.45527155615272)(-1.93484933035714,1.45635659724884)(-1.93470982142857,1.45744256962283)(-1.9345703125,1.45852947088956)(-1.93443080357143,1.45961729866212)(-1.93429129464286,1.46070605055176)(-1.93415178571429,1.46179572416795)(-1.93401227678571,1.46288631711837)(-1.93387276785714,1.4639778270089)(-1.93373325892857,1.46507025144364)(-1.93359375,1.46616358802491)(-1.93345424107143,1.46725783435326)(-1.93331473214286,1.46835298802746)(-1.93317522321429,1.46944904664453)(-1.93303571428571,1.4705460077997)(-1.93289620535714,1.47164386908647)(-1.93275669642857,1.47274262809657)(-1.9326171875,1.47384228242001)(-1.93247767857143,1.47494282964502)(-1.93233816964286,1.47604426735812)(-1.93219866071429,1.47714659314408)(-1.93205915178571,1.47824980458597)(-1.93191964285714,1.4793538992651)(-1.93178013392857,1.48045887476107)(-1.931640625,1.48156472865179)(-1.93150111607143,1.48267145851343)(-1.93136160714286,1.48377906192048)(-1.93122209821429,1.4848875364457)(-1.93108258928571,1.48599687966019)(-1.93094308035714,1.48710708913334)(-1.93080357142857,1.48821816243284)(-1.9306640625,1.48933009712473)(-1.93052455357143,1.49044289077336)(-1.93038504464286,1.49155654094139)(-1.93024553571429,1.49267104518985)(-1.93010602678571,1.49378640107807)(-1.92996651785714,1.49490260616374)(-1.92982700892857,1.49601965800291)(-1.9296875,1.49713755414996)(-1.92954799107143,1.49825629215764)(-1.92940848214286,1.49937586957704)(-1.92926897321429,1.50049628395766)(-1.92912946428571,1.50161753284733)(-1.92898995535714,1.50273961379227)(-1.92885044642857,1.50386252433708)(-1.9287109375,1.50498626202475)(-1.92857142857143,1.50611082439665)(-1.92843191964286,1.50723620899255)(-1.92829241071429,1.50836241335064)(-1.92815290178571,1.50948943500747)(-1.92801339285714,1.51061727149805)(-1.92787388392857,1.51174592035575)(-1.927734375,1.51287537911241)(-1.92759486607143,1.51400564529826)(-1.92745535714286,1.51513671644198)(-1.92731584821429,1.51626859007066)(-1.92717633928571,1.51740126370986)(-1.92703683035714,1.51853473488354)(-1.92689732142857,1.51966900111415)(-1.9267578125,1.52080405992258)(-1.92661830357143,1.52193990882815)(-1.92647879464286,1.52307654534868)(-1.92633928571429,1.52421396700044)(-1.92619977678571,1.52535217129818)(-1.92606026785714,1.5264911557551)(-1.92592075892857,1.52763091788292)(-1.92578125,1.52877145519182)(-1.92564174107143,1.52991276519048)(-1.92550223214286,1.53105484538608)(-1.92536272321429,1.53219769328429)(-1.92522321428571,1.5333413063893)(-1.92508370535714,1.5344856822038)(-1.92494419642857,1.535630818229)(-1.9248046875,1.53677671196463)(-1.92466517857143,1.53792336090894)(-1.92452566964286,1.53907076255871)(-1.92438616071429,1.54021891440927)(-1.92424665178571,1.54136781395448)(-1.92410714285714,1.54251745868674)(-1.92396763392857,1.543667846097)(-1.923828125,1.54481897367478)(-1.92368861607143,1.54597083890814)(-1.92354910714286,1.5471234392837)(-1.92340959821429,1.54827677228669)(-1.92327008928571,1.54943083540086)(-1.92313058035714,1.55058562610859)(-1.92299107142857,1.55174114189079)(-1.9228515625,1.552897380227)(-1.92271205357143,1.55405433859535)(-1.92257254464286,1.55521201447255)(-1.92243303571429,1.55637040533392)(-1.92229352678571,1.5575295086534)(-1.92215401785714,1.55868932190353)(-1.92201450892857,1.55984984255548)(-1.921875,1.56101106807903)(-1.92173549107143,1.56217299594259)(-1.92159598214286,1.56333562361321)(-1.92145647321429,1.56449894855659)(-1.92131696428571,1.56566296823704)(-1.92117745535714,1.56682768011755)(-1.92103794642857,1.56799308165974)(-1.9208984375,1.5691591703239)(-1.92075892857143,1.57032594356899)(-1.92061941964286,1.5714933988526)(-1.92047991071429,1.57266153363105)(-1.92034040178571,1.57383034535929)(-1.92020089285714,1.57499983149097)(-1.92006138392857,1.57616998947844)(-1.919921875,1.57734081677271)(-1.91978236607143,1.57851231082353)(-1.91964285714286,1.57968446907931)(-1.91950334821429,1.58085728898719)(-1.91936383928571,1.58203076799303)(-1.91922433035714,1.58320490354139)(-1.91908482142857,1.58437969307555)(-1.9189453125,1.58555513403753)(-1.91880580357143,1.58673122386808)(-1.91866629464286,1.58790796000669)(-1.91852678571429,1.58908533989158)(-1.91838727678571,1.59026336095972)(-1.91824776785714,1.59144202064684)(-1.91810825892857,1.59262131638743)(-1.91796875,1.59380124561472)(-1.91782924107143,1.59498180576074)(-1.91768973214286,1.59616299425625)(-1.91755022321429,1.59734480853084)(-1.91741071428571,1.59852724601283)(-1.91727120535714,1.59971030412937)(-1.91713169642857,1.60089398030636)(-1.9169921875,1.60207827196854)(-1.91685267857143,1.60326317653942)(-1.91671316964286,1.60444869144133)(-1.91657366071429,1.60563481409541)(-1.91643415178571,1.60682154192162)(-1.91629464285714,1.60800887233875)(-1.91615513392857,1.60919680276439)(-1.916015625,1.61038533061499)(-1.91587611607143,1.61157445330582)(-1.91573660714286,1.61276416825101)(-1.91559709821429,1.61395447286352)(-1.91545758928571,1.61514536455516)(-1.91531808035714,1.61633684073662)(-1.91517857142857,1.61752889881743)(-1.9150390625,1.61872153620601)(-1.91489955357143,1.61991475030962)(-1.91476004464286,1.62110853853442)(-1.91462053571429,1.62230289828546)(-1.91448102678571,1.62349782696665)(-1.91434151785714,1.62469332198083)(-1.91420200892857,1.62588938072969)(-1.9140625,1.62708600061388)(-1.91392299107143,1.6282831790329)(-1.91378348214286,1.62948091338521)(-1.91364397321429,1.63067920106815)(-1.91350446428571,1.63187803947802)(-1.91336495535714,1.63307742601003)(-1.91322544642857,1.6342773580583)(-1.9130859375,1.63547783301593)(-1.91294642857143,1.63667884827494)(-1.91280691964286,1.63788040122631)(-1.91266741071429,1.63908248925995)(-1.91252790178571,1.64028510976476)(-1.91238839285714,1.64148826012858)(-1.91224888392857,1.64269193773823)(-1.912109375,1.64389613997951)(-1.91196986607143,1.64510086423718)(-1.91183035714286,1.64630610789499)(-1.91169084821429,1.64751186833569)(-1.91155133928571,1.64871814294102)(-1.91141183035714,1.64992492909172)(-1.91127232142857,1.65113222416751)(-1.9111328125,1.65234002554716)(-1.91099330357143,1.65354833060844)(-1.91085379464286,1.65475713672812)(-1.91071428571429,1.65596644128201)(-1.91057477678571,1.65717624164496)(-1.91043526785714,1.65838653519084)(-1.91029575892857,1.65959731929257)(-1.91015625,1.6608085913221)(-1.91001674107143,1.66202034865044)(-1.90987723214286,1.66323258864766)(-1.90973772321429,1.6644453086829)(-1.90959821428571,1.66565850612433)(-1.90945870535714,1.66687217833923)(-1.90931919642857,1.66808632269392)(-1.9091796875,1.66930093655384)(-1.90904017857143,1.67051601728349)(-1.90890066964286,1.67173156224645)(-1.90876116071429,1.67294756880545)(-1.90862165178571,1.67416403432225)(-1.90848214285714,1.67538095615778)(-1.90834263392857,1.67659833167204)(-1.908203125,1.67781615822416)(-1.90806361607143,1.6790344331724)(-1.90792410714286,1.68025315387413)(-1.90778459821429,1.68147231768588)(-1.90764508928571,1.68269192196329)(-1.90750558035714,1.68391196406115)(-1.90736607142857,1.68513244133341)(-1.9072265625,1.68635335113317)(-1.90708705357143,1.68757469081267)(-1.90694754464286,1.68879645772332)(-1.90680803571429,1.69001864921573)(-1.90666852678571,1.69124126263963)(-1.90652901785714,1.69246429534398)(-1.90638950892857,1.69368774467688)(-1.90625,1.69491160798566)(-1.90611049107143,1.69613588261682)(-1.90597098214286,1.69736056591605)(-1.90583147321429,1.69858565522827)(-1.90569196428571,1.69981114789759)(-1.90555245535714,1.70103704126736)(-1.90541294642857,1.70226333268011)(-1.9052734375,1.70349001947764)(-1.90513392857143,1.70471709900094)(-1.90499441964286,1.70594456859027)(-1.90485491071429,1.70717242558511)(-1.90471540178571,1.70840066732419)(-1.90457589285714,1.7096292911455)(-1.90443638392857,1.71085829438627)(-1.904296875,1.712087674383)(-1.90415736607143,1.71331742847147)(-1.90401785714286,1.7145475539867)(-1.90387834821429,1.71577804826302)(-1.90373883928571,1.71700890863403)(-1.90359933035714,1.71824013243261)(-1.90345982142857,1.71947171699093)(-1.9033203125,1.72070365964049)(-1.90318080357143,1.72193595771205)(-1.90304129464286,1.72316860853571)(-1.90290178571429,1.72440160944086)(-1.90276227678571,1.72563495775623)(-1.90262276785714,1.72686865080986)(-1.90248325892857,1.72810268592912)(-1.90234375,1.72933706044073)(-1.90220424107143,1.73057177167072)(-1.90206473214286,1.73180681694448)(-1.90192522321429,1.73304219358677)(-1.90178571428571,1.73427789892168)(-1.90164620535714,1.73551393027265)(-1.90150669642857,1.73675028496252)(-1.9013671875,1.73798696031348)(-1.90122767857143,1.7392239536471)(-1.90108816964286,1.74046126228431)(-1.90094866071429,1.74169888354547)(-1.90080915178571,1.7429368147503)(-1.90066964285714,1.74417505321792)(-1.90053013392857,1.74541359626686)(-1.900390625,1.74665244121506)(-1.90025111607143,1.74789158537985)(-1.90011160714286,1.749131026078)(-1.89997209821429,1.75037076062571)(-1.89983258928571,1.75161078633858)(-1.89969308035714,1.75285110053166)(-1.89955357142857,1.75409170051944)(-1.8994140625,1.75533258361584)(-1.89927455357143,1.75657374713426)(-1.89913504464286,1.75781518838751)(-1.89899553571429,1.7590569046879)(-1.89885602678571,1.76029889334718)(-1.89871651785714,1.76154115167658)(-1.89857700892857,1.7627836769868)(-1.8984375,1.76402646658802)(-1.89829799107143,1.76526951778992)(-1.89815848214286,1.76651282790164)(-1.89801897321429,1.76775639423186)(-1.89787946428571,1.76900021408872)(-1.89773995535714,1.77024428477989)(-1.89760044642857,1.77148860361254)(-1.8974609375,1.77273316789336)(-1.89732142857143,1.77397797492858)(-1.89718191964286,1.77522302202393)(-1.89704241071429,1.77646830648469)(-1.89690290178571,1.77771382561566)(-1.89676339285714,1.77895957672121)(-1.89662388392857,1.78020555710525)(-1.896484375,1.78145176407122)(-1.89634486607143,1.78269819492215)(-1.89620535714286,1.78394484696061)(-1.89606584821429,1.78519171748877)(-1.89592633928571,1.78643880380834)(-1.89578683035714,1.78768610322064)(-1.89564732142857,1.78893361302654)(-1.8955078125,1.79018133052655)(-1.89536830357143,1.79142925302072)(-1.89522879464286,1.79267737780874)(-1.89508928571429,1.79392570218991)(-1.89494977678571,1.7951742234631)(-1.89481026785714,1.79642293892684)(-1.89467075892857,1.79767184587925)(-1.89453125,1.79892094161812)(-1.89439174107143,1.80017022344082)(-1.89425223214286,1.80141968864439)(-1.89411272321429,1.8026693345255)(-1.89397321428571,1.8039191583805)(-1.89383370535714,1.80516915750534)(-1.89369419642857,1.80641932919566)(-1.8935546875,1.80766967074677)(-1.89341517857143,1.80892017945364)(-1.89327566964286,1.81017085261092)(-1.89313616071429,1.81142168751292)(-1.89299665178571,1.81267268145366)(-1.89285714285714,1.81392383172684)(-1.89271763392857,1.81517513562585)(-1.892578125,1.8164265904438)(-1.89243861607143,1.81767819347348)(-1.89229910714286,1.81892994200742)(-1.89215959821429,1.82018183333784)(-1.89202008928571,1.82143386475669)(-1.89188058035714,1.82268603355567)(-1.89174107142857,1.82393833702618)(-1.8916015625,1.82519077245937)(-1.89146205357143,1.82644333714615)(-1.89132254464286,1.82769602837715)(-1.89118303571429,1.82894884344277)(-1.89104352678571,1.83020177963317)(-1.89090401785714,1.83145483423827)(-1.89076450892857,1.83270800454776)(-1.890625,1.8339612878511)(-1.89048549107143,1.83521468143754)(-1.89034598214286,1.8364681825961)(-1.89020647321429,1.83772178861562)(-1.89006696428571,1.8389754967847)(-1.88992745535714,1.84022930439177)(-1.88978794642857,1.84148320872504)(-1.8896484375,1.84273720707257)(-1.88950892857143,1.84399129672219)(-1.88936941964286,1.84524547496158)(-1.88922991071429,1.84649973907826)(-1.88909040178571,1.84775408635954)(-1.88895089285714,1.84900851409262)(-1.88881138392857,1.85026301956449)(-1.888671875,1.85151760006205)(-1.88853236607143,1.85277225287199)(-1.88839285714286,1.8540269752809)(-1.88825334821429,1.85528176457523)(-1.88811383928571,1.85653661804129)(-1.88797433035714,1.85779153296527)(-1.88783482142857,1.85904650663323)(-1.8876953125,1.86030153633114)(-1.88755580357143,1.86155661934484)(-1.88741629464286,1.86281175296006)(-1.88727678571429,1.86406693446246)(-1.88713727678571,1.86532216113759)(-1.88699776785714,1.86657743027091)(-1.88685825892857,1.86783273914779)(-1.88671875,1.86908808505355)(-1.88657924107143,1.8703434652734)(-1.88643973214286,1.87159887709252)(-1.88630022321429,1.87285431779601)(-1.88616071428571,1.87410978466891)(-1.88602120535714,1.87536527499622)(-1.88588169642857,1.87662078606289)(-1.8857421875,1.87787631515383)(-1.88560267857143,1.8791318595539)(-1.88546316964286,1.88038741654796)(-1.88532366071429,1.88164298342083)(-1.88518415178571,1.8828985574573)(-1.88504464285714,1.88415413594217)(-1.88490513392857,1.8854097161602)(-1.884765625,1.88666529539619)(-1.88462611607143,1.88792087093489)(-1.88448660714286,1.88917644006109)(-1.88434709821429,1.8904320000596)(-1.88420758928571,1.89168754821523)(-1.88406808035714,1.8929430818128)(-1.88392857142857,1.89419859813719)(-1.8837890625,1.8954540944733)(-1.88364955357143,1.89670956810606)(-1.88351004464286,1.89796501632046)(-1.88337053571429,1.89922043640152)(-1.88323102678571,1.90047582563434)(-1.88309151785714,1.90173118130406)(-1.88295200892857,1.90298650069589)(-1.8828125,1.90424178109512)(-1.88267299107143,1.90549701978709)(-1.88253348214286,1.90675221405726)(-1.88239397321429,1.90800736119114)(-1.88225446428571,1.90926245847436)(-1.88211495535714,1.91051750319262)(-1.88197544642857,1.91177249263175)(-1.8818359375,1.91302742407767)(-1.88169642857143,1.91428229481642)(-1.88155691964286,1.91553710213415)(-1.88141741071429,1.91679184331714)(-1.88127790178571,1.9180465156518)(-1.88113839285714,1.91930111642467)(-1.88099888392857,1.92055564292243)(-1.880859375,1.92181009243191)(-1.88071986607143,1.92306446224008)(-1.88058035714286,1.92431874963406)(-1.88044084821429,1.92557295190117)(-1.88030133928571,1.92682706632884)(-1.88016183035714,1.92808109020471)(-1.88002232142857,1.92933502081658)(-1.8798828125,1.93058885545243)(-1.87974330357143,1.93184259140044)(-1.87960379464286,1.93309622594897)(-1.87946428571429,1.93434975638659)(-1.87932477678571,1.93560318000205)(-1.87918526785714,1.93685649408433)(-1.87904575892857,1.93810969592261)(-1.87890625,1.9393627828063)(-1.87876674107143,1.94061575202503)(-1.87862723214286,1.94186860086864)(-1.87848772321429,1.94312132662723)(-1.87834821428571,1.94437392659114)(-1.87820870535714,1.94562639805094)(-1.87806919642857,1.94687873829744)(-1.8779296875,1.94813094462173)(-1.87779017857143,1.94938301431516)(-1.87765066964286,1.95063494466933)(-1.87751116071429,1.95188673297611)(-1.87737165178571,1.95313837652767)(-1.87723214285714,1.95438987261644)(-1.87709263392857,1.95564121853513)(-1.876953125,1.95689241157678)(-1.87681361607143,1.95814344903469)(-1.87667410714286,1.95939432820247)(-1.87653459821429,1.96064504637406)(-1.87639508928571,1.96189560084368)(-1.87625558035714,1.9631459889059)(-1.87611607142857,1.96439620785559)(-1.8759765625,1.96564625498797)(-1.87583705357143,1.96689612759857)(-1.87569754464286,1.96814582298328)(-1.87555803571429,1.96939533843833)(-1.87541852678571,1.9706446712603)(-1.87527901785714,1.97189381874612)(-1.87513950892857,1.97314277819308)(-1.875,1.97439154689885)(-1.87486049107143,1.97564012216145)(-1.87472098214286,1.9768885012793)(-1.87458147321429,1.97813668155118)(-1.87444196428571,1.97938466027627)(-1.87430245535714,1.98063243475415)(-1.87416294642857,1.98188000228477)(-1.8740234375,1.9831273601685)(-1.87388392857143,1.98437450570614)(-1.87374441964286,1.98562143619885)(-1.87360491071429,1.98686814894826)(-1.87346540178571,1.98811464125639)(-1.87332589285714,1.98936091042572)(-1.87318638392857,1.99060695375913)(-1.873046875,1.99185276855996)(-1.87290736607143,1.993098352132)(-1.87276785714286,1.99434370177946)(-1.87262834821429,1.99558881480705)(-1.87248883928571,1.9968336885199)(-1.87234933035714,1.99807832022363)(-1.87220982142857,1.99932270722432)(-1.8720703125,2.00056684682853)(-1.87193080357143,2.00181073634331)(-1.87179129464286,2.00305437307617)(-1.87165178571429,2.00429775433514)(-1.87151227678571,2.00554087742874)(-1.87137276785714,2.00678373966599)(-1.87123325892857,2.00802633835641)(-1.87109375,2.00926867081004)(-1.87095424107143,2.01051073433745)(-1.87081473214286,2.0117525262497)(-1.87067522321429,2.01299404385842)(-1.87053571428571,2.01423528447575)(-1.87039620535714,2.01547624541438)(-1.87025669642857,2.01671692398752)(-1.8701171875,2.01795731750896)(-1.86997767857143,2.01919742329302)(-1.86983816964286,2.02043723865461)(-1.86969866071429,2.02167676090917)(-1.86955915178571,2.02291598737274)(-1.86941964285714,2.02415491536192)(-1.86928013392857,2.02539354219389)(-1.869140625,2.02663186518643)(-1.86900111607143,2.02786988165789)(-1.86886160714286,2.02910758892723)(-1.86872209821429,2.03034498431402)(-1.86858258928571,2.03158206513842)(-1.86844308035714,2.03281882872122)(-1.86830357142857,2.03405527238379)(-1.8681640625,2.03529139344818)(-1.86802455357143,2.03652718923702)(-1.86788504464286,2.0377626570736)(-1.86774553571429,2.03899779428183)(-1.86760602678571,2.04023259818627)(-1.86746651785714,2.04146706611215)(-1.86732700892857,2.04270119538531)(-1.8671875,2.04393498333229)(-1.86704799107143,2.04516842728028)(-1.86690848214286,2.04640152455712)(-1.86676897321429,2.04763427249137)(-1.86662946428571,2.04886666841221)(-1.86648995535714,2.05009870964957)(-1.86635044642857,2.05133039353401)(-1.8662109375,2.05256171739683)(-1.86607142857143,2.05379267857001)(-1.86593191964286,2.05502327438624)(-1.86579241071429,2.05625350217892)(-1.86565290178571,2.05748335928216)(-1.86551339285714,2.05871284303082)(-1.86537388392857,2.05994195076044)(-1.865234375,2.06117067980734)(-1.86509486607143,2.06239902750854)(-1.86495535714286,2.06362699120182)(-1.86481584821429,2.0648545682257)(-1.86467633928571,2.06608175591946)(-1.86453683035714,2.06730855162315)(-1.86439732142857,2.06853495267753)(-1.8642578125,2.0697609564242)(-1.86411830357143,2.07098656020548)(-1.86397879464286,2.07221176136449)(-1.86383928571429,2.07343655724514)(-1.86369977678571,2.07466094519211)(-1.86356026785714,2.07588492255088)(-1.86342075892857,2.07710848666774)(-1.86328125,2.07833163488978)(-1.86314174107143,2.07955436456489)(-1.86300223214286,2.08077667304177)(-1.86286272321429,2.08199855766997)(-1.86272321428571,2.08322001579983)(-1.86258370535714,2.08444104478256)(-1.86244419642857,2.08566164197014)(-1.8623046875,2.08688180471547)(-1.86216517857143,2.08810153037224)(-1.86202566964286,2.089320816295)(-1.86188616071429,2.09053965983917)(-1.86174665178571,2.09175805836103)(-1.86160714285714,2.09297600921772)(-1.86146763392857,2.09419350976724)(-1.861328125,2.09541055736849)(-1.86118861607143,2.09662714938124)(-1.86104910714286,2.09784328316613)(-1.86090959821429,2.09905895608473)(-1.86077008928571,2.10027416549946)(-1.86063058035714,2.1014889087737)(-1.86049107142857,2.10270318327167)(-1.8603515625,2.10391698635856)(-1.86021205357143,2.10513031540045)(-1.86007254464286,2.10634316776434)(-1.85993303571429,2.10755554081817)(-1.85979352678571,2.1087674319308)(-1.85965401785714,2.10997883847206)(-1.85951450892857,2.11118975781266)(-1.859375,2.11240018732433)(-1.85923549107143,2.1136101243797)(-1.85909598214286,2.11481956635238)(-1.85895647321429,2.11602851061696)(-1.85881696428571,2.11723695454896)(-1.85867745535714,2.11844489552492)(-1.85853794642857,2.1196523309223)(-1.8583984375,2.1208592581196)(-1.85825892857143,2.12206567449629)(-1.85811941964286,2.12327157743281)(-1.85797991071429,2.12447696431063)(-1.85784040178571,2.12568183251223)(-1.85770089285714,2.12688617942107)(-1.85756138392857,2.12809000242163)(-1.857421875,2.12929329889943)(-1.85728236607143,2.13049606624101)(-1.85714285714286,2.13169830183391)(-1.85700334821429,2.13290000306675)(-1.85686383928571,2.13410116732915)(-1.85672433035714,2.13530179201182)(-1.85658482142857,2.13650187450646)(-1.8564453125,2.13770141220588)(-1.85630580357143,2.13890040250392)(-1.85616629464286,2.1400988427955)(-1.85602678571429,2.1412967304766)(-1.85588727678571,2.14249406294427)(-1.85574776785714,2.14369083759668)(-1.85560825892857,2.14488705183302)(-1.85546875,2.14608270305363)(-1.85532924107143,2.14727778865992)(-1.85518973214286,2.1484723060544)(-1.85505022321429,2.14966625264069)(-1.85491071428571,2.15085962582352)(-1.85477120535714,2.15205242300875)(-1.85463169642857,2.15324464160332)(-1.8544921875,2.15443627901535)(-1.85435267857143,2.15562733265404)(-1.85421316964286,2.15681779992977)(-1.85407366071429,2.15800767825404)(-1.85393415178571,2.15919696503949)(-1.85379464285714,2.16038565769993)(-1.85365513392857,2.16157375365031)(-1.853515625,2.16276125030674)(-1.85337611607143,2.16394814508652)(-1.85323660714286,2.16513443540809)(-1.85309709821429,2.1663201186911)(-1.85295758928571,2.16750519235634)(-1.85281808035714,2.16868965382583)(-1.85267857142857,2.16987350052275)(-1.8525390625,2.17105672987149)(-1.85239955357143,2.17223933929764)(-1.85226004464286,2.17342132622799)(-1.85212053571429,2.17460268809056)(-1.85198102678571,2.17578342231456)(-1.85184151785714,2.17696352633045)(-1.85170200892857,2.17814299756988)(-1.8515625,2.17932183346578)(-1.85142299107143,2.18050003145227)(-1.85128348214286,2.18167758896473)(-1.85114397321429,2.18285450343981)(-1.85100446428571,2.18403077231537)(-1.85086495535714,2.18520639303056)(-1.85072544642857,2.18638136302576)(-1.8505859375,2.18755567974264)(-1.85044642857143,2.18872934062415)(-1.85030691964286,2.18990234311449)(-1.85016741071429,2.19107468465915)(-1.85002790178571,2.19224636270492)(-1.84988839285714,2.19341737469986)(-1.84974888392857,2.19458771809335)(-1.849609375,2.19575739033605)(-1.84946986607143,2.19692638887993)(-1.84933035714286,2.19809471117829)(-1.84919084821429,2.19926235468573)(-1.84905133928571,2.20042931685816)(-1.84891183035714,2.20159559515284)(-1.84877232142857,2.20276118702835)(-1.8486328125,2.20392608994461)(-1.84849330357143,2.20509030136287)(-1.84835379464286,2.20625381874573)(-1.84821428571429,2.20741663955716)(-1.84807477678571,2.20857876126245)(-1.84793526785714,2.20974018132829)(-1.84779575892857,2.21090089722271)(-1.84765625,2.2120609064151)(-1.84751674107143,2.21322020637626)(-1.84737723214286,2.21437879457835)(-1.84723772321429,2.21553666849491)(-1.84709821428571,2.21669382560089)(-1.84695870535714,2.21785026337261)(-1.84681919642857,2.21900597928781)(-1.8466796875,2.22016097082562)(-1.84654017857143,2.2213152354666)(-1.84640066964286,2.2224687706927)(-1.84626116071429,2.22362157398732)(-1.84612165178571,2.22477364283524)(-1.84598214285714,2.22592497472272)(-1.84584263392857,2.22707556713742)(-1.845703125,2.22822541756845)(-1.84556361607143,2.22937452350636)(-1.84542410714286,2.23052288244316)(-1.84528459821429,2.23167049187231)(-1.84514508928571,2.23281734928871)(-1.84500558035714,2.23396345218875)(-1.84486607142857,2.23510879807027)(-1.8447265625,2.2362533844326)(-1.84458705357143,2.23739720877654)(-1.84444754464286,2.23854026860436)(-1.84430803571429,2.23968256141984)(-1.84416852678571,2.24082408472824)(-1.84402901785714,2.24196483603632)(-1.84388950892857,2.24310481285234)(-1.84375,2.24424401268608)(-1.84361049107143,2.24538243304881)(-1.84347098214286,2.24652007145334)(-1.84333147321429,2.24765692541398)(-1.84319196428571,2.24879299244659)(-1.84305245535714,2.24992827006855)(-1.84291294642857,2.25106275579877)(-1.8427734375,2.25219644715771)(-1.84263392857143,2.25332934166738)(-1.84249441964286,2.25446143685131)(-1.84235491071429,2.25559273023464)(-1.84221540178571,2.25672321934403)(-1.84207589285714,2.25785290170772)(-1.84193638392857,2.2589817748555)(-1.841796875,2.26010983631877)(-1.84165736607143,2.26123708363048)(-1.84151785714286,2.26236351432517)(-1.84137834821429,2.263489125939)(-1.84123883928571,2.26461391600967)(-1.84109933035714,2.26573788207653)(-1.84095982142857,2.2668610216805)(-1.8408203125,2.26798333236412)(-1.84068080357143,2.26910481167155)(-1.84054129464286,2.27022545714855)(-1.84040178571429,2.27134526634253)(-1.84026227678571,2.2724642368025)(-1.84012276785714,2.27358236607913)(-1.83998325892857,2.27469965172469)(-1.83984375,2.27581609129313)(-1.83970424107143,2.27693168234002)(-1.83956473214286,2.2780464224226)(-1.83942522321429,2.27916030909976)(-1.83928571428571,2.28027333993203)(-1.83914620535714,2.28138551248165)(-1.83900669642857,2.28249682431249)(-1.8388671875,2.28360727299011)(-1.83872767857143,2.28471685608176)(-1.83858816964286,2.28582557115635)(-1.83844866071429,2.28693341578451)(-1.83830915178571,2.28804038753854)(-1.83816964285714,2.28914648399246)(-1.83803013392857,2.29025170272196)(-1.837890625,2.29135604130447)(-1.83775111607143,2.29245949731912)(-1.83761160714286,2.29356206834676)(-1.83747209821429,2.29466375196996)(-1.83733258928571,2.29576454577303)(-1.83719308035714,2.296864447342)(-1.83705357142857,2.29796345426462)(-1.8369140625,2.29906156413041)(-1.83677455357143,2.30015877453064)(-1.83663504464286,2.30125508305829)(-1.83649553571429,2.30235048730814)(-1.83635602678571,2.30344498487669)(-1.83621651785714,2.30453857336225)(-1.83607700892857,2.30563125036486)(-1.8359375,2.30672301348636)(-1.83579799107143,2.30781386033035)(-1.83565848214286,2.30890378850221)(-1.83551897321429,2.30999279560913)(-1.83537946428571,2.31108087926009)(-1.83523995535714,2.31216803706585)(-1.83510044642857,2.31325426663897)(-1.8349609375,2.31433956559384)(-1.83482142857143,2.31542393154665)(-1.83468191964286,2.3165073621154)(-1.83454241071429,2.31758985491992)(-1.83440290178571,2.31867140758185)(-1.83426339285714,2.31975201772468)(-1.83412388392857,2.32083168297371)(-1.833984375,2.32191040095611)(-1.83384486607143,2.32298816930087)(-1.83370535714286,2.32406498563883)(-1.83356584821429,2.32514084760269)(-1.83342633928571,2.32621575282701)(-1.83328683035714,2.32728969894822)(-1.83314732142857,2.32836268360457)(-1.8330078125,2.32943470443625)(-1.83286830357143,2.33050575908527)(-1.83272879464286,2.33157584519555)(-1.83258928571429,2.33264496041289)(-1.83244977678571,2.33371310238496)(-1.83231026785714,2.33478026876137)(-1.83217075892857,2.33584645719357)(-1.83203125,2.33691166533496)(-1.83189174107143,2.33797589084082)(-1.83175223214286,2.33903913136836)(-1.83161272321429,2.34010138457671)(-1.83147321428571,2.34116264812689)(-1.83133370535714,2.34222291968188)(-1.83119419642857,2.34328219690658)(-1.8310546875,2.34434047746782)(-1.83091517857143,2.34539775903438)(-1.83077566964286,2.34645403927697)(-1.83063616071429,2.34750931586826)(-1.83049665178571,2.34856358648287)(-1.83035714285714,2.34961684879738)(-1.83021763392857,2.35066910049033)(-1.830078125,2.35172033924223)(-1.82993861607143,2.35277056273555)(-1.82979910714286,2.35381976865475)(-1.82965959821429,2.35486795468627)(-1.82952008928571,2.35591511851852)(-1.82938058035714,2.35696125784193)(-1.82924107142857,2.35800637034888)(-1.8291015625,2.35905045373378)(-1.82896205357143,2.36009350569304)(-1.82882254464286,2.36113552392507)(-1.82868303571429,2.3621765061303)(-1.82854352678571,2.36321645001117)(-1.82840401785714,2.36425535327213)(-1.82826450892857,2.36529321361968)(-1.828125,2.36633002876233)(-1.82798549107143,2.36736579641063)(-1.82784598214286,2.36840051427718)(-1.82770647321429,2.36943418007661)(-1.82756696428571,2.37046679152559)(-1.82742745535714,2.37149834634286)(-1.82728794642857,2.37252884224922)(-1.8271484375,2.3735582769675)(-1.82700892857143,2.37458664822263)(-1.82686941964286,2.37561395374159)(-1.82672991071429,2.37664019125344)(-1.82659040178571,2.37766535848932)(-1.82645089285714,2.37868945318245)(-1.82631138392857,2.37971247306814)(-1.826171875,2.38073441588379)(-1.82603236607143,2.38175527936889)(-1.82589285714286,2.38277506126505)(-1.82575334821429,2.38379375931597)(-1.82561383928571,2.38481137126745)(-1.82547433035714,2.38582789486743)(-1.82533482142857,2.38684332786595)(-1.8251953125,2.38785766801517)(-1.82505580357143,2.3888709130694)(-1.82491629464286,2.38988306078505)(-1.82477678571429,2.39089410892068)(-1.82463727678571,2.391904055237)(-1.82449776785714,2.39291289749685)(-1.82435825892857,2.39392063346523)(-1.82421875,2.39492726090928)(-1.82407924107143,2.39593277759831)(-1.82393973214286,2.39693718130379)(-1.82380022321429,2.39794046979934)(-1.82366071428571,2.39894264086077)(-1.82352120535714,2.39994369226606)(-1.82338169642857,2.40094362179537)(-1.8232421875,2.40194242723104)(-1.82310267857143,2.40294010635759)(-1.82296316964286,2.40393665696174)(-1.82282366071429,2.40493207683242)(-1.82268415178571,2.40592636376074)(-1.82254464285714,2.40691951554003)(-1.82240513392857,2.40791152996581)(-1.822265625,2.40890240483584)(-1.82212611607143,2.40989213795007)(-1.82198660714286,2.41088072711069)(-1.82184709821429,2.41186817012211)(-1.82170758928571,2.41285446479098)(-1.82156808035714,2.41383960892618)(-1.82142857142857,2.41482360033882)(-1.8212890625,2.41580643684226)(-1.82114955357143,2.41678811625211)(-1.82101004464286,2.41776863638623)(-1.82087053571429,2.41874799506475)(-1.82073102678571,2.41972619011002)(-1.82059151785714,2.42070321934671)(-1.82045200892857,2.42167908060171)(-1.8203125,2.42265377170422)(-1.82017299107143,2.42362729048569)(-1.82003348214286,2.42459963477986)(-1.81989397321429,2.42557080242278)(-1.81975446428571,2.42654079125275)(-1.81961495535714,2.42750959911039)(-1.81947544642857,2.42847722383862)(-1.8193359375,2.42944366328264)(-1.81919642857143,2.43040891528998)(-1.81905691964286,2.43137297771047)(-1.81891741071429,2.43233584839626)(-1.81877790178571,2.43329752520181)(-1.81863839285714,2.43425800598393)(-1.81849888392857,2.43521728860172)(-1.818359375,2.43617537091663)(-1.81821986607143,2.43713225079247)(-1.81808035714286,2.43808792609534)(-1.81794084821429,2.43904239469373)(-1.81780133928571,2.43999565445844)(-1.81766183035714,2.44094770326267)(-1.81752232142857,2.44189853898193)(-1.8173828125,2.44284815949412)(-1.81724330357143,2.44379656267949)(-1.81710379464286,2.44474374642067)(-1.81696428571429,2.44568970860266)(-1.81682477678571,2.44663444711282)(-1.81668526785714,2.44757795984094)(-1.81654575892857,2.44852024467913)(-1.81640625,2.44946129952193)(-1.81626674107143,2.45040112226629)(-1.81612723214286,2.4513397108115)(-1.81598772321429,2.45227706305932)(-1.81584821428571,2.45321317691386)(-1.81570870535714,2.45414805028168)(-1.81556919642857,2.45508168107173)(-1.8154296875,2.45601406719538)(-1.81529017857143,2.45694520656645)(-1.81515066964286,2.45787509710113)(-1.81501116071429,2.45880373671811)(-1.81487165178571,2.45973112333846)(-1.81473214285714,2.46065725488573)(-1.81459263392857,2.46158212928586)(-1.814453125,2.4625057444673)(-1.81431361607143,2.4634280983609)(-1.81417410714286,2.46434918889999)(-1.81403459821429,2.46526901402036)(-1.81389508928571,2.46618757166024)(-1.81375558035714,2.46710485976036)(-1.81361607142857,2.46802087626389)(-1.8134765625,2.46893561911649)(-1.81333705357143,2.46984908626631)(-1.81319754464286,2.47076127566396)(-1.81305803571429,2.47167218526255)(-1.81291852678571,2.47258181301767)(-1.81277901785714,2.47349015688743)(-1.81263950892857,2.4743972148324)(-1.8125,2.47530298481569)(-1.81236049107143,2.4762074648029)(-1.81222098214286,2.47711065276213)(-1.81208147321429,2.47801254666402)(-1.81194196428571,2.47891314448171)(-1.81180245535714,2.47981244419087)(-1.81166294642857,2.48071044376969)(-1.8115234375,2.48160714119889)(-1.81138392857143,2.48250253446174)(-1.81124441964286,2.48339662154402)(-1.81110491071429,2.48428940043409)(-1.81096540178571,2.48518086912282)(-1.81082589285714,2.48607102560366)(-1.81068638392857,2.48695986787258)(-1.810546875,2.48784739392814)(-1.81040736607143,2.48873360177145)(-1.81026785714286,2.48961848940617)(-1.81012834821429,2.49050205483857)(-1.80998883928571,2.49138429607744)(-1.80984933035714,2.49226521113418)(-1.80970982142857,2.49314479802278)(-1.8095703125,2.49402305475979)(-1.80943080357143,2.49489997936436)(-1.80929129464286,2.49577556985823)(-1.80915178571429,2.49664982426575)(-1.80901227678571,2.49752274061384)(-1.80887276785714,2.49839431693206)(-1.80873325892857,2.49926455125255)(-1.80859375,2.50013344161008)(-1.80845424107143,2.50100098604203)(-1.80831473214286,2.50186718258839)(-1.80817522321429,2.50273202929179)(-1.80803571428571,2.50359552419748)(-1.80789620535714,2.50445766535333)(-1.80775669642857,2.50531845080987)(-1.8076171875,2.50617787862023)(-1.80747767857143,2.50703594684023)(-1.80733816964286,2.5078926535283)(-1.80719866071429,2.50874799674553)(-1.80705915178571,2.50960197455567)(-1.80691964285714,2.51045458502512)(-1.80678013392857,2.51130582622294)(-1.806640625,2.51215569622087)(-1.80650111607143,2.5130041930933)(-1.80636160714286,2.51385131491731)(-1.80622209821429,2.51469705977263)(-1.80608258928571,2.51554142574169)(-1.80594308035714,2.51638441090962)(-1.80580357142857,2.51722601336419)(-1.8056640625,2.5180662311959)(-1.80552455357143,2.51890506249794)(-1.80538504464286,2.51974250536619)(-1.80524553571429,2.52057855789922)(-1.80510602678571,2.52141321819834)(-1.80496651785714,2.52224648436754)(-1.80482700892857,2.52307835451353)(-1.8046875,2.52390882674575)(-1.80454799107143,2.52473789917634)(-1.80440848214286,2.52556556992018)(-1.80426897321429,2.52639183709488)(-1.80412946428571,2.52721669882077)(-1.80398995535714,2.52804015322092)(-1.80385044642857,2.52886219842114)(-1.8037109375,2.52968283254998)(-1.80357142857143,2.53050205373875)(-1.80343191964286,2.53131986012148)(-1.80329241071429,2.53213624983499)(-1.80315290178571,2.53295122101882)(-1.80301339285714,2.53376477181532)(-1.80287388392857,2.53457690036954)(-1.802734375,2.53538760482936)(-1.80259486607143,2.5361968833454)(-1.80245535714286,2.53700473407104)(-1.80231584821429,2.53781115516249)(-1.80217633928571,2.53861614477868)(-1.80203683035714,2.53941970108138)(-1.80189732142857,2.54022182223511)(-1.8017578125,2.54102250640722)(-1.80161830357143,2.54182175176782)(-1.80147879464286,2.54261955648984)(-1.80133928571429,2.54341591874902)(-1.80119977678571,2.54421083672388)(-1.80106026785714,2.5450043085958)(-1.80092075892857,2.54579633254891)(-1.80078125,2.54658690677021)(-1.80064174107143,2.5473760294495)(-1.80050223214286,2.54816369877941)(-1.80036272321429,2.5489499129554)(-1.80022321428571,2.54973467017575)(-1.80008370535714,2.5505179686416)(-1.79994419642857,2.5512998065569)(-1.7998046875,2.55208018212847)(-1.79966517857143,2.55285909356594)(-1.79952566964286,2.55363653908184)(-1.79938616071429,2.5544125168915)(-1.79924665178571,2.55518702521316)(-1.79910714285714,2.55596006226787)(-1.79896763392857,2.55673162627957)(-1.798828125,2.55750171547507)(-1.79868861607143,2.55827032808404)(-1.79854910714286,2.55903746233902)(-1.79840959821429,2.55980311647544)(-1.79827008928571,2.5605672887316)(-1.79813058035714,2.56132997734871)(-1.79799107142857,2.56209118057081)(-1.7978515625,2.5628508966449)(-1.79771205357143,2.56360912382083)(-1.79757254464286,2.56436586035135)(-1.79743303571429,2.56512110449214)(-1.79729352678571,2.56587485450175)(-1.79715401785714,2.56662710864166)(-1.79701450892857,2.56737786517625)(-1.796875,2.56812712237283)(-1.79673549107143,2.56887487850161)(-1.79659598214286,2.56962113183573)(-1.79645647321429,2.57036588065125)(-1.79631696428571,2.57110912322717)(-1.79617745535714,2.57185085784542)(-1.79603794642857,2.57259108279084)(-1.7958984375,2.57332979635124)(-1.79575892857143,2.57406699681735)(-1.79561941964286,2.57480268248287)(-1.79547991071429,2.57553685164441)(-1.79534040178571,2.57626950260157)(-1.79520089285714,2.57700063365688)(-1.79506138392857,2.57773024311584)(-1.794921875,2.5784583292869)(-1.79478236607143,2.57918489048149)(-1.79464285714286,2.579909925014)(-1.79450334821429,2.58063343120179)(-1.79436383928571,2.5813554073652)(-1.79422433035714,2.58207585182754)(-1.79408482142857,2.5827947629151)(-1.7939453125,2.58351213895717)(-1.79380580357143,2.58422797828601)(-1.79366629464286,2.58494227923689)(-1.79352678571429,2.58565504014804)(-1.79338727678571,2.58636625936073)(-1.79324776785714,2.5870759352192)(-1.79310825892857,2.58778406607071)(-1.79296875,2.58849065026551)(-1.79282924107143,2.58919568615689)(-1.79268973214286,2.58989917210111)(-1.79255022321429,2.5906011064575)(-1.79241071428571,2.59130148758835)(-1.79227120535714,2.59200031385903)(-1.79213169642857,2.59269758363789)(-1.7919921875,2.59339329529635)(-1.79185267857143,2.59408744720882)(-1.79171316964286,2.59478003775279)(-1.79157366071429,2.59547106530875)(-1.79143415178571,2.59616052826025)(-1.79129464285714,2.59684842499389)(-1.79115513392857,2.59753475389931)(-1.791015625,2.59821951336921)(-1.79087611607143,2.59890270179932)(-1.79073660714286,2.59958431758847)(-1.79059709821429,2.6002643591385)(-1.79045758928571,2.60094282485436)(-1.79031808035714,2.60161971314404)(-1.79017857142857,2.60229502241861)(-1.7900390625,2.60296875109221)(-1.78989955357143,2.60364089758206)(-1.78976004464286,2.60431146030844)(-1.78962053571429,2.60498043769475)(-1.78948102678571,2.60564782816745)(-1.78934151785714,2.60631363015608)(-1.78920200892857,2.6069778420933)(-1.7890625,2.60764046241484)(-1.78892299107143,2.60830148955954)(-1.78878348214286,2.60896092196933)(-1.78864397321429,2.60961875808926)(-1.78850446428571,2.61027499636748)(-1.78836495535714,2.61092963525524)(-1.78822544642857,2.61158267320692)(-1.7880859375,2.61223410868)(-1.78794642857143,2.61288394013509)(-1.78780691964286,2.61353216603591)(-1.78766741071429,2.61417878484932)(-1.78752790178571,2.61482379504529)(-1.78738839285714,2.61546719509695)(-1.78724888392857,2.61610898348053)(-1.787109375,2.61674915867541)(-1.78696986607143,2.61738771916412)(-1.78683035714286,2.61802466343232)(-1.78669084821429,2.61865998996882)(-1.78655133928571,2.61929369726557)(-1.78641183035714,2.6199257838177)(-1.78627232142857,2.62055624812345)(-1.7861328125,2.62118508868426)(-1.78599330357143,2.6218123040047)(-1.78585379464286,2.62243789259252)(-1.78571428571429,2.62306185295863)(-1.78557477678571,2.6236841836171)(-1.78543526785714,2.6243048830852)(-1.78529575892857,2.62492394988335)(-1.78515625,2.62554138253514)(-1.78501674107143,2.62615717956737)(-1.78487723214286,2.62677133951)(-1.78473772321429,2.62738386089619)(-1.78459821428571,2.62799474226228)(-1.78445870535714,2.62860398214781)(-1.78431919642857,2.62921157909549)(-1.7841796875,2.62981753165128)(-1.78404017857143,2.63042183836428)(-1.78390066964286,2.63102449778683)(-1.78376116071429,2.63162550847447)(-1.78362165178571,2.63222486898595)(-1.78348214285714,2.63282257788322)(-1.78334263392857,2.63341863373145)(-1.783203125,2.63401303509904)(-1.78306361607143,2.63460578055759)(-1.78292410714286,2.63519686868193)(-1.78278459821429,2.63578629805013)(-1.78264508928571,2.63637406724347)(-1.78250558035714,2.63696017484647)(-1.78236607142857,2.63754461944688)(-1.7822265625,2.63812739963569)(-1.78208705357143,2.63870851400712)(-1.78194754464286,2.63928796115865)(-1.78180803571429,2.63986573969099)(-1.78166852678571,2.6404418482081)(-1.78152901785714,2.6410162853172)(-1.78138950892857,2.64158904962875)(-1.78125,2.64216013975647)(-1.78111049107143,2.64272955431734)(-1.78097098214286,2.6432972919316)(-1.78083147321429,2.64386335122275)(-1.78069196428571,2.64442773081758)(-1.78055245535714,2.64499042934612)(-1.78041294642857,2.64555144544167)(-1.7802734375,2.64611077774083)(-1.78013392857143,2.64666842488346)(-1.77999441964286,2.6472243855127)(-1.77985491071429,2.64777865827499)(-1.77971540178571,2.64833124182002)(-1.77957589285714,2.6488821348008)(-1.77943638392857,2.64943133587362)(-1.779296875,2.64997884369806)(-1.77915736607143,2.650524656937)(-1.77901785714286,2.65106877425661)(-1.77887834821429,2.65161119432636)(-1.77873883928571,2.65215191581905)(-1.77859933035714,2.65269093741075)(-1.77845982142857,2.65322825778085)(-1.7783203125,2.65376387561207)(-1.77818080357143,2.65429778959041)(-1.77804129464286,2.65482999840522)(-1.77790178571429,2.65536050074915)(-1.77776227678571,2.65588929531817)(-1.77762276785714,2.65641638081159)(-1.77748325892857,2.65694175593202)(-1.77734375,2.65746541938541)(-1.77720424107143,2.65798736988107)(-1.77706473214286,2.65850760613159)(-1.77692522321429,2.65902612685295)(-1.77678571428571,2.65954293076443)(-1.77664620535714,2.66005801658866)(-1.77650669642857,2.66057138305164)(-1.7763671875,2.66108302888268)(-1.77622767857143,2.66159295281445)(-1.77608816964286,2.66210115358299)(-1.77594866071429,2.66260762992767)(-1.77580915178571,2.66311238059123)(-1.77566964285714,2.66361540431977)(-1.77553013392857,2.66411669986273)(-1.775390625,2.66461626597295)(-1.77525111607143,2.66511410140659)(-1.77511160714286,2.66561020492323)(-1.77497209821429,2.66610457528577)(-1.77483258928571,2.66659721126052)(-1.77469308035714,2.66708811161715)(-1.77455357142857,2.66757727512872)(-1.7744140625,2.66806470057164)(-1.77427455357143,2.66855038672575)(-1.77413504464286,2.66903433237423)(-1.77399553571429,2.66951653630367)(-1.77385602678571,2.66999699730406)(-1.77371651785714,2.67047571416875)(-1.77357700892857,2.67095268569451)(-1.7734375,2.67142791068151)(-1.77329799107143,2.6719013879333)(-1.77315848214286,2.67237311625684)(-1.77301897321429,2.67284309446249)(-1.77287946428571,2.67331132136404)(-1.77273995535714,2.67377779577866)(-1.77260044642857,2.67424251652693)(-1.7724609375,2.67470548243287)(-1.77232142857143,2.67516669232388)(-1.77218191964286,2.67562614503082)(-1.77204241071429,2.67608383938793)(-1.77190290178571,2.67653977423289)(-1.77176339285714,2.67699394840682)(-1.77162388392857,2.67744636075422)(-1.771484375,2.67789701012308)(-1.77134486607143,2.67834589536477)(-1.77120535714286,2.67879301533412)(-1.77106584821429,2.6792383688894)(-1.77092633928571,2.67968195489229)(-1.77078683035714,2.68012377220793)(-1.77064732142857,2.68056381970491)(-1.7705078125,2.68100209625526)(-1.77036830357143,2.68143860073443)(-1.77022879464286,2.68187333202135)(-1.77008928571429,2.6823062889984)(-1.76994977678571,2.68273747055139)(-1.76981026785714,2.68316687556961)(-1.76967075892857,2.68359450294579)(-1.76953125,2.68402035157614)(-1.76939174107143,2.68444442036031)(-1.76925223214286,2.68486670820142)(-1.76911272321429,2.68528721400607)(-1.76897321428571,2.68570593668431)(-1.76883370535714,2.68612287514967)(-1.76869419642857,2.68653802831916)(-1.7685546875,2.68695139511324)(-1.76841517857143,2.68736297445587)(-1.76827566964286,2.68777276527449)(-1.76813616071429,2.6881807665)(-1.76799665178571,2.6885869770668)(-1.76785714285714,2.68899139591277)(-1.76771763392857,2.68939402197927)(-1.767578125,2.68979485421117)(-1.76743861607143,2.69019389155682)(-1.76729910714286,2.69059113296804)(-1.76715959821429,2.69098657740019)(-1.76702008928571,2.69138022381209)(-1.76688058035714,2.69177207116609)(-1.76674107142857,2.692162118428)(-1.7666015625,2.69255036456718)(-1.76646205357143,2.69293680855647)(-1.76632254464286,2.69332144937221)(-1.76618303571429,2.69370428599428)(-1.76604352678571,2.69408531740603)(-1.76590401785714,2.69446454259437)(-1.76576450892857,2.69484196054968)(-1.765625,2.69521757026589)(-1.76548549107143,2.69559137074044)(-1.76534598214286,2.69596336097428)(-1.76520647321429,2.69633353997191)(-1.76506696428571,2.69670190674131)(-1.76492745535714,2.69706846029404)(-1.76478794642857,2.69743319964516)(-1.7646484375,2.69779612381325)(-1.76450892857143,2.69815723182045)(-1.76436941964286,2.69851652269242)(-1.76422991071429,2.69887399545836)(-1.76409040178571,2.699229649151)(-1.76395089285714,2.69958348280663)(-1.76381138392857,2.69993549546506)(-1.763671875,2.70028568616967)(-1.76353236607143,2.70063405396735)(-1.76339285714286,2.70098059790856)(-1.76325334821429,2.70132531704733)(-1.76311383928571,2.7016682104412)(-1.76297433035714,2.70200927715128)(-1.76283482142857,2.70234851624226)(-1.7626953125,2.70268592678234)(-1.76255580357143,2.70302150784332)(-1.76241629464286,2.70335525850054)(-1.76227678571429,2.70368717783291)(-1.76213727678571,2.7040172649229)(-1.76199776785714,2.70434551885654)(-1.76185825892857,2.70467193872345)(-1.76171875,2.70499652361679)(-1.76157924107143,2.70531927263332)(-1.76143973214286,2.70564018487335)(-1.76130022321429,2.70595925944078)(-1.76116071428571,2.70627649544308)(-1.76102120535714,2.70659189199129)(-1.76088169642857,2.70690544820005)(-1.7607421875,2.70721716318757)(-1.76060267857143,2.70752703607564)(-1.76046316964286,2.70783506598965)(-1.76032366071429,2.70814125205855)(-1.76018415178571,2.70844559341491)(-1.76004464285714,2.70874808919488)(-1.75990513392857,2.70904873853818)(-1.759765625,2.70934754058815)(-1.75962611607143,2.70964449449172)(-1.75948660714286,2.70993959939941)(-1.75934709821429,2.71023285446534)(-1.75920758928571,2.71052425884724)(-1.75906808035714,2.71081381170643)(-1.75892857142857,2.71110151220784)(-1.7587890625,2.71138735952)(-1.75864955357143,2.71167135281506)(-1.75851004464286,2.71195349126875)(-1.75837053571429,2.71223377406045)(-1.75823102678571,2.71251220037312)(-1.75809151785714,2.71278876939335)(-1.75795200892857,2.71306348031132)(-1.7578125,2.71333633232087)(-1.75767299107143,2.71360732461941)(-1.75753348214286,2.71387645640801)(-1.75739397321429,2.71414372689133)(-1.75725446428571,2.71440913527767)(-1.75711495535714,2.71467268077896)(-1.75697544642857,2.71493436261074)(-1.7568359375,2.71519417999217)(-1.75669642857143,2.71545213214607)(-1.75655691964286,2.71570821829886)(-1.75641741071429,2.71596243768061)(-1.75627790178571,2.71621478952501)(-1.75613839285714,2.7164652730694)(-1.75599888392857,2.71671388755474)(-1.755859375,2.71696063222565)(-1.75571986607143,2.71720550633036)(-1.75558035714286,2.71744850912075)(-1.75544084821429,2.71768963985237)(-1.75530133928571,2.71792889778437)(-1.75516183035714,2.71816628217958)(-1.75502232142857,2.71840179230445)(-1.7548828125,2.7186354274291)(-1.75474330357143,2.71886718682729)(-1.75460379464286,2.71909706977643)(-1.75446428571429,2.71932507555757)(-1.75432477678571,2.71955120345544)(-1.75418526785714,2.7197754527584)(-1.75404575892857,2.71999782275848)(-1.75390625,2.72021831275138)(-1.75376674107143,2.72043692203642)(-1.75362723214286,2.72065364991661)(-1.75348772321429,2.72086849569863)(-1.75334821428571,2.72108145869279)(-1.75320870535714,2.7212925382131)(-1.75306919642857,2.7215017335772)(-1.7529296875,2.72170904410644)(-1.75279017857143,2.72191446912579)(-1.75265066964286,2.72211800796393)(-1.75251116071429,2.72231965995318)(-1.75237165178571,2.72251942442957)(-1.75223214285714,2.72271730073276)(-1.75209263392857,2.72291328820612)(-1.751953125,2.72310738619668)(-1.75181361607143,2.72329959405514)(-1.75167410714286,2.72348991113589)(-1.75153459821429,2.72367833679701)(-1.75139508928571,2.72386487040024)(-1.75125558035714,2.72404951131102)(-1.75111607142857,2.72423225889845)(-1.7509765625,2.72441311253535)(-1.75083705357143,2.72459207159819)(-1.75069754464286,2.72476913546716)(-1.75055803571429,2.72494430352611)(-1.75041852678571,2.72511757516259)(-1.75027901785714,2.72528894976785)(-1.75013950892857,2.72545842673683)(-1.75,2.72562600546816)(-1.74986049107143,2.72579168536415)(-1.74972098214286,2.72595546583083)(-1.74958147321429,2.72611734627791)(-1.74944196428571,2.7262773261188)(-1.74930245535714,2.72643540477063)(-1.74916294642857,2.72659158165419)(-1.7490234375,2.72674585619402)(-1.74888392857143,2.72689822781832)(-1.74874441964286,2.72704869595901)(-1.74860491071429,2.72719726005172)(-1.74846540178571,2.72734391953577)(-1.74832589285714,2.72748867385422)(-1.74818638392857,2.72763152245381)(-1.748046875,2.72777246478498)(-1.74790736607143,2.7279115003019)(-1.74776785714286,2.72804862846246)(-1.74762834821429,2.72818384872823)(-1.74748883928571,2.72831716056452)(-1.74734933035714,2.72844856344035)(-1.74720982142857,2.72857805682845)(-1.7470703125,2.72870564020527)(-1.74693080357143,2.72883131305096)(-1.74679129464286,2.72895507484942)(-1.74665178571429,2.72907692508825)(-1.74651227678571,2.72919686325877)(-1.74637276785714,2.72931488885603)(-1.74623325892857,2.72943100137879)(-1.74609375,2.72954520032954)(-1.74595424107143,2.7296574852145)(-1.74581473214286,2.72976785554361)(-1.74567522321429,2.72987631083054)(-1.74553571428571,2.72998285059268)(-1.74539620535714,2.73008747435115)(-1.74525669642857,2.7301901816308)(-1.7451171875,2.73029097196022)(-1.74497767857143,2.73038984487171)(-1.74483816964286,2.73048679990132)(-1.74469866071429,2.73058183658882)(-1.74455915178571,2.73067495447773)(-1.74441964285714,2.7307661531153)(-1.74428013392857,2.73085543205249)(-1.744140625,2.73094279084402)(-1.74400111607143,2.73102822904836)(-1.74386160714286,2.73111174622769)(-1.74372209821429,2.73119334194793)(-1.74358258928571,2.73127301577877)(-1.74344308035714,2.73135076729361)(-1.74330357142857,2.7314265960696)(-1.7431640625,2.73150050168764)(-1.74302455357143,2.73157248373237)(-1.74288504464286,2.73164254179215)(-1.74274553571429,2.73171067545913)(-1.74260602678571,2.73177688432916)(-1.74246651785714,2.73184116800187)(-1.74232700892857,2.73190352608063)(-1.7421875,2.73196395817253)(-1.74204799107143,2.73202246388845)(-1.74190848214286,2.73207904284299)(-1.74176897321429,2.7321336946545)(-1.74162946428571,2.73218641894511)(-1.74148995535714,2.73223721534068)(-1.74135044642857,2.7322860834708)(-1.7412109375,2.73233302296887)(-1.74107142857143,2.73237803347198)(-1.74093191964286,2.73242111462102)(-1.74079241071429,2.73246226606062)(-1.74065290178571,2.73250148743916)(-1.74051339285714,2.73253877840879)(-1.74037388392857,2.7325741386254)(-1.740234375,2.73260756774865)(-1.74009486607143,2.73263906544195)(-1.73995535714286,2.73266863137249)(-1.73981584821429,2.73269626521119)(-1.73967633928571,2.73272196663275)(-1.73953683035714,2.73274573531562)(-1.73939732142857,2.73276757094202)(-1.7392578125,2.73278747319794)(-1.73911830357143,2.7328054417731)(-1.73897879464286,2.73282147636103)(-1.73883928571429,2.73283557665898)(-1.73869977678571,2.73284774236799)(-1.73856026785714,2.73285797319287)(-1.73842075892857,2.73286626884217)(-1.73828125,2.73287262902824)(-1.73814174107143,2.73287705346716)(-1.73800223214286,2.7328795418788)(-1.73786272321429,2.7328800939868)(-1.73772321428571,2.73287870951857)(-1.73758370535714,2.73287538820526)(-1.73744419642857,2.73287012978183)(-1.7373046875,2.73286293398699)(-1.73716517857143,2.73285380056322)(-1.73702566964286,2.73284272925677)(-1.73688616071429,2.73282971981767)(-1.73674665178571,2.73281477199971)(-1.73660714285714,2.73279788556047)(-1.73646763392857,2.73277906026129)(-1.736328125,2.73275829586728)(-1.73618861607143,2.73273559214733)(-1.73604910714286,2.73271094887412)(-1.73590959821429,2.73268436582408)(-1.73577008928571,2.73265584277742)(-1.73563058035714,2.73262537951814)(-1.73549107142857,2.73259297583401)(-1.7353515625,2.73255863151656)(-1.73521205357143,2.73252234636112)(-1.73507254464286,2.73248412016679)(-1.73493303571429,2.73244395273645)(-1.73479352678571,2.73240184387674)(-1.73465401785714,2.7323577933981)(-1.73451450892857,2.73231180111474)(-1.734375,2.73226386684465)(-1.73423549107143,2.73221399040961)(-1.73409598214286,2.73216217163517)(-1.73395647321429,2.73210841035064)(-1.73381696428571,2.73205270638916)(-1.73367745535714,2.73199505958761)(-1.73353794642857,2.73193546978665)(-1.7333984375,2.73187393683076)(-1.73325892857143,2.73181046056816)(-1.73311941964286,2.73174504085088)(-1.73297991071429,2.73167767753472)(-1.73284040178571,2.73160837047926)(-1.73270089285714,2.73153711954787)(-1.73256138392857,2.7314639246077)(-1.732421875,2.73138878552969)(-1.73228236607143,2.73131170218856)(-1.73214285714286,2.7312326744628)(-1.73200334821429,2.73115170223471)(-1.73186383928571,2.73106878539036)(-1.73172433035714,2.7309839238196)(-1.73158482142857,2.73089711741607)(-1.7314453125,2.73080836607721)(-1.73130580357143,2.73071766970421)(-1.73116629464286,2.73062502820209)(-1.73102678571429,2.73053044147962)(-1.73088727678571,2.73043390944937)(-1.73074776785714,2.7303354320277)(-1.73060825892857,2.73023500913475)(-1.73046875,2.73013264069445)(-1.73032924107143,2.7300283266345)(-1.73018973214286,2.72992206688642)(-1.73005022321429,2.72981386138549)(-1.72991071428571,2.72970371007079)(-1.72977120535714,2.72959161288517)(-1.72963169642857,2.72947756977529)(-1.7294921875,2.72936158069158)(-1.72935267857143,2.72924364558827)(-1.72921316964286,2.72912376442337)(-1.72907366071429,2.72900193715868)(-1.72893415178571,2.72887816375978)(-1.72879464285714,2.72875244419605)(-1.72865513392857,2.72862477844065)(-1.728515625,2.72849516647053)(-1.72837611607143,2.72836360826642)(-1.72823660714286,2.72823010381287)(-1.72809709821429,2.72809465309817)(-1.72795758928571,2.72795725611443)(-1.72781808035714,2.72781791285754)(-1.72767857142857,2.72767662332717)(-1.7275390625,2.7275333875268)(-1.72739955357143,2.72738820546367)(-1.72726004464286,2.72724107714883)(-1.72712053571429,2.72709200259711)(-1.72698102678571,2.72694098182712)(-1.72684151785714,2.72678801486127)(-1.72670200892857,2.72663310172575)(-1.7265625,2.72647624245055)(-1.72642299107143,2.72631743706942)(-1.72628348214286,2.72615668561994)(-1.72614397321429,2.72599398814343)(-1.72600446428571,2.72582934468504)(-1.72586495535714,2.72566275529368)(-1.72572544642857,2.72549422002205)(-1.7255859375,2.72532373892666)(-1.72544642857143,2.72515131206777)(-1.72530691964286,2.72497693950947)(-1.72516741071429,2.7248006213196)(-1.72502790178571,2.7246223575698)(-1.72488839285714,2.7244421483355)(-1.72474888392857,2.72425999369591)(-1.724609375,2.72407589373404)(-1.72446986607143,2.72388984853668)(-1.72433035714286,2.72370185819438)(-1.72419084821429,2.72351192280152)(-1.72405133928571,2.72332004245623)(-1.72391183035714,2.72312621726045)(-1.72377232142857,2.72293044731988)(-1.7236328125,2.72273273274403)(-1.72349330357143,2.72253307364619)(-1.72335379464286,2.72233147014342)(-1.72321428571429,2.72212792235657)(-1.72307477678571,2.72192243041028)(-1.72293526785714,2.72171499443297)(-1.72279575892857,2.72150561455685)(-1.72265625,2.7212942909179)(-1.72251674107143,2.72108102365591)(-1.72237723214286,2.72086581291441)(-1.72223772321429,2.72064865884074)(-1.72209821428571,2.72042956158603)(-1.72195870535714,2.72020852130518)(-1.72181919642857,2.71998553815687)(-1.7216796875,2.71976061230355)(-1.72154017857143,2.71953374391149)(-1.72140066964286,2.71930493315069)(-1.72126116071429,2.71907418019497)(-1.72112165178571,2.71884148522192)(-1.72098214285714,2.71860684841289)(-1.72084263392857,2.71837026995304)(-1.720703125,2.71813175003129)(-1.72056361607143,2.71789128884033)(-1.72042410714286,2.71764888657666)(-1.72028459821429,2.71740454344053)(-1.72014508928571,2.71715825963597)(-1.72000558035714,2.7169100353708)(-1.71986607142857,2.71665987085661)(-1.7197265625,2.71640776630877)(-1.71958705357143,2.71615372194641)(-1.71944754464286,2.71589773799246)(-1.71930803571429,2.7156398146736)(-1.71916852678571,2.7153799522203)(-1.71902901785714,2.71511815086681)(-1.71888950892857,2.71485441085114)(-1.71875,2.71458873241507)(-1.71861049107143,2.71432111580417)(-1.71847098214286,2.71405156126777)(-1.71833147321429,2.71378006905898)(-1.71819196428571,2.71350663943466)(-1.71805245535714,2.71323127265547)(-1.71791294642857,2.71295396898582)(-1.7177734375,2.7126747286939)(-1.71763392857143,2.71239355205166)(-1.71749441964286,2.71211043933483)(-1.71735491071429,2.7118253908229)(-1.71721540178571,2.71153840679913)(-1.71707589285714,2.71124948755055)(-1.71693638392857,2.71095863336795)(-1.716796875,2.71066584454589)(-1.71665736607143,2.71037112138269)(-1.71651785714286,2.71007446418045)(-1.71637834821429,2.70977587324501)(-1.71623883928571,2.709475348886)(-1.71609933035714,2.70917289141679)(-1.71595982142857,2.70886850115453)(-1.7158203125,2.70856217842013)(-1.71568080357143,2.70825392353824)(-1.71554129464286,2.70794373683729)(-1.71540178571429,2.70763161864947)(-1.71526227678571,2.70731756931072)(-1.71512276785714,2.70700158916075)(-1.71498325892857,2.70668367854302)(-1.71484375,2.70636383780474)(-1.71470424107143,2.7060420672969)(-1.71456473214286,2.70571836737421)(-1.71442522321429,2.70539273839518)(-1.71428571428571,2.70506518072204)(-1.71414620535714,2.70473569472077)(-1.71400669642857,2.70440428076114)(-1.7138671875,2.70407093921664)(-1.71372767857143,2.70373567046453)(-1.71358816964286,2.70339847488579)(-1.71344866071429,2.7030593528652)(-1.71330915178571,2.70271830479124)(-1.71316964285714,2.70237533105617)(-1.71303013392857,2.70203043205598)(-1.712890625,2.70168360819043)(-1.71275111607143,2.701334859863)(-1.71261160714286,2.70098418748094)(-1.71247209821429,2.70063159145521)(-1.71233258928571,2.70027707220056)(-1.71219308035714,2.69992063013544)(-1.71205357142857,2.69956226568207)(-1.7119140625,2.6992019792664)(-1.71177455357143,2.69883977131813)(-1.71163504464286,2.69847564227069)(-1.71149553571429,2.69810959256124)(-1.71135602678571,2.69774162263071)(-1.71121651785714,2.69737173292374)(-1.71107700892857,2.69699992388871)(-1.7109375,2.69662619597774)(-1.71079799107143,2.69625054964668)(-1.71065848214286,2.69587298535514)(-1.71051897321429,2.69549350356642)(-1.71037946428571,2.69511210474757)(-1.71023995535714,2.69472878936939)(-1.71010044642857,2.69434355790639)(-1.7099609375,2.6939564108368)(-1.70982142857143,2.69356734864261)(-1.70968191964286,2.69317637180951)(-1.70954241071429,2.69278348082692)(-1.70940290178571,2.692388676188)(-1.70926339285714,2.69199195838962)(-1.70912388392857,2.69159332793238)(-1.708984375,2.69119278532061)(-1.70884486607143,2.69079033106234)(-1.70870535714286,2.69038596566934)(-1.70856584821429,2.68997968965709)(-1.70842633928571,2.6895715035448)(-1.70828683035714,2.68916140785537)(-1.70814732142857,2.68874940311546)(-1.7080078125,2.68833548985541)(-1.70786830357143,2.68791966860929)(-1.70772879464286,2.68750193991487)(-1.70758928571429,2.68708230431365)(-1.70744977678571,2.68666076235083)(-1.70731026785714,2.68623731457533)(-1.70717075892857,2.68581196153976)(-1.70703125,2.68538470380046)(-1.70689174107143,2.68495554191747)(-1.70675223214286,2.68452447645453)(-1.70661272321429,2.68409150797908)(-1.70647321428571,2.68365663706229)(-1.70633370535714,2.68321986427901)(-1.70619419642857,2.68278119020779)(-1.7060546875,2.68234061543089)(-1.70591517857143,2.68189814053427)(-1.70577566964286,2.68145376610758)(-1.70563616071429,2.68100749274417)(-1.70549665178571,2.6805593210411)(-1.70535714285714,2.6801092515991)(-1.70521763392857,2.67965728502261)(-1.705078125,2.67920342191976)(-1.70493861607143,2.67874766290236)(-1.70479910714286,2.67829000858593)(-1.70465959821429,2.67783045958967)(-1.70452008928571,2.67736901653646)(-1.70438058035714,2.67690568005287)(-1.70424107142857,2.67644045076915)(-1.7041015625,2.67597332931926)(-1.70396205357143,2.67550431634081)(-1.70382254464286,2.67503341247511)(-1.70368303571429,2.67456061836715)(-1.70354352678571,2.67408593466558)(-1.70340401785714,2.67360936202276)(-1.70326450892857,2.67313090109469)(-1.703125,2.67265055254107)(-1.70298549107143,2.67216831702528)(-1.70284598214286,2.67168419521434)(-1.70270647321429,2.67119818777897)(-1.70256696428571,2.67071029539356)(-1.70242745535714,2.67022051873614)(-1.70228794642857,2.66972885848844)(-1.7021484375,2.66923531533584)(-1.70200892857143,2.66873988996739)(-1.70186941964286,2.6682425830758)(-1.70172991071429,2.66774339535744)(-1.70159040178571,2.66724232751234)(-1.70145089285714,2.6667393802442)(-1.70131138392857,2.66623455426038)(-1.701171875,2.66572785027186)(-1.70103236607143,2.66521926899332)(-1.70089285714286,2.66470881114308)(-1.70075334821429,2.6641964774431)(-1.70061383928571,2.66368226861899)(-1.70047433035714,2.66316618540004)(-1.70033482142857,2.66264822851915)(-1.7001953125,2.66212839871289)(-1.70005580357143,2.66160669672147)(-1.69991629464286,2.66108312328874)(-1.69977678571429,2.66055767916219)(-1.69963727678571,2.66003036509296)(-1.69949776785714,2.65950118183583)(-1.69935825892857,2.65897013014921)(-1.69921875,2.65843721079514)(-1.69907924107143,2.65790242453931)(-1.69893973214286,2.65736577215105)(-1.69880022321429,2.65682725440329)(-1.69866071428571,2.65628687207263)(-1.69852120535714,2.65574462593926)(-1.69838169642857,2.65520051678703)(-1.6982421875,2.6546545454034)(-1.69810267857143,2.65410671257946)(-1.69796316964286,2.65355701910991)(-1.69782366071429,2.65300546579309)(-1.69768415178571,2.65245205343095)(-1.69754464285714,2.65189678282905)(-1.69740513392857,2.6513396547966)(-1.697265625,2.65078067014637)(-1.69712611607143,2.6502198296948)(-1.69698660714286,2.64965713426191)(-1.69684709821429,2.64909258467132)(-1.69670758928571,2.6485261817503)(-1.69656808035714,2.64795792632969)(-1.69642857142857,2.64738781924395)(-1.6962890625,2.64681586133114)(-1.69614955357143,2.64624205343292)(-1.69601004464286,2.64566639639457)(-1.69587053571429,2.64508889106494)(-1.69573102678571,2.6445095382965)(-1.69559151785714,2.64392833894529)(-1.69545200892857,2.64334529387098)(-1.6953125,2.6427604039368)(-1.69517299107143,2.6421736700096)(-1.69503348214286,2.64158509295978)(-1.69489397321429,2.64099467366136)(-1.69475446428571,2.64040241299194)(-1.69461495535714,2.63980831183269)(-1.69447544642857,2.63921237106838)(-1.6943359375,2.63861459158734)(-1.69419642857143,2.6380149742815)(-1.69405691964286,2.63741352004635)(-1.69391741071429,2.63681022978097)(-1.69377790178571,2.636205104388)(-1.69363839285714,2.63559814477365)(-1.69349888392857,2.63498935184772)(-1.693359375,2.63437872652356)(-1.69321986607143,2.63376626971809)(-1.69308035714286,2.6331519823518)(-1.69294084821429,2.63253586534874)(-1.69280133928571,2.63191791963651)(-1.69266183035714,2.63129814614628)(-1.69252232142857,2.63067654581279)(-1.6923828125,2.63005311957431)(-1.69224330357143,2.62942786837268)(-1.69210379464286,2.62880079315329)(-1.69196428571429,2.62817189486507)(-1.69182477678571,2.62754117446051)(-1.69168526785714,2.62690863289563)(-1.69154575892857,2.62627427113003)(-1.69140625,2.62563809012682)(-1.69126674107143,2.62500009085265)(-1.69112723214286,2.62436027427773)(-1.69098772321429,2.62371864137579)(-1.69084821428571,2.6230751931241)(-1.69070870535714,2.62242993050347)(-1.69056919642857,2.62178285449823)(-1.6904296875,2.62113396609624)(-1.69029017857143,2.6204832662889)(-1.69015066964286,2.61983075607113)(-1.69001116071429,2.61917643644136)(-1.68987165178571,2.61852030840156)(-1.68973214285714,2.61786237295721)(-1.68959263392857,2.61720263111732)(-1.689453125,2.61654108389439)(-1.68931361607143,2.61587773230445)(-1.68917410714286,2.61521257736705)(-1.68903459821429,2.61454562010524)(-1.68889508928571,2.61387686154558)(-1.68875558035714,2.61320630271813)(-1.68861607142857,2.61253394465646)(-1.6884765625,2.61185978839764)(-1.68833705357143,2.61118383498223)(-1.68819754464286,2.61050608545432)(-1.68805803571429,2.60982654086146)(-1.68791852678571,2.60914520225471)(-1.68777901785714,2.60846207068862)(-1.68763950892857,2.60777714722123)(-1.6875,2.60709043291405)(-1.68736049107143,2.60640192883211)(-1.68722098214286,2.6057116360439)(-1.68708147321429,2.6050195556214)(-1.68694196428571,2.60432568864006)(-1.68680245535714,2.6036300361788)(-1.68666294642857,2.60293259932006)(-1.6865234375,2.60223337914969)(-1.68638392857143,2.60153237675705)(-1.68624441964286,2.60082959323497)(-1.68610491071429,2.60012502967973)(-1.68596540178571,2.59941868719107)(-1.68582589285714,2.59871056687222)(-1.68568638392857,2.59800066982985)(-1.685546875,2.59728899717408)(-1.68540736607143,2.59657555001852)(-1.68526785714286,2.59586032948018)(-1.68512834821429,2.59514333667958)(-1.68498883928571,2.59442457274065)(-1.68484933035714,2.59370403879078)(-1.68470982142857,2.5929817359608)(-1.6845703125,2.59225766538499)(-1.68443080357143,2.59153182820107)(-1.68429129464286,2.59080422555019)(-1.68415178571429,2.59007485857694)(-1.68401227678571,2.58934372842936)(-1.68387276785714,2.58861083625889)(-1.68373325892857,2.58787618322042)(-1.68359375,2.58713977047226)(-1.68345424107143,2.58640159917616)(-1.68331473214286,2.58566167049728)(-1.68317522321429,2.58491998560419)(-1.68303571428571,2.5841765456689)(-1.68289620535714,2.58343135186681)(-1.68275669642857,2.58268440537677)(-1.6826171875,2.581935707381)(-1.68247767857143,2.58118525906516)(-1.68233816964286,2.58043306161831)(-1.68219866071429,2.57967911623289)(-1.68205915178571,2.57892342410479)(-1.68191964285714,2.57816598643324)(-1.68178013392857,2.57740680442092)(-1.681640625,2.57664587927388)(-1.68150111607143,2.57588321220155)(-1.68136160714286,2.57511880441679)(-1.68122209821429,2.57435265713582)(-1.68108258928571,2.57358477157823)(-1.68094308035714,2.57281514896704)(-1.68080357142857,2.5720437905286)(-1.6806640625,2.57127069749268)(-1.68052455357143,2.57049587109239)(-1.68038504464286,2.56971931256425)(-1.68024553571429,2.56894102314813)(-1.68010602678571,2.56816100408726)(-1.67996651785714,2.56737925662826)(-1.67982700892857,2.56659578202109)(-1.6796875,2.5658105815191)(-1.67954799107143,2.56502365637896)(-1.67940848214286,2.56423500786075)(-1.67926897321429,2.56344463722784)(-1.67912946428571,2.56265254574701)(-1.67898995535714,2.56185873468835)(-1.67885044642857,2.56106320532532)(-1.6787109375,2.56026595893471)(-1.67857142857143,2.55946699679666)(-1.67843191964286,2.55866632019465)(-1.67829241071429,2.55786393041549)(-1.67815290178571,2.55705982874932)(-1.67801339285714,2.55625401648962)(-1.67787388392857,2.55544649493322)(-1.677734375,2.55463726538023)(-1.67759486607143,2.55382632913412)(-1.67745535714286,2.55301368750167)(-1.67731584821429,2.55219934179299)(-1.67717633928571,2.55138329332148)(-1.67703683035714,2.55056554340388)(-1.67689732142857,2.54974609336024)(-1.6767578125,2.5489249445139)(-1.67661830357143,2.54810209819152)(-1.67647879464286,2.54727755572307)(-1.67633928571429,2.54645131844181)(-1.67619977678571,2.54562338768429)(-1.67606026785714,2.54479376479037)(-1.67592075892857,2.54396245110322)(-1.67578125,2.54312944796926)(-1.67564174107143,2.54229475673823)(-1.67550223214286,2.54145837876314)(-1.67536272321429,2.54062031540029)(-1.67522321428571,2.53978056800926)(-1.67508370535714,2.5389391379529)(-1.67494419642857,2.53809602659735)(-1.6748046875,2.53725123531201)(-1.67466517857143,2.53640476546954)(-1.67452566964286,2.5355566184459)(-1.67438616071429,2.53470679562028)(-1.67424665178571,2.53385529837516)(-1.67410714285714,2.53300212809624)(-1.67396763392857,2.53214728617252)(-1.673828125,2.53129077399622)(-1.67368861607143,2.53043259296284)(-1.67354910714286,2.5295727444711)(-1.67340959821429,2.52871122992299)(-1.67327008928571,2.52784805072372)(-1.67313058035714,2.52698320828175)(-1.67299107142857,2.52611670400878)(-1.6728515625,2.52524853931975)(-1.67271205357143,2.52437871563282)(-1.67257254464286,2.52350723436937)(-1.67243303571429,2.52263409695404)(-1.67229352678571,2.52175930481465)(-1.67215401785714,2.52088285938228)(-1.67201450892857,2.5200047620912)(-1.671875,2.51912501437891)(-1.67173549107143,2.51824361768611)(-1.67159598214286,2.51736057345672)(-1.67145647321429,2.51647588313787)(-1.67131696428571,2.51558954817989)(-1.67117745535714,2.51470157003629)(-1.67103794642857,2.51381195016381)(-1.6708984375,2.51292069002237)(-1.67075892857143,2.51202779107508)(-1.67061941964286,2.51113325478824)(-1.67047991071429,2.51023708263135)(-1.67034040178571,2.50933927607708)(-1.67020089285714,2.50843983660127)(-1.67006138392857,2.50753876568298)(-1.669921875,2.5066360648044)(-1.66978236607143,2.50573173545092)(-1.66964285714286,2.50482577911108)(-1.66950334821429,2.5039181972766)(-1.66936383928571,2.50300899144236)(-1.66922433035714,2.5020981631064)(-1.66908482142857,2.50118571376993)(-1.6689453125,2.50027164493729)(-1.66880580357143,2.49935595811599)(-1.66866629464286,2.49843865481669)(-1.66852678571429,2.49751973655318)(-1.66838727678571,2.49659920484241)(-1.66824776785714,2.49567706120446)(-1.66810825892857,2.49475330716257)(-1.66796875,2.49382794424307)(-1.66782924107143,2.49290097397546)(-1.66768973214286,2.49197239789235)(-1.66755022321429,2.49104221752949)(-1.66741071428571,2.49011043442574)(-1.66727120535714,2.48917705012307)(-1.66713169642857,2.4882420661666)(-1.6669921875,2.48730548410452)(-1.66685267857143,2.48636730548817)(-1.66671316964286,2.48542753187197)(-1.66657366071429,2.48448616481346)(-1.66643415178571,2.48354320587328)(-1.66629464285714,2.48259865661514)(-1.66615513392857,2.4816525186059)(-1.666015625,2.48070479341546)(-1.66587611607143,2.47975548261684)(-1.66573660714286,2.47880458778613)(-1.66559709821429,2.47785211050252)(-1.66545758928571,2.47689805234825)(-1.66531808035714,2.47594241490867)(-1.66517857142857,2.47498519977219)(-1.6650390625,2.47402640853029)(-1.66489955357143,2.4730660427775)(-1.66476004464286,2.47210410411146)(-1.66462053571429,2.47114059413283)(-1.66448102678571,2.47017551444533)(-1.66434151785714,2.46920886665577)(-1.66420200892857,2.46824065237397)(-1.6640625,2.46727087321283)(-1.66392299107143,2.46629953078827)(-1.66378348214286,2.46532662671928)(-1.66364397321429,2.46435216262787)(-1.66350446428571,2.46337614013909)(-1.66336495535714,2.46239856088101)(-1.66322544642857,2.46141942648477)(-1.6630859375,2.4604387385845)(-1.66294642857143,2.45945649881737)(-1.66280691964286,2.45847270882355)(-1.66266741071429,2.45748737024625)(-1.66252790178571,2.4565004847317)(-1.66238839285714,2.4555120539291)(-1.66224888392857,2.45452207949071)(-1.662109375,2.45353056307175)(-1.66196986607143,2.45253750633047)(-1.66183035714286,2.45154291092811)(-1.66169084821429,2.45054677852889)(-1.66155133928571,2.44954911080004)(-1.66141183035714,2.44854990941178)(-1.66127232142857,2.4475491760373)(-1.6611328125,2.44654691235277)(-1.66099330357143,2.44554312003736)(-1.66085379464286,2.4445378007732)(-1.66071428571429,2.44353095624539)(-1.66057477678571,2.44252258814201)(-1.66043526785714,2.44151269815409)(-1.66029575892857,2.44050128797563)(-1.66015625,2.43948835930359)(-1.66001674107143,2.43847391383788)(-1.65987723214286,2.43745795328138)(-1.65973772321429,2.43644047933989)(-1.65959821428571,2.43542149372218)(-1.65945870535714,2.43440099813994)(-1.65931919642857,2.43337899430783)(-1.6591796875,2.43235548394341)(-1.65904017857143,2.43133046876719)(-1.65890066964286,2.43030395050262)(-1.65876116071429,2.42927593087606)(-1.65862165178571,2.42824641161678)(-1.65848214285714,2.427215394457)(-1.65834263392857,2.42618288113183)(-1.658203125,2.42514887337931)(-1.65806361607143,2.42411337294037)(-1.65792410714286,2.42307638155887)(-1.65778459821429,2.42203790098153)(-1.65764508928571,2.42099793295802)(-1.65750558035714,2.41995647924087)(-1.65736607142857,2.41891354158551)(-1.6572265625,2.41786912175026)(-1.65708705357143,2.41682322149633)(-1.65694754464286,2.4157758425878)(-1.65680803571429,2.41472698679163)(-1.65666852678571,2.41367665587768)(-1.65652901785714,2.41262485161862)(-1.65638950892857,2.41157157579007)(-1.65625,2.41051683017044)(-1.65611049107143,2.40946061654104)(-1.65597098214286,2.40840293668603)(-1.65583147321429,2.40734379239242)(-1.65569196428571,2.40628318545007)(-1.65555245535714,2.40522111765169)(-1.65541294642857,2.40415759079285)(-1.6552734375,2.40309260667191)(-1.65513392857143,2.40202616709012)(-1.65499441964286,2.40095827385154)(-1.65485491071429,2.39988892876305)(-1.65471540178571,2.39881813363437)(-1.65457589285714,2.39774589027804)(-1.65443638392857,2.39667220050942)(-1.654296875,2.39559706614667)(-1.65415736607143,2.39452048901079)(-1.65401785714286,2.39344247092555)(-1.65387834821429,2.39236301371757)(-1.65373883928571,2.39128211921623)(-1.65359933035714,2.39019978925372)(-1.65345982142857,2.38911602566504)(-1.6533203125,2.38803083028797)(-1.65318080357143,2.38694420496306)(-1.65304129464286,2.38585615153368)(-1.65290178571429,2.38476667184593)(-1.65276227678571,2.38367576774874)(-1.65262276785714,2.38258344109376)(-1.65248325892857,2.38148969373546)(-1.65234375,2.38039452753103)(-1.65220424107143,2.37929794434044)(-1.65206473214286,2.37819994602644)(-1.65192522321429,2.37710053445449)(-1.65178571428571,2.37599971149283)(-1.65164620535714,2.37489747901244)(-1.65150669642857,2.37379383888706)(-1.6513671875,2.37268879299313)(-1.65122767857143,2.37158234320985)(-1.65108816964286,2.37047449141917)(-1.65094866071429,2.36936523950574)(-1.65080915178571,2.36825458935694)(-1.65066964285714,2.36714254286289)(-1.65053013392857,2.3660291019164)(-1.650390625,2.36491426841302)(-1.65025111607143,2.36379804425099)(-1.65011160714286,2.36268043133127)(-1.64997209821429,2.36156143155752)(-1.64983258928571,2.36044104683609)(-1.64969308035714,2.35931927907604)(-1.64955357142857,2.35819613018911)(-1.6494140625,2.35707160208974)(-1.64927455357143,2.35594569669503)(-1.64913504464286,2.3548184159248)(-1.64899553571429,2.35368976170151)(-1.64885602678571,2.35255973595032)(-1.64871651785714,2.35142834059902)(-1.64857700892857,2.35029557757813)(-1.6484375,2.34916144882076)(-1.64829799107143,2.34802595626274)(-1.64815848214286,2.34688910184251)(-1.64801897321429,2.34575088750118)(-1.64787946428571,2.34461131518251)(-1.64773995535714,2.34347038683288)(-1.64760044642857,2.34232810440136)(-1.6474609375,2.34118446983959)(-1.64732142857143,2.34003948510188)(-1.64718191964286,2.33889315214518)(-1.64704241071429,2.33774547292902)(-1.64690290178571,2.33659644941559)(-1.64676339285714,2.33544608356968)(-1.64662388392857,2.33429437735869)(-1.646484375,2.33314133275264)(-1.64634486607143,2.33198695172414)(-1.64620535714286,2.33083123624842)(-1.64606584821429,2.32967418830329)(-1.64592633928571,2.32851580986915)(-1.64578683035714,2.32735610292901)(-1.64564732142857,2.32619506946845)(-1.6455078125,2.32503271147564)(-1.64536830357143,2.32386903094133)(-1.64522879464286,2.32270402985883)(-1.64508928571429,2.32153771022402)(-1.64494977678571,2.32037007403537)(-1.64481026785714,2.31920112329389)(-1.64467075892857,2.31803086000317)(-1.64453125,2.31685928616931)(-1.64439174107143,2.31568640380102)(-1.64425223214286,2.31451221490951)(-1.64411272321429,2.31333672150856)(-1.64397321428571,2.31215992561448)(-1.64383370535714,2.3109818292461)(-1.64369419642857,2.30980243442482)(-1.6435546875,2.30862174317452)(-1.64341517857143,2.30743975752163)(-1.64327566964286,2.30625647949511)(-1.64313616071429,2.3050719111264)(-1.64299665178571,2.30388605444949)(-1.64285714285714,2.30269891150085)(-1.64271763392857,2.30151048431946)(-1.642578125,2.30032077494681)(-1.64243861607143,2.29912978542687)(-1.64229910714286,2.29793751780612)(-1.64215959821429,2.2967439741335)(-1.64202008928571,2.29554915646047)(-1.64188058035714,2.29435306684093)(-1.64174107142857,2.29315570733129)(-1.6416015625,2.29195707999041)(-1.64146205357143,2.29075718687963)(-1.64132254464286,2.28955603006274)(-1.64118303571429,2.288353611606)(-1.64104352678571,2.28714993357813)(-1.64090401785714,2.28594499805028)(-1.64076450892857,2.28473880709607)(-1.640625,2.28353136279157)(-1.64048549107143,2.28232266721525)(-1.64034598214286,2.28111272244805)(-1.64020647321429,2.27990153057333)(-1.64006696428571,2.27868909367689)(-1.63992745535714,2.27747541384692)(-1.63978794642857,2.27626049317407)(-1.6396484375,2.27504433375138)(-1.63950892857143,2.2738269376743)(-1.63936941964286,2.2726083070407)(-1.63922991071429,2.27138844395085)(-1.63909040178571,2.2701673505074)(-1.63895089285714,2.26894502881542)(-1.63881138392857,2.26772148098237)(-1.638671875,2.26649670911807)(-1.63853236607143,2.26527071533474)(-1.63839285714286,2.26404350174699)(-1.63825334821429,2.26281507047177)(-1.63811383928571,2.26158542362845)(-1.63797433035714,2.26035456333871)(-1.63783482142857,2.25912249172663)(-1.6376953125,2.25788921091864)(-1.63755580357143,2.25665472304351)(-1.63741629464286,2.25541903023238)(-1.63727678571429,2.25418213461871)(-1.63713727678571,2.25294403833833)(-1.63699776785714,2.25170474352937)(-1.63685825892857,2.25046425233235)(-1.63671875,2.24922256689005)(-1.63657924107143,2.24797968934762)(-1.63643973214286,2.24673562185252)(-1.63630022321429,2.24549036655451)(-1.63616071428571,2.2442439256057)(-1.63602120535714,2.24299630116045)(-1.63588169642857,2.24174749537549)(-1.6357421875,2.24049751040979)(-1.63560267857143,2.23924634842466)(-1.63546316964286,2.23799401158366)(-1.63532366071429,2.23674050205268)(-1.63518415178571,2.23548582199985)(-1.63504464285714,2.23422997359561)(-1.63490513392857,2.23297295901266)(-1.634765625,2.23171478042597)(-1.63462611607143,2.23045544001278)(-1.63448660714286,2.2291949399526)(-1.63434709821429,2.22793328242716)(-1.63420758928571,2.22667046962048)(-1.63406808035714,2.22540650371882)(-1.63392857142857,2.22414138691069)(-1.6337890625,2.22287512138681)(-1.63364955357143,2.22160770934018)(-1.63351004464286,2.22033915296599)(-1.63337053571429,2.21906945446168)(-1.63323102678571,2.21779861602691)(-1.63309151785714,2.21652663986355)(-1.63295200892857,2.21525352817571)(-1.6328125,2.21397928316968)(-1.63267299107143,2.21270390705396)(-1.63253348214286,2.21142740203927)(-1.63239397321429,2.21014977033851)(-1.63225446428571,2.20887101416678)(-1.63211495535714,2.20759113574136)(-1.63197544642857,2.20631013728173)(-1.6318359375,2.20502802100955)(-1.63169642857143,2.20374478914862)(-1.63155691964286,2.20246044392497)(-1.63141741071429,2.20117498756674)(-1.63127790178571,2.19988842230427)(-1.63113839285714,2.19860075037003)(-1.63099888392857,2.19731197399868)(-1.630859375,2.196022095427)(-1.63071986607143,2.19473111689393)(-1.63058035714286,2.19343904064053)(-1.63044084821429,2.19214586891003)(-1.63030133928571,2.19085160394776)(-1.63016183035714,2.1895562480012)(-1.63002232142857,2.18825980331994)(-1.6298828125,2.18696227215569)(-1.62974330357143,2.18566365676229)(-1.62960379464286,2.18436395939567)(-1.62946428571429,2.18306318231388)(-1.62932477678571,2.18176132777706)(-1.62918526785714,2.18045839804745)(-1.62904575892857,2.1791543953894)(-1.62890625,2.17784932206932)(-1.62876674107143,2.17654318035571)(-1.62862723214286,2.17523597251918)(-1.62848772321429,2.17392770083238)(-1.62834821428571,2.17261836757005)(-1.62820870535714,2.17130797500896)(-1.62806919642857,2.16999652542801)(-1.6279296875,2.16868402110809)(-1.62779017857143,2.16737046433217)(-1.62765066964286,2.16605585738529)(-1.62751116071429,2.1647402025545)(-1.62737165178571,2.1634235021289)(-1.62723214285714,2.16210575839962)(-1.62709263392857,2.16078697365984)(-1.626953125,2.15946715020476)(-1.62681361607143,2.15814629033157)(-1.62667410714286,2.15682439633953)(-1.62653459821429,2.15550147052986)(-1.62639508928571,2.15417751520584)(-1.62625558035714,2.1528525326727)(-1.62611607142857,2.15152652523771)(-1.6259765625,2.15019949521013)(-1.62583705357143,2.14887144490118)(-1.62569754464286,2.1475423766241)(-1.62555803571429,2.1462122926941)(-1.62541852678571,2.14488119542836)(-1.62527901785714,2.14354908714604)(-1.62513950892857,2.14221597016826)(-1.625,2.14088184681811)(-1.62486049107143,2.13954671942064)(-1.62472098214286,2.13821059030285)(-1.62458147321429,2.13687346179369)(-1.62444196428571,2.13553533622405)(-1.62430245535714,2.13419621592677)(-1.62416294642857,2.13285610323663)(-1.6240234375,2.13151500049032)(-1.62388392857143,2.13017291002649)(-1.62374441964286,2.12882983418567)(-1.62360491071429,2.12748577531034)(-1.62346540178571,2.12614073574489)(-1.62332589285714,2.12479471783561)(-1.62318638392857,2.12344772393069)(-1.623046875,2.12209975638023)(-1.62290736607143,2.12075081753622)(-1.62276785714286,2.11940090975256)(-1.62262834821429,2.11805003538499)(-1.62248883928571,2.11669819679117)(-1.62234933035714,2.11534539633062)(-1.62220982142857,2.11399163636476)(-1.6220703125,2.11263691925682)(-1.62193080357143,2.11128124737196)(-1.62179129464286,2.10992462307715)(-1.62165178571429,2.10856704874123)(-1.62151227678571,2.10720852673489)(-1.62137276785714,2.10584905943066)(-1.62123325892857,2.10448864920293)(-1.62109375,2.10312729842789)(-1.62095424107143,2.10176500948358)(-1.62081473214286,2.10040178474987)(-1.62067522321429,2.09903762660845)(-1.62053571428571,2.09767253744281)(-1.62039620535714,2.09630651963828)(-1.62025669642857,2.09493957558197)(-1.6201171875,2.09357170766281)(-1.61997767857143,2.09220291827152)(-1.61983816964286,2.09083320980062)(-1.61969866071429,2.08946258464442)(-1.61955915178571,2.088091045199)(-1.61941964285714,2.08671859386222)(-1.61928013392857,2.08534523303375)(-1.619140625,2.08397096511498)(-1.61900111607143,2.0825957925091)(-1.61886160714286,2.08121971762105)(-1.61872209821429,2.07984274285752)(-1.61858258928571,2.07846487062697)(-1.61844308035714,2.07708610333957)(-1.61830357142857,2.07570644340729)(-1.6181640625,2.07432589324377)(-1.61802455357143,2.07294445526443)(-1.61788504464286,2.07156213188641)(-1.61774553571429,2.07017892552856)(-1.61760602678571,2.06879483861146)(-1.61746651785714,2.06740987355739)(-1.61732700892857,2.06602403279037)(-1.6171875,2.06463731873607)(-1.61704799107143,2.06324973382192)(-1.61690848214286,2.06186128047701)(-1.61676897321429,2.06047196113213)(-1.61662946428571,2.05908177821975)(-1.61648995535714,2.05769073417401)(-1.61635044642857,2.05629883143077)(-1.6162109375,2.05490607242751)(-1.61607142857143,2.0535124596034)(-1.61593191964286,2.05211799539927)(-1.61579241071429,2.05072268225761)(-1.61565290178571,2.04932652262257)(-1.61551339285714,2.04792951893991)(-1.61537388392857,2.04653167365708)(-1.615234375,2.04513298922314)(-1.61509486607143,2.04373346808879)(-1.61495535714286,2.04233311270636)(-1.61481584821429,2.0409319255298)(-1.61467633928571,2.03952990901468)(-1.61453683035714,2.03812706561819)(-1.61439732142857,2.03672339779913)(-1.6142578125,2.03531890801789)(-1.61411830357143,2.03391359873647)(-1.61397879464286,2.03250747241847)(-1.61383928571429,2.03110053152906)(-1.61369977678571,2.02969277853503)(-1.61356026785714,2.02828421590471)(-1.61342075892857,2.02687484610805)(-1.61328125,2.02546467161652)(-1.61314174107143,2.0240536949032)(-1.61300223214286,2.02264191844272)(-1.61286272321429,2.02122934471124)(-1.61272321428571,2.01981597618651)(-1.61258370535714,2.01840181534779)(-1.61244419642857,2.01698686467593)(-1.6123046875,2.01557112665327)(-1.61216517857143,2.0141546037637)(-1.61202566964286,2.01273729849266)(-1.61188616071429,2.01131921332706)(-1.61174665178571,2.00990035075538)(-1.61160714285714,2.00848071326758)(-1.61146763392857,2.00706030335516)(-1.611328125,2.00563912351109)(-1.61118861607143,2.00421717622985)(-1.61104910714286,2.00279446400743)(-1.61090959821429,2.00137098934127)(-1.61077008928571,1.99994675473035)(-1.61063058035714,1.99852176267507)(-1.61049107142857,1.99709601567736)(-1.6103515625,1.99566951624056)(-1.61021205357143,1.99424226686953)(-1.61007254464286,1.99281427007056)(-1.60993303571429,1.9913855283514)(-1.60979352678571,1.98995604422124)(-1.60965401785714,1.98852582019073)(-1.60951450892857,1.98709485877197)(-1.609375,1.98566316247846)(-1.60923549107143,1.98423073382516)(-1.60909598214286,1.98279757532845)(-1.60895647321429,1.98136368950611)(-1.60881696428571,1.97992907887737)(-1.60867745535714,1.97849374596284)(-1.60853794642857,1.97705769328455)(-1.6083984375,1.97562092336594)(-1.60825892857143,1.97418343873182)(-1.60811941964286,1.97274524190842)(-1.60797991071429,1.97130633542333)(-1.60784040178571,1.96986672180555)(-1.60770089285714,1.96842640358541)(-1.60756138392857,1.96698538329468)(-1.607421875,1.96554366346642)(-1.60728236607143,1.96410124663511)(-1.60714285714286,1.96265813533657)(-1.60700334821429,1.96121433210795)(-1.60686383928571,1.95976983948778)(-1.60672433035714,1.9583246600159)(-1.60658482142857,1.95687879623352)(-1.6064453125,1.95543225068315)(-1.60630580357143,1.95398502590864)(-1.60616629464286,1.95253712445518)(-1.60602678571429,1.95108854886924)(-1.60588727678571,1.94963930169864)(-1.60574776785714,1.94818938549247)(-1.60560825892857,1.94673880280116)(-1.60546875,1.9452875561764)(-1.60532924107143,1.94383564817121)(-1.60518973214286,1.94238308133987)(-1.60505022321429,1.94092985823794)(-1.60491071428571,1.93947598142229)(-1.60477120535714,1.93802145345101)(-1.60463169642857,1.93656627688352)(-1.6044921875,1.93511045428044)(-1.60435267857143,1.9336539882037)(-1.60421316964286,1.93219688121644)(-1.60407366071429,1.93073913588308)(-1.60393415178571,1.92928075476927)(-1.60379464285714,1.92782174044188)(-1.60365513392857,1.92636209546904)(-1.603515625,1.9249018224201)(-1.60337611607143,1.92344092386563)(-1.60323660714286,1.92197940237741)(-1.60309709821429,1.92051726052843)(-1.60295758928571,1.91905450089291)(-1.60281808035714,1.91759112604625)(-1.60267857142857,1.91612713856506)(-1.6025390625,1.91466254102713)(-1.60239955357143,1.91319733601146)(-1.60226004464286,1.91173152609821)(-1.60212053571429,1.91026511386872)(-1.60198102678571,1.90879810190552)(-1.60184151785714,1.90733049279227)(-1.60170200892857,1.90586228911384)(-1.6015625,1.90439349345622)(-1.60142299107143,1.90292410840658)(-1.60128348214286,1.90145413655322)(-1.60114397321429,1.89998358048557)(-1.60100446428571,1.89851244279422)(-1.60086495535714,1.89704072607089)(-1.60072544642857,1.89556843290842)(-1.6005859375,1.89409556590077)(-1.60044642857143,1.89262212764301)(-1.60030691964286,1.89114812073136)(-1.60016741071429,1.88967354776309)(-1.60002790178571,1.88819841133661)(-1.59988839285714,1.88672271405142)(-1.59974888392857,1.8852464585081)(-1.599609375,1.88376964730834)(-1.59946986607143,1.88229228305488)(-1.59933035714286,1.88081436835157)(-1.59919084821429,1.87933590580329)(-1.59905133928571,1.87785689801604)(-1.59891183035714,1.87637734759682)(-1.59877232142857,1.87489725715374)(-1.5986328125,1.87341662929594)(-1.59849330357143,1.87193546663359)(-1.59835379464286,1.87045377177792)(-1.59821428571429,1.8689715473412)(-1.59807477678571,1.86748879593671)(-1.59793526785714,1.86600552017877)(-1.59779575892857,1.86452172268273)(-1.59765625,1.86303740606493)(-1.59751674107143,1.86155257294274)(-1.59737723214286,1.86006722593453)(-1.59723772321429,1.85858136765967)(-1.59709821428571,1.85709500073852)(-1.59695870535714,1.85560812779243)(-1.59681919642857,1.85412075144375)(-1.5966796875,1.8526328743158)(-1.59654017857143,1.85114449903287)(-1.59640066964286,1.84965562822021)(-1.59626116071429,1.84816626450407)(-1.59612165178571,1.84667641051162)(-1.59598214285714,1.84518606887101)(-1.59584263392857,1.84369524221133)(-1.595703125,1.84220393316261)(-1.59556361607143,1.84071214435583)(-1.59542410714286,1.83921987842289)(-1.59528459821429,1.83772713799663)(-1.59514508928571,1.8362339257108)(-1.59500558035714,1.83474024420009)(-1.59486607142857,1.83324609610008)(-1.5947265625,1.83175148404728)(-1.59458705357143,1.83025641067908)(-1.59444754464286,1.82876087863379)(-1.59430803571429,1.82726489055059)(-1.59416852678571,1.82576844906957)(-1.59402901785714,1.82427155683169)(-1.59388950892857,1.82277421647879)(-1.59375,1.82127643065358)(-1.59361049107143,1.81977820199964)(-1.59347098214286,1.81827953316141)(-1.59333147321429,1.81678042678419)(-1.59319196428571,1.81528088551412)(-1.59305245535714,1.81378091199821)(-1.59291294642857,1.81228050888428)(-1.5927734375,1.81077967882101)(-1.59263392857143,1.80927842445791)(-1.59249441964286,1.8077767484453)(-1.59235491071429,1.80627465343434)(-1.59221540178571,1.80477214207698)(-1.59207589285714,1.80326921702601)(-1.59193638392857,1.80176588093501)(-1.591796875,1.80026213645835)(-1.59165736607143,1.79875798625122)(-1.59151785714286,1.79725343296958)(-1.59137834821429,1.79574847927018)(-1.59123883928571,1.79424312781054)(-1.59109933035714,1.79273738124898)(-1.59095982142857,1.79123124224456)(-1.5908203125,1.78972471345712)(-1.59068080357143,1.78821779754726)(-1.59054129464286,1.78671049717632)(-1.59040178571429,1.78520281500641)(-1.59026227678571,1.78369475370035)(-1.59012276785714,1.78218631592173)(-1.58998325892857,1.78067750433486)(-1.58984375,1.77916832160478)(-1.58970424107143,1.77765877039724)(-1.58956473214286,1.77614885337873)(-1.58942522321429,1.77463857321643)(-1.58928571428571,1.77312793257825)(-1.58914620535714,1.77161693413276)(-1.58900669642857,1.77010558054928)(-1.5888671875,1.76859387449779)(-1.58872767857143,1.76708181864895)(-1.58858816964286,1.76556941567413)(-1.58844866071429,1.76405666824534)(-1.58830915178571,1.76254357903529)(-1.58816964285714,1.76103015071734)(-1.58803013392857,1.75951638596551)(-1.587890625,1.75800228745448)(-1.58775111607143,1.75648785785958)(-1.58761160714286,1.75497309985678)(-1.58747209821429,1.75345801612269)(-1.58733258928571,1.75194260933456)(-1.58719308035714,1.75042688217026)(-1.58705357142857,1.74891083730828)(-1.5869140625,1.74739447742776)(-1.58677455357143,1.7458778052084)(-1.58663504464286,1.74436082333056)(-1.58649553571429,1.74284353447516)(-1.58635602678571,1.74132594132375)(-1.58621651785714,1.73980804655846)(-1.58607700892857,1.73828985286199)(-1.5859375,1.73677136291766)(-1.58579799107143,1.73525257940934)(-1.58565848214286,1.73373350502146)(-1.58551897321429,1.73221414243905)(-1.58537946428571,1.73069449434767)(-1.58523995535714,1.72917456343345)(-1.58510044642857,1.72765435238308)(-1.5849609375,1.72613386388378)(-1.58482142857143,1.72461310062331)(-1.58468191964286,1.72309206528997)(-1.58454241071429,1.72157076057259)(-1.58440290178571,1.72004918916052)(-1.58426339285714,1.71852735374364)(-1.58412388392857,1.71700525701233)(-1.583984375,1.71548290165748)(-1.58384486607143,1.7139602903705)(-1.58370535714286,1.71243742584328)(-1.58356584821429,1.7109143107682)(-1.58342633928571,1.70939094783815)(-1.58328683035714,1.70786733974647)(-1.58314732142857,1.70634348918702)(-1.5830078125,1.7048193988541)(-1.58286830357143,1.70329507144248)(-1.58272879464286,1.70177050964739)(-1.58258928571429,1.70024571616453)(-1.58244977678571,1.69872069369005)(-1.58231026785714,1.69719544492052)(-1.58217075892857,1.69566997255298)(-1.58203125,1.6941442792849)(-1.58189174107143,1.69261836781417)(-1.58175223214286,1.69109224083911)(-1.58161272321429,1.68956590105845)(-1.58147321428571,1.68803935117135)(-1.58133370535714,1.68651259387737)(-1.58119419642857,1.68498563187647)(-1.5810546875,1.68345846786902)(-1.58091517857143,1.68193110455578)(-1.58077566964286,1.68040354463789)(-1.58063616071429,1.67887579081688)(-1.58049665178571,1.67734784579464)(-1.58035714285714,1.67581971227347)(-1.58021763392857,1.674291392956)(-1.580078125,1.67276289054523)(-1.57993861607143,1.67123420774454)(-1.57979910714286,1.66970534725763)(-1.57965959821429,1.66817631178855)(-1.57952008928571,1.66664710404172)(-1.57938058035714,1.66511772672186)(-1.57924107142857,1.66358818253404)(-1.5791015625,1.66205847418365)(-1.57896205357143,1.66052860437639)(-1.57882254464286,1.65899857581829)(-1.57868303571429,1.65746839121569)(-1.57854352678571,1.65593805327521)(-1.57840401785714,1.65440756470379)(-1.57826450892857,1.65287692820866)(-1.578125,1.65134614649733)(-1.57798549107143,1.6498152222776)(-1.57784598214286,1.64828415825755)(-1.57770647321429,1.64675295714553)(-1.57756696428571,1.64522162165014)(-1.57742745535714,1.64369015448026)(-1.57728794642857,1.64215855834503)(-1.5771484375,1.64062683595382)(-1.57700892857143,1.63909499001628)(-1.57686941964286,1.63756302324225)(-1.57672991071429,1.63603093834186)(-1.57659040178571,1.63449873802543)(-1.57645089285714,1.63296642500352)(-1.57631138392857,1.63143400198692)(-1.576171875,1.6299014716866)(-1.57603236607143,1.62836883681378)(-1.57589285714286,1.62683610007986)(-1.57575334821429,1.62530326419644)(-1.57561383928571,1.62377033187532)(-1.57547433035714,1.62223730582847)(-1.57533482142857,1.62070418876808)(-1.5751953125,1.61917098340647)(-1.57505580357143,1.61763769245616)(-1.57491629464286,1.61610431862984)(-1.57477678571429,1.61457086464035)(-1.57463727678571,1.61303733320069)(-1.57449776785714,1.611503727024)(-1.57435825892857,1.60997004882358)(-1.57421875,1.60843630131287)(-1.57407924107143,1.60690248720544)(-1.57393973214286,1.60536860921498)(-1.57380022321429,1.60383467005532)(-1.57366071428571,1.60230067244041)(-1.57352120535714,1.60076661908429)(-1.57338169642857,1.59923251270114)(-1.5732421875,1.59769835600522)(-1.57310267857143,1.59616415171091)(-1.57296316964286,1.59462990253265)(-1.57282366071429,1.593095611185)(-1.57268415178571,1.59156128038258)(-1.57254464285714,1.5900269128401)(-1.57240513392857,1.58849251127234)(-1.572265625,1.58695807839413)(-1.57212611607143,1.58542361692039)(-1.57198660714286,1.58388912956607)(-1.57184709821429,1.58235461904619)(-1.57170758928571,1.58082008807579)(-1.57156808035714,1.57928553936997)(-1.57142857142857,1.57775097564387)(-1.5712890625,1.57621639961263)(-1.57114955357143,1.57468181399145)(-1.57101004464286,1.57314722149553)(-1.57087053571429,1.57161262484007)(-1.57073102678571,1.5700780267403)(-1.57059151785714,1.56854342991145)(-1.57045200892857,1.56700883706874)(-1.5703125,1.56547425092738)(-1.57017299107143,1.56393967420259)(-1.57003348214286,1.56240510960953)(-1.56989397321429,1.56087055986337)(-1.56975446428571,1.55933602767925)(-1.56961495535714,1.55780151577226)(-1.56947544642857,1.55626702685745)(-1.5693359375,1.55473256364985)(-1.56919642857143,1.55319812886441)(-1.56905691964286,1.55166372521605)(-1.56891741071429,1.5501293554196)(-1.56877790178571,1.54859502218985)(-1.56863839285714,1.54706072824151)(-1.56849888392857,1.54552647628921)(-1.568359375,1.54399226904751)(-1.56821986607143,1.54245810923087)(-1.56808035714286,1.54092399955366)(-1.56794084821429,1.53938994273016)(-1.56780133928571,1.53785594147454)(-1.56766183035714,1.53632199850086)(-1.56752232142857,1.53478811652307)(-1.5673828125,1.53325429825502)(-1.56724330357143,1.53172054641039)(-1.56710379464286,1.53018686370277)(-1.56696428571429,1.52865325284561)(-1.56682477678571,1.52711971655221)(-1.56668526785714,1.52558625753573)(-1.56654575892857,1.52405287850918)(-1.56640625,1.5225195821854)(-1.56626674107143,1.5209863712771)(-1.56612723214286,1.51945324849679)(-1.56598772321429,1.51792021655684)(-1.56584821428571,1.51638727816941)(-1.56570870535714,1.51485443604651)(-1.56556919642857,1.51332169289994)(-1.5654296875,1.51178905144132)(-1.56529017857143,1.51025651438206)(-1.56515066964286,1.50872408443338)(-1.56501116071429,1.50719176430629)(-1.56487165178571,1.50565955671157)(-1.56473214285714,1.5041274643598)(-1.56459263392857,1.50259548996132)(-1.564453125,1.50106363622627)(-1.56431361607143,1.49953190586452)(-1.56417410714286,1.49800030158571)(-1.56403459821429,1.49646882609925)(-1.56389508928571,1.49493748211427)(-1.56375558035714,1.49340627233968)(-1.56361607142857,1.49187519948411)(-1.5634765625,1.49034426625592)(-1.56333705357143,1.4888134753632)(-1.56319754464286,1.48728282951377)(-1.56305803571429,1.48575233141515)(-1.56291852678571,1.48422198377461)(-1.56277901785714,1.48269178929908)(-1.56263950892857,1.48116175069523)(-1.5625,1.4796318706694)(-1.56236049107143,1.47810215192765)(-1.56222098214286,1.47657259717569)(-1.56208147321429,1.47504320911894)(-1.56194196428571,1.47351399046248)(-1.56180245535714,1.47198494391107)(-1.56166294642857,1.47045607216913)(-1.5615234375,1.46892737794074)(-1.56138392857143,1.46739886392963)(-1.56124441964286,1.4658705328392)(-1.56110491071429,1.46434238737246)(-1.56096540178571,1.46281443023208)(-1.56082589285714,1.46128666412037)(-1.56068638392857,1.45975909173925)(-1.560546875,1.45823171579027)(-1.56040736607143,1.45670453897461)(-1.56026785714286,1.45517756399305)(-1.56012834821429,1.45365079354597)(-1.55998883928571,1.45212423033336)(-1.55984933035714,1.45059787705481)(-1.55970982142857,1.44907173640951)(-1.5595703125,1.44754581109622)(-1.55943080357143,1.44602010381327)(-1.55929129464286,1.4444946172586)(-1.55915178571429,1.44296935412969)(-1.55901227678571,1.44144431712361)(-1.55887276785714,1.43991950893696)(-1.55873325892857,1.43839493226593)(-1.55859375,1.43687058980623)(-1.55845424107143,1.43534648425313)(-1.55831473214286,1.43382261830143)(-1.55817522321429,1.43229899464547)(-1.55803571428571,1.43077561597911)(-1.55789620535714,1.42925248499575)(-1.55775669642857,1.42772960438829)(-1.5576171875,1.42620697684915)(-1.55747767857143,1.42468460507027)(-1.55733816964286,1.42316249174306)(-1.55719866071429,1.42164063955846)(-1.55705915178571,1.42011905120689)(-1.55691964285714,1.41859772937824)(-1.55678013392857,1.41707667676191)(-1.556640625,1.41555589604676)(-1.55650111607143,1.41403538992112)(-1.55636160714286,1.41251516107279)(-1.55622209821429,1.41099521218903)(-1.55608258928571,1.40947554595655)(-1.55594308035714,1.40795616506152)(-1.55580357142857,1.40643707218955)(-1.5556640625,1.40491827002568)(-1.55552455357143,1.4033997612544)(-1.55538504464286,1.40188154855961)(-1.55524553571429,1.40036363462466)(-1.55510602678571,1.39884602213229)(-1.55496651785714,1.39732871376467)(-1.55482700892857,1.39581171220339)(-1.5546875,1.39429502012941)(-1.55454799107143,1.39277864022311)(-1.55440848214286,1.39126257516426)(-1.55426897321429,1.38974682763201)(-1.55412946428571,1.38823140030491)(-1.55398995535714,1.38671629586085)(-1.55385044642857,1.38520151697715)(-1.5537109375,1.38368706633043)(-1.55357142857143,1.38217294659672)(-1.55343191964286,1.38065916045139)(-1.55329241071429,1.37914571056914)(-1.55315290178571,1.37763259962406)(-1.55301339285714,1.37611983028954)(-1.55287388392857,1.37460740523833)(-1.552734375,1.3730953271425)(-1.55259486607143,1.37158359867344)(-1.55245535714286,1.37007222250187)(-1.55231584821429,1.36856120129781)(-1.55217633928571,1.36705053773061)(-1.55203683035714,1.3655402344689)(-1.55189732142857,1.36403029418063)(-1.5517578125,1.36252071953304)(-1.55161830357143,1.36101151319264)(-1.55147879464286,1.35950267782525)(-1.55133928571429,1.35799421609594)(-1.55119977678571,1.35648613066907)(-1.55106026785714,1.35497842420827)(-1.55092075892857,1.35347109937642)(-1.55078125,1.35196415883568)(-1.55064174107143,1.35045760524743)(-1.55050223214286,1.34895144127232)(-1.55036272321429,1.34744566957023)(-1.55022321428571,1.34594029280029)(-1.55008370535714,1.34443531362084)(-1.54994419642857,1.34293073468948)(-1.5498046875,1.34142655866299)(-1.54966517857143,1.33992278819739)(-1.54952566964286,1.33841942594791)(-1.54938616071429,1.33691647456898)(-1.54924665178571,1.33541393671424)(-1.54910714285714,1.3339118150365)(-1.54896763392857,1.33241011218779)(-1.548828125,1.33090883081932)(-1.54868861607143,1.32940797358146)(-1.54854910714286,1.32790754312376)(-1.54840959821429,1.32640754209496)(-1.54827008928571,1.32490797314295)(-1.54813058035714,1.32340883891477)(-1.54799107142857,1.32191014205663)(-1.5478515625,1.32041188521388)(-1.54771205357143,1.31891407103101)(-1.54757254464286,1.31741670215166)(-1.54743303571429,1.3159197812186)(-1.54729352678571,1.31442331087372)(-1.54715401785714,1.31292729375804)(-1.54701450892857,1.31143173251169)(-1.546875,1.30993662977393)(-1.54673549107143,1.30844198818312)(-1.54659598214286,1.3069478103767)(-1.54645647321429,1.30545409899124)(-1.54631696428571,1.30396085666239)(-1.54617745535714,1.30246808602488)(-1.54603794642857,1.30097578971253)(-1.5458984375,1.29948397035824)(-1.54575892857143,1.29799263059397)(-1.54561941964286,1.29650177305075)(-1.54547991071429,1.29501140035868)(-1.54534040178571,1.29352151514691)(-1.54520089285714,1.29203212004363)(-1.54506138392857,1.29054321767611)(-1.544921875,1.28905481067063)(-1.54478236607143,1.28756690165252)(-1.54464285714286,1.28607949324612)(-1.54450334821429,1.28459258807482)(-1.54436383928571,1.28310618876102)(-1.54422433035714,1.28162029792614)(-1.54408482142857,1.28013491819062)(-1.5439453125,1.27865005217387)(-1.54380580357143,1.27716570249434)(-1.54366629464286,1.27568187176945)(-1.54352678571429,1.27419856261562)(-1.54338727678571,1.27271577764825)(-1.54324776785714,1.27123351948173)(-1.54310825892857,1.26975179072942)(-1.54296875,1.26827059400364)(-1.54282924107143,1.26678993191568)(-1.54268973214286,1.2653098070758)(-1.54255022321429,1.2638302220932)(-1.54241071428571,1.26235117957603)(-1.54227120535714,1.2608726821314)(-1.54213169642857,1.25939473236534)(-1.5419921875,1.25791733288283)(-1.54185267857143,1.25644048628776)(-1.54171316964286,1.25496419518297)(-1.54157366071429,1.25348846217019)(-1.54143415178571,1.25201328985008)(-1.54129464285714,1.25053868082222)(-1.54115513392857,1.24906463768507)(-1.541015625,1.24759116303599)(-1.54087611607143,1.24611825947126)(-1.54073660714286,1.24464592958603)(-1.54059709821429,1.24317417597431)(-1.54045758928571,1.24170300122905)(-1.54031808035714,1.240232407942)(-1.54017857142857,1.23876239870384)(-1.5400390625,1.23729297610408)(-1.53989955357143,1.2358241427311)(-1.53976004464286,1.23435590117212)(-1.53962053571429,1.23288825401323)(-1.53948102678571,1.23142120383933)(-1.53934151785714,1.2299547532342)(-1.53920200892857,1.22848890478042)(-1.5390625,1.22702366105942)(-1.53892299107143,1.22555902465144)(-1.53878348214286,1.22409499813553)(-1.53864397321429,1.22263158408957)(-1.53850446428571,1.22116878509025)(-1.53836495535714,1.21970660371304)(-1.53822544642857,1.21824504253223)(-1.5380859375,1.2167841041209)(-1.53794642857143,1.21532379105091)(-1.53780691964286,1.21386410589291)(-1.53766741071429,1.21240505121632)(-1.53752790178571,1.21094662958934)(-1.53738839285714,1.20948884357893)(-1.53724888392857,1.20803169575082)(-1.537109375,1.2065751886695)(-1.53696986607143,1.2051193248982)(-1.53683035714286,1.20366410699892)(-1.53669084821429,1.20220953753239)(-1.53655133928571,1.20075561905806)(-1.53641183035714,1.19930235413415)(-1.53627232142857,1.19784974531757)(-1.5361328125,1.196397795164)(-1.53599330357143,1.19494650622778)(-1.53585379464286,1.19349588106201)(-1.53571428571429,1.19204592221848)(-1.53557477678571,1.19059663224768)(-1.53543526785714,1.1891480136988)(-1.53529575892857,1.18770006911973)(-1.53515625,1.18625280105705)(-1.53501674107143,1.18480621205601)(-1.53487723214286,1.18336030466054)(-1.53473772321429,1.18191508141326)(-1.53459821428571,1.18047054485545)(-1.53445870535714,1.17902669752703)(-1.53431919642857,1.17758354196663)(-1.5341796875,1.17614108071148)(-1.53404017857143,1.17469931629749)(-1.53390066964286,1.17325825125921)(-1.53376116071429,1.17181788812981)(-1.53362165178571,1.17037822944113)(-1.53348214285714,1.16893927772359)(-1.53334263392857,1.16750103550629)(-1.533203125,1.16606350531691)(-1.53306361607143,1.16462668968175)(-1.53292410714286,1.16319059112572)(-1.53278459821429,1.16175521217235)(-1.53264508928571,1.16032055534375)(-1.53250558035714,1.15888662316064)(-1.53236607142857,1.15745341814231)(-1.5322265625,1.15602094280666)(-1.53208705357143,1.15458919967015)(-1.53194754464286,1.15315819124782)(-1.53180803571429,1.15172792005329)(-1.53166852678571,1.15029838859872)(-1.53152901785714,1.14886959939487)(-1.53138950892857,1.14744155495101)(-1.53125,1.146014257775)(-1.53111049107143,1.14458771037322)(-1.53097098214286,1.1431619152506)(-1.53083147321429,1.14173687491061)(-1.53069196428571,1.14031259185524)(-1.53055245535714,1.13888906858501)(-1.53041294642857,1.13746630759897)(-1.5302734375,1.13604431139468)(-1.53013392857143,1.13462308246821)(-1.52999441964286,1.13320262331413)(-1.52985491071429,1.13178293642554)(-1.52971540178571,1.13036402429399)(-1.52957589285714,1.12894588940957)(-1.52943638392857,1.12752853426083)(-1.529296875,1.1261119613348)(-1.52915736607143,1.12469617311701)(-1.52901785714286,1.12328117209145)(-1.52887834821429,1.12186696074056)(-1.52873883928571,1.12045354154527)(-1.52859933035714,1.11904091698495)(-1.52845982142857,1.11762908953745)(-1.5283203125,1.11621806167904)(-1.52818080357143,1.11480783588444)(-1.52804129464286,1.11339841462682)(-1.52790178571429,1.11198980037777)(-1.52776227678571,1.11058199560732)(-1.52762276785714,1.10917500278392)(-1.52748325892857,1.10776882437445)(-1.52734375,1.10636346284418)(-1.52720424107143,1.10495892065681)(-1.52706473214286,1.10355520027445)(-1.52692522321429,1.10215230415759)(-1.52678571428571,1.10075023476514)(-1.52664620535714,1.09934899455437)(-1.52650669642857,1.09794858598098)(-1.5263671875,1.096549011499)(-1.52622767857143,1.09515027356089)(-1.52608816964286,1.09375237461743)(-1.52594866071429,1.09235531711779)(-1.52580915178571,1.09095910350953)(-1.52566964285714,1.08956373623852)(-1.52553013392857,1.08816921774901)(-1.525390625,1.08677555048359)(-1.52525111607143,1.0853827368832)(-1.52511160714286,1.08399077938711)(-1.52497209821429,1.08259968043293)(-1.52483258928571,1.08120944245659)(-1.52469308035714,1.07982006789237)(-1.52455357142857,1.07843155917283)(-1.5244140625,1.07704391872887)(-1.52427455357143,1.07565714898971)(-1.52413504464286,1.07427125238285)(-1.52399553571429,1.07288623133409)(-1.52385602678571,1.07150208826756)(-1.52371651785714,1.07011882560565)(-1.52357700892857,1.06873644576904)(-1.5234375,1.0673549511767)(-1.52329799107143,1.06597434424588)(-1.52315848214286,1.06459462739209)(-1.52301897321429,1.06321580302911)(-1.52287946428571,1.061837873569)(-1.52273995535714,1.06046084142205)(-1.52260044642857,1.05908470899683)(-1.5224609375,1.05770947870015)(-1.52232142857143,1.05633515293706)(-1.52218191964286,1.05496173411085)(-1.52204241071429,1.05358922462303)(-1.52190290178571,1.05221762687339)(-1.52176339285714,1.05084694325988)(-1.52162388392857,1.04947717617873)(-1.521484375,1.04810832802434)(-1.52134486607143,1.04674040118935)(-1.52120535714286,1.0453733980646)(-1.52106584821429,1.04400732103912)(-1.52092633928571,1.04264217250015)(-1.52078683035714,1.04127795483312)(-1.52064732142857,1.03991467042166)(-1.5205078125,1.03855232164756)(-1.52036830357143,1.0371909108908)(-1.52022879464286,1.03583044052954)(-1.52008928571429,1.03447091294011)(-1.51994977678571,1.033112330497)(-1.51981026785714,1.03175469557284)(-1.51967075892857,1.03039801053847)(-1.51953125,1.02904227776282)(-1.51939174107143,1.02768749961301)(-1.51925223214286,1.02633367845428)(-1.51911272321429,1.02498081665002)(-1.51897321428571,1.02362891656174)(-1.51883370535714,1.02227798054909)(-1.51869419642857,1.02092801096983)(-1.5185546875,1.01957901017987)(-1.51841517857143,1.01823098053319)(-1.51827566964286,1.01688392438191)(-1.51813616071429,1.01553784407624)(-1.51799665178571,1.01419274196451)(-1.51785714285714,1.01284862039313)(-1.51771763392857,1.01150548170661)(-1.517578125,1.01016332824753)(-1.51743861607143,1.00882216235658)(-1.51729910714286,1.00748198637251)(-1.51715959821429,1.00614280263214)(-1.51702008928571,1.00480461347037)(-1.51688058035714,1.00346742122017)(-1.51674107142857,1.00213122821254)(-1.5166015625,1.00079603677657)(-1.51646205357143,0.999461849239375)(-1.51632254464286,0.998128667926135)(-1.51618303571429,0.996796495160059)(-1.51604352678571,0.9954653332624)(-1.51590401785714,0.994135184552437)(-1.51576450892857,0.992806051347486)(-1.515625,0.99147793596288)(-1.51548549107143,0.990150840711971)(-1.51534598214286,0.988824767906129)(-1.51520647321429,0.987499719854729)(-1.51506696428571,0.986175698865158)(-1.51492745535714,0.984852707242797)(-1.51478794642857,0.983530747291033)(-1.5146484375,0.982209821311237)(-1.51450892857143,0.980889931602769)(-1.51436941964286,0.979571080462974)(-1.51422991071429,0.978253270187172)(-1.51409040178571,0.976936503068665)(-1.51395089285714,0.975620781398714)(-1.51381138392857,0.974306107466558)(-1.513671875,0.972992483559385)(-1.51353236607143,0.971679911962347)(-1.51339285714286,0.970368394958542)(-1.51325334821429,0.96905793482902)(-1.51311383928571,0.967748533852774)(-1.51297433035714,0.96644019430673)(-1.51283482142857,0.965132918465758)(-1.5126953125,0.96382670860265)(-1.51255580357143,0.962521566988126)(-1.51241629464286,0.961217495890824)(-1.51227678571429,0.959914497577302)(-1.51213727678571,0.95861257431203)(-1.51199776785714,0.957311728357381)(-1.51185825892857,0.95601196197364)(-1.51171875,0.954713277418984)(-1.51157924107143,0.953415676949486)(-1.51143973214286,0.952119162819108)(-1.51130022321429,0.950823737279698)(-1.51116071428571,0.949529402580989)(-1.51102120535714,0.948236160970584)(-1.51088169642857,0.946944014693969)(-1.5107421875,0.945652965994488)(-1.51060267857143,0.944363017113354)(-1.51046316964286,0.943074170289636)(-1.51032366071429,0.941786427760261)(-1.51018415178571,0.940499791760008)(-1.51004464285714,0.939214264521496)(-1.50990513392857,0.937929848275196)(-1.509765625,0.936646545249408)(-1.50962611607143,0.935364357670269)(-1.50948660714286,0.934083287761745)(-1.50934709821429,0.932803337745624)(-1.50920758928571,0.93152450984152)(-1.50906808035714,0.930246806266854)(-1.50892857142857,0.928970229236872)(-1.5087890625,0.927694780964616)(-1.50864955357143,0.926420463660935)(-1.50851004464286,0.925147279534477)(-1.50837053571429,0.923875230791683)(-1.50823102678571,0.922604319636788)(-1.50809151785714,0.921334548271807)(-1.50795200892857,0.920065918896545)(-1.5078125,0.918798433708577)(-1.50767299107143,0.917532094903254)(-1.50753348214286,0.916266904673695)(-1.50739397321429,0.915002865210783)(-1.50725446428571,0.913739978703165)(-1.50711495535714,0.912478247337237)(-1.50697544642857,0.911217673297157)(-1.5068359375,0.909958258764821)(-1.50669642857143,0.908700005919873)(-1.50655691964286,0.907442916939695)(-1.50641741071429,0.906186993999402)(-1.50627790178571,0.904932239271843)(-1.50613839285714,0.903678654927589)(-1.50599888392857,0.902426243134942)(-1.505859375,0.901175006059911)(-1.50571986607143,0.899924945866224)(-1.50558035714286,0.898676064715318)(-1.50544084821429,0.897428364766334)(-1.50530133928571,0.896181848176118)(-1.50516183035714,0.894936517099205)(-1.50502232142857,0.893692373687834)(-1.5048828125,0.892449420091923)(-1.50474330357143,0.891207658459078)(-1.50460379464286,0.889967090934583)(-1.50446428571429,0.8887277196614)(-1.50432477678571,0.887489546780164)(-1.50418526785714,0.886252574429172)(-1.50404575892857,0.885016804744394)(-1.50390625,0.883782239859451)(-1.50376674107143,0.882548881905621)(-1.50362723214286,0.881316733011834)(-1.50348772321429,0.880085795304665)(-1.50334821428571,0.878856070908336)(-1.50320870535714,0.877627561944701)(-1.50306919642857,0.876400270533258)(-1.5029296875,0.875174198791126)(-1.50279017857143,0.873949348833055)(-1.50265066964286,0.872725722771414)(-1.50251116071429,0.871503322716192)(-1.50237165178571,0.870282150774993)(-1.50223214285714,0.869062209053026)(-1.50209263392857,0.867843499653115)(-1.501953125,0.866626024675676)(-1.50181361607143,0.865409786218727)(-1.50167410714286,0.864194786377877)(-1.50153459821429,0.862981027246326)(-1.50139508928571,0.861768510914861)(-1.50125558035714,0.860557239471844)(-1.50111607142857,0.859347215003225)(-1.5009765625,0.858138439592515)(-1.50083705357143,0.856930915320802)(-1.50069754464286,0.855724644266735)(-1.50055803571429,0.854519628506523)(-1.50041852678571,0.853315870113939)(-1.50027901785714,0.852113371160298)(-1.50013950892857,0.850912133714475)(-1.5,0.849712159842883)(-1.49986049107143,0.848513451609476)(-1.49972098214286,0.847316011075747)(-1.49958147321429,0.846119840300716)(-1.49944196428571,0.844924941340945)(-1.49930245535714,0.843731316250503)(-1.49916294642857,0.842538967080997)(-1.4990234375,0.841347895881534)(-1.49888392857143,0.840158104698749)(-1.49874441964286,0.838969595576775)(-1.49860491071429,0.837782370557253)(-1.49846540178571,0.83659643167933)(-1.49832589285714,0.835411780979639)(-1.49818638392857,0.834228420492319)(-1.498046875,0.833046352248984)(-1.49790736607143,0.831865578278745)(-1.49776785714286,0.830686100608186)(-1.49762834821429,0.82950792126137)(-1.49748883928571,0.82833104225984)(-1.49734933035714,0.827155465622593)(-1.49720982142857,0.825981193366111)(-1.4970703125,0.824808227504317)(-1.49693080357143,0.823636570048608)(-1.49679129464286,0.822466223007825)(-1.49665178571429,0.821297188388257)(-1.49651227678571,0.820129468193653)(-1.49637276785714,0.818963064425183)(-1.49623325892857,0.817797979081474)(-1.49609375,0.816634214158572)(-1.49595424107143,0.815471771649965)(-1.49581473214286,0.814310653546558)(-1.49567522321429,0.813150861836681)(-1.49553571428571,0.81199239850609)(-1.49539620535714,0.810835265537941)(-1.49525669642857,0.809679464912818)(-1.4951171875,0.808524998608695)(-1.49497767857143,0.807371868600962)(-1.49483816964286,0.806220076862401)(-1.49469866071429,0.805069625363189)(-1.49455915178571,0.803920516070904)(-1.49441964285714,0.802772750950496)(-1.49428013392857,0.801626331964317)(-1.494140625,0.800481261072082)(-1.49400111607143,0.799337540230894)(-1.49386160714286,0.798195171395223)(-1.49372209821429,0.797054156516906)(-1.49358258928571,0.795914497545156)(-1.49344308035714,0.794776196426529)(-1.49330357142857,0.793639255104957)(-1.4931640625,0.792503675521709)(-1.49302455357143,0.791369459615416)(-1.49288504464286,0.790236609322049)(-1.49274553571429,0.789105126574919)(-1.49260602678571,0.787975013304685)(-1.49246651785714,0.786846271439328)(-1.49232700892857,0.785718902904172)(-1.4921875,0.784592909621857)(-1.49204799107143,0.783468293512355)(-1.49190848214286,0.782345056492953)(-1.49176897321429,0.781223200478253)(-1.49162946428571,0.780102727380178)(-1.49148995535714,0.778983639107944)(-1.49135044642857,0.77786593756809)(-1.4912109375,0.776749624664439)(-1.49107142857143,0.775634702298123)(-1.49093191964286,0.774521172367563)(-1.49079241071429,0.773409036768467)(-1.49065290178571,0.77229829739384)(-1.49051339285714,0.771188956133954)(-1.49037388392857,0.770081014876377)(-1.490234375,0.768974475505936)(-1.49009486607143,0.767869339904741)(-1.48995535714286,0.766765609952165)(-1.48981584821429,0.765663287524843)(-1.48967633928571,0.764562374496682)(-1.48953683035714,0.763462872738827)(-1.48939732142857,0.762364784119697)(-1.4892578125,0.761268110504942)(-1.48911830357143,0.760172853757473)(-1.48897879464286,0.759079015737433)(-1.48883928571429,0.757986598302207)(-1.48869977678571,0.756895603306421)(-1.48856026785714,0.755806032601921)(-1.48842075892857,0.754717888037794)(-1.48828125,0.753631171460339)(-1.48814174107143,0.752545884713084)(-1.48800223214286,0.751462029636772)(-1.48786272321429,0.750379608069356)(-1.48772321428571,0.749298621846008)(-1.48758370535714,0.748219072799093)(-1.48744419642857,0.747140962758195)(-1.4873046875,0.746064293550081)(-1.48716517857143,0.744989066998727)(-1.48702566964286,0.743915284925293)(-1.48688616071429,0.742842949148129)(-1.48674665178571,0.741772061482776)(-1.48660714285714,0.740702623741947)(-1.48646763392857,0.739634637735543)(-1.486328125,0.73856810527063)(-1.48618861607143,0.737503028151453)(-1.48604910714286,0.73643940817942)(-1.48590959821429,0.7353772471531)(-1.48577008928571,0.734316546868236)(-1.48563058035714,0.733257309117709)(-1.48549107142857,0.73219953569157)(-1.4853515625,0.731143228377009)(-1.48521205357143,0.730088388958369)(-1.48507254464286,0.729035019217132)(-1.48493303571429,0.727983120931921)(-1.48479352678571,0.726932695878499)(-1.48465401785714,0.725883745829753)(-1.48451450892857,0.724836272555711)(-1.484375,0.723790277823513)(-1.48423549107143,0.722745763397434)(-1.48409598214286,0.721702731038861)(-1.48395647321429,0.720661182506294)(-1.48381696428571,0.719621119555357)(-1.48367745535714,0.718582543938766)(-1.48353794642857,0.71754545740636)(-1.4833984375,0.716509861705062)(-1.48325892857143,0.715475758578908)(-1.48311941964286,0.714443149769022)(-1.48297991071429,0.713412037013617)(-1.48284040178571,0.712382422048006)(-1.48270089285714,0.71135430660457)(-1.48256138392857,0.710327692412789)(-1.482421875,0.709302581199207)(-1.48228236607143,0.708278974687451)(-1.48214285714286,0.707256874598216)(-1.48200334821429,0.706236282649265)(-1.48186383928571,0.705217200555431)(-1.48172433035714,0.704199630028598)(-1.48158482142857,0.703183572777722)(-1.4814453125,0.702169030508798)(-1.48130580357143,0.701156004924886)(-1.48116629464286,0.700144497726087)(-1.48102678571429,0.699134510609545)(-1.48088727678571,0.698126045269456)(-1.48074776785714,0.69711910339704)(-1.48060825892857,0.696113686680566)(-1.48046875,0.69510979680532)(-1.48032924107143,0.69410743545363)(-1.48018973214286,0.693106604304841)(-1.48005022321429,0.692107305035318)(-1.47991071428571,0.691109539318457)(-1.47977120535714,0.690113308824652)(-1.47963169642857,0.689118615221325)(-1.4794921875,0.688125460172893)(-1.47935267857143,0.68713384534079)(-1.47921316964286,0.686143772383446)(-1.47907366071429,0.685155242956288)(-1.47893415178571,0.684168258711749)(-1.47879464285714,0.683182821299241)(-1.47865513392857,0.682198932365179)(-1.478515625,0.681216593552953)(-1.47837611607143,0.680235806502943)(-1.47823660714286,0.679256572852505)(-1.47809709821429,0.678278894235973)(-1.47795758928571,0.677302772284658)(-1.47781808035714,0.676328208626833)(-1.47767857142857,0.675355204887749)(-1.4775390625,0.67438376268961)(-1.47739955357143,0.673413883651589)(-1.47726004464286,0.672445569389812)(-1.47712053571429,0.67147882151736)(-1.47698102678571,0.670513641644271)(-1.47684151785714,0.669550031377521)(-1.47670200892857,0.668587992321043)(-1.4765625,0.6676275260757)(-1.47642299107143,0.666668634239303)(-1.47628348214286,0.665711318406595)(-1.47614397321429,0.664755580169249)(-1.47600446428571,0.663801421115878)(-1.47586495535714,0.662848842832006)(-1.47572544642857,0.661897846900095)(-1.4755859375,0.660948434899517)(-1.47544642857143,0.660000608406567)(-1.47530691964286,0.659054368994451)(-1.47516741071429,0.658109718233287)(-1.47502790178571,0.657166657690105)(-1.47488839285714,0.656225188928831)(-1.47474888392857,0.655285313510305)(-1.474609375,0.654347032992253)(-1.47446986607143,0.653410348929307)(-1.47433035714286,0.652475262872988)(-1.47419084821429,0.651541776371703)(-1.47405133928571,0.650609890970756)(-1.47391183035714,0.649679608212321)(-1.47377232142857,0.648750929635469)(-1.4736328125,0.647823856776134)(-1.47349330357143,0.646898391167133)(-1.47335379464286,0.645974534338153)(-1.47321428571429,0.645052287815748)(-1.47307477678571,0.644131653123344)(-1.47293526785714,0.643212631781219)(-1.47279575892857,0.642295225306525)(-1.47265625,0.641379435213258)(-1.47251674107143,0.640465263012277)(-1.47237723214286,0.639552710211288)(-1.47223772321429,0.638641778314842)(-1.47209821428571,0.637732468824348)(-1.47195870535714,0.636824783238039)(-1.47181919642857,0.635918723051005)(-1.4716796875,0.635014289755159)(-1.47154017857143,0.634111484839257)(-1.47140066964286,0.63321030978888)(-1.47126116071429,0.632310766086438)(-1.47112165178571,0.631412855211171)(-1.47098214285714,0.63051657863913)(-1.47084263392857,0.6296219378432)(-1.470703125,0.628728934293068)(-1.47056361607143,0.627837569455244)(-1.47042410714286,0.626947844793045)(-1.47028459821429,0.626059761766594)(-1.47014508928571,0.625173321832826)(-1.47000558035714,0.624288526445467)(-1.46986607142857,0.623405377055055)(-1.4697265625,0.622523875108912)(-1.46958705357143,0.621644022051164)(-1.46944754464286,0.620765819322722)(-1.46930803571429,0.619889268361283)(-1.46916852678571,0.61901437060134)(-1.46902901785714,0.618141127474152)(-1.46888950892857,0.617269540407776)(-1.46875,0.616399610827029)(-1.46861049107143,0.615531340153513)(-1.46847098214286,0.614664729805598)(-1.46833147321429,0.613799781198419)(-1.46819196428571,0.612936495743887)(-1.46805245535714,0.612074874850661)(-1.46791294642857,0.611214919924177)(-1.4677734375,0.610356632366614)(-1.46763392857143,0.609500013576914)(-1.46749441964286,0.60864506495077)(-1.46735491071429,0.60779178788062)(-1.46721540178571,0.606940183755659)(-1.46707589285714,0.606090253961812)(-1.46693638392857,0.605241999881758)(-1.466796875,0.604395422894905)(-1.46665736607143,0.603550524377404)(-1.46651785714286,0.602707305702135)(-1.46637834821429,0.601865768238708)(-1.46623883928571,0.601025913353468)(-1.46609933035714,0.600187742409473)(-1.46595982142857,0.599351256766518)(-1.4658203125,0.598516457781104)(-1.46568080357143,0.59768334680646)(-1.46554129464286,0.596851925192524)(-1.46540178571429,0.596022194285946)(-1.46526227678571,0.595194155430091)(-1.46512276785714,0.594367809965022)(-1.46498325892857,0.593543159227516)(-1.46484375,0.592720204551042)(-1.46470424107143,0.591898947265775)(-1.46456473214286,0.591079388698583)(-1.46442522321429,0.590261530173028)(-1.46428571428571,0.589445373009368)(-1.46414620535714,0.58863091852454)(-1.46400669642857,0.587818168032179)(-1.4638671875,0.587007122842591)(-1.46372767857143,0.586197784262775)(-1.46358816964286,0.585390153596399)(-1.46344866071429,0.584584232143812)(-1.46330915178571,0.583780021202038)(-1.46316964285714,0.582977522064763)(-1.46303013392857,0.582176736022355)(-1.462890625,0.581377664361834)(-1.46275111607143,0.580580308366894)(-1.46261160714286,0.579784669317883)(-1.46247209821429,0.578990748491809)(-1.46233258928571,0.57819854716234)(-1.46219308035714,0.577408066599791)(-1.46205357142857,0.576619308071135)(-1.4619140625,0.575832272839984)(-1.46177455357143,0.575046962166606)(-1.46163504464286,0.574263377307906)(-1.46149553571429,0.573481519517432)(-1.46135602678571,0.572701390045373)(-1.46121651785714,0.571922990138547)(-1.46107700892857,0.571146321040418)(-1.4609375,0.570371383991067)(-1.46079799107143,0.569598180227216)(-1.46065848214286,0.568826710982207)(-1.46051897321429,0.568056977486005)(-1.46037946428571,0.567288980965206)(-1.46023995535714,0.566522722643012)(-1.46010044642857,0.565758203739256)(-1.4599609375,0.564995425470371)(-1.45982142857143,0.564234389049416)(-1.45968191964286,0.56347509568605)(-1.45954241071429,0.562717546586543)(-1.45940290178571,0.561961742953775)(-1.45926339285714,0.561207685987217)(-1.45912388392857,0.560455376882956)(-1.458984375,0.559704816833662)(-1.45884486607143,0.558956007028611)(-1.45870535714286,0.558208948653669)(-1.45856584821429,0.557463642891292)(-1.45842633928571,0.55672009092053)(-1.45828683035714,0.555978293917012)(-1.45814732142857,0.555238253052959)(-1.4580078125,0.554499969497168)(-1.45786830357143,0.55376344441502)(-1.45772879464286,0.553028678968472)(-1.45758928571429,0.552295674316054)(-1.45744977678571,0.551564431612874)(-1.45731026785714,0.550834952010605)(-1.45717075892857,0.550107236657495)(-1.45703125,0.549381286698352)(-1.45689174107143,0.548657103274552)(-1.45675223214286,0.547934687524031)(-1.45661272321429,0.547214040581284)(-1.45647321428571,0.546495163577369)(-1.45633370535714,0.54577805763989)(-1.45619419642857,0.545062723893013)(-1.4560546875,0.544349163457447)(-1.45591517857143,0.543637377450457)(-1.45577566964286,0.542927366985848)(-1.45563616071429,0.542219133173972)(-1.45549665178571,0.541512677121728)(-1.45535714285714,0.540807999932544)(-1.45521763392857,0.540105102706398)(-1.455078125,0.539403986539792)(-1.45493861607143,0.538704652525773)(-1.45479910714286,0.538007101753911)(-1.45465959821429,0.537311335310306)(-1.45452008928571,0.536617354277594)(-1.45438058035714,0.535925159734922)(-1.45424107142857,0.535234752757974)(-1.4541015625,0.534546134418944)(-1.45396205357143,0.533859305786552)(-1.45382254464286,0.53317426792603)(-1.45368303571429,0.532491021899126)(-1.45354352678571,0.531809568764106)(-1.45340401785714,0.531129909575735)(-1.45326450892857,0.5304520453853)(-1.453125,0.529775977240583)(-1.45298549107143,0.529101706185877)(-1.45284598214286,0.528429233261976)(-1.45270647321429,0.527758559506171)(-1.45256696428571,0.527089685952259)(-1.45242745535714,0.526422613630523)(-1.45228794642857,0.525757343567752)(-1.4521484375,0.525093876787215)(-1.45200892857143,0.524432214308681)(-1.45186941964286,0.523772357148403)(-1.45172991071429,0.52311430631912)(-1.45159040178571,0.52245806283006)(-1.45145089285714,0.521803627686925)(-1.45131138392857,0.521151001891907)(-1.451171875,0.520500186443669)(-1.45103236607143,0.519851182337354)(-1.45089285714286,0.51920399056458)(-1.45075334821429,0.518558612113435)(-1.45061383928571,0.517915047968483)(-1.45047433035714,0.517273299110749)(-1.45033482142857,0.516633366517734)(-1.4501953125,0.515995251163396)(-1.45005580357143,0.515358954018161)(-1.44991629464286,0.514724476048915)(-1.44977678571429,0.514091818219002)(-1.44963727678571,0.513460981488227)(-1.44949776785714,0.512831966812845)(-1.44935825892857,0.512204775145574)(-1.44921875,0.511579407435572)(-1.44907924107143,0.510955864628456)(-1.44893973214286,0.510334147666289)(-1.44880022321429,0.509714257487578)(-1.44866071428571,0.509096195027279)(-1.44852120535714,0.508479961216785)(-1.44838169642857,0.507865556983937)(-1.4482421875,0.507252983253007)(-1.44810267857143,0.506642240944712)(-1.44796316964286,0.5060333309762)(-1.44782366071429,0.505426254261052)(-1.44768415178571,0.504821011709288)(-1.44754464285714,0.504217604227347)(-1.44740513392857,0.503616032718109)(-1.447265625,0.503016298080868)(-1.44712611607143,0.502418401211353)(-1.44698660714286,0.501822343001712)(-1.44684709821429,0.501228124340513)(-1.44670758928571,0.50063574611275)(-1.44656808035714,0.500045209199825)(-1.44642857142857,0.499456514479566)(-1.4462890625,0.49886966282621)(-1.44614955357143,0.498284655110409)(-1.44601004464286,0.497701492199225)(-1.44587053571429,0.497120174956129)(-1.44573102678571,0.496540704241005)(-1.44559151785714,0.495963080910134)(-1.44545200892857,0.495387305816211)(-1.4453125,0.494813379808326)(-1.44517299107143,0.494241303731976)(-1.44503348214286,0.493671078429054)(-1.44489397321429,0.493102704737851)(-1.44475446428571,0.492536183493059)(-1.44461495535714,0.491971515525755)(-1.44447544642857,0.491408701663421)(-1.4443359375,0.49084774272992)(-1.44419642857143,0.490288639545511)(-1.44405691964286,0.489731392926839)(-1.44391741071429,0.489176003686934)(-1.44377790178571,0.488622472635217)(-1.44363839285714,0.488070800577485)(-1.44349888392857,0.487520988315922)(-1.443359375,0.48697303664909)(-1.44321986607143,0.486426946371931)(-1.44308035714286,0.485882718275762)(-1.44294084821429,0.485340353148278)(-1.44280133928571,0.48479985177355)(-1.44266183035714,0.484261214932015)(-1.44252232142857,0.483724443400489)(-1.4423828125,0.48318953795215)(-1.44224330357143,0.482656499356551)(-1.44210379464286,0.482125328379608)(-1.44196428571429,0.4815960257836) 
};
\addplot [
color=blue,
solid,
forget plot
]
coordinates{
 (-1.44196428571429,0.4815960257836)(-1.44182477678571,0.481068592327176)(-1.44168526785714,0.480543028765339)(-1.44154575892857,0.480019335849463)(-1.44140625,0.479497514327269)(-1.44126674107143,0.478977564942846)(-1.44112723214286,0.478459488436634)(-1.44098772321429,0.477943285545427)(-1.44084821428571,0.47742895700238)(-1.44070870535714,0.476916503536988)(-1.44056919642857,0.47640592587511)(-1.4404296875,0.475897224738941)(-1.44029017857143,0.475390400847035)(-1.44015066964286,0.474885454914285)(-1.44001116071429,0.47438238765193)(-1.43987165178571,0.473881199767559)(-1.43973214285714,0.473381891965093)(-1.43959263392857,0.472884464944803)(-1.439453125,0.472388919403293)(-1.43931361607143,0.471895256033509)(-1.43917410714286,0.471403475524732)(-1.43903459821429,0.470913578562579)(-1.43889508928571,0.470425565829003)(-1.43875558035714,0.469939438002284)(-1.43861607142857,0.469455195757041)(-1.4384765625,0.468972839764216)(-1.43833705357143,0.468492370691086)(-1.43819754464286,0.468013789201252)(-1.43805803571429,0.467537095954641)(-1.43791852678571,0.467062291607508)(-1.43777901785714,0.466589376812427)(-1.43763950892857,0.466118352218301)(-1.4375,0.465649218470346)(-1.43736049107143,0.465181976210105)(-1.43722098214286,0.464716626075436)(-1.43708147321429,0.464253168700514)(-1.43694196428571,0.463791604715834)(-1.43680245535714,0.4633319347482)(-1.43666294642857,0.462874159420736)(-1.4365234375,0.462418279352873)(-1.43638392857143,0.461964295160359)(-1.43624441964286,0.461512207455246)(-1.43610491071429,0.461062016845899)(-1.43596540178571,0.460613723936991)(-1.43582589285714,0.460167329329497)(-1.43568638392857,0.459722833620704)(-1.435546875,0.459280237404197)(-1.43540736607143,0.458839541269869)(-1.43526785714286,0.458400745803912)(-1.43512834821429,0.457963851588817)(-1.43498883928571,0.457528859203382)(-1.43484933035714,0.457095769222695)(-1.43470982142857,0.456664582218148)(-1.4345703125,0.456235298757423)(-1.43443080357143,0.455807919404503)(-1.43429129464286,0.455382444719662)(-1.43415178571429,0.454958875259467)(-1.43401227678571,0.45453721157678)(-1.43387276785714,0.454117454220747)(-1.43373325892857,0.453699603736813)(-1.43359375,0.453283660666704)(-1.43345424107143,0.452869625548437)(-1.43331473214286,0.452457498916315)(-1.43317522321429,0.452047281300927)(-1.43303571428571,0.451638973229148)(-1.43289620535714,0.451232575224131)(-1.43275669642857,0.450828087805318)(-1.4326171875,0.450425511488429)(-1.43247767857143,0.450024846785465)(-1.43233816964286,0.449626094204706)(-1.43219866071429,0.449229254250712)(-1.43205915178571,0.448834327424321)(-1.43191964285714,0.448441314222642)(-1.43178013392857,0.448050215139069)(-1.431640625,0.447661030663261)(-1.43150111607143,0.447273761281157)(-1.43136160714286,0.446888407474967)(-1.43122209821429,0.446504969723171)(-1.43108258928571,0.446123448500523)(-1.43094308035714,0.445743844278042)(-1.43080357142857,0.445366157523024)(-1.4306640625,0.444990388699024)(-1.43052455357143,0.444616538265871)(-1.43038504464286,0.444244606679657)(-1.43024553571429,0.443874594392739)(-1.43010602678571,0.443506501853742)(-1.42996651785714,0.443140329507549)(-1.42982700892857,0.442776077795313)(-1.4296875,0.442413747154442)(-1.42954799107143,0.442053338018609)(-1.42940848214286,0.441694850817747)(-1.42926897321429,0.441338285978046)(-1.42912946428571,0.440983643921958)(-1.42898995535714,0.44063092506819)(-1.42885044642857,0.440280129831707)(-1.4287109375,0.43993125862373)(-1.42857142857143,0.439584311851736)(-1.42843191964286,0.439239289919455)(-1.42829241071429,0.438896193226871)(-1.42815290178571,0.438555022170224)(-1.42801339285714,0.438215777142)(-1.42787388392857,0.437878458530944)(-1.427734375,0.437543066722045)(-1.42759486607143,0.437209602096547)(-1.42745535714286,0.436878065031939)(-1.42731584821429,0.43654845590196)(-1.42717633928571,0.436220775076599)(-1.42703683035714,0.435895022922088)(-1.42689732142857,0.43557119980091)(-1.4267578125,0.435249306071788)(-1.42661830357143,0.434929342089695)(-1.42647879464286,0.434611308205844)(-1.42633928571429,0.434295204767694)(-1.42619977678571,0.433981032118948)(-1.42606026785714,0.433668790599547)(-1.42592075892857,0.433358480545678)(-1.42578125,0.433050102289765)(-1.42564174107143,0.432743656160475)(-1.42550223214286,0.432439142482713)(-1.42536272321429,0.432136561577623)(-1.42522321428571,0.431835913762589)(-1.42508370535714,0.431537199351231)(-1.42494419642857,0.431240418653406)(-1.4248046875,0.430945571975208)(-1.42466517857143,0.430652659618966)(-1.42452566964286,0.430361681883247)(-1.42438616071429,0.430072639062849)(-1.42424665178571,0.429785531448808)(-1.42410714285714,0.429500359328389)(-1.42396763392857,0.429217122985094)(-1.423828125,0.428935822698655)(-1.42368861607143,0.428656458745037)(-1.42354910714286,0.428379031396437)(-1.42340959821429,0.428103540921281)(-1.42327008928571,0.427829987584228)(-1.42313058035714,0.427558371646162)(-1.42299107142857,0.427288693364203)(-1.4228515625,0.427020952991693)(-1.42271205357143,0.426755150778208)(-1.42257254464286,0.426491286969548)(-1.42243303571429,0.426229361807741)(-1.42229352678571,0.425969375531045)(-1.42215401785714,0.425711328373939)(-1.42201450892857,0.425455220567133)(-1.421875,0.425201052337559)(-1.42173549107143,0.424948823908377)(-1.42159598214286,0.424698535498968)(-1.42145647321429,0.42445018732494)(-1.42131696428571,0.424203779598125)(-1.42117745535714,0.423959312526575)(-1.42103794642857,0.423716786314571)(-1.4208984375,0.423476201162608)(-1.42075892857143,0.423237557267412)(-1.42061941964286,0.423000854821926)(-1.42047991071429,0.422766094015313)(-1.42034040178571,0.422533275032962)(-1.42020089285714,0.422302398056478)(-1.42006138392857,0.422073463263689)(-1.419921875,0.421846470828642)(-1.41978236607143,0.421621420921604)(-1.41964285714286,0.421398313709062)(-1.41950334821429,0.421177149353719)(-1.41936383928571,0.420957928014502)(-1.41922433035714,0.42074064984655)(-1.41908482142857,0.420525315001225)(-1.4189453125,0.420311923626104)(-1.41880580357143,0.420100475864983)(-1.41866629464286,0.419890971857876)(-1.41852678571429,0.41968341174101)(-1.41838727678571,0.419477795646833)(-1.41824776785714,0.419274123704007)(-1.41810825892857,0.419072396037411)(-1.41796875,0.418872612768139)(-1.41782924107143,0.418674774013503)(-1.41768973214286,0.418478879887028)(-1.41755022321429,0.418284930498454)(-1.41741071428571,0.418092925953738)(-1.41727120535714,0.417902866355051)(-1.41713169642857,0.417714751800779)(-1.4169921875,0.41752858238552)(-1.41685267857143,0.417344358200089)(-1.41671316964286,0.417162079331515)(-1.41657366071429,0.416981745863039)(-1.41643415178571,0.416803357874117)(-1.41629464285714,0.416626915440418)(-1.41615513392857,0.416452418633826)(-1.416015625,0.416279867522435)(-1.41587611607143,0.416109262170555)(-1.41573660714286,0.415940602638709)(-1.41559709821429,0.415773888983631)(-1.41545758928571,0.41560912125827)(-1.41531808035714,0.415446299511785)(-1.41517857142857,0.41528542378955)(-1.4150390625,0.41512649413315)(-1.41489955357143,0.414969510580385)(-1.41476004464286,0.414814473165263)(-1.41462053571429,0.414661381918007)(-1.41448102678571,0.414510236865054)(-1.41434151785714,0.414361038029048)(-1.41420200892857,0.414213785428851)(-1.4140625,0.414068479079533)(-1.41392299107143,0.413925118992377)(-1.41378348214286,0.41378370517488)(-1.41364397321429,0.413644237630747)(-1.41350446428571,0.4135067163599)(-1.41336495535714,0.413371141358468)(-1.41322544642857,0.413237512618797)(-1.4130859375,0.413105830129441)(-1.41294642857143,0.412976093875168)(-1.41280691964286,0.412848303836957)(-1.41266741071429,0.412722459992001)(-1.41252790178571,0.412598562313704)(-1.41238839285714,0.412476610771681)(-1.41224888392857,0.412356605331762)(-1.412109375,0.412238545955986)(-1.41196986607143,0.412122432602608)(-1.41183035714286,0.412008265226093)(-1.41169084821429,0.41189604377712)(-1.41155133928571,0.411785768202579)(-1.41141183035714,0.411677438445575)(-1.41127232142857,0.411571054445425)(-1.4111328125,0.411466616137657)(-1.41099330357143,0.411364123454016)(-1.41085379464286,0.411263576322457)(-1.41071428571429,0.411164974667151)(-1.41057477678571,0.411068318408479)(-1.41043526785714,0.41097360746304)(-1.41029575892857,0.410880841743645)(-1.41015625,0.410790021159317)(-1.41001674107143,0.410701145615296)(-1.40987723214286,0.410614215013037)(-1.40973772321429,0.410529229250206)(-1.40959821428571,0.410446188220687)(-1.40945870535714,0.410365091814579)(-1.40931919642857,0.410285939918193)(-1.4091796875,0.41020873241406)(-1.40904017857143,0.410133469180923)(-1.40890066964286,0.410060150093742)(-1.40876116071429,0.409988775023694)(-1.40862165178571,0.409919343838173)(-1.40848214285714,0.409851856400786)(-1.40834263392857,0.409786312571362)(-1.408203125,0.409722712205943)(-1.40806361607143,0.409661055156791)(-1.40792410714286,0.409601341272384)(-1.40778459821429,0.40954357039742)(-1.40764508928571,0.409487742372814)(-1.40750558035714,0.4094338570357)(-1.40736607142857,0.409381914219431)(-1.4072265625,0.40933191375358)(-1.40708705357143,0.409283855463938)(-1.40694754464286,0.409237739172517)(-1.40680803571429,0.40919356469755)(-1.40666852678571,0.409151331853491)(-1.40652901785714,0.409111040451012)(-1.40638950892857,0.40907269029701)(-1.40625,0.409036281194601)(-1.40611049107143,0.409001812943126)(-1.40597098214286,0.408969285338146)(-1.40583147321429,0.408938698171447)(-1.40569196428571,0.408910051231037)(-1.40555245535714,0.408883344301148)(-1.40541294642857,0.408858577162237)(-1.4052734375,0.408835749590984)(-1.40513392857143,0.408814861360295)(-1.40499441964286,0.408795912239304)(-1.40485491071429,0.408778901993366)(-1.40471540178571,0.408763830384066)(-1.40457589285714,0.408750697169216)(-1.40443638392857,0.408739502102852)(-1.404296875,0.408730244935242)(-1.40415736607143,0.408722925412879)(-1.40401785714286,0.408717543278488)(-1.40387834821429,0.40871409827102)(-1.40373883928571,0.408712590125657)(-1.40359933035714,0.408713018573814)(-1.40345982142857,0.408715383343132)(-1.4033203125,0.408719684157487)(-1.40318080357143,0.408725920736986)(-1.40304129464286,0.408734092797968)(-1.40290178571429,0.408744200053006)(-1.40276227678571,0.408756242210906)(-1.40262276785714,0.408770218976707)(-1.40248325892857,0.408786130051684)(-1.40234375,0.408803975133348)(-1.40220424107143,0.408823753915444)(-1.40206473214286,0.408845466087955)(-1.40192522321429,0.408869111337102)(-1.40178571428571,0.40889468934534)(-1.40164620535714,0.408922199791365)(-1.40150669642857,0.408951642350113)(-1.4013671875,0.408983016692757)(-1.40122767857143,0.409016322486711)(-1.40108816964286,0.409051559395631)(-1.40094866071429,0.409088727079415)(-1.40080915178571,0.409127825194199)(-1.40066964285714,0.409168853392367)(-1.40053013392857,0.409211811322544)(-1.400390625,0.409256698629599)(-1.40025111607143,0.409303514954646)(-1.40011160714286,0.409352259935046)(-1.39997209821429,0.409402933204405)(-1.39983258928571,0.409455534392577)(-1.39969308035714,0.409510063125662)(-1.39955357142857,0.40956651902601)(-1.3994140625,0.40962490171222)(-1.39927455357143,0.409685210799141)(-1.39913504464286,0.409747445897871)(-1.39899553571429,0.409811606615763)(-1.39885602678571,0.409877692556419)(-1.39871651785714,0.409945703319695)(-1.39857700892857,0.410015638501701)(-1.3984375,0.410087497694801)(-1.39829799107143,0.410161280487615)(-1.39815848214286,0.410236986465019)(-1.39801897321429,0.410314615208147)(-1.39787946428571,0.410394166294388)(-1.39773995535714,0.410475639297392)(-1.39760044642857,0.410559033787067)(-1.3974609375,0.410644349329583)(-1.39732142857143,0.410731585487369)(-1.39718191964286,0.410820741819118)(-1.39704241071429,0.410911817879784)(-1.39690290178571,0.411004813220585)(-1.39676339285714,0.411099727389005)(-1.39662388392857,0.411196559928792)(-1.396484375,0.411295310379962)(-1.39634486607143,0.411395978278795)(-1.39620535714286,0.411498563157842)(-1.39606584821429,0.411603064545923)(-1.39592633928571,0.411709481968126)(-1.39578683035714,0.411817814945811)(-1.39564732142857,0.411928062996611)(-1.3955078125,0.412040225634429)(-1.39536830357143,0.412154302369444)(-1.39522879464286,0.412270292708109)(-1.39508928571429,0.412388196153152)(-1.39494977678571,0.412508012203579)(-1.39481026785714,0.412629740354671)(-1.39467075892857,0.41275338009799)(-1.39453125,0.412878930921378)(-1.39439174107143,0.413006392308953)(-1.39425223214286,0.41313576374112)(-1.39411272321429,0.413267044694565)(-1.39397321428571,0.413400234642254)(-1.39383370535714,0.413535333053442)(-1.39369419642857,0.413672339393668)(-1.3935546875,0.413811253124757)(-1.39341517857143,0.413952073704822)(-1.39327566964286,0.414094800588266)(-1.39313616071429,0.414239433225781)(-1.39299665178571,0.414385971064348)(-1.39285714285714,0.414534413547244)(-1.39271763392857,0.414684760114034)(-1.392578125,0.414837010200582)(-1.39243861607143,0.414991163239044)(-1.39229910714286,0.415147218657874)(-1.39215959821429,0.415305175881823)(-1.39202008928571,0.41546503433194)(-1.39188058035714,0.415626793425575)(-1.39174107142857,0.415790452576375)(-1.3916015625,0.415956011194295)(-1.39146205357143,0.41612346868559)(-1.39132254464286,0.416292824452817)(-1.39118303571429,0.416464077894843)(-1.39104352678571,0.416637228406838)(-1.39090401785714,0.416812275380282)(-1.39076450892857,0.416989218202962)(-1.390625,0.417168056258977)(-1.39048549107143,0.417348788928737)(-1.39034598214286,0.417531415588963)(-1.39020647321429,0.417715935612692)(-1.39006696428571,0.417902348369275)(-1.38992745535714,0.418090653224378)(-1.38978794642857,0.418280849539986)(-1.3896484375,0.418472936674404)(-1.38950892857143,0.418666913982253)(-1.38936941964286,0.418862780814478)(-1.38922991071429,0.419060536518347)(-1.38909040178571,0.41926018043745)(-1.38895089285714,0.419461711911703)(-1.38881138392857,0.419665130277347)(-1.388671875,0.419870434866953)(-1.38853236607143,0.420077625009418)(-1.38839285714286,0.420286700029972)(-1.38825334821429,0.420497659250175)(-1.38811383928571,0.420710501987919)(-1.38797433035714,0.420925227557433)(-1.38783482142857,0.421141835269277)(-1.3876953125,0.421360324430353)(-1.38755580357143,0.421580694343897)(-1.38741629464286,0.421802944309488)(-1.38727678571429,0.422027073623043)(-1.38713727678571,0.422253081576822)(-1.38699776785714,0.422480967459431)(-1.38685825892857,0.422710730555816)(-1.38671875,0.422942370147275)(-1.38657924107143,0.42317588551145)(-1.38643973214286,0.423411275922335)(-1.38630022321429,0.423648540650273)(-1.38616071428571,0.423887678961958)(-1.38602120535714,0.42412869012044)(-1.38588169642857,0.424371573385121)(-1.3857421875,0.424616328011764)(-1.38560267857143,0.424862953252484)(-1.38546316964286,0.42511144835576)(-1.38532366071429,0.42536181256643)(-1.38518415178571,0.425614045125693)(-1.38504464285714,0.425868145271114)(-1.38490513392857,0.42612411223662)(-1.384765625,0.426381945252509)(-1.38462611607143,0.426641643545443)(-1.38448660714286,0.426903206338456)(-1.38434709821429,0.427166632850954)(-1.38420758928571,0.427431922298712)(-1.38406808035714,0.427699073893883)(-1.38392857142857,0.427968086844993)(-1.3837890625,0.428238960356949)(-1.38364955357143,0.428511693631032)(-1.38351004464286,0.428786285864906)(-1.38337053571429,0.429062736252619)(-1.38323102678571,0.429341043984598)(-1.38309151785714,0.429621208247658)(-1.38295200892857,0.429903228225)(-1.3828125,0.430187103096214)(-1.38267299107143,0.430472832037278)(-1.38253348214286,0.430760414220563)(-1.38239397321429,0.431049848814835)(-1.38225446428571,0.431341134985248)(-1.38211495535714,0.431634271893362)(-1.38197544642857,0.431929258697125)(-1.3818359375,0.432226094550892)(-1.38169642857143,0.432524778605417)(-1.38155691964286,0.432825310007854)(-1.38141741071429,0.433127687901767)(-1.38127790178571,0.433431911427121)(-1.38113839285714,0.433737979720293)(-1.38099888392857,0.434045891914065)(-1.380859375,0.434355647137636)(-1.38071986607143,0.434667244516612)(-1.38058035714286,0.434980683173019)(-1.38044084821429,0.435295962225298)(-1.38030133928571,0.435613080788303)(-1.38016183035714,0.435932037973316)(-1.38002232142857,0.436252832888033)(-1.3798828125,0.436575464636578)(-1.37974330357143,0.4368999323195)(-1.37960379464286,0.437226235033772)(-1.37946428571429,0.437554371872799)(-1.37932477678571,0.43788434192641)(-1.37918526785714,0.438216144280876)(-1.37904575892857,0.438549778018891)(-1.37890625,0.438885242219593)(-1.37876674107143,0.439222535958553)(-1.37862723214286,0.439561658307783)(-1.37848772321429,0.439902608335736)(-1.37834821428571,0.440245385107306)(-1.37820870535714,0.440589987683835)(-1.37806919642857,0.440936415123106)(-1.3779296875,0.441284666479357)(-1.37779017857143,0.441634740803271)(-1.37765066964286,0.441986637141986)(-1.37751116071429,0.442340354539094)(-1.37737165178571,0.442695892034638)(-1.37723214285714,0.443053248665126)(-1.37709263392857,0.443412423463518)(-1.376953125,0.443773415459242)(-1.37681361607143,0.444136223678184)(-1.37667410714286,0.444500847142698)(-1.37653459821429,0.444867284871606)(-1.37639508928571,0.445235535880194)(-1.37625558035714,0.445605599180225)(-1.37611607142857,0.44597747377993)(-1.3759765625,0.446351158684018)(-1.37583705357143,0.446726652893673)(-1.37569754464286,0.447103955406557)(-1.37555803571429,0.447483065216815)(-1.37541852678571,0.447863981315071)(-1.37527901785714,0.448246702688438)(-1.37513950892857,0.448631228320509)(-1.375,0.449017557191373)(-1.37486049107143,0.449405688277603)(-1.37472098214286,0.449795620552269)(-1.37458147321429,0.450187352984934)(-1.37444196428571,0.450580884541653)(-1.37430245535714,0.450976214184988)(-1.37416294642857,0.451373340873993)(-1.3740234375,0.451772263564231)(-1.37388392857143,0.452172981207764)(-1.37374441964286,0.452575492753165)(-1.37360491071429,0.452979797145515)(-1.37346540178571,0.4533858933264)(-1.37332589285714,0.453793780233928)(-1.37318638392857,0.454203456802713)(-1.373046875,0.454614921963892)(-1.37290736607143,0.455028174645117)(-1.37276785714286,0.455443213770563)(-1.37262834821429,0.45586003826093)(-1.37248883928571,0.456278647033437)(-1.37234933035714,0.456699039001837)(-1.37220982142857,0.457121213076408)(-1.3720703125,0.457545168163962)(-1.37193080357143,0.457970903167842)(-1.37179129464286,0.458398416987931)(-1.37165178571429,0.458827708520647)(-1.37151227678571,0.459258776658947)(-1.37137276785714,0.459691620292334)(-1.37123325892857,0.46012623830685)(-1.37109375,0.46056262958509)(-1.37095424107143,0.461000793006193)(-1.37081473214286,0.46144072744585)(-1.37067522321429,0.461882431776308)(-1.37053571428571,0.462325904866363)(-1.37039620535714,0.462771145581376)(-1.37025669642857,0.463218152783262)(-1.3701171875,0.463666925330501)(-1.36997767857143,0.464117462078135)(-1.36983816964286,0.464569761877774)(-1.36969866071429,0.465023823577598)(-1.36955915178571,0.465479646022353)(-1.36941964285714,0.465937228053365)(-1.36928013392857,0.466396568508527)(-1.369140625,0.466857666222317)(-1.36900111607143,0.467320520025789)(-1.36886160714286,0.46778512874658)(-1.36872209821429,0.468251491208914)(-1.36858258928571,0.468719606233596)(-1.36844308035714,0.469189472638027)(-1.36830357142857,0.469661089236192)(-1.3681640625,0.470134454838675)(-1.36802455357143,0.470609568252655)(-1.36788504464286,0.471086428281908)(-1.36774553571429,0.471565033726813)(-1.36760602678571,0.472045383384346)(-1.36746651785714,0.472527476048098)(-1.36732700892857,0.473011310508257)(-1.3671875,0.473496885551629)(-1.36704799107143,0.473984199961628)(-1.36690848214286,0.474473252518285)(-1.36676897321429,0.474964041998249)(-1.36662946428571,0.475456567174782)(-1.36648995535714,0.475950826817778)(-1.36635044642857,0.476446819693745)(-1.3662109375,0.476944544565826)(-1.36607142857143,0.477444000193788)(-1.36593191964286,0.477945185334031)(-1.36579241071429,0.478448098739591)(-1.36565290178571,0.478952739160134)(-1.36551339285714,0.479459105341974)(-1.36537388392857,0.479967196028056)(-1.365234375,0.480477009957978)(-1.36509486607143,0.480988545867978)(-1.36495535714286,0.481501802490945)(-1.36481584821429,0.48201677855642)(-1.36467633928571,0.482533472790592)(-1.36453683035714,0.483051883916315)(-1.36439732142857,0.483572010653092)(-1.3642578125,0.484093851717094)(-1.36411830357143,0.484617405821151)(-1.36397879464286,0.485142671674762)(-1.36383928571429,0.485669647984095)(-1.36369977678571,0.486198333451984)(-1.36356026785714,0.486728726777942)(-1.36342075892857,0.487260826658153)(-1.36328125,0.487794631785486)(-1.36314174107143,0.488330140849485)(-1.36300223214286,0.488867352536381)(-1.36286272321429,0.489406265529092)(-1.36272321428571,0.48994687850722)(-1.36258370535714,0.490489190147064)(-1.36244419642857,0.491033199121612)(-1.3623046875,0.491578904100554)(-1.36216517857143,0.492126303750274)(-1.36202566964286,0.49267539673386)(-1.36188616071429,0.493226181711107)(-1.36174665178571,0.49377865733851)(-1.36160714285714,0.49433282226928)(-1.36146763392857,0.494888675153336)(-1.361328125,0.495446214637316)(-1.36118861607143,0.496005439364571)(-1.36104910714286,0.496566347975174)(-1.36090959821429,0.497128939105923)(-1.36077008928571,0.497693211390335)(-1.36063058035714,0.498259163458664)(-1.36049107142857,0.498826793937884)(-1.3603515625,0.499396101451713)(-1.36021205357143,0.499967084620598)(-1.36007254464286,0.500539742061727)(-1.35993303571429,0.50111407238903)(-1.35979352678571,0.501690074213179)(-1.35965401785714,0.502267746141597)(-1.35951450892857,0.502847086778449)(-1.359375,0.503428094724661)(-1.35923549107143,0.504010768577907)(-1.35909598214286,0.504595106932624)(-1.35895647321429,0.505181108380007)(-1.35881696428571,0.505768771508012)(-1.35867745535714,0.506358094901366)(-1.35853794642857,0.506949077141557)(-1.3583984375,0.507541716806854)(-1.35825892857143,0.508136012472293)(-1.35811941964286,0.50873196270969)(-1.35797991071429,0.509329566087642)(-1.35784040178571,0.509928821171522)(-1.35770089285714,0.510529726523498)(-1.35756138392857,0.511132280702516)(-1.357421875,0.511736482264322)(-1.35728236607143,0.51234232976145)(-1.35714285714286,0.512949821743233)(-1.35700334821429,0.513558956755804)(-1.35686383928571,0.514169733342093)(-1.35672433035714,0.514782150041844)(-1.35658482142857,0.515396205391598)(-1.3564453125,0.516011897924718)(-1.35630580357143,0.516629226171372)(-1.35616629464286,0.517248188658548)(-1.35602678571429,0.517868783910056)(-1.35588727678571,0.51849101044652)(-1.35574776785714,0.5191148667854)(-1.35560825892857,0.519740351440972)(-1.35546875,0.520367462924354)(-1.35532924107143,0.52099619974349)(-1.35518973214286,0.521626560403164)(-1.35505022321429,0.522258543405001)(-1.35491071428571,0.522892147247463)(-1.35477120535714,0.523527370425865)(-1.35463169642857,0.524164211432361)(-1.3544921875,0.524802668755966)(-1.35435267857143,0.525442740882543)(-1.35421316964286,0.526084426294813)(-1.35407366071429,0.526727723472361)(-1.35393415178571,0.527372630891627)(-1.35379464285714,0.528019147025927)(-1.35365513392857,0.528667270345436)(-1.353515625,0.529316999317211)(-1.35337611607143,0.529968332405176)(-1.35323660714286,0.530621268070136)(-1.35309709821429,0.531275804769779)(-1.35295758928571,0.531931940958672)(-1.35281808035714,0.532589675088274)(-1.35267857142857,0.533249005606929)(-1.3525390625,0.53390993095988)(-1.35239955357143,0.534572449589261)(-1.35226004464286,0.535236559934106)(-1.35212053571429,0.535902260430355)(-1.35198102678571,0.536569549510845)(-1.35184151785714,0.537238425605332)(-1.35170200892857,0.537908887140472)(-1.3515625,0.538580932539844)(-1.35142299107143,0.53925456022394)(-1.35128348214286,0.539929768610173)(-1.35114397321429,0.540606556112883)(-1.35100446428571,0.541284921143328)(-1.35086495535714,0.541964862109709)(-1.35072544642857,0.542646377417144)(-1.3505859375,0.543329465467701)(-1.35044642857143,0.544014124660377)(-1.35030691964286,0.544700353391118)(-1.35016741071429,0.545388150052814)(-1.35002790178571,0.546077513035296)(-1.34988839285714,0.546768440725359)(-1.34974888392857,0.547460931506739)(-1.349609375,0.548154983760143)(-1.34946986607143,0.548850595863228)(-1.34933035714286,0.549547766190623)(-1.34919084821429,0.550246493113922)(-1.34905133928571,0.550946775001683)(-1.34891183035714,0.55164861021945)(-1.34877232142857,0.552351997129731)(-1.3486328125,0.553056934092024)(-1.34849330357143,0.553763419462804)(-1.34835379464286,0.554471451595535)(-1.34821428571429,0.555181028840672)(-1.34807477678571,0.555892149545656)(-1.34793526785714,0.556604812054934)(-1.34779575892857,0.557319014709943)(-1.34765625,0.55803475584913)(-1.34751674107143,0.558752033807941)(-1.34737723214286,0.559470846918836)(-1.34723772321429,0.560191193511287)(-1.34709821428571,0.560913071911775)(-1.34695870535714,0.56163648044381)(-1.34681919642857,0.562361417427913)(-1.3466796875,0.563087881181639)(-1.34654017857143,0.563815870019566)(-1.34640066964286,0.564545382253308)(-1.34626116071429,0.565276416191512)(-1.34612165178571,0.56600897013986)(-1.34598214285714,0.566743042401083)(-1.34584263392857,0.567478631274948)(-1.345703125,0.568215735058279)(-1.34556361607143,0.568954352044946)(-1.34542410714286,0.569694480525877)(-1.34528459821429,0.570436118789057)(-1.34514508928571,0.571179265119528)(-1.34500558035714,0.571923917799407)(-1.34486607142857,0.572670075107868)(-1.3447265625,0.573417735321166)(-1.34458705357143,0.574166896712625)(-1.34444754464286,0.57491755755265)(-1.34430803571429,0.575669716108729)(-1.34416852678571,0.576423370645428)(-1.34402901785714,0.577178519424411)(-1.34388950892857,0.577935160704425)(-1.34375,0.57869329274132)(-1.34361049107143,0.579452913788039)(-1.34347098214286,0.580214022094629)(-1.34333147321429,0.580976615908245)(-1.34319196428571,0.581740693473142)(-1.34305245535714,0.5825062530307)(-1.34291294642857,0.583273292819402)(-1.3427734375,0.584041811074861)(-1.34263392857143,0.584811806029804)(-1.34249441964286,0.585583275914089)(-1.34235491071429,0.586356218954704)(-1.34221540178571,0.587130633375761)(-1.34207589285714,0.587906517398523)(-1.34193638392857,0.588683869241378)(-1.341796875,0.589462687119868)(-1.34165736607143,0.590242969246676)(-1.34151785714286,0.591024713831637)(-1.34137834821429,0.591807919081741)(-1.34123883928571,0.592592583201129)(-1.34109933035714,0.593378704391114)(-1.34095982142857,0.594166280850159)(-1.3408203125,0.594955310773908)(-1.34068080357143,0.595745792355166)(-1.34054129464286,0.596537723783919)(-1.34040178571429,0.597331103247331)(-1.34026227678571,0.598125928929741)(-1.34012276785714,0.598922199012682)(-1.33998325892857,0.599719911674868)(-1.33984375,0.600519065092214)(-1.33970424107143,0.601319657437822)(-1.33956473214286,0.602121686881999)(-1.33942522321429,0.602925151592255)(-1.33928571428571,0.603730049733302)(-1.33914620535714,0.604536379467069)(-1.33900669642857,0.60534413895269)(-1.3388671875,0.606153326346527)(-1.33872767857143,0.606963939802153)(-1.33858816964286,0.60777597747037)(-1.33844866071429,0.608589437499213)(-1.33830915178571,0.609404318033935)(-1.33816964285714,0.610220617217041)(-1.33803013392857,0.61103833318826)(-1.337890625,0.611857464084574)(-1.33775111607143,0.612678008040207)(-1.33761160714286,0.613499963186632)(-1.33747209821429,0.61432332765258)(-1.33733258928571,0.61514809956403)(-1.33719308035714,0.615974277044234)(-1.33705357142857,0.616801858213695)(-1.3369140625,0.617630841190197)(-1.33677455357143,0.618461224088785)(-1.33663504464286,0.619293005021787)(-1.33649553571429,0.620126182098808)(-1.33635602678571,0.620960753426729)(-1.33621651785714,0.62179671710973)(-1.33607700892857,0.622634071249268)(-1.3359375,0.623472813944105)(-1.33579799107143,0.624312943290293)(-1.33565848214286,0.625154457381188)(-1.33551897321429,0.625997354307453)(-1.33537946428571,0.626841632157051)(-1.33523995535714,0.627687289015272)(-1.33510044642857,0.628534322964705)(-1.3349609375,0.629382732085274)(-1.33482142857143,0.630232514454217)(-1.33468191964286,0.631083668146103)(-1.33454241071429,0.631936191232833)(-1.33440290178571,0.632790081783636)(-1.33426339285714,0.633645337865091)(-1.33412388392857,0.634501957541107)(-1.333984375,0.63535993887295)(-1.33384486607143,0.636219279919228)(-1.33370535714286,0.637079978735905)(-1.33356584821429,0.637942033376308)(-1.33342633928571,0.638805441891111)(-1.33328683035714,0.639670202328372)(-1.33314732142857,0.640536312733498)(-1.3330078125,0.641403771149287)(-1.33286830357143,0.642272575615899)(-1.33272879464286,0.643142724170883)(-1.33258928571429,0.644014214849171)(-1.33244977678571,0.644887045683075)(-1.33231026785714,0.645761214702311)(-1.33217075892857,0.646636719933978)(-1.33203125,0.647513559402586)(-1.33189174107143,0.648391731130039)(-1.33175223214286,0.649271233135653)(-1.33161272321429,0.650152063436157)(-1.33147321428571,0.651034220045685)(-1.33133370535714,0.651917700975803)(-1.33119419642857,0.652802504235486)(-1.3310546875,0.653688627831147)(-1.33091517857143,0.654576069766622)(-1.33077566964286,0.655464828043183)(-1.33063616071429,0.656354900659545)(-1.33049665178571,0.657246285611853)(-1.33035714285714,0.658138980893712)(-1.33021763392857,0.659032984496165)(-1.330078125,0.659928294407717)(-1.32993861607143,0.660824908614325)(-1.32979910714286,0.661722825099411)(-1.32965959821429,0.662622041843863)(-1.32952008928571,0.66352255682603)(-1.32938058035714,0.664424368021749)(-1.32924107142857,0.665327473404319)(-1.3291015625,0.666231870944531)(-1.32896205357143,0.667137558610656)(-1.32882254464286,0.668044534368455)(-1.32868303571429,0.668952796181185)(-1.32854352678571,0.669862342009591)(-1.32840401785714,0.670773169811934)(-1.32826450892857,0.671685277543961)(-1.328125,0.672598663158947)(-1.32798549107143,0.673513324607665)(-1.32784598214286,0.674429259838414)(-1.32770647321429,0.675346466797011)(-1.32756696428571,0.676264943426794)(-1.32742745535714,0.677184687668639)(-1.32728794642857,0.678105697460942)(-1.3271484375,0.679027970739649)(-1.32700892857143,0.679951505438239)(-1.32686941964286,0.680876299487739)(-1.32672991071429,0.681802350816725)(-1.32659040178571,0.68272965735132)(-1.32645089285714,0.683658217015216)(-1.32631138392857,0.684588027729651)(-1.326171875,0.685519087413444)(-1.32603236607143,0.686451393982971)(-1.32589285714286,0.687384945352186)(-1.32575334821429,0.688319739432623)(-1.32561383928571,0.689255774133388)(-1.32547433035714,0.690193047361186)(-1.32533482142857,0.691131557020297)(-1.3251953125,0.692071301012607)(-1.32505580357143,0.693012277237591)(-1.32491629464286,0.693954483592331)(-1.32477678571429,0.694897917971515)(-1.32463727678571,0.695842578267432)(-1.32449776785714,0.696788462369999)(-1.32435825892857,0.697735568166738)(-1.32421875,0.698683893542805)(-1.32407924107143,0.699633436380972)(-1.32393973214286,0.700584194561647)(-1.32380022321429,0.701536165962876)(-1.32366071428571,0.702489348460333)(-1.32352120535714,0.703443739927347)(-1.32338169642857,0.704399338234882)(-1.3232421875,0.705356141251565)(-1.32310267857143,0.70631414684367)(-1.32296316964286,0.707273352875135)(-1.32282366071429,0.70823375720756)(-1.32268415178571,0.709195357700209)(-1.32254464285714,0.710158152210028)(-1.32240513392857,0.711122138591624)(-1.322265625,0.712087314697301)(-1.32212611607143,0.713053678377035)(-1.32198660714286,0.714021227478497)(-1.32184709821429,0.71498995984705)(-1.32170758928571,0.715959873325749)(-1.32156808035714,0.716930965755359)(-1.32142857142857,0.71790323497434)(-1.3212890625,0.718876678818874)(-1.32114955357143,0.719851295122844)(-1.32101004464286,0.720827081717861)(-1.32087053571429,0.721804036433256)(-1.32073102678571,0.722782157096079)(-1.32059151785714,0.723761441531125)(-1.32045200892857,0.724741887560906)(-1.3203125,0.725723493005692)(-1.32017299107143,0.726706255683482)(-1.32003348214286,0.72769017341003)(-1.31989397321429,0.728675243998842)(-1.31975446428571,0.729661465261173)(-1.31961495535714,0.730648835006051)(-1.31947544642857,0.731637351040255)(-1.3193359375,0.732627011168344)(-1.31919642857143,0.733617813192643)(-1.31905691964286,0.734609754913258)(-1.31891741071429,0.735602834128081)(-1.31877790178571,0.736597048632776)(-1.31863839285714,0.737592396220817)(-1.31849888392857,0.738588874683453)(-1.318359375,0.739586481809748)(-1.31821986607143,0.74058521538656)(-1.31808035714286,0.741585073198559)(-1.31794084821429,0.742586053028227)(-1.31780133928571,0.743588152655854)(-1.31766183035714,0.744591369859565)(-1.31752232142857,0.745595702415295)(-1.3173828125,0.746601148096822)(-1.31724330357143,0.747607704675746)(-1.31710379464286,0.748615369921511)(-1.31696428571429,0.749624141601405)(-1.31682477678571,0.750634017480553)(-1.31668526785714,0.751644995321945)(-1.31654575892857,0.752657072886412)(-1.31640625,0.753670247932656)(-1.31626674107143,0.754684518217236)(-1.31612723214286,0.755699881494583)(-1.31598772321429,0.756716335517001)(-1.31584821428571,0.757733878034664)(-1.31570870535714,0.758752506795641)(-1.31556919642857,0.75977221954587)(-1.3154296875,0.760793014029196)(-1.31529017857143,0.761814887987346)(-1.31515066964286,0.762837839159954)(-1.31501116071429,0.763861865284554)(-1.31487165178571,0.764886964096585)(-1.31473214285714,0.765913133329407)(-1.31459263392857,0.766940370714283)(-1.314453125,0.767968673980414)(-1.31431361607143,0.768998040854912)(-1.31417410714286,0.770028469062827)(-1.31403459821429,0.771059956327142)(-1.31389508928571,0.772092500368773)(-1.31375558035714,0.773126098906592)(-1.31361607142857,0.774160749657401)(-1.3134765625,0.775196450335971)(-1.31333705357143,0.77623319865502)(-1.31319754464286,0.777270992325228)(-1.31305803571429,0.778309829055247)(-1.31291852678571,0.779349706551686)(-1.31277901785714,0.780390622519143)(-1.31263950892857,0.781432574660182)(-1.3125,0.782475560675361)(-1.31236049107143,0.783519578263218)(-1.31222098214286,0.784564625120289)(-1.31208147321429,0.785610698941104)(-1.31194196428571,0.78665779741819)(-1.31180245535714,0.787705918242091)(-1.31166294642857,0.788755059101347)(-1.3115234375,0.789805217682525)(-1.31138392857143,0.790856391670205)(-1.31124441964286,0.791908578746991)(-1.31110491071429,0.79296177659352)(-1.31096540178571,0.794015982888452)(-1.31082589285714,0.795071195308497)(-1.31068638392857,0.796127411528393)(-1.310546875,0.797184629220938)(-1.31040736607143,0.79824284605697)(-1.31026785714286,0.799302059705389)(-1.31012834821429,0.800362267833154)(-1.30998883928571,0.801423468105281)(-1.30984933035714,0.80248565818487)(-1.30970982142857,0.803548835733076)(-1.3095703125,0.80461299840915)(-1.30943080357143,0.805678143870413)(-1.30929129464286,0.806744269772279)(-1.30915178571429,0.807811373768255)(-1.30901227678571,0.808879453509935)(-1.30887276785714,0.80994850664703)(-1.30873325892857,0.811018530827337)(-1.30859375,0.812089523696783)(-1.30845424107143,0.813161482899394)(-1.30831473214286,0.814234406077323)(-1.30817522321429,0.815308290870849)(-1.30803571428571,0.816383134918368)(-1.30789620535714,0.817458935856424)(-1.30775669642857,0.818535691319684)(-1.3076171875,0.819613398940972)(-1.30747767857143,0.820692056351245)(-1.30733816964286,0.821771661179621)(-1.30719866071429,0.822852211053373)(-1.30705915178571,0.823933703597926)(-1.30691964285714,0.825016136436885)(-1.30678013392857,0.826099507192008)(-1.306640625,0.827183813483243)(-1.30650111607143,0.828269052928707)(-1.30636160714286,0.829355223144706)(-1.30622209821429,0.830442321745735)(-1.30608258928571,0.831530346344473)(-1.30594308035714,0.832619294551812)(-1.30580357142857,0.83370916397683)(-1.3056640625,0.834799952226826)(-1.30552455357143,0.835891656907303)(-1.30538504464286,0.836984275621981)(-1.30524553571429,0.838077805972807)(-1.30510602678571,0.83917224555994)(-1.30496651785714,0.840267591981789)(-1.30482700892857,0.841363842834977)(-1.3046875,0.842460995714386)(-1.30454799107143,0.843559048213127)(-1.30440848214286,0.84465799792257)(-1.30426897321429,0.845757842432338)(-1.30412946428571,0.846858579330302)(-1.30398995535714,0.847960206202613)(-1.30385044642857,0.849062720633672)(-1.3037109375,0.850166120206169)(-1.30357142857143,0.851270402501059)(-1.30343191964286,0.852375565097586)(-1.30329241071429,0.853481605573281)(-1.30315290178571,0.854588521503956)(-1.30301339285714,0.855696310463736)(-1.30287388392857,0.85680497002503)(-1.302734375,0.857914497758565)(-1.30259486607143,0.859024891233373)(-1.30245535714286,0.8601361480168)(-1.30231584821429,0.861248265674518)(-1.30217633928571,0.862361241770511)(-1.30203683035714,0.86347507386711)(-1.30189732142857,0.864589759524961)(-1.3017578125,0.865705296303064)(-1.30161830357143,0.866821681758753)(-1.30147879464286,0.867938913447715)(-1.30133928571429,0.869056988923993)(-1.30119977678571,0.870175905739974)(-1.30106026785714,0.871295661446426)(-1.30092075892857,0.872416253592468)(-1.30078125,0.873537679725604)(-1.30064174107143,0.874659937391705)(-1.30050223214286,0.87578302413503)(-1.30036272321429,0.876906937498225)(-1.30022321428571,0.878031675022316)(-1.30008370535714,0.879157234246743)(-1.29994419642857,0.880283612709327)(-1.2998046875,0.881410807946313)(-1.29966517857143,0.882538817492344)(-1.29952566964286,0.883667638880482)(-1.29938616071429,0.884797269642213)(-1.29924665178571,0.885927707307435)(-1.29910714285714,0.887058949404494)(-1.29896763392857,0.888190993460152)(-1.298828125,0.889323836999626)(-1.29868861607143,0.890457477546564)(-1.29854910714286,0.891591912623072)(-1.29840959821429,0.892727139749708)(-1.29827008928571,0.893863156445479)(-1.29813058035714,0.894999960227872)(-1.29799107142857,0.896137548612825)(-1.2978515625,0.897275919114763)(-1.29771205357143,0.898415069246579)(-1.29757254464286,0.899554996519654)(-1.29743303571429,0.900695698443858)(-1.29729352678571,0.901837172527541)(-1.29715401785714,0.902979416277571)(-1.29701450892857,0.904122427199295)(-1.296875,0.905266202796587)(-1.29673549107143,0.90641074057182)(-1.29659598214286,0.907556038025887)(-1.29645647321429,0.908702092658208)(-1.29631696428571,0.909848901966715)(-1.29617745535714,0.910996463447892)(-1.29603794642857,0.912144774596736)(-1.2958984375,0.913293832906805)(-1.29575892857143,0.91444363587019)(-1.29561941964286,0.915594180977539)(-1.29547991071429,0.916745465718056)(-1.29534040178571,0.917897487579496)(-1.29520089285714,0.919050244048195)(-1.29506138392857,0.920203732609042)(-1.294921875,0.921357950745517)(-1.29478236607143,0.922512895939668)(-1.29464285714286,0.923668565672136)(-1.29450334821429,0.924824957422151)(-1.29436383928571,0.925982068667527)(-1.29422433035714,0.927139896884697)(-1.29408482142857,0.928298439548676)(-1.2939453125,0.92945769413311)(-1.29380580357143,0.930617658110245)(-1.29366629464286,0.931778328950951)(-1.29352678571429,0.932939704124726)(-1.29338727678571,0.934101781099686)(-1.29324776785714,0.935264557342597)(-1.29310825892857,0.936428030318847)(-1.29296875,0.937592197492483)(-1.29282924107143,0.938757056326192)(-1.29268973214286,0.939922604281316)(-1.29255022321429,0.941088838817862)(-1.29241071428571,0.942255757394487)(-1.29227120535714,0.943423357468536)(-1.29213169642857,0.944591636496006)(-1.2919921875,0.945760591931591)(-1.29185267857143,0.946930221228659)(-1.29171316964286,0.948100521839268)(-1.29157366071429,0.949271491214173)(-1.29143415178571,0.950443126802818)(-1.29129464285714,0.951615426053366)(-1.29115513392857,0.952788386412668)(-1.291015625,0.953962005326309)(-1.29087611607143,0.955136280238578)(-1.29073660714286,0.956311208592492)(-1.29059709821429,0.9574867878298)(-1.29045758928571,0.958663015390974)(-1.29031808035714,0.959839888715238)(-1.29017857142857,0.961017405240543)(-1.2900390625,0.962195562403604)(-1.28989955357143,0.963374357639879)(-1.28976004464286,0.964553788383587)(-1.28962053571429,0.965733852067714)(-1.28948102678571,0.966914546124003)(-1.28934151785714,0.968095867982987)(-1.28920200892857,0.969277815073958)(-1.2890625,0.970460384825008)(-1.28892299107143,0.971643574663005)(-1.28878348214286,0.972827382013618)(-1.28864397321429,0.974011804301313)(-1.28850446428571,0.975196838949351)(-1.28836495535714,0.976382483379815)(-1.28822544642857,0.977568735013587)(-1.2880859375,0.978755591270379)(-1.28794642857143,0.979943049568718)(-1.28780691964286,0.981131107325963)(-1.28766741071429,0.98231976195831)(-1.28752790178571,0.983509010880781)(-1.28738839285714,0.984698851507259)(-1.28724888392857,0.985889281250455)(-1.287109375,0.987080297521954)(-1.28696986607143,0.988271897732183)(-1.28683035714286,0.989464079290443)(-1.28669084821429,0.9906568396049)(-1.28655133928571,0.991850176082588)(-1.28641183035714,0.993044086129434)(-1.28627232142857,0.994238567150228)(-1.2861328125,0.995433616548671)(-1.28599330357143,0.99662923172734)(-1.28585379464286,0.997825410087721)(-1.28571428571429,0.999022149030202)(-1.28557477678571,1.00021944595407)(-1.28543526785714,1.00141729825756)(-1.28529575892857,1.00261570333776)(-1.28515625,1.00381465859076)(-1.28501674107143,1.00501416141153)(-1.28487723214286,1.00621420919397)(-1.28473772321429,1.00741479933096)(-1.28459821428571,1.00861592921429)(-1.28445870535714,1.00981759623472)(-1.28431919642857,1.01101979778193)(-1.2841796875,1.01222253124461)(-1.28404017857143,1.01342579401037)(-1.28390066964286,1.01462958346582)(-1.28376116071429,1.01583389699653)(-1.28362165178571,1.01703873198705)(-1.28348214285714,1.01824408582093)(-1.28334263392857,1.01944995588068)(-1.283203125,1.02065633954784)(-1.28306361607143,1.02186323420293)(-1.28292410714286,1.02307063722548)(-1.28278459821429,1.02427854599403)(-1.28264508928571,1.02548695788614)(-1.28250558035714,1.02669587027838)(-1.28236607142857,1.02790528054636)(-1.2822265625,1.0291151860647)(-1.28208705357143,1.03032558420709)(-1.28194754464286,1.03153647234623)(-1.28180803571429,1.03274784785388)(-1.28166852678571,1.03395970810084)(-1.28152901785714,1.035172050457)(-1.28138950892857,1.03638487229125)(-1.28125,1.03759817097162)(-1.28111049107143,1.03881194386516)(-1.28097098214286,1.04002618833801)(-1.28083147321429,1.0412409017554)(-1.28069196428571,1.04245608148164)(-1.28055245535714,1.04367172488014)(-1.28041294642857,1.0448878293134)(-1.2802734375,1.04610439214302)(-1.28013392857143,1.04732141072972)(-1.27999441964286,1.04853888243332)(-1.27985491071429,1.04975680461278)(-1.27971540178571,1.05097517462614)(-1.27957589285714,1.05219398983062)(-1.27943638392857,1.05341324758254)(-1.279296875,1.05463294523737)(-1.27915736607143,1.05585308014971)(-1.27901785714286,1.05707364967334)(-1.27887834821429,1.05829465116116)(-1.27873883928571,1.05951608196526)(-1.27859933035714,1.06073793943686)(-1.27845982142857,1.06196022092638)(-1.2783203125,1.0631829237834)(-1.27818080357143,1.06440604535668)(-1.27804129464286,1.06562958299417)(-1.27790178571429,1.06685353404302)(-1.27776227678571,1.06807789584954)(-1.27762276785714,1.06930266575929)(-1.27748325892857,1.07052784111698)(-1.27734375,1.07175341926659)(-1.27720424107143,1.07297939755127)(-1.27706473214286,1.07420577331341)(-1.27692522321429,1.07543254389463)(-1.27678571428571,1.07665970663577)(-1.27664620535714,1.07788725887692)(-1.27650669642857,1.0791151979574)(-1.2763671875,1.08034352121578)(-1.27622767857143,1.08157222598988)(-1.27608816964286,1.08280130961679)(-1.27594866071429,1.08403076943285)(-1.27580915178571,1.08526060277368)(-1.27566964285714,1.08649080697416)(-1.27553013392857,1.08772137936844)(-1.275390625,1.08895231728999)(-1.27525111607143,1.09018361807152)(-1.27511160714286,1.09141527904508)(-1.27497209821429,1.092647297542)(-1.27483258928571,1.09387967089289)(-1.27469308035714,1.09511239642771)(-1.27455357142857,1.09634547147571)(-1.2744140625,1.09757889336546)(-1.27427455357143,1.09881265942487)(-1.27413504464286,1.10004676698117)(-1.27399553571429,1.10128121336092)(-1.27385602678571,1.10251599589003)(-1.27371651785714,1.10375111189376)(-1.27357700892857,1.10498655869669)(-1.2734375,1.10622233362281)(-1.27329799107143,1.10745843399542)(-1.27315848214286,1.10869485713722)(-1.27301897321429,1.10993160037027)(-1.27287946428571,1.11116866101599)(-1.27273995535714,1.11240603639522)(-1.27260044642857,1.11364372382814)(-1.2724609375,1.11488172063438)(-1.27232142857143,1.11612002413291)(-1.27218191964286,1.11735863164215)(-1.27204241071429,1.11859754047989)(-1.27190290178571,1.11983674796337)(-1.27176339285714,1.12107625140922)(-1.27162388392857,1.1223160481335)(-1.271484375,1.12355613545171)(-1.27134486607143,1.12479651067877)(-1.27120535714286,1.12603717112906)(-1.27106584821429,1.12727811411639)(-1.27092633928571,1.12851933695402)(-1.27078683035714,1.12976083695467)(-1.27064732142857,1.13100261143051)(-1.2705078125,1.13224465769319)(-1.27036830357143,1.13348697305383)(-1.27022879464286,1.13472955482301)(-1.27008928571429,1.13597240031082)(-1.26994977678571,1.1372155068268)(-1.26981026785714,1.13845887168002)(-1.26967075892857,1.13970249217902)(-1.26953125,1.14094636563185)(-1.26939174107143,1.14219048934607)(-1.26925223214286,1.14343486062875)(-1.26911272321429,1.1446794767865)(-1.26897321428571,1.1459243351254)(-1.26883370535714,1.14716943295112)(-1.26869419642857,1.14841476756882)(-1.2685546875,1.14966033628322)(-1.26841517857143,1.15090613639858)(-1.26827566964286,1.15215216521872)(-1.26813616071429,1.15339842004699)(-1.26799665178571,1.15464489818631)(-1.26785714285714,1.15589159693919)(-1.26771763392857,1.15713851360767)(-1.267578125,1.1583856454934)(-1.26743861607143,1.15963298989758)(-1.26729910714286,1.16088054412104)(-1.26715959821429,1.16212830546415)(-1.26702008928571,1.16337627122692)(-1.26688058035714,1.16462443870894)(-1.26674107142857,1.16587280520941)(-1.2666015625,1.16712136802714)(-1.26646205357143,1.16837012446056)(-1.26632254464286,1.16961907180774)(-1.26618303571429,1.17086820736635)(-1.26604352678571,1.17211752843372)(-1.26590401785714,1.17336703230679)(-1.26576450892857,1.17461671628218)(-1.265625,1.17586657765613)(-1.26548549107143,1.17711661372454)(-1.26534598214286,1.17836682178298)(-1.26520647321429,1.17961719912668)(-1.26506696428571,1.18086774305055)(-1.26492745535714,1.18211845084916)(-1.26478794642857,1.18336931981675)(-1.2646484375,1.18462034724729)(-1.26450892857143,1.18587153043441)(-1.26436941964286,1.18712286667143)(-1.26422991071429,1.18837435325139)(-1.26409040178571,1.18962598746704)(-1.26395089285714,1.19087776661083)(-1.26381138392857,1.19212968797493)(-1.263671875,1.19338174885123)(-1.26353236607143,1.19463394653137)(-1.26339285714286,1.19588627830669)(-1.26325334821429,1.1971387414683)(-1.26311383928571,1.19839133330704)(-1.26297433035714,1.1996440511135)(-1.26283482142857,1.20089689217801)(-1.2626953125,1.2021498537907)(-1.26255580357143,1.20340293324143)(-1.26241629464286,1.20465612781983)(-1.26227678571429,1.20590943481534)(-1.26213727678571,1.20716285151715)(-1.26199776785714,1.20841637521424)(-1.26185825892857,1.20967000319539)(-1.26171875,1.21092373274917)(-1.26157924107143,1.21217756116397)(-1.26143973214286,1.21343148572796)(-1.26130022321429,1.21468550372914)(-1.26116071428571,1.21593961245533)(-1.26102120535714,1.21719380919417)(-1.26088169642857,1.21844809123312)(-1.2607421875,1.21970245585948)(-1.26060267857143,1.2209569003604)(-1.26046316964286,1.22221142202286)(-1.26032366071429,1.22346601813369)(-1.26018415178571,1.22472068597958)(-1.26004464285714,1.2259754228471)(-1.25990513392857,1.22723022602263)(-1.259765625,1.22848509279249)(-1.25962611607143,1.22974002044283)(-1.25948660714286,1.23099500625969)(-1.25934709821429,1.23225004752901)(-1.25920758928571,1.23350514153661)(-1.25906808035714,1.23476028556821)(-1.25892857142857,1.23601547690943)(-1.2587890625,1.23727071284581)(-1.25864955357143,1.23852599066279)(-1.25851004464286,1.23978130764573)(-1.25837053571429,1.24103666107992)(-1.25823102678571,1.24229204825058)(-1.25809151785714,1.24354746644286)(-1.25795200892857,1.24480291294184)(-1.2578125,1.24605838503256)(-1.25767299107143,1.24731388000001)(-1.25753348214286,1.24856939512912)(-1.25739397321429,1.2498249277048)(-1.25725446428571,1.25108047501192)(-1.25711495535714,1.25233603433531)(-1.25697544642857,1.25359160295979)(-1.2568359375,1.25484717817015)(-1.25669642857143,1.25610275725119)(-1.25655691964286,1.25735833748767)(-1.25641741071429,1.25861391616437)(-1.25627790178571,1.25986949056608)(-1.25613839285714,1.26112505797757)(-1.25599888392857,1.26238061568364)(-1.255859375,1.26363616096911)(-1.25571986607143,1.26489169111883)(-1.25558035714286,1.26614720341767)(-1.25544084821429,1.26740269515053)(-1.25530133928571,1.26865816360236)(-1.25516183035714,1.26991360605816)(-1.25502232142857,1.27116901980295)(-1.2548828125,1.27242440212186)(-1.25474330357143,1.27367975030002)(-1.25460379464286,1.27493506162268)(-1.25446428571429,1.27619033337513)(-1.25432477678571,1.27744556284273)(-1.25418526785714,1.27870074731097)(-1.25404575892857,1.27995588406536)(-1.25390625,1.28121097039155)(-1.25376674107143,1.28246600357528)(-1.25362723214286,1.28372098090239)(-1.25348772321429,1.28497589965883)(-1.25334821428571,1.28623075713065)(-1.25320870535714,1.28748555060404)(-1.25306919642857,1.28874027736531)(-1.2529296875,1.28999493470088)(-1.25279017857143,1.29124951989734)(-1.25265066964286,1.29250403024138)(-1.25251116071429,1.29375846301988)(-1.25237165178571,1.29501281551983)(-1.25223214285714,1.29626708502841)(-1.25209263392857,1.29752126883292)(-1.251953125,1.29877536422088)(-1.25181361607143,1.30002936847994)(-1.25167410714286,1.30128327889794)(-1.25153459821429,1.30253709276291)(-1.25139508928571,1.30379080736306)(-1.25125558035714,1.30504441998681)(-1.25111607142857,1.30629792792273)(-1.2509765625,1.30755132845966)(-1.25083705357143,1.30880461888661)(-1.25069754464286,1.3100577964928)(-1.25055803571429,1.31131085856768)(-1.25041852678571,1.31256380240094)(-1.25027901785714,1.31381662528249)(-1.25013950892857,1.31506932450244)(-1.25,1.3163218973512)(-1.24986049107143,1.31757434111939)(-1.24972098214286,1.31882665309788)(-1.24958147321429,1.32007883057782)(-1.24944196428571,1.3213308708506)(-1.24930245535714,1.32258277120789)(-1.24916294642857,1.32383452894161)(-1.2490234375,1.32508614134399)(-1.24888392857143,1.32633760570752)(-1.24874441964286,1.32758891932499)(-1.24860491071429,1.32884007948948)(-1.24846540178571,1.33009108349436)(-1.24832589285714,1.33134192863332)(-1.24818638392857,1.33259261220033)(-1.248046875,1.33384313148972)(-1.24790736607143,1.33509348379611)(-1.24776785714286,1.33634366641443)(-1.24762834821429,1.33759367663998)(-1.24748883928571,1.33884351176837)(-1.24734933035714,1.34009316909555)(-1.24720982142857,1.34134264591783)(-1.2470703125,1.34259193953186)(-1.24693080357143,1.34384104723464)(-1.24679129464286,1.34508996632356)(-1.24665178571429,1.34633869409635)(-1.24651227678571,1.34758722785113)(-1.24637276785714,1.34883556488637)(-1.24623325892857,1.35008370250096)(-1.24609375,1.35133163799415)(-1.24595424107143,1.35257936866559)(-1.24581473214286,1.35382689181534)(-1.24567522321429,1.35507420474386)(-1.24553571428571,1.35632130475201)(-1.24539620535714,1.35756818914106)(-1.24525669642857,1.35881485521272)(-1.2451171875,1.36006130026911)(-1.24497767857143,1.36130752161279)(-1.24483816964286,1.36255351654673)(-1.24469866071429,1.36379928237439)(-1.24455915178571,1.36504481639962)(-1.24441964285714,1.36629011592677)(-1.24428013392857,1.3675351782606)(-1.244140625,1.36878000070637)(-1.24400111607143,1.37002458056979)(-1.24386160714286,1.37126891515704)(-1.24372209821429,1.37251300177479)(-1.24358258928571,1.37375683773018)(-1.24344308035714,1.37500042033084)(-1.24330357142857,1.37624374688489)(-1.2431640625,1.37748681470096)(-1.24302455357143,1.37872962108818)(-1.24288504464286,1.37997216335618)(-1.24274553571429,1.38121443881512)(-1.24260602678571,1.38245644477565)(-1.24246651785714,1.38369817854899)(-1.24232700892857,1.38493963744683)(-1.2421875,1.38618081878146)(-1.24204799107143,1.38742171986567)(-1.24190848214286,1.38866233801279)(-1.24176897321429,1.38990267053673)(-1.24162946428571,1.39114271475194)(-1.24148995535714,1.39238246797343)(-1.24135044642857,1.39362192751677)(-1.2412109375,1.39486109069812)(-1.24107142857143,1.39609995483419)(-1.24093191964286,1.39733851724231)(-1.24079241071429,1.39857677524037)(-1.24065290178571,1.39981472614686)(-1.24051339285714,1.40105236728085)(-1.24037388392857,1.40228969596205)(-1.240234375,1.40352670951074)(-1.24009486607143,1.40476340524784)(-1.23995535714286,1.40599978049487)(-1.23981584821429,1.407235832574)(-1.23967633928571,1.408471558808)(-1.23953683035714,1.40970695652028)(-1.23939732142857,1.4109420230349)(-1.2392578125,1.41217675567657)(-1.23911830357143,1.41341115177063)(-1.23897879464286,1.41464520864308)(-1.23883928571429,1.4158789236206)(-1.23869977678571,1.41711229403051)(-1.23856026785714,1.41834531720081)(-1.23842075892857,1.41957799046018)(-1.23828125,1.42081031113797)(-1.23814174107143,1.42204227656424)(-1.23800223214286,1.42327388406971)(-1.23786272321429,1.42450513098582)(-1.23772321428571,1.4257360146447)(-1.23758370535714,1.4269665323792)(-1.23744419642857,1.42819668152285)(-1.2373046875,1.42942645940995)(-1.23716517857143,1.43065586337547)(-1.23702566964286,1.43188489075515)(-1.23688616071429,1.43311353888543)(-1.23674665178571,1.4343418051035)(-1.23660714285714,1.43556968674731)(-1.23646763392857,1.43679718115552)(-1.236328125,1.43802428566759)(-1.23618861607143,1.43925099762371)(-1.23604910714286,1.44047731436483)(-1.23590959821429,1.44170323323269)(-1.23577008928571,1.44292875156978)(-1.23563058035714,1.44415386671941)(-1.23549107142857,1.44537857602562)(-1.2353515625,1.44660287683328)(-1.23521205357143,1.44782676648805)(-1.23507254464286,1.44905024233637)(-1.23493303571429,1.4502733017255)(-1.23479352678571,1.45149594200351)(-1.23465401785714,1.4527181605193)(-1.23451450892857,1.45393995462255)(-1.234375,1.45516132166381)(-1.23423549107143,1.45638225899443)(-1.23409598214286,1.45760276396662)(-1.23395647321429,1.45882283393342)(-1.23381696428571,1.46004246624871)(-1.23367745535714,1.46126165826724)(-1.23353794642857,1.46248040734459)(-1.2333984375,1.46369871083723)(-1.23325892857143,1.46491656610248)(-1.23311941964286,1.46613397049854)(-1.23297991071429,1.46735092138449)(-1.23284040178571,1.46856741612028)(-1.23270089285714,1.46978345206676)(-1.23256138392857,1.47099902658566)(-1.232421875,1.47221413703962)(-1.23228236607143,1.47342878079219)(-1.23214285714286,1.47464295520781)(-1.23200334821429,1.47585665765185)(-1.23186383928571,1.47706988549057)(-1.23172433035714,1.4782826360912)(-1.23158482142857,1.47949490682185)(-1.2314453125,1.4807066950516)(-1.23130580357143,1.48191799815045)(-1.23116629464286,1.48312881348935)(-1.23102678571429,1.48433913844021)(-1.23088727678571,1.48554897037587)(-1.23074776785714,1.48675830667015)(-1.23060825892857,1.48796714469782)(-1.23046875,1.48917548183464)(-1.23032924107143,1.49038331545732)(-1.23018973214286,1.49159064294357)(-1.23005022321429,1.49279746167208)(-1.22991071428571,1.49400376902252)(-1.22977120535714,1.49520956237557)(-1.22963169642857,1.49641483911289)(-1.2294921875,1.49761959661718)(-1.22935267857143,1.49882383227212)(-1.22921316964286,1.50002754346241)(-1.22907366071429,1.50123072757378)(-1.22893415178571,1.50243338199298)(-1.22879464285714,1.5036355041078)(-1.22865513392857,1.50483709130704)(-1.228515625,1.50603814098059)(-1.22837611607143,1.50723865051932)(-1.22823660714286,1.50843861731522)(-1.22809709821429,1.50963803876128)(-1.22795758928571,1.51083691225159)(-1.22781808035714,1.51203523518129)(-1.22767857142857,1.51323300494657)(-1.2275390625,1.51443021894475)(-1.22739955357143,1.51562687457419)(-1.22726004464286,1.51682296923435)(-1.22712053571429,1.51801850032577)(-1.22698102678571,1.5192134652501)(-1.22684151785714,1.5204078614101)(-1.22670200892857,1.52160168620961)(-1.2265625,1.52279493705361)(-1.22642299107143,1.52398761134818)(-1.22628348214286,1.52517970650052)(-1.22614397321429,1.52637121991897)(-1.22600446428571,1.527562149013)(-1.22586495535714,1.52875249119321)(-1.22572544642857,1.52994224387134)(-1.2255859375,1.53113140446029)(-1.22544642857143,1.53231997037411)(-1.22530691964286,1.53350793902799)(-1.22516741071429,1.5346953078383)(-1.22502790178571,1.53588207422257)(-1.22488839285714,1.53706823559952)(-1.22474888392857,1.53825378938901)(-1.224609375,1.53943873301212)(-1.22446986607143,1.5406230638911)(-1.22433035714286,1.54180677944939)(-1.22419084821429,1.54298987711165)(-1.22405133928571,1.5441723543037)(-1.22391183035714,1.54535420845261)(-1.22377232142857,1.54653543698664)(-1.2236328125,1.54771603733528)(-1.22349330357143,1.54889600692922)(-1.22335379464286,1.55007534320042)(-1.22321428571429,1.55125404358202)(-1.22307477678571,1.55243210550843)(-1.22293526785714,1.55360952641531)(-1.22279575892857,1.55478630373954)(-1.22265625,1.55596243491927)(-1.22251674107143,1.55713791739391)(-1.22237723214286,1.55831274860412)(-1.22223772321429,1.55948692599183)(-1.22209821428571,1.56066044700026)(-1.22195870535714,1.56183330907389)(-1.22181919642857,1.56300550965847)(-1.2216796875,1.56417704620108)(-1.22154017857143,1.56534791615005)(-1.22140066964286,1.56651811695503)(-1.22126116071429,1.56768764606696)(-1.22112165178571,1.5688565009381)(-1.22098214285714,1.57002467902202)(-1.22084263392857,1.57119217777359)(-1.220703125,1.57235899464902)(-1.22056361607143,1.57352512710585)(-1.22042410714286,1.57469057260294)(-1.22028459821429,1.57585532860049)(-1.22014508928571,1.57701939256004)(-1.22000558035714,1.57818276194448)(-1.21986607142857,1.57934543421806)(-1.2197265625,1.58050740684638)(-1.21958705357143,1.58166867729639)(-1.21944754464286,1.58282924303642)(-1.21930803571429,1.58398910153618)(-1.21916852678571,1.58514825026673)(-1.21902901785714,1.58630668670054)(-1.21888950892857,1.58746440831144)(-1.21875,1.58862141257467)(-1.21861049107143,1.58977769696686)(-1.21847098214286,1.59093325896604)(-1.21833147321429,1.59208809605165)(-1.21819196428571,1.59324220570454)(-1.21805245535714,1.59439558540698)(-1.21791294642857,1.59554823264263)(-1.2177734375,1.59670014489663)(-1.21763392857143,1.5978513196555)(-1.21749441964286,1.59900175440722)(-1.21735491071429,1.60015144664122)(-1.21721540178571,1.60130039384835)(-1.21707589285714,1.60244859352092)(-1.21693638392857,1.6035960431527)(-1.216796875,1.60474274023892)(-1.21665736607143,1.60588868227626)(-1.21651785714286,1.60703386676289)(-1.21637834821429,1.60817829119844)(-1.21623883928571,1.60932195308403)(-1.21609933035714,1.61046484992226)(-1.21595982142857,1.6116069792172)(-1.2158203125,1.61274833847445)(-1.21568080357143,1.61388892520108)(-1.21554129464286,1.61502873690568)(-1.21540178571429,1.61616777109833)(-1.21526227678571,1.61730602529065)(-1.21512276785714,1.61844349699576)(-1.21498325892857,1.61958018372829)(-1.21484375,1.62071608300444)(-1.21470424107143,1.62185119234189)(-1.21456473214286,1.6229855092599)(-1.21442522321429,1.62411903127926)(-1.21428571428571,1.62525175592228)(-1.21414620535714,1.62638368071288)(-1.21400669642857,1.62751480317647)(-1.2138671875,1.62864512084006)(-1.21372767857143,1.62977463123224)(-1.21358816964286,1.63090333188313)(-1.21344866071429,1.63203122032445)(-1.21330915178571,1.63315829408951)(-1.21316964285714,1.63428455071318)(-1.21303013392857,1.63540998773194)(-1.212890625,1.63653460268385)(-1.21275111607143,1.63765839310859)(-1.21261160714286,1.63878135654743)(-1.21247209821429,1.63990349054326)(-1.21233258928571,1.64102479264056)(-1.21219308035714,1.64214526038546)(-1.21205357142857,1.6432648913257)(-1.2119140625,1.64438368301064)(-1.21177455357143,1.64550163299129)(-1.21163504464286,1.6466187388203)(-1.21149553571429,1.64773499805193)(-1.21135602678571,1.64885040824213)(-1.21121651785714,1.64996496694848)(-1.21107700892857,1.65107867173021)(-1.2109375,1.65219152014824)(-1.21079799107143,1.65330350976512)(-1.21065848214286,1.65441463814512)(-1.21051897321429,1.65552490285413)(-1.21037946428571,1.65663430145976)(-1.21023995535714,1.65774283153129)(-1.21010044642857,1.6588504906397)(-1.2099609375,1.65995727635766)(-1.20982142857143,1.66106318625953)(-1.20968191964286,1.66216821792139)(-1.20954241071429,1.66327236892103)(-1.20940290178571,1.66437563683793)(-1.20926339285714,1.66547801925332)(-1.20912388392857,1.66657951375014)(-1.208984375,1.66768011791305)(-1.20884486607143,1.66877982932845)(-1.20870535714286,1.66987864558449)(-1.20856584821429,1.67097656427104)(-1.20842633928571,1.67207358297972)(-1.20828683035714,1.67316969930393)(-1.20814732142857,1.67426491083878)(-1.2080078125,1.67535921518119)(-1.20786830357143,1.6764526099298)(-1.20772879464286,1.67754509268506)(-1.20758928571429,1.67863666104916)(-1.20744977678571,1.6797273126261)(-1.20731026785714,1.68081704502165)(-1.20717075892857,1.68190585584336)(-1.20703125,1.68299374270059)(-1.20689174107143,1.68408070320449)(-1.20675223214286,1.68516673496801)(-1.20661272321429,1.68625183560593)(-1.20647321428571,1.68733600273481)(-1.20633370535714,1.68841923397304)(-1.20619419642857,1.68950152694083)(-1.2060546875,1.69058287926024)(-1.20591517857143,1.69166328855512)(-1.20577566964286,1.69274275245117)(-1.20563616071429,1.69382126857596)(-1.20549665178571,1.69489883455886)(-1.20535714285714,1.69597544803111)(-1.20521763392857,1.69705110662581)(-1.205078125,1.69812580797791)(-1.20493861607143,1.69919954972422)(-1.20479910714286,1.70027232950343)(-1.20465959821429,1.70134414495609)(-1.20452008928571,1.70241499372462)(-1.20438058035714,1.70348487345335)(-1.20424107142857,1.70455378178847)(-1.2041015625,1.70562171637807)(-1.20396205357143,1.70668867487213)(-1.20382254464286,1.70775465492255)(-1.20368303571429,1.7088196541831)(-1.20354352678571,1.70988367030948)(-1.20340401785714,1.71094670095931)(-1.20326450892857,1.71200874379211)(-1.203125,1.71306979646934)(-1.20298549107143,1.71412985665437)(-1.20284598214286,1.71518892201252)(-1.20270647321429,1.71624699021102)(-1.20256696428571,1.71730405891908)(-1.20242745535714,1.71836012580781)(-1.20228794642857,1.7194151885503)(-1.2021484375,1.72046924482159)(-1.20200892857143,1.72152229229867)(-1.20186941964286,1.72257432866051)(-1.20172991071429,1.72362535158802)(-1.20159040178571,1.72467535876411)(-1.20145089285714,1.72572434787366)(-1.20131138392857,1.72677231660351)(-1.201171875,1.72781926264252)(-1.20103236607143,1.72886518368151)(-1.20089285714286,1.72991007741333)(-1.20075334821429,1.73095394153278)(-1.20061383928571,1.73199677373672)(-1.20047433035714,1.73303857172397)(-1.20033482142857,1.73407933319539)(-1.2001953125,1.73511905585386)(-1.20005580357143,1.73615773740425)(-1.19991629464286,1.7371953755535)(-1.19977678571429,1.73823196801055)(-1.19963727678571,1.73926751248639)(-1.19949776785714,1.74030200669404)(-1.19935825892857,1.74133544834857)(-1.19921875,1.74236783516709)(-1.19907924107143,1.74339916486878)(-1.19893973214286,1.74442943517487)(-1.19880022321429,1.74545864380863)(-1.19866071428571,1.74648678849544)(-1.19852120535714,1.74751386696271)(-1.19838169642857,1.74853987693995)(-1.1982421875,1.74956481615874)(-1.19810267857143,1.75058868235273)(-1.19796316964286,1.7516114732577)(-1.19782366071429,1.75263318661148)(-1.19768415178571,1.75365382015401)(-1.19754464285714,1.75467337162735)(-1.19740513392857,1.75569183877563)(-1.197265625,1.75670921934513)(-1.19712611607143,1.75772551108422)(-1.19698660714286,1.7587407117434)(-1.19684709821429,1.75975481907528)(-1.19670758928571,1.76076783083461)(-1.19656808035714,1.76177974477827)(-1.19642857142857,1.76279055866528)(-1.1962890625,1.76380027025679)(-1.19614955357143,1.76480887731611)(-1.19601004464286,1.76581637760869)(-1.19587053571429,1.76682276890213)(-1.19573102678571,1.76782804896619)(-1.19559151785714,1.76883221557281)(-1.19545200892857,1.76983526649608)(-1.1953125,1.77083719951227)(-1.19517299107143,1.77183801239981)(-1.19503348214286,1.77283770293933)(-1.19489397321429,1.77383626891363)(-1.19475446428571,1.77483370810772)(-1.19461495535714,1.77583001830877)(-1.19447544642857,1.77682519730616)(-1.1943359375,1.7778192428915)(-1.19419642857143,1.77881215285857)(-1.19405691964286,1.77980392500337)(-1.19391741071429,1.78079455712412)(-1.19377790178571,1.78178404702126)(-1.19363839285714,1.78277239249744)(-1.19349888392857,1.78375959135755)(-1.193359375,1.78474564140872)(-1.19321986607143,1.78573054046029)(-1.19308035714286,1.78671428632386)(-1.19294084821429,1.78769687681326)(-1.19280133928571,1.78867830974459)(-1.19266183035714,1.78965858293619)(-1.19252232142857,1.79063769420864)(-1.1923828125,1.79161564138482)(-1.19224330357143,1.79259242228985)(-1.19210379464286,1.79356803475111)(-1.19196428571429,1.79454247659828)(-1.19182477678571,1.7955157456633)(-1.19168526785714,1.79648783978042)(-1.19154575892857,1.79745875678613)(-1.19140625,1.79842849451924)(-1.19126674107143,1.79939705082087)(-1.19112723214286,1.80036442353441)(-1.19098772321429,1.80133061050557)(-1.19084821428571,1.80229560958235)(-1.19070870535714,1.80325941861509)(-1.19056919642857,1.80422203545643)(-1.1904296875,1.80518345796132)(-1.19029017857143,1.80614368398706)(-1.19015066964286,1.80710271139326)(-1.19001116071429,1.80806053804186)(-1.18987165178571,1.80901716179715)(-1.18973214285714,1.80997258052576)(-1.18959263392857,1.81092679209665)(-1.189453125,1.81187979438115)(-1.18931361607143,1.81283158525293)(-1.18917410714286,1.81378216258803)(-1.18903459821429,1.81473152426483)(-1.18889508928571,1.8156796681641)(-1.18875558035714,1.81662659216897)(-1.18861607142857,1.81757229416493)(-1.1884765625,1.81851677203988)(-1.18833705357143,1.81946002368408)(-1.18819754464286,1.82040204699017)(-1.18805803571429,1.82134283985321)(-1.18791852678571,1.82228240017062)(-1.18777901785714,1.82322072584225)(-1.18763950892857,1.82415781477033)(-1.1875,1.82509366485951)(-1.18736049107143,1.82602827401684)(-1.18722098214286,1.8269616401518)(-1.18708147321429,1.82789376117628)(-1.18694196428571,1.8288246350046)(-1.18680245535714,1.82975425955349)(-1.18666294642857,1.83068263274213)(-1.1865234375,1.83160975249213)(-1.18638392857143,1.83253561672753)(-1.18624441964286,1.83346022337483)(-1.18610491071429,1.83438357036297)(-1.18596540178571,1.83530565562335)(-1.18582589285714,1.8362264770898)(-1.18568638392857,1.83714603269863)(-1.185546875,1.83806432038861)(-1.18540736607143,1.83898133810099)(-1.18526785714286,1.83989708377947)(-1.18512834821429,1.84081155537024)(-1.18498883928571,1.84172475082196)(-1.18484933035714,1.84263666808579)(-1.18470982142857,1.84354730511535)(-1.1845703125,1.84445665986678)(-1.18443080357143,1.84536473029871)(-1.18429129464286,1.84627151437224)(-1.18415178571429,1.84717701005102)(-1.18401227678571,1.84808121530118)(-1.18387276785714,1.84898412809136)(-1.18373325892857,1.84988574639273)(-1.18359375,1.85078606817897)(-1.18345424107143,1.85168509142627)(-1.18331473214286,1.85258281411338)(-1.18317522321429,1.85347923422156)(-1.18303571428571,1.8543743497346)(-1.18289620535714,1.85526815863885)(-1.18275669642857,1.85616065892318)(-1.1826171875,1.85705184857902)(-1.18247767857143,1.85794172560035)(-1.18233816964286,1.85883028798371)(-1.18219866071429,1.85971753372818)(-1.18205915178571,1.86060346083542)(-1.18191964285714,1.86148806730964)(-1.18178013392857,1.86237135115765)(-1.181640625,1.86325331038881)(-1.18150111607143,1.86413394301505)(-1.18136160714286,1.8650132470509)(-1.18122209821429,1.86589122051349)(-1.18108258928571,1.86676786142249)(-1.18094308035714,1.86764316780022)(-1.18080357142857,1.86851713767156)(-1.1806640625,1.86938976906401)(-1.18052455357143,1.87026106000766)(-1.18038504464286,1.87113100853521)(-1.18024553571429,1.871999612682)(-1.18010602678571,1.87286687048594)(-1.17996651785714,1.87373277998762)(-1.17982700892857,1.87459733923018)(-1.1796875,1.87546054625946)(-1.17954799107143,1.87632239912389)(-1.17940848214286,1.87718289587455)(-1.17926897321429,1.87804203456514)(-1.17912946428571,1.87889981325204)(-1.17898995535714,1.87975622999424)(-1.17885044642857,1.8806112828534)(-1.1787109375,1.88146496989383)(-1.17857142857143,1.8823172891825)(-1.17843191964286,1.88316823878903)(-1.17829241071429,1.88401781678574)(-1.17815290178571,1.88486602124758)(-1.17801339285714,1.88571285025219)(-1.17787388392857,1.88655830187989)(-1.177734375,1.88740237421367)(-1.17759486607143,1.88824506533922)(-1.17745535714286,1.88908637334491)(-1.17731584821429,1.88992629632179)(-1.17717633928571,1.89076483236363)(-1.17703683035714,1.89160197956689)(-1.17689732142857,1.89243773603071)(-1.1767578125,1.89327209985697)(-1.17661830357143,1.89410506915025)(-1.17647879464286,1.89493664201784)(-1.17633928571429,1.89576681656974)(-1.17619977678571,1.89659559091869)(-1.17606026785714,1.89742296318014)(-1.17592075892857,1.89824893147227)(-1.17578125,1.89907349391601)(-1.17564174107143,1.89989664863501)(-1.17550223214286,1.90071839375565)(-1.17536272321429,1.90153872740708)(-1.17522321428571,1.90235764772118)(-1.17508370535714,1.90317515283258)(-1.17494419642857,1.90399124087868)(-1.1748046875,1.90480590999961)(-1.17466517857143,1.90561915833829)(-1.17452566964286,1.90643098404039)(-1.17438616071429,1.90724138525436)(-1.17424665178571,1.9080503601314)(-1.17410714285714,1.90885790682551)(-1.17396763392857,1.90966402349346)(-1.173828125,1.9104687082948)(-1.17368861607143,1.91127195939188)(-1.17354910714286,1.91207377494982)(-1.17340959821429,1.91287415313655)(-1.17327008928571,1.91367309212279)(-1.17313058035714,1.91447059008206)(-1.17299107142857,1.9152666451907)(-1.1728515625,1.91606125562783)(-1.17271205357143,1.91685441957541)(-1.17257254464286,1.9176461352182)(-1.17243303571429,1.91843640074379)(-1.17229352678571,1.91922521434257)(-1.17215401785714,1.92001257420778)(-1.17201450892857,1.92079847853547)(-1.171875,1.92158292552454)(-1.17173549107143,1.92236591337672)(-1.17159598214286,1.92314744029657)(-1.17145647321429,1.92392750449152)(-1.17131696428571,1.92470610417181)(-1.17117745535714,1.92548323755055)(-1.17103794642857,1.92625890284372)(-1.1708984375,1.92703309827012)(-1.17075892857143,1.92780582205145)(-1.17061941964286,1.92857707241224)(-1.17047991071429,1.92934684757991)(-1.17034040178571,1.93011514578475)(-1.17020089285714,1.93088196525991)(-1.17006138392857,1.93164730424142)(-1.169921875,1.93241116096822)(-1.16978236607143,1.93317353368209)(-1.16964285714286,1.93393442062774)(-1.16950334821429,1.93469382005274)(-1.16936383928571,1.93545173020757)(-1.16922433035714,1.93620814934561)(-1.16908482142857,1.93696307572315)(-1.1689453125,1.93771650759935)(-1.16880580357143,1.93846844323632)(-1.16866629464286,1.93921888089906)(-1.16852678571429,1.93996781885549)(-1.16838727678571,1.94071525537645)(-1.16824776785714,1.9414611887357)(-1.16810825892857,1.94220561720993)(-1.16796875,1.94294853907875)(-1.16782924107143,1.94368995262471)(-1.16768973214286,1.9444298561333)(-1.16755022321429,1.94516824789293)(-1.16741071428571,1.94590512619498)(-1.16727120535714,1.94664048933375)(-1.16713169642857,1.9473743356065)(-1.1669921875,1.94810666331345)(-1.16685267857143,1.94883747075776)(-1.16671316964286,1.94956675624556)(-1.16657366071429,1.95029451808593)(-1.16643415178571,1.95102075459093)(-1.16629464285714,1.95174546407558)(-1.16615513392857,1.95246864485788)(-1.166015625,1.9531902952588)(-1.16587611607143,1.95391041360227)(-1.16573660714286,1.95462899821524)(-1.16559709821429,1.95534604742762)(-1.16545758928571,1.95606155957232)(-1.16531808035714,1.95677553298522)(-1.16517857142857,1.95748796600521)(-1.1650390625,1.95819885697419)(-1.16489955357143,1.95890820423704)(-1.16476004464286,1.95961600614166)(-1.16462053571429,1.96032226103895)(-1.16448102678571,1.96102696728282)(-1.16434151785714,1.96173012323019)(-1.16420200892857,1.96243172724101)(-1.1640625,1.96313177767825)(-1.16392299107143,1.96383027290789)(-1.16378348214286,1.96452721129894)(-1.16364397321429,1.96522259122345)(-1.16350446428571,1.9659164110565)(-1.16336495535714,1.9666086691762)(-1.16322544642857,1.96729936396371)(-1.1630859375,1.96798849380322)(-1.16294642857143,1.96867605708197)(-1.16280691964286,1.96936205219026)(-1.16266741071429,1.97004647752142)(-1.16252790178571,1.97072933147186)(-1.16238839285714,1.97141061244103)(-1.16224888392857,1.97209031883146)(-1.162109375,1.97276844904871)(-1.16196986607143,1.97344500150145)(-1.16183035714286,1.9741199746014)(-1.16169084821429,1.97479336676335)(-1.16155133928571,1.97546517640516)(-1.16141183035714,1.97613540194781)(-1.16127232142857,1.97680404181531)(-1.1611328125,1.97747109443479)(-1.16099330357143,1.97813655823646)(-1.16085379464286,1.97880043165362)(-1.16071428571429,1.97946271312268)(-1.16057477678571,1.98012340108312)(-1.16043526785714,1.98078249397755)(-1.16029575892857,1.98143999025166)(-1.16015625,1.98209588835427)(-1.16001674107143,1.9827501867373)(-1.15987723214286,1.98340288385578)(-1.15973772321429,1.98405397816786)(-1.15959821428571,1.98470346813481)(-1.15945870535714,1.98535135222103)(-1.15931919642857,1.98599762889402)(-1.1591796875,1.98664229662444)(-1.15904017857143,1.98728535388606)(-1.15890066964286,1.9879267991558)(-1.15876116071429,1.98856663091371)(-1.15862165178571,1.98920484764297)(-1.15848214285714,1.98984144782992)(-1.15834263392857,1.99047642996403)(-1.158203125,1.99110979253794)(-1.15806361607143,1.99174153404743)(-1.15792410714286,1.99237165299142)(-1.15778459821429,1.99300014787202)(-1.15764508928571,1.99362701719448)(-1.15750558035714,1.99425225946722)(-1.15736607142857,1.99487587320181)(-1.1572265625,1.99549785691303)(-1.15708705357143,1.99611820911879)(-1.15694754464286,1.99673692834019)(-1.15680803571429,1.99735401310151)(-1.15666852678571,1.99796946193023)(-1.15652901785714,1.99858327335698)(-1.15638950892857,1.99919544591559)(-1.15625,1.99980597814309)(-1.15611049107143,2.00041486857968)(-1.15597098214286,2.0010221157688)(-1.15583147321429,2.00162771825703)(-1.15569196428571,2.00223167459419)(-1.15555245535714,2.00283398333329)(-1.15541294642857,2.00343464303055)(-1.1552734375,2.0040336522454)(-1.15513392857143,2.00463100954048)(-1.15499441964286,2.00522671348166)(-1.15485491071429,2.00582076263799)(-1.15471540178571,2.00641315558179)(-1.15457589285714,2.00700389088856)(-1.15443638392857,2.00759296713705)(-1.154296875,2.00818038290924)(-1.15415736607143,2.00876613679033)(-1.15401785714286,2.00935022736876)(-1.15387834821429,2.00993265323621)(-1.15373883928571,2.01051341298759)(-1.15359933035714,2.01109250522107)(-1.15345982142857,2.01166992853805)(-1.1533203125,2.01224568154318)(-1.15318080357143,2.01281976284437)(-1.15304129464286,2.01339217105278)(-1.15290178571429,2.01396290478281)(-1.15276227678571,2.01453196265216)(-1.15262276785714,2.01509934328174)(-1.15248325892857,2.01566504529576)(-1.15234375,2.01622906732169)(-1.15220424107143,2.01679140799025)(-1.15206473214286,2.01735206593547)(-1.15192522321429,2.01791103979462)(-1.15178571428571,2.01846832820826)(-1.15164620535714,2.01902392982024)(-1.15150669642857,2.01957784327768)(-1.1513671875,2.020130067231)(-1.15122767857143,2.02068060033388)(-1.15108816964286,2.02122944124332)(-1.15094866071429,2.0217765886196)(-1.15080915178571,2.0223220411263)(-1.15066964285714,2.0228657974303)(-1.15053013392857,2.02340785620177)(-1.150390625,2.02394821611421)(-1.15025111607143,2.02448687584439)(-1.15011160714286,2.02502383407242)(-1.14997209821429,2.0255590894817)(-1.14983258928571,2.02609264075896)(-1.14969308035714,2.02662448659425)(-1.14955357142857,2.02715462568091)(-1.1494140625,2.02768305671564)(-1.14927455357143,2.02820977839843)(-1.14913504464286,2.02873478943263)(-1.14899553571429,2.02925808852488)(-1.14885602678571,2.0297796743852)(-1.14871651785714,2.0302995457269)(-1.14857700892857,2.03081770126665)(-1.1484375,2.03133413972445)(-1.14829799107143,2.03184885982366)(-1.14815848214286,2.03236186029096)(-1.14801897321429,2.0328731398564)(-1.14787946428571,2.03338269725335)(-1.14773995535714,2.03389053121856)(-1.14760044642857,2.03439664049212)(-1.1474609375,2.03490102381749)(-1.14732142857143,2.03540367994146)(-1.14718191964286,2.03590460761422)(-1.14704241071429,2.0364038055893)(-1.14690290178571,2.03690127262359)(-1.14676339285714,2.03739700747738)(-1.14662388392857,2.03789100891429)(-1.146484375,2.03838327570136)(-1.14634486607143,2.03887380660896)(-1.14620535714286,2.03936260041086)(-1.14606584821429,2.03984965588423)(-1.14592633928571,2.04033497180959)(-1.14578683035714,2.04081854697086)(-1.14564732142857,2.04130038015535)(-1.1455078125,2.04178047015376)(-1.14536830357143,2.04225881576018)(-1.14522879464286,2.04273541577209)(-1.14508928571429,2.04321026899039)(-1.14494977678571,2.04368337421934)(-1.14481026785714,2.04415473026664)(-1.14467075892857,2.04462433594338)(-1.14453125,2.04509219006405)(-1.14439174107143,2.04555829144655)(-1.14425223214286,2.04602263891221)(-1.14411272321429,2.04648523128576)(-1.14397321428571,2.04694606739533)(-1.14383370535714,2.04740514607251)(-1.14369419642857,2.04786246615225)(-1.1435546875,2.04831802647299)(-1.14341517857143,2.04877182587654)(-1.14327566964286,2.04922386320817)(-1.14313616071429,2.04967413731657)(-1.14299665178571,2.05012264705386)(-1.14285714285714,2.05056939127559)(-1.14271763392857,2.05101436884075)(-1.142578125,2.05145757861178)(-1.14243861607143,2.05189901945454)(-1.14229910714286,2.05233869023834)(-1.14215959821429,2.05277658983595)(-1.14202008928571,2.05321271712356)(-1.14188058035714,2.05364707098084)(-1.14174107142857,2.05407965029088)(-1.1416015625,2.05451045394026)(-1.14146205357143,2.05493948081897)(-1.14132254464286,2.0553667298205)(-1.14118303571429,2.05579219984179)(-1.14104352678571,2.05621588978322)(-1.14090401785714,2.05663779854866)(-1.14076450892857,2.05705792504543)(-1.140625,2.05747626818434)(-1.14048549107143,2.05789282687964)(-1.14034598214286,2.05830760004908)(-1.14020647321429,2.05872058661388)(-1.14006696428571,2.05913178549873)(-1.13992745535714,2.05954119563179)(-1.13978794642857,2.05994881594472)(-1.1396484375,2.06035464537267)(-1.13950892857143,2.06075868285424)(-1.13936941964286,2.06116092733155)(-1.13922991071429,2.06156137775021)(-1.13909040178571,2.06196003305929)(-1.13895089285714,2.06235689221139)(-1.13881138392857,2.06275195416259)(-1.138671875,2.06314521787247)(-1.13853236607143,2.06353668230409)(-1.13839285714286,2.06392634642406)(-1.13825334821429,2.06431420920244)(-1.13811383928571,2.06470026961283)(-1.13797433035714,2.06508452663234)(-1.13783482142857,2.06546697924155)(-1.1376953125,2.0658476264246)(-1.13755580357143,2.06622646716913)(-1.13741629464286,2.06660350046627)(-1.13727678571429,2.06697872531069)(-1.13713727678571,2.06735214070059)(-1.13699776785714,2.06772374563768)(-1.13685825892857,2.06809353912718)(-1.13671875,2.06846152017785)(-1.13657924107143,2.06882768780198)(-1.13643973214286,2.06919204101538)(-1.13630022321429,2.0695545788374)(-1.13616071428571,2.06991530029093)(-1.13602120535714,2.07027420440236)(-1.13588169642857,2.07063129020167)(-1.1357421875,2.07098655672233)(-1.13560267857143,2.07134000300137)(-1.13546316964286,2.07169162807938)(-1.13532366071429,2.07204143100047)(-1.13518415178571,2.07238941081231)(-1.13504464285714,2.07273556656611)(-1.13490513392857,2.07307989731663)(-1.134765625,2.07342240212219)(-1.13462611607143,2.07376308004465)(-1.13448660714286,2.07410193014945)(-1.13434709821429,2.07443895150557)(-1.13420758928571,2.07477414318554)(-1.13406808035714,2.07510750426547)(-1.13392857142857,2.07543903382503)(-1.1337890625,2.07576873094744)(-1.13364955357143,2.0760965947195)(-1.13351004464286,2.07642262423158)(-1.13337053571429,2.07674681857761)(-1.13323102678571,2.0770691768551)(-1.13309151785714,2.07738969816513)(-1.13295200892857,2.07770838161234)(-1.1328125,2.07802522630499)(-1.13267299107143,2.07834023135487)(-1.13253348214286,2.07865339587737)(-1.13239397321429,2.07896471899148)(-1.13225446428571,2.07927419981975)(-1.13211495535714,2.07958183748831)(-1.13197544642857,2.07988763112691)(-1.1318359375,2.08019157986886)(-1.13169642857143,2.08049368285107)(-1.13155691964286,2.08079393921404)(-1.13141741071429,2.08109234810188)(-1.13127790178571,2.08138890866226)(-1.13113839285714,2.08168362004648)(-1.13099888392857,2.08197648140944)(-1.130859375,2.08226749190961)(-1.13071986607143,2.08255665070909)(-1.13058035714286,2.08284395697358)(-1.13044084821429,2.08312940987237)(-1.13030133928571,2.08341300857837)(-1.13016183035714,2.08369475226811)(-1.13002232142857,2.0839746401217)(-1.1298828125,2.08425267132289)(-1.12974330357143,2.08452884505902)(-1.12960379464286,2.08480316052108)(-1.12946428571429,2.08507561690363)(-1.12932477678571,2.08534621340489)(-1.12918526785714,2.08561494922667)(-1.12904575892857,2.08588182357443)(-1.12890625,2.08614683565722)(-1.12876674107143,2.08640998468774)(-1.12862723214286,2.0866712698823)(-1.12848772321429,2.08693069046086)(-1.12834821428571,2.08718824564698)(-1.12820870535714,2.08744393466787)(-1.12806919642857,2.08769775675437)(-1.1279296875,2.08794971114094)(-1.12779017857143,2.08819979706569)(-1.12765066964286,2.08844801377036)(-1.12751116071429,2.08869436050034)(-1.12737165178571,2.08893883650464)(-1.12723214285714,2.08918144103592)(-1.12709263392857,2.08942217335049)(-1.126953125,2.08966103270829)(-1.12681361607143,2.08989801837291)(-1.12667410714286,2.09013312961159)(-1.12653459821429,2.09036636569521)(-1.12639508928571,2.09059772589833)(-1.12625558035714,2.09082720949911)(-1.12611607142857,2.0910548157794)(-1.1259765625,2.0912805440247)(-1.12583705357143,2.09150439352415)(-1.12569754464286,2.09172636357056)(-1.12555803571429,2.09194645346039)(-1.12541852678571,2.09216466249376)(-1.12527901785714,2.09238098997445)(-1.12513950892857,2.09259543520992)(-1.125,2.09280799751126)(-1.12486049107143,2.09301867619326)(-1.12472098214286,2.09322747057434)(-1.12458147321429,2.09343437997663)(-1.12444196428571,2.09363940372588)(-1.12430245535714,2.09384254115155)(-1.12416294642857,2.09404379158676)(-1.1240234375,2.09424315436829)(-1.12388392857143,2.0944406288366)(-1.12374441964286,2.09463621433584)(-1.12360491071429,2.09482991021381)(-1.12346540178571,2.09502171582202)(-1.12332589285714,2.09521163051562)(-1.12318638392857,2.09539965365348)(-1.123046875,2.09558578459813)(-1.12290736607143,2.09577002271577)(-1.12276785714286,2.09595236737632)(-1.12262834821429,2.09613281795335)(-1.12248883928571,2.09631137382414)(-1.12234933035714,2.09648803436965)(-1.12220982142857,2.09666279897452)(-1.1220703125,2.09683566702709)(-1.12193080357143,2.09700663791939)(-1.12179129464286,2.09717571104714)(-1.12165178571429,2.09734288580975)(-1.12151227678571,2.09750816161035)(-1.12137276785714,2.09767153785572)(-1.12123325892857,2.09783301395638)(-1.12109375,2.09799258932652)(-1.12095424107143,2.09815026338404)(-1.12081473214286,2.09830603555055)(-1.12067522321429,2.09845990525135)(-1.12053571428571,2.09861187191544)(-1.12039620535714,2.09876193497553)(-1.12025669642857,2.09891009386803)(-1.1201171875,2.09905634803307)(-1.11997767857143,2.09920069691447)(-1.11983816964286,2.09934313995976)(-1.11969866071429,2.09948367662019)(-1.11955915178571,2.09962230635072)(-1.11941964285714,2.09975902861001)(-1.11928013392857,2.09989384286044)(-1.119140625,2.10002674856809)(-1.11900111607143,2.10015774520279)(-1.11886160714286,2.10028683223804)(-1.11872209821429,2.1004140091511)(-1.11858258928571,2.10053927542291)(-1.11844308035714,2.10066263053814)(-1.11830357142857,2.10078407398521)(-1.1181640625,2.10090360525621)(-1.11802455357143,2.10102122384698)(-1.11788504464286,2.1011369292571)(-1.11774553571429,2.10125072098983)(-1.11760602678571,2.10136259855219)(-1.11746651785714,2.10147256145491)(-1.11732700892857,2.10158060921244)(-1.1171875,2.10168674134298)(-1.11704799107143,2.10179095736845)(-1.11690848214286,2.10189325681448)(-1.11676897321429,2.10199363921044)(-1.11662946428571,2.10209210408946)(-1.11648995535714,2.10218865098836)(-1.11635044642857,2.10228327944771)(-1.1162109375,2.10237598901182)(-1.11607142857143,2.10246677922873)(-1.11593191964286,2.10255564965022)(-1.11579241071429,2.10264259983179)(-1.11565290178571,2.10272762933269)(-1.11551339285714,2.10281073771591)(-1.11537388392857,2.10289192454817)(-1.115234375,2.10297118939995)(-1.11509486607143,2.10304853184544)(-1.11495535714286,2.1031239514626)(-1.11481584821429,2.10319744783312)(-1.11467633928571,2.10326902054242)(-1.11453683035714,2.10333866917969)(-1.11439732142857,2.10340639333785)(-1.1142578125,2.10347219261357)(-1.11411830357143,2.10353606660725)(-1.11397879464286,2.10359801492307)(-1.11383928571429,2.10365803716892)(-1.11369977678571,2.10371613295647)(-1.11356026785714,2.10377230190113)(-1.11342075892857,2.10382654362205)(-1.11328125,2.10387885774214)(-1.11314174107143,2.10392924388807)(-1.11300223214286,2.10397770169023)(-1.11286272321429,2.1040242307828)(-1.11272321428571,2.1040688308037)(-1.11258370535714,2.1041115013946)(-1.11244419642857,2.10415224220093)(-1.1123046875,2.10419105287187)(-1.11216517857143,2.10422793306036)(-1.11202566964286,2.10426288242311)(-1.11188616071429,2.10429590062057)(-1.11174665178571,2.10432698731696)(-1.11160714285714,2.10435614218024)(-1.11146763392857,2.10438336488216)(-1.111328125,2.1044086550982)(-1.11118861607143,2.10443201250763)(-1.11104910714286,2.10445343679346)(-1.11090959821429,2.10447292764247)(-1.11077008928571,2.10449048474521)(-1.11063058035714,2.10450610779597)(-1.11049107142857,2.10451979649284)(-1.1103515625,2.10453155053764)(-1.11021205357143,2.10454136963598)(-1.11007254464286,2.10454925349722)(-1.10993303571429,2.1045552018345)(-1.10979352678571,2.10455921436472)(-1.10965401785714,2.10456129080855)(-1.10951450892857,2.10456143089041)(-1.109375,2.10455963433853)(-1.10923549107143,2.10455590088486)(-1.10909598214286,2.10455023026516)(-1.10895647321429,2.10454262221893)(-1.10881696428571,2.10453307648947)(-1.10867745535714,2.10452159282383)(-1.10853794642857,2.10450817097284)(-1.1083984375,2.10449281069109)(-1.10825892857143,2.10447551173696)(-1.10811941964286,2.1044562738726)(-1.10797991071429,2.10443509686392)(-1.10784040178571,2.10441198048062)(-1.10770089285714,2.10438692449617)(-1.10756138392857,2.10435992868781)(-1.107421875,2.10433099283656)(-1.10728236607143,2.10430011672721)(-1.10714285714286,2.10426730014833)(-1.10700334821429,2.10423254289228)(-1.10686383928571,2.10419584475518)(-1.10672433035714,2.10415720553692)(-1.10658482142857,2.10411662504119)(-1.1064453125,2.10407410307545)(-1.10630580357143,2.10402963945093)(-1.10616629464286,2.10398323398265)(-1.10602678571429,2.1039348864894)(-1.10588727678571,2.10388459679376)(-1.10574776785714,2.10383236472209)(-1.10560825892857,2.10377819010452)(-1.10546875,2.10372207277496)(-1.10532924107143,2.10366401257111)(-1.10518973214286,2.10360400933446)(-1.10505022321429,2.10354206291026)(-1.10491071428571,2.10347817314755)(-1.10477120535714,2.10341233989916)(-1.10463169642857,2.10334456302169)(-1.1044921875,2.10327484237554)(-1.10435267857143,2.10320317782487)(-1.10421316964286,2.10312956923765)(-1.10407366071429,2.10305401648562)(-1.10393415178571,2.10297651944429)(-1.10379464285714,2.10289707799297)(-1.10365513392857,2.10281569201477)(-1.103515625,2.10273236139654)(-1.10337611607143,2.10264708602897)(-1.10323660714286,2.10255986580648)(-1.10309709821429,2.10247070062732)(-1.10295758928571,2.10237959039351)(-1.10281808035714,2.10228653501083)(-1.10267857142857,2.10219153438889)(-1.1025390625,2.10209458844106)(-1.10239955357143,2.1019956970845)(-1.10226004464286,2.10189486024015)(-1.10212053571429,2.10179207783275)(-1.10198102678571,2.10168734979082)(-1.10184151785714,2.10158067604667)(-1.10170200892857,2.10147205653639)(-1.1015625,2.10136149119986)(-1.10142299107143,2.10124897998075)(-1.10128348214286,2.10113452282651)(-1.10114397321429,2.10101811968839)(-1.10100446428571,2.10089977052141)(-1.10086495535714,2.1007794752844)(-1.10072544642857,2.10065723393995)(-1.1005859375,2.10053304645447)(-1.10044642857143,2.10040691279813)(-1.10030691964286,2.10027883294489)(-1.10016741071429,2.10014880687252)(-1.10002790178571,2.10001683456256)(-1.09988839285714,2.09988291600034)(-1.09974888392857,2.09974705117498)(-1.099609375,2.09960924007939)(-1.09946986607143,2.09946948271026)(-1.09933035714286,2.09932777906808)(-1.09919084821429,2.09918412915712)(-1.09905133928571,2.09903853298543)(-1.09891183035714,2.09889099056487)(-1.09877232142857,2.09874150191107)(-1.0986328125,2.09859006704345)(-1.09849330357143,2.09843668598523)(-1.09835379464286,2.0982813587634)(-1.09821428571429,2.09812408540875)(-1.09807477678571,2.09796486595586)(-1.09793526785714,2.09780370044308)(-1.09779575892857,2.09764058891257)(-1.09765625,2.09747553141027)(-1.09751674107143,2.09730852798589)(-1.09737723214286,2.09713957869296)(-1.09723772321429,2.09696868358876)(-1.09709821428571,2.09679584273439)(-1.09695870535714,2.09662105619471)(-1.09681919642857,2.0964443240384)(-1.0966796875,2.09626564633789)(-1.09654017857143,2.09608502316942)(-1.09640066964286,2.095902454613)(-1.09626116071429,2.09571794075245)(-1.09612165178571,2.09553148167535)(-1.09598214285714,2.09534307747309)(-1.09584263392857,2.09515272824083)(-1.095703125,2.09496043407751)(-1.09556361607143,2.09476619508587)(-1.09542410714286,2.09457001137243)(-1.09528459821429,2.09437188304749)(-1.09514508928571,2.09417181022515)(-1.09500558035714,2.09396979302328)(-1.09486607142857,2.09376583156353)(-1.0947265625,2.09355992597135)(-1.09458705357143,2.09335207637597)(-1.09444754464286,2.09314228291039)(-1.09430803571429,2.09293054571141)(-1.09416852678571,2.0927168649196)(-1.09402901785714,2.09250124067933)(-1.09388950892857,2.09228367313872)(-1.09375,2.09206416244972)(-1.09361049107143,2.09184270876801)(-1.09347098214286,2.0916193122531)(-1.09333147321429,2.09139397306824)(-1.09319196428571,2.09116669138049)(-1.09305245535714,2.09093746736067)(-1.09291294642857,2.0907063011834)(-1.0927734375,2.09047319302707)(-1.09263392857143,2.09023814307384)(-1.09249441964286,2.09000115150967)(-1.09235491071429,2.08976221852428)(-1.09221540178571,2.08952134431118)(-1.09207589285714,2.08927852906766)(-1.09193638392857,2.08903377299477)(-1.091796875,2.08878707629736)(-1.09165736607143,2.08853843918404)(-1.09151785714286,2.0882878618672)(-1.09137834821429,2.08803534456302)(-1.09123883928571,2.08778088749145)(-1.09109933035714,2.08752449087619)(-1.09095982142857,2.08726615494475)(-1.0908203125,2.0870058799284)(-1.09068080357143,2.08674366606218)(-1.09054129464286,2.08647951358491)(-1.09040178571429,2.08621342273917)(-1.09026227678571,2.08594539377134)(-1.09012276785714,2.08567542693154)(-1.08998325892857,2.08540352247369)(-1.08984375,2.08512968065546)(-1.08970424107143,2.0848539017383)(-1.08956473214286,2.08457618598743)(-1.08942522321429,2.08429653367184)(-1.08928571428571,2.08401494506428)(-1.08914620535714,2.08373142044128)(-1.08900669642857,2.08344596008313)(-1.0888671875,2.08315856427391)(-1.08872767857143,2.08286923330142)(-1.08858816964286,2.08257796745728)(-1.08844866071429,2.08228476703683)(-1.08830915178571,2.08198963233921)(-1.08816964285714,2.08169256366731)(-1.08803013392857,2.08139356132777)(-1.087890625,2.08109262563103)(-1.08775111607143,2.08078975689125)(-1.08761160714286,2.08048495542638)(-1.08747209821429,2.08017822155813)(-1.08733258928571,2.07986955561197)(-1.08719308035714,2.0795589579171)(-1.08705357142857,2.07924642880654)(-1.0869140625,2.07893196861701)(-1.08677455357143,2.07861557768902)(-1.08663504464286,2.07829725636683)(-1.08649553571429,2.07797700499846)(-1.08635602678571,2.07765482393569)(-1.08621651785714,2.07733071353403)(-1.08607700892857,2.07700467415279)(-1.0859375,2.076676706155)(-1.08579799107143,2.07634680990744)(-1.08565848214286,2.07601498578068)(-1.08551897321429,2.075681234149)(-1.08537946428571,2.07534555539045)(-1.08523995535714,2.07500794988683)(-1.08510044642857,2.07466841802371)(-1.0849609375,2.07432696019036)(-1.08482142857143,2.07398357677985)(-1.08468191964286,2.07363826818897)(-1.08454241071429,2.07329103481825)(-1.08440290178571,2.072941877072)(-1.08426339285714,2.07259079535823)(-1.08412388392857,2.07223779008874)(-1.083984375,2.07188286167904)(-1.08384486607143,2.0715260105484)(-1.08370535714286,2.07116723711983)(-1.08356584821429,2.07080654182007)(-1.08342633928571,2.07044392507962)(-1.08328683035714,2.0700793873327)(-1.08314732142857,2.06971292901728)(-1.0830078125,2.06934455057506)(-1.08286830357143,2.0689742524515)(-1.08272879464286,2.06860203509576)(-1.08258928571429,2.06822789896077)(-1.08244977678571,2.06785184450317)(-1.08231026785714,2.06747387218334)(-1.08217075892857,2.0670939824654)(-1.08203125,2.06671217581719)(-1.08189174107143,2.0663284527103)(-1.08175223214286,2.06594281362003)(-1.08161272321429,2.06555525902542)(-1.08147321428571,2.06516578940923)(-1.08133370535714,2.06477440525796)(-1.08119419642857,2.06438110706181)(-1.0810546875,2.06398589531475)(-1.08091517857143,2.06358877051442)(-1.08077566964286,2.06318973316223)(-1.08063616071429,2.06278878376329)(-1.08049665178571,2.06238592282643)(-1.08035714285714,2.06198115086421)(-1.08021763392857,2.0615744683929)(-1.080078125,2.0611658759325)(-1.07993861607143,2.06075537400671)(-1.07979910714286,2.06034296314297)(-1.07965959821429,2.05992864387241)(-1.07952008928571,2.05951241672988)(-1.07938058035714,2.05909428225397)(-1.07924107142857,2.05867424098695)(-1.0791015625,2.05825229347481)(-1.07896205357143,2.05782844026726)(-1.07882254464286,2.05740268191771)(-1.07868303571429,2.05697501898327)(-1.07854352678571,2.05654545202478)(-1.07840401785714,2.05611398160677)(-1.07826450892857,2.05568060829747)(-1.078125,2.05524533266883)(-1.07798549107143,2.05480815529648)(-1.07784598214286,2.05436907675978)(-1.07770647321429,2.05392809764176)(-1.07756696428571,2.05348521852917)(-1.07742745535714,2.05304044001245)(-1.07728794642857,2.05259376268575)(-1.0771484375,2.05214518714688)(-1.07700892857143,2.05169471399739)(-1.07686941964286,2.05124234384249)(-1.07672991071429,2.0507880772911)(-1.07659040178571,2.05033191495582)(-1.07645089285714,2.04987385745294)(-1.07631138392857,2.04941390540246)(-1.076171875,2.04895205942803)(-1.07603236607143,2.04848832015701)(-1.07589285714286,2.04802268822045)(-1.07575334821429,2.04755516425307)(-1.07561383928571,2.04708574889328)(-1.07547433035714,2.04661444278316)(-1.07533482142857,2.04614124656849)(-1.0751953125,2.04566616089871)(-1.07505580357143,2.04518918642694)(-1.07491629464286,2.04471032380999)(-1.07477678571429,2.04422957370833)(-1.07463727678571,2.0437469367861)(-1.07449776785714,2.04326241371114)(-1.07435825892857,2.04277600515493)(-1.07421875,2.04228771179263)(-1.07407924107143,2.04179753430307)(-1.07393973214286,2.04130547336876)(-1.07380022321429,2.04081152967585)(-1.07366071428571,2.04031570391417)(-1.07352120535714,2.03981799677721)(-1.07338169642857,2.03931840896212)(-1.0732421875,2.03881694116972)(-1.07310267857143,2.03831359410447)(-1.07296316964286,2.03780836847451)(-1.07282366071429,2.03730126499161)(-1.07268415178571,2.03679228437123)(-1.07254464285714,2.03628142733244)(-1.07240513392857,2.03576869459801)(-1.072265625,2.03525408689431)(-1.07212611607143,2.03473760495141)(-1.07198660714286,2.03421924950298)(-1.07184709821429,2.03369902128637)(-1.07170758928571,2.03317692104257)(-1.07156808035714,2.03265294951619)(-1.07142857142857,2.03212710745552)(-1.0712890625,2.03159939561246)(-1.07114955357143,2.03106981474255)(-1.07101004464286,2.03053836560499)(-1.07087053571429,2.0300050489626)(-1.07073102678571,2.02946986558184)(-1.07059151785714,2.0289328162328)(-1.07045200892857,2.02839390168919)(-1.0703125,2.02785312272838)(-1.07017299107143,2.02731048013134)(-1.07003348214286,2.02676597468269)(-1.06989397321429,2.02621960717065)(-1.06975446428571,2.02567137838708)(-1.06961495535714,2.02512128912748)(-1.06947544642857,2.02456934019093)(-1.0693359375,2.02401553238015)(-1.06919642857143,2.0234598665015)(-1.06905691964286,2.02290234336492)(-1.06891741071429,2.02234296378398)(-1.06877790178571,2.02178172857587)(-1.06863839285714,2.02121863856137)(-1.06849888392857,2.02065369456491)(-1.068359375,2.02008689741448)(-1.06821986607143,2.01951824794172)(-1.06808035714286,2.01894774698183)(-1.06794084821429,2.01837539537366)(-1.06780133928571,2.01780119395963)(-1.06766183035714,2.01722514358578)(-1.06752232142857,2.01664724510173)(-1.0673828125,2.0160674993607)(-1.06724330357143,2.01548590721953)(-1.06710379464286,2.01490246953862)(-1.06696428571429,2.01431718718198)(-1.06682477678571,2.01373006101722)(-1.06668526785714,2.0131410919155)(-1.06654575892857,2.01255028075162)(-1.06640625,2.01195762840392)(-1.06626674107143,2.01136313575434)(-1.06612723214286,2.01076680368842)(-1.06598772321429,2.01016863309524)(-1.06584821428571,2.0095686248675)(-1.06570870535714,2.00896677990145)(-1.06556919642857,2.00836309909693)(-1.0654296875,2.00775758335733)(-1.06529017857143,2.00715023358964)(-1.06515066964286,2.0065410507044)(-1.06501116071429,2.00593003561573)(-1.06487165178571,2.0053171892413)(-1.06473214285714,2.00470251250237)(-1.06459263392857,2.00408600632373)(-1.064453125,2.00346767163375)(-1.06431361607143,2.00284750936437)(-1.06417410714286,2.00222552045106)(-1.06403459821429,2.00160170583286)(-1.06389508928571,2.00097606645237)(-1.06375558035714,2.00034860325572)(-1.06361607142857,1.99971931719261)(-1.0634765625,1.99908820921628)(-1.06333705357143,1.99845528028351)(-1.06319754464286,1.99782053135465)(-1.06305803571429,1.99718396339354)(-1.06291852678571,1.99654557736763)(-1.06277901785714,1.99590537424784)(-1.06263950892857,1.99526335500868)(-1.0625,1.99461952062815)(-1.06236049107143,1.99397387208783)(-1.06222098214286,1.9933264103728)(-1.06208147321429,1.99267713647166)(-1.06194196428571,1.99202605137656)(-1.06180245535714,1.99137315608318)(-1.06166294642857,1.9907184515907)(-1.0615234375,1.99006193890183)(-1.06138392857143,1.9894036190228)(-1.06124441964286,1.98874349296336)(-1.06110491071429,1.98808156173678)(-1.06096540178571,1.98741782635983)(-1.06082589285714,1.98675228785279)(-1.06068638392857,1.98608494723948)(-1.060546875,1.98541580554719)(-1.06040736607143,1.98474486380673)(-1.06026785714286,1.98407212305242)(-1.06012834821429,1.98339758432206)(-1.05998883928571,1.98272124865699)(-1.05984933035714,1.982043117102)(-1.05970982142857,1.98136319070541)(-1.0595703125,1.98068147051901)(-1.05943080357143,1.9799979575981)(-1.05929129464286,1.97931265300145)(-1.05915178571429,1.97862555779134)(-1.05901227678571,1.97793667303352)(-1.05887276785714,1.97724599979723)(-1.05873325892857,1.97655353915517)(-1.05859375,1.97585929218355)(-1.05845424107143,1.97516325996204)(-1.05831473214286,1.97446544357378)(-1.05817522321429,1.9737658441054)(-1.05803571428571,1.97306446264698)(-1.05789620535714,1.97236130029208)(-1.05775669642857,1.97165635813773)(-1.0576171875,1.97094963728441)(-1.05747767857143,1.97024113883607)(-1.05733816964286,1.96953086390013)(-1.05719866071429,1.96881881358744)(-1.05705915178571,1.96810498901234)(-1.05691964285714,1.96738939129259)(-1.05678013392857,1.96667202154943)(-1.056640625,1.96595288090752)(-1.05650111607143,1.96523197049499)(-1.05636160714286,1.96450929144342)(-1.05622209821429,1.9637848448878)(-1.05608258928571,1.96305863196658)(-1.05594308035714,1.96233065382165)(-1.05580357142857,1.96160091159834)(-1.0556640625,1.9608694064454)(-1.05552455357143,1.96013613951502)(-1.05538504464286,1.9594011119628)(-1.05524553571429,1.9586643249478)(-1.05510602678571,1.95792577963248)(-1.05496651785714,1.95718547718273)(-1.05482700892857,1.95644341876786)(-1.0546875,1.95569960556059)(-1.05454799107143,1.95495403873707)(-1.05440848214286,1.95420671947684)(-1.05426897321429,1.95345764896289)(-1.05412946428571,1.95270682838157)(-1.05398995535714,1.95195425892267)(-1.05385044642857,1.95119994177938)(-1.0537109375,1.95044387814828)(-1.05357142857143,1.94968606922934)(-1.05343191964286,1.94892651622597)(-1.05329241071429,1.94816522034491)(-1.05315290178571,1.94740218279636)(-1.05301339285714,1.94663740479386)(-1.05287388392857,1.94587088755436)(-1.052734375,1.94510263229819)(-1.05259486607143,1.94433264024906)(-1.05245535714286,1.94356091263407)(-1.05231584821429,1.94278745068367)(-1.05217633928571,1.94201225563173)(-1.05203683035714,1.94123532871546)(-1.05189732142857,1.94045667117545)(-1.0517578125,1.93967628425566)(-1.05161830357143,1.93889416920341)(-1.05147879464286,1.9381103272694)(-1.05133928571429,1.93732475970765)(-1.05119977678571,1.9365374677756)(-1.05106026785714,1.93574845273399)(-1.05092075892857,1.93495771584695)(-1.05078125,1.93416525838194)(-1.05064174107143,1.93337108160978)(-1.05050223214286,1.93257518680463)(-1.05036272321429,1.93177757524401)(-1.05022321428571,1.93097824820875)(-1.05008370535714,1.93017720698304)(-1.04994419642857,1.92937445285441)(-1.0498046875,1.92856998711371)(-1.04966517857143,1.92776381105514)(-1.04952566964286,1.9269559259762)(-1.04938616071429,1.92614633317774)(-1.04924665178571,1.92533503396394)(-1.04910714285714,1.92452202964226)(-1.04896763392857,1.92370732152353)(-1.048828125,1.92289091092187)(-1.04868861607143,1.9220727991547)(-1.04854910714286,1.92125298754279)(-1.04840959821429,1.92043147741017)(-1.04827008928571,1.91960827008422)(-1.04813058035714,1.9187833668956)(-1.04799107142857,1.91795676917827)(-1.0478515625,1.9171284782695)(-1.04771205357143,1.91629849550984)(-1.04757254464286,1.91546682224315)(-1.04743303571429,1.91463345981656)(-1.04729352678571,1.91379840958051)(-1.04715401785714,1.91296167288871)(-1.04701450892857,1.91212325109816)(-1.046875,1.91128314556912)(-1.04673549107143,1.91044135766516)(-1.04659598214286,1.9095978887531)(-1.04645647321429,1.90875274020305)(-1.04631696428571,1.90790591338836)(-1.04617745535714,1.90705740968567)(-1.04603794642857,1.90620723047488)(-1.0458984375,1.90535537713916)(-1.04575892857143,1.90450185106491)(-1.04561941964286,1.90364665364181)(-1.04547991071429,1.90278978626278)(-1.04534040178571,1.901931250324)(-1.04520089285714,1.9010710472249)(-1.04506138392857,1.90020917836815)(-1.044921875,1.89934564515964)(-1.04478236607143,1.89848044900854)(-1.04464285714286,1.89761359132723)(-1.04450334821429,1.89674507353132)(-1.04436383928571,1.89587489703968)(-1.04422433035714,1.89500306327437)(-1.04408482142857,1.89412957366071)(-1.0439453125,1.89325442962721)(-1.04380580357143,1.89237763260564)(-1.04366629464286,1.89149918403094)(-1.04352678571429,1.89061908534131)(-1.04338727678571,1.88973733797813)(-1.04324776785714,1.888853943386)(-1.04310825892857,1.88796890301272)(-1.04296875,1.88708221830931)(-1.04282924107143,1.88619389072998)(-1.04268973214286,1.88530392173212)(-1.04255022321429,1.88441231277634)(-1.04241071428571,1.88351906532643)(-1.04227120535714,1.88262418084938)(-1.04213169642857,1.88172766081535)(-1.0419921875,1.88082950669769)(-1.04185267857143,1.87992971997293)(-1.04171316964286,1.87902830212079)(-1.04157366071429,1.87812525462414)(-1.04143415178571,1.87722057896905)(-1.04129464285714,1.87631427664473)(-1.04115513392857,1.87540634914357)(-1.041015625,1.87449679796113)(-1.04087611607143,1.87358562459612)(-1.04073660714286,1.87267283055042)(-1.04059709821429,1.87175841732904)(-1.04045758928571,1.87084238644016)(-1.04031808035714,1.8699247393951)(-1.04017857142857,1.86900547770834)(-1.0400390625,1.86808460289749)(-1.03989955357143,1.8671621164833)(-1.03976004464286,1.86623801998966)(-1.03962053571429,1.86531231494359)(-1.03948102678571,1.86438500287524)(-1.03934151785714,1.8634560853179)(-1.03920200892857,1.86252556380796)(-1.0390625,1.86159343988497)(-1.03892299107143,1.86065971509155)(-1.03878348214286,1.85972439097348)(-1.03864397321429,1.85878746907963)(-1.03850446428571,1.85784895096198)(-1.03836495535714,1.85690883817562)(-1.03822544642857,1.85596713227875)(-1.0380859375,1.85502383483266)(-1.03794642857143,1.85407894740175)(-1.03780691964286,1.8531324715535)(-1.03766741071429,1.85218440885849)(-1.03752790178571,1.85123476089039)(-1.03738839285714,1.85028352922594)(-1.03724888392857,1.84933071544499)(-1.037109375,1.84837632113045)(-1.03696986607143,1.84742034786831)(-1.03683035714286,1.84646279724762)(-1.03669084821429,1.84550367086054)(-1.03655133928571,1.84454297030225)(-1.03641183035714,1.84358069717101)(-1.03627232142857,1.84261685306816)(-1.0361328125,1.84165143959807)(-1.03599330357143,1.84068445836819)(-1.03585379464286,1.83971591098899)(-1.03571428571429,1.83874579907402)(-1.03557477678571,1.83777412423985)(-1.03543526785714,1.83680088810611)(-1.03529575892857,1.83582609229547)(-1.03515625,1.8348497384336)(-1.03501674107143,1.83387182814926)(-1.03487723214286,1.83289236307418)(-1.03473772321429,1.83191134484317)(-1.03459821428571,1.83092877509403)(-1.03445870535714,1.82994465546757)(-1.03431919642857,1.82895898760766)(-1.0341796875,1.82797177316114)(-1.03404017857143,1.82698301377788)(-1.03390066964286,1.82599271111075)(-1.03376116071429,1.82500086681564)(-1.03362165178571,1.82400748255142)(-1.03348214285714,1.82301255997997)(-1.03334263392857,1.82201610076615)(-1.033203125,1.82101810657783)(-1.03306361607143,1.82001857908585)(-1.03292410714286,1.81901751996404)(-1.03278459821429,1.81801493088922)(-1.03264508928571,1.81701081354118)(-1.03250558035714,1.81600516960267)(-1.03236607142857,1.81499800075945)(-1.0322265625,1.81398930870021)(-1.03208705357143,1.81297909511662)(-1.03194754464286,1.8119673617033)(-1.03180803571429,1.81095411015786)(-1.03166852678571,1.80993934218082)(-1.03152901785714,1.80892305947568)(-1.03138950892857,1.8079052637489)(-1.03125,1.80688595670984)(-1.03111049107143,1.80586514007084)(-1.03097098214286,1.80484281554717)(-1.03083147321429,1.80381898485702)(-1.03069196428571,1.80279364972154)(-1.03055245535714,1.80176681186477)(-1.03041294642857,1.8007384730137)(-1.0302734375,1.79970863489825)(-1.03013392857143,1.79867729925123)(-1.02999441964286,1.79764446780839)(-1.02985491071429,1.79661014230837)(-1.02971540178571,1.79557432449273)(-1.02957589285714,1.79453701610594)(-1.02943638392857,1.79349821889535)(-1.029296875,1.79245793461124)(-1.02915736607143,1.79141616500675)(-1.02901785714286,1.79037291183793)(-1.02887834821429,1.78932817686371)(-1.02873883928571,1.78828196184592)(-1.02859933035714,1.78723426854925)(-1.02845982142857,1.78618509874128)(-1.0283203125,1.78513445419245)(-1.02818080357143,1.78408233667608)(-1.02804129464286,1.78302874796837)(-1.02790178571429,1.78197368984836)(-1.02776227678571,1.78091716409796)(-1.02762276785714,1.77985917250193)(-1.02748325892857,1.7787997168479)(-1.02734375,1.77773879892633)(-1.02720424107143,1.77667642053054)(-1.02706473214286,1.77561258345668)(-1.02692522321429,1.77454728950375)(-1.02678571428571,1.77348054047359)(-1.02664620535714,1.77241233817084)(-1.02650669642857,1.77134268440301)(-1.0263671875,1.77027158098041)(-1.02622767857143,1.76919902971619)(-1.02608816964286,1.76812503242628)(-1.02594866071429,1.76704959092948)(-1.02580915178571,1.76597270704735)(-1.02566964285714,1.76489438260428)(-1.02553013392857,1.76381461942748)(-1.025390625,1.76273341934692)(-1.02525111607143,1.7616507841954)(-1.02511160714286,1.7605667158085)(-1.02497209821429,1.75948121602459)(-1.02483258928571,1.75839428668483)(-1.02469308035714,1.75730592963316)(-1.02455357142857,1.75621614671629)(-1.0244140625,1.75512493978372)(-1.02427455357143,1.75403231068772)(-1.02413504464286,1.75293826128332)(-1.02399553571429,1.75184279342831)(-1.02385602678571,1.75074590898325)(-1.02371651785714,1.74964760981146)(-1.02357700892857,1.74854789777901)(-1.0234375,1.74744677475471)(-1.02329799107143,1.74634424261012)(-1.02315848214286,1.74524030321957)(-1.02301897321429,1.74413495846008)(-1.02287946428571,1.74302821021144)(-1.02273995535714,1.74192006035617)(-1.02260044642857,1.7408105107795)(-1.0224609375,1.73969956336941)(-1.02232142857143,1.73858722001656)(-1.02218191964286,1.73747348261438)(-1.02204241071429,1.73635835305896)(-1.02190290178571,1.73524183324914)(-1.02176339285714,1.73412392508645)(-1.02162388392857,1.73300463047512)(-1.021484375,1.73188395132208)(-1.02134486607143,1.73076188953696)(-1.02120535714286,1.72963844703208)(-1.02106584821429,1.72851362572243)(-1.02092633928571,1.72738742752571)(-1.02078683035714,1.72625985436228)(-1.02064732142857,1.72513090815519)(-1.0205078125,1.72400059083015)(-1.02036830357143,1.72286890431555)(-1.02022879464286,1.72173585054243)(-1.02008928571429,1.72060143144452)(-1.01994977678571,1.71946564895816)(-1.01981026785714,1.7183285050224)(-1.01967075892857,1.71719000157889)(-1.01953125,1.71605014057195)(-1.01939174107143,1.71490892394855)(-1.01925223214286,1.71376635365827)(-1.01911272321429,1.71262243165336)(-1.01897321428571,1.71147715988867)(-1.01883370535714,1.7103305403217)(-1.01869419642857,1.70918257491255)(-1.0185546875,1.70803326562397)(-1.01841517857143,1.70688261442129)(-1.01827566964286,1.70573062327248)(-1.01813616071429,1.70457729414811)(-1.01799665178571,1.70342262902134)(-1.01785714285714,1.70226662986795)(-1.01771763392857,1.70110929866631)(-1.017578125,1.69995063739738)(-1.01743861607143,1.6987906480447)(-1.01729910714286,1.69762933259442)(-1.01715959821429,1.69646669303525)(-1.01702008928571,1.69530273135847)(-1.01688058035714,1.69413744955796)(-1.01674107142857,1.69297084963015)(-1.0166015625,1.69180293357404)(-1.01646205357143,1.69063370339119)(-1.01632254464286,1.68946316108573)(-1.01618303571429,1.68829130866431)(-1.01604352678571,1.68711814813618)(-1.01590401785714,1.68594368151309)(-1.01576450892857,1.68476791080936)(-1.015625,1.68359083804184)(-1.01548549107143,1.68241246522991)(-1.01534598214286,1.68123279439548)(-1.01520647321429,1.680051827563)(-1.01506696428571,1.67886956675942)(-1.01492745535714,1.67768601401423)(-1.01478794642857,1.67650117135943)(-1.0146484375,1.6753150408295)(-1.01450892857143,1.67412762446148)(-1.01436941964286,1.67293892429488)(-1.01422991071429,1.6717489423717)(-1.01409040178571,1.67055768073646)(-1.01395089285714,1.66936514143615)(-1.01381138392857,1.66817132652026)(-1.013671875,1.66697623804076)(-1.01353236607143,1.6657798780521)(-1.01339285714286,1.66458224861119)(-1.01325334821429,1.66338335177743)(-1.01311383928571,1.66218318961268)(-1.01297433035714,1.66098176418126)(-1.01283482142857,1.65977907754995)(-1.0126953125,1.65857513178798)(-1.01255580357143,1.65736992896705)(-1.01241629464286,1.65616347116127)(-1.01227678571429,1.65495576044723)(-1.01213727678571,1.65374679890395)(-1.01199776785714,1.65253658861285)(-1.01185825892857,1.65132513165784)(-1.01171875,1.65011243012521)(-1.01157924107143,1.64889848610368)(-1.01143973214286,1.64768330168441)(-1.01130022321429,1.64646687896094)(-1.01116071428571,1.64524922002925)(-1.01102120535714,1.64403032698772)(-1.01088169642857,1.64281020193712)(-1.0107421875,1.64158884698063)(-1.01060267857143,1.64036626422381)(-1.01046316964286,1.63914245577463)(-1.01032366071429,1.63791742374342)(-1.01018415178571,1.63669117024292)(-1.01004464285714,1.63546369738821)(-1.00990513392857,1.6342350072968)(-1.009765625,1.6330051020885)(-1.00962611607143,1.63177398388554)(-1.00948660714286,1.63054165481249)(-1.00934709821429,1.62930811699627)(-1.00920758928571,1.62807337256616)(-1.00906808035714,1.62683742365379)(-1.00892857142857,1.62560027239314)(-1.0087890625,1.62436192092051)(-1.00864955357143,1.62312237137455)(-1.00851004464286,1.62188162589625)(-1.00837053571429,1.62063968662891)(-1.00823102678571,1.61939655571817)(-1.00809151785714,1.61815223531196)(-1.00795200892857,1.61690672756057)(-1.0078125,1.61566003461656)(-1.00767299107143,1.61441215863483)(-1.00753348214286,1.61316310177255)(-1.00739397321429,1.6119128661892)(-1.00725446428571,1.61066145404658)(-1.00711495535714,1.60940886750874)(-1.00697544642857,1.60815510874204)(-1.0068359375,1.60690017991511)(-1.00669642857143,1.60564408319887)(-1.00655691964286,1.60438682076651)(-1.00641741071429,1.60312839479347)(-1.00627790178571,1.60186880745747)(-1.00613839285714,1.6006080609385)(-1.00599888392857,1.5993461574188)(-1.005859375,1.59808309908283)(-1.00571986607143,1.59681888811736)(-1.00558035714286,1.59555352671134)(-1.00544084821429,1.594287017056)(-1.00530133928571,1.59301936134479)(-1.00516183035714,1.59175056177339)(-1.00502232142857,1.59048062053971)(-1.0048828125,1.58920953984389)(-1.00474330357143,1.58793732188826)(-1.00460379464286,1.5866639688774)(-1.00446428571429,1.58538948301807)(-1.00432477678571,1.58411386651925)(-1.00418526785714,1.58283712159211)(-1.00404575892857,1.58155925045003)(-1.00390625,1.58028025530858)(-1.00376674107143,1.5790001383855)(-1.00362723214286,1.57771890190073)(-1.00348772321429,1.57643654807638)(-1.00334821428571,1.57515307913675)(-1.00320870535714,1.57386849730829)(-1.00306919642857,1.57258280481964)(-1.0029296875,1.57129600390156)(-1.00279017857143,1.57000809678702)(-1.00265066964286,1.5687190857111)(-1.00251116071429,1.56742897291106)(-1.00237165178571,1.56613776062628)(-1.00223214285714,1.56484545109829)(-1.00209263392857,1.56355204657076)(-1.001953125,1.56225754928949)(-1.00181361607143,1.56096196150239)(-1.00167410714286,1.55966528545953)(-1.00153459821429,1.55836752341305)(-1.00139508928571,1.55706867761724)(-1.00125558035714,1.55576875032848)(-1.00111607142857,1.55446774380527)(-1.0009765625,1.55316566030819)(-1.00083705357143,1.55186250209995)(-1.00069754464286,1.55055827144531)(-1.00055803571429,1.54925297061114)(-1.00041852678571,1.5479466018664)(-1.00027901785714,1.54663916748211)(-1.00013950892857,1.54533066973138)(-1,1.54402111088937)(-0.999860491071429,1.54271049323333)(-0.999720982142857,1.54139881904256)(-0.999581473214286,1.5400860905984)(-0.999441964285714,1.53877231018426)(-0.999302455357143,1.53745748008561)(-0.999162946428571,1.53614160258993)(-0.9990234375,1.53482467998676)(-0.998883928571429,1.53350671456767)(-0.998744419642857,1.53218770862627)(-0.998604910714286,1.53086766445816)(-0.998465401785714,1.529546584361)(-0.998325892857143,1.52822447063445)(-0.998186383928571,1.52690132558018)(-0.998046875,1.52557715150188)(-0.997907366071429,1.52425195070522)(-0.997767857142857,1.52292572549788)(-0.997628348214286,1.52159847818954)(-0.997488839285714,1.52027021109187)(-0.997349330357143,1.51894092651852)(-0.997209821428571,1.5176106267851)(-0.9970703125,1.51627931420923)(-0.996930803571429,1.51494699111049)(-0.996791294642857,1.51361365981041)(-0.996651785714286,1.5122793226325)(-0.996512276785714,1.51094398190223)(-0.996372767857143,1.50960763994701)(-0.996233258928571,1.5082702990962)(-0.99609375,1.50693196168112)(-0.995954241071429,1.50559263003501)(-0.995814732142857,1.50425230649306)(-0.995675223214286,1.50291099339238)(-0.995535714285714,1.50156869307201)(-0.995396205357143,1.50022540787291)(-0.995256696428571,1.49888114013795)(-0.9951171875,1.49753589221195)(-0.994977678571429,1.49618966644158)(-0.994838169642857,1.49484246517545)(-0.994698660714286,1.49349429076405)(-0.994559151785714,1.4921451455598)(-0.994419642857143,1.49079503191696)(-0.994280133928571,1.4894439521917)(-0.994140625,1.48809190874208)(-0.994001116071429,1.48673890392802)(-0.993861607142857,1.48538494011132)(-0.993722098214286,1.48403001965562)(-0.993582589285714,1.48267414492647)(-0.993443080357143,1.48131731829123)(-0.993303571428571,1.47995954211915)(-0.9931640625,1.47860081878131)(-0.993024553571429,1.47724115065062)(-0.992885044642857,1.47588054010186)(-0.992745535714286,1.47451898951162)(-0.992606026785714,1.47315650125833)(-0.992466517857143,1.47179307772224)(-0.992327008928571,1.47042872128542)(-0.9921875,1.46906343433177)(-0.992047991071429,1.46769721924699)(-0.991908482142857,1.46633007841857)(-0.991768973214286,1.46496201423583)(-0.991629464285714,1.46359302908988)(-0.991489955357143,1.4622231253736)(-0.991350446428571,1.46085230548168)(-0.9912109375,1.4594805718106)(-0.991071428571429,1.4581079267586)(-0.990931919642857,1.45673437272571)(-0.990792410714286,1.45535991211369)(-0.990652901785714,1.45398454732613)(-0.990513392857143,1.45260828076832)(-0.990373883928571,1.45123111484734)(-0.990234375,1.449853051972)(-0.990094866071429,1.44847409455288)(-0.989955357142857,1.44709424500227)(-0.989815848214286,1.44571350573422)(-0.989676339285714,1.44433187916451)(-0.989536830357143,1.44294936771063)(-0.989397321428571,1.44156597379181)(-0.9892578125,1.44018169982899)(-0.989118303571429,1.43879654824482)(-0.988978794642857,1.43741052146367)(-0.988839285714286,1.43602362191159)(-0.988699776785714,1.43463585201636)(-0.988560267857143,1.43324721420743)(-0.988420758928571,1.43185771091594)(-0.98828125,1.43046734457474)(-0.988141741071429,1.42907611761832)(-0.988002232142857,1.42768403248288)(-0.987862723214286,1.42629109160627)(-0.987723214285714,1.42489729742802)(-0.987583705357143,1.4235026523893)(-0.987444196428571,1.42210715893296)(-0.9873046875,1.42071081950348)(-0.987165178571429,1.419313636547)(-0.987025669642857,1.4179156125113)(-0.986886160714286,1.41651674984579)(-0.986746651785714,1.41511705100153)(-0.986607142857143,1.41371651843118)(-0.986467633928571,1.41231515458904)(-0.986328125,1.41091296193103)(-0.986188616071429,1.40950994291468)(-0.986049107142857,1.40810609999912)(-0.985909598214286,1.40670143564509)(-0.985770089285714,1.40529595231494)(-0.985630580357143,1.40388965247259)(-0.985491071428571,1.40248253858356)(-0.9853515625,1.40107461311497)(-0.985212053571429,1.3996658785355)(-0.985072544642857,1.39825633731541)(-0.984933035714286,1.39684599192653)(-0.984793526785714,1.39543484484225)(-0.984654017857143,1.39402289853753)(-0.984514508928571,1.39261015548889)(-0.984375,1.39119661817438)(-0.984235491071429,1.38978228907362)(-0.984095982142857,1.38836717066775)(-0.983956473214286,1.38695126543946)(-0.983816964285714,1.38553457587296)(-0.983677455357143,1.384117104454)(-0.983537946428571,1.38269885366984)(-0.9833984375,1.38127982600927)(-0.983258928571429,1.37986002396257)(-0.983119419642857,1.37843945002156)(-0.982979910714286,1.37701810667952)(-0.982840401785714,1.37559599643127)(-0.982700892857143,1.3741731217731)(-0.982561383928571,1.37274948520278)(-0.982421875,1.37132508921959)(-0.982282366071429,1.36989993632426)(-0.982142857142857,1.36847402901901)(-0.982003348214286,1.36704736980752)(-0.981863839285714,1.36561996119494)(-0.981724330357143,1.36419180568787)(-0.981584821428571,1.36276290579438)(-0.9814453125,1.36133326402397)(-0.981305803571429,1.35990288288759)(-0.981166294642857,1.35847176489764)(-0.981026785714286,1.35703991256793)(-0.980887276785714,1.35560732841373)(-0.980747767857143,1.35417401495171)(-0.980608258928571,1.35273997469997)(-0.98046875,1.35130521017802)(-0.980329241071429,1.34986972390678)(-0.980189732142857,1.34843351840859)(-0.980050223214286,1.34699659620716)(-0.979910714285714,1.34555895982762)(-0.979771205357143,1.34412061179648)(-0.979631696428571,1.34268155464163)(-0.9794921875,1.34124179089236)(-0.979352678571429,1.33980132307932)(-0.979213169642857,1.33836015373452)(-0.979073660714286,1.33691828539136)(-0.978934151785714,1.33547572058458)(-0.978794642857143,1.33403246185028)(-0.978655133928571,1.33258851172592)(-0.978515625,1.33114387275031)(-0.978376116071429,1.32969854746358)(-0.978236607142857,1.3282525384072)(-0.978097098214286,1.32680584812399)(-0.977957589285714,1.32535847915807)(-0.977818080357143,1.3239104340549)(-0.977678571428571,1.32246171536125)(-0.9775390625,1.32101232562521)(-0.977399553571429,1.31956226739615)(-0.977260044642857,1.31811154322476)(-0.977120535714286,1.31666015566304)(-0.976981026785714,1.31520810726424)(-0.976841517857143,1.31375540058295)(-0.976702008928571,1.31230203817499)(-0.9765625,1.31084802259748)(-0.976422991071429,1.30939335640882)(-0.976283482142857,1.30793804216867)(-0.976143973214286,1.30648208243793)(-0.976004464285714,1.30502547977879)(-0.975864955357143,1.30356823675466)(-0.975725446428571,1.30211035593022)(-0.9755859375,1.30065183987139)(-0.975446428571429,1.29919269114531)(-0.975306919642857,1.29773291232036)(-0.975167410714286,1.29627250596614)(-0.975027901785714,1.2948114746535)(-0.974888392857143,1.29334982095446)(-0.974748883928571,1.29188754744229)(-0.974609375,1.29042465669145)(-0.974469866071429,1.2889611512776)(-0.974330357142857,1.2874970337776)(-0.974190848214286,1.2860323067695)(-0.974051339285714,1.28456697283255)(-0.973911830357143,1.28310103454715)(-0.973772321428571,1.2816344944949)(-0.9736328125,1.28016735525858)(-0.973493303571429,1.2786996194221)(-0.973353794642857,1.27723128957057)(-0.973214285714286,1.27576236829023)(-0.973074776785714,1.27429285816849)(-0.972935267857143,1.27282276179388)(-0.972795758928571,1.2713520817561)(-0.97265625,1.26988082064598)(-0.972516741071429,1.26840898105546)(-0.972377232142857,1.26693656557763)(-0.972237723214286,1.26546357680668)(-0.972098214285714,1.26399001733794)(-0.971958705357143,1.26251588976783)(-0.971819196428571,1.26104119669388)(-0.9716796875,1.25956594071473)(-0.971540178571429,1.25809012443011)(-0.971400669642857,1.25661375044084)(-0.971261160714286,1.25513682134882)(-0.971121651785714,1.25365933975704)(-0.970982142857143,1.25218130826956)(-0.970842633928571,1.25070272949151)(-0.970703125,1.24922360602909)(-0.970563616071429,1.24774394048956)(-0.970424107142857,1.24626373548123)(-0.970284598214286,1.24478299361345)(-0.970145089285714,1.24330171749664)(-0.970005580357143,1.24181990974224)(-0.969866071428571,1.24033757296272)(-0.9697265625,1.23885470977162)(-0.969587053571429,1.23737132278345)(-0.969447544642857,1.23588741461378)(-0.969308035714286,1.23440298787917)(-0.969168526785714,1.23291804519719)(-0.969029017857143,1.23143258918644)(-0.968889508928571,1.2299466224665)(-0.96875,1.22846014765793)(-0.968610491071429,1.22697316738231)(-0.968470982142857,1.22548568426218)(-0.968331473214286,1.22399770092107)(-0.968191964285714,1.22250921998348)(-0.968052455357143,1.22102024407488)(-0.967912946428571,1.2195307758217)(-0.9677734375,1.21804081785133)(-0.967633928571429,1.21655037279211)(-0.967494419642857,1.21505944327335)(-0.967354910714286,1.21356803192526)(-0.967215401785714,1.21207614137902)(-0.967075892857143,1.21058377426675)(-0.966936383928571,1.20909093322146)(-0.966796875,1.20759762087712)(-0.966657366071429,1.20610383986859)(-0.966517857142857,1.20460959283167)(-0.966378348214286,1.20311488240304)(-0.966238839285714,1.2016197112203)(-0.966099330357143,1.20012408192193)(-0.965959821428571,1.19862799714732)(-0.9658203125,1.19713145953674)(-0.965680803571429,1.19563447173133)(-0.965541294642857,1.19413703637312)(-0.965401785714286,1.19263915610501)(-0.965262276785714,1.19114083357076)(-0.965122767857143,1.18964207141499)(-0.964983258928571,1.18814287228319)(-0.96484375,1.18664323882167)(-0.964704241071429,1.18514317367762)(-0.964564732142857,1.18364267949906)(-0.964425223214286,1.18214175893482)(-0.964285714285714,1.1806404146346)(-0.964146205357143,1.17913864924889)(-0.964006696428571,1.17763646542901)(-0.9638671875,1.17613386582712)(-0.963727678571429,1.17463085309615)(-0.963588169642857,1.17312742988986)(-0.963448660714286,1.1716235988628)(-0.963309151785714,1.17011936267031)(-0.963169642857143,1.16861472396853)(-0.963030133928571,1.16710968541436)(-0.962890625,1.16560424966551)(-0.962751116071429,1.16409841938045)(-0.962611607142857,1.1625921972184)(-0.962472098214286,1.16108558583938)(-0.962332589285714,1.15957858790412)(-0.962193080357143,1.15807120607415)(-0.962053571428571,1.15656344301171)(-0.9619140625,1.1550553013798)(-0.961774553571429,1.15354678384216)(-0.961635044642857,1.15203789306326)(-0.961495535714286,1.15052863170828)(-0.961356026785714,1.14901900244314)(-0.961216517857143,1.14750900793446)(-0.961077008928571,1.14599865084959)(-0.9609375,1.14448793385658)(-0.960797991071429,1.14297685962416)(-0.960658482142857,1.14146543082179)(-0.960518973214286,1.13995365011959)(-0.960379464285714,1.13844152018838)(-0.960239955357143,1.13692904369966)(-0.960100446428571,1.13541622332559)(-0.9599609375,1.13390306173901)(-0.959821428571429,1.13238956161345)(-0.959681919642857,1.13087572562304)(-0.959542410714286,1.12936155644261)(-0.959402901785714,1.12784705674763)(-0.959263392857143,1.12633222921421)(-0.959123883928571,1.12481707651908)(-0.958984375,1.12330160133964)(-0.958844866071429,1.12178580635389)(-0.958705357142857,1.12026969424045)(-0.958565848214286,1.11875326767857)(-0.958426339285714,1.11723652934812)(-0.958286830357143,1.11571948192956)(-0.958147321428571,1.11420212810395)(-0.9580078125,1.11268447055296)(-0.957868303571429,1.11116651195884)(-0.957728794642857,1.10964825500444)(-0.957589285714286,1.10812970237317)(-0.957449776785714,1.10661085674903)(-0.957310267857143,1.10509172081659)(-0.957170758928571,1.10357229726098)(-0.95703125,1.10205258876789)(-0.956891741071429,1.10053259802357)(-0.956752232142857,1.09901232771481)(-0.956612723214286,1.09749178052896)(-0.956473214285714,1.09597095915389)(-0.956333705357143,1.09444986627801)(-0.956194196428571,1.09292850459027)(-0.9560546875,1.09140687678012)(-0.955915178571429,1.08988498553756)(-0.955775669642857,1.08836283355307)(-0.955636160714286,1.08684042351766)(-0.955496651785714,1.08531775812283)(-0.955357142857143,1.08379484006058)(-0.955217633928571,1.0822716720234)(-0.955078125,1.08074825670428)(-0.954938616071429,1.07922459679668)(-0.954799107142857,1.07770069499453)(-0.954659598214286,1.07617655399223)(-0.954520089285714,1.07465217648467)(-0.954380580357143,1.07312756516718)(-0.954241071428571,1.07160272273554)(-0.9541015625,1.07007765188599)(-0.953962053571429,1.06855235531522)(-0.953822544642857,1.06702683572035)(-0.953683035714286,1.06550109579893)(-0.953543526785714,1.06397513824896)(-0.953404017857143,1.06244896576883)(-0.953264508928571,1.06092258105737)(-0.953125,1.05939598681383)(-0.952985491071429,1.05786918573785)(-0.952845982142857,1.05634218052949)(-0.952706473214286,1.05481497388918)(-0.952566964285714,1.05328756851778)(-0.952427455357143,1.0517599671165)(-0.952287946428571,1.05023217238696)(-0.9521484375,1.04870418703114)(-0.952008928571429,1.04717601375139)(-0.951869419642857,1.04564765525044)(-0.951729910714286,1.04411911423136)(-0.951590401785714,1.0425903933976)(-0.951450892857143,1.04106149545294)(-0.951311383928571,1.03953242310151)(-0.951171875,1.03800317904778)(-0.951032366071429,1.03647376599657)(-0.950892857142857,1.03494418665299)(-0.950753348214286,1.03341444372251)(-0.950613839285714,1.03188453991091)(-0.950474330357143,1.03035447792428)(-0.950334821428571,1.028824260469)(-0.9501953125,1.02729389025179)(-0.950055803571429,1.02576336997963)(-0.949916294642857,1.02423270235983)(-0.949776785714286,1.02270189009994)(-0.949637276785714,1.02117093590783)(-0.949497767857143,1.01963984249164)(-0.949358258928571,1.01810861255977)(-0.94921875,1.01657724882088)(-0.949079241071429,1.01504575398391)(-0.948939732142857,1.01351413075805)(-0.948800223214286,1.01198238185273)(-0.948660714285714,1.01045050997763)(-0.948521205357143,1.00891851784266)(-0.948381696428571,1.00738640815799)(-0.9482421875,1.00585418363399)(-0.948102678571429,1.00432184698128)(-0.947963169642857,1.00278940091067)(-0.947823660714286,1.00125684813321)(-0.947684151785714,0.999724191360131)(-0.947544642857143,0.998191433302888)(-0.947405133928571,0.99665857667312)(-0.947265625,0.99512562418267)(-0.947126116071429,0.993592578543559)(-0.946986607142857,0.992059442467988)(-0.946847098214286,0.990526218668338)(-0.946707589285714,0.988992909857161)(-0.946568080357143,0.987459518747173)(-0.946428571428571,0.985926048051246)(-0.9462890625,0.984392500482419)(-0.946149553571429,0.982858878753869)(-0.946010044642857,0.981325185578921)(-0.945870535714286,0.979791423671038)(-0.945731026785714,0.978257595743821)(-0.945591517857143,0.976723704510995)(-0.945452008928571,0.975189752686406)(-0.9453125,0.97365574298403)(-0.945172991071429,0.972121678117942)(-0.945033482142857,0.970587560802331)(-0.944893973214286,0.969053393751484)(-0.944754464285714,0.967519179679792)(-0.944614955357143,0.96598492130173)(-0.944475446428571,0.96445062133186)(-0.9443359375,0.962916282484834)(-0.944196428571429,0.96138190747537)(-0.944056919642857,0.959847499018258)(-0.943917410714286,0.958313059828354)(-0.943777901785714,0.956778592620576)(-0.943638392857143,0.955244100109895)(-0.943498883928571,0.953709585011326)(-0.943359375,0.952175050039941)(-0.943219866071429,0.950640497910838)(-0.943080357142857,0.949105931339154)(-0.942940848214286,0.947571353040051)(-0.942801339285714,0.94603676572872)(-0.942661830357143,0.944502172120363)(-0.942522321428571,0.942967574930193)(-0.9423828125,0.941432976873442)(-0.942243303571429,0.939898380665331)(-0.942103794642857,0.938363789021082)(-0.941964285714286,0.936829204655906)(-0.941824776785714,0.935294630285006)(-0.941685267857143,0.933760068623559)(-0.941545758928571,0.932225522386715)(-0.94140625,0.930690994289608)(-0.941266741071429,0.929156487047322)(-0.941127232142857,0.927622003374906)(-0.940987723214286,0.926087545987361)(-0.940848214285714,0.924553117599642)(-0.940708705357143,0.923018720926643)(-0.940569196428571,0.921484358683194)(-0.9404296875,0.919950033584068)(-0.940290178571429,0.918415748343956)(-0.940150669642857,0.916881505677475)(-0.940011160714286,0.915347308299156)(-0.939871651785714,0.913813158923448)(-0.939732142857143,0.912279060264702)(-0.939592633928571,0.910745015037168)(-0.939453125,0.909211025955002)(-0.939313616071429,0.907677095732242)(-0.939174107142857,0.906143227082812)(-0.939034598214286,0.904609422720517)(-0.938895089285714,0.90307568535904)(-0.938755580357143,0.901542017711931)(-0.938616071428571,0.9000084224926)(-0.9384765625,0.898474902414329)(-0.938337053571429,0.89694146019024)(-0.938197544642857,0.895408098533311)(-0.938058035714286,0.893874820156357)(-0.937918526785714,0.89234162777204)(-0.937779017857143,0.890808524092847)(-0.937639508928571,0.889275511831092)(-0.9375,0.887742593698921)(-0.937360491071429,0.886209772408287)(-0.937220982142857,0.884677050670956)(-0.937081473214286,0.8831444311985)(-0.936941964285714,0.881611916702298)(-0.936802455357143,0.880079509893517)(-0.936662946428571,0.878547213483116)(-0.9365234375,0.877015030181846)(-0.936383928571429,0.875482962700231)(-0.936244419642857,0.87395101374857)(-0.936104910714286,0.872419186036931)(-0.935965401785714,0.870887482275151)(-0.935825892857143,0.86935590517282)(-0.935686383928571,0.867824457439282)(-0.935546875,0.866293141783636)(-0.935407366071429,0.864761960914717)(-0.935267857142857,0.8632309175411)(-0.935128348214286,0.861700014371088)(-0.934988839285714,0.860169254112721)(-0.934849330357143,0.858638639473752)(-0.934709821428571,0.85710817316165)(-0.9345703125,0.855577857883607)(-0.934430803571429,0.854047696346508)(-0.934291294642857,0.852517691256945)(-0.934151785714286,0.850987845321202)(-0.934012276785714,0.849458161245259)(-0.933872767857143,0.847928641734775)(-0.933733258928571,0.846399289495089)(-0.93359375,0.844870107231223)(-0.933454241071429,0.843341097647859)(-0.933314732142857,0.841812263449344)(-0.933175223214286,0.840283607339685)(-0.933035714285714,0.838755132022546)(-0.932896205357143,0.837226840201234)(-0.932756696428571,0.835698734578697)(-0.9326171875,0.834170817857532)(-0.932477678571429,0.832643092739957)(-0.932338169642857,0.831115561927821)(-0.932198660714286,0.829588228122592)(-0.932059151785714,0.828061094025361)(-0.931919642857143,0.826534162336826)(-0.931780133928571,0.825007435757286)(-0.931640625,0.823480916986655)(-0.931501116071429,0.82195460872443)(-0.931361607142857,0.820428513669702)(-0.931222098214286,0.818902634521145)(-0.931082589285714,0.81737697397702)(-0.930943080357143,0.815851534735154)(-0.930803571428571,0.814326319492945)(-0.9306640625,0.812801330947362)(-0.930524553571429,0.811276571794927)(-0.930385044642857,0.809752044731714)(-0.930245535714286,0.808227752453346)(-0.930106026785714,0.806703697654994)(-0.929966517857143,0.805179883031363)(-0.929827008928571,0.803656311276685)(-0.9296875,0.802132985084735)(-0.929547991071429,0.800609907148795)(-0.929408482142857,0.799087080161669)(-0.929268973214286,0.797564506815672)(-0.929129464285714,0.796042189802629)(-0.928989955357143,0.794520131813861)(-0.928850446428571,0.792998335540184)(-0.9287109375,0.791476803671914)(-0.928571428571429,0.789955538898844)(-0.928431919642857,0.788434543910248)(-0.928292410714286,0.786913821394874)(-0.928152901785714,0.785393374040947)(-0.928013392857143,0.783873204536148)(-0.927873883928571,0.782353315567617)(-0.927734375,0.780833709821959)(-0.927594866071429,0.779314389985217)(-0.927455357142857,0.777795358742881)(-0.927315848214286,0.776276618779877)(-0.927176339285714,0.774758172780569)(-0.927036830357143,0.773240023428745)(-0.926897321428571,0.771722173407614)(-0.9267578125,0.770204625399811)(-0.926618303571429,0.768687382087374)(-0.926478794642857,0.767170446151752)(-0.926339285714286,0.765653820273793)(-0.926199776785714,0.764137507133747)(-0.926060267857143,0.76262150941125)(-0.925920758928571,0.761105829785323)(-0.92578125,0.759590470934378)(-0.925641741071429,0.758075435536192)(-0.925502232142857,0.756560726267917)(-0.925362723214286,0.755046345806065)(-0.925223214285714,0.753532296826519)(-0.925083705357143,0.752018582004507)(-0.924944196428571,0.750505204014606)(-0.9248046875,0.74899216553075)(-0.924665178571429,0.747479469226199)(-0.924525669642857,0.745967117773551)(-0.924386160714286,0.744455113844733)(-0.924246651785714,0.742943460110999)(-0.924107142857143,0.741432159242917)(-0.923967633928571,0.739921213910365)(-0.923828125,0.738410626782542)(-0.923688616071429,0.736900400527938)(-0.923549107142857,0.735390537814344)(-0.923409598214286,0.733881041308842)(-0.923270089285714,0.732371913677806)(-0.923130580357143,0.730863157586887)(-0.922991071428571,0.729354775701013)(-0.9228515625,0.727846770684391)(-0.922712053571429,0.726339145200486)(-0.922572544642857,0.724831901912027)(-0.922433035714286,0.723325043480999)(-0.922293526785714,0.721818572568641)(-0.922154017857143,0.720312491835434)(-0.922014508928571,0.718806803941097)(-0.921875,0.717301511544596)(-0.921735491071429,0.715796617304116)(-0.921595982142857,0.714292123877069)(-0.921456473214286,0.712788033920089)(-0.921316964285714,0.711284350089027)(-0.921177455357143,0.709781075038939)(-0.921037946428571,0.708278211424083)(-0.9208984375,0.706775761897929)(-0.920758928571429,0.705273729113129)(-0.920619419642857,0.703772115721526)(-0.920479910714286,0.702270924374147)(-0.920340401785714,0.700770157721204)(-0.920200892857143,0.699269818412073)(-0.920061383928571,0.6977699090953)(-0.919921875,0.696270432418605)(-0.919782366071429,0.694771391028852)(-0.919642857142857,0.693272787572065)(-0.919503348214286,0.691774624693411)(-0.919363839285714,0.690276905037207)(-0.919224330357143,0.688779631246901)(-0.919084821428571,0.687282805965071)(-0.9189453125,0.685786431833435)(-0.918805803571429,0.684290511492818)(-0.918666294642857,0.682795047583171)(-0.918526785714286,0.68130004274355)(-0.918387276785714,0.679805499612125)(-0.918247767857143,0.678311420826162)(-0.918108258928571,0.676817809022021)(-0.91796875,0.675324666835164)(-0.917829241071429,0.673831996900129)(-0.917689732142857,0.672339801850538)(-0.917550223214286,0.670848084319086)(-0.917410714285714,0.669356846937544)(-0.917271205357143,0.667866092336746)(-0.917131696428571,0.666375823146581)(-0.9169921875,0.664886041996006)(-0.916852678571429,0.663396751513016)(-0.916713169642857,0.661907954324657)(-0.916573660714286,0.66041965305701)(-0.916434151785714,0.658931850335199)(-0.916294642857143,0.65744454878337)(-0.916155133928571,0.655957751024694)(-0.916015625,0.654471459681371)(-0.915876116071429,0.652985677374604)(-0.915736607142857,0.65150040672461)(-0.915597098214286,0.650015650350608)(-0.915457589285714,0.648531410870821)(-0.915318080357143,0.647047690902462)(-0.915178571428571,0.645564493061729)(-0.9150390625,0.644081819963816)(-0.914899553571429,0.642599674222886)(-0.914760044642857,0.641118058452076)(-0.914620535714286,0.639636975263493)(-0.914481026785714,0.63815642726821)(-0.914341517857143,0.636676417076256)(-0.914202008928571,0.635196947296608)(-0.9140625,0.633718020537206)(-0.913922991071429,0.632239639404919)(-0.913783482142857,0.630761806505558)(-0.913643973214286,0.629284524443869)(-0.913504464285714,0.627807795823525)(-0.913364955357143,0.626331623247122)(-0.913225446428571,0.624856009316169)(-0.9130859375,0.6233809566311)(-0.912946428571429,0.621906467791245)(-0.912806919642857,0.62043254539484)(-0.912667410714286,0.618959192039018)(-0.912527901785714,0.617486410319807)(-0.912388392857143,0.616014202832122)(-0.912248883928571,0.614542572169754)(-0.912109375,0.613071520925384)(-0.911969866071429,0.611601051690554)(-0.911830357142857,0.610131167055678)(-0.911690848214286,0.608661869610029)(-0.911551339285714,0.607193161941743)(-0.911411830357143,0.605725046637804)(-0.911272321428571,0.604257526284039)(-0.9111328125,0.602790603465129)(-0.910993303571429,0.601324280764583)(-0.910853794642857,0.599858560764741)(-0.910714285714286,0.598393446046774)(-0.910574776785714,0.596928939190675)(-0.910435267857143,0.595465042775253)(-0.910295758928571,0.594001759378123)(-0.91015625,0.592539091575722)(-0.910016741071429,0.591077041943274)(-0.909877232142857,0.589615613054806)(-0.909737723214286,0.588154807483135)(-0.909598214285714,0.586694627799868)(-0.909458705357143,0.585235076575392)(-0.909319196428571,0.583776156378866)(-0.9091796875,0.582317869778233)(-0.909040178571429,0.580860219340192)(-0.908900669642857,0.579403207630207)(-0.908761160714286,0.577946837212498)(-0.908621651785714,0.57649111065004)(-0.908482142857143,0.575036030504553)(-0.908342633928571,0.573581599336494)(-0.908203125,0.57212781970507)(-0.908063616071429,0.570674694168207)(-0.907924107142857,0.569222225282562)(-0.907784598214286,0.567770415603515)(-0.907645089285714,0.566319267685164)(-0.907505580357143,0.564868784080317)(-0.907366071428571,0.563418967340486)(-0.9072265625,0.561969820015896)(-0.907087053571429,0.560521344655457)(-0.906947544642857,0.559073543806778)(-0.906808035714286,0.55762642001615)(-0.906668526785714,0.556179975828553)(-0.906529017857143,0.554734213787641)(-0.906389508928571,0.553289136435735)(-0.90625,0.551844746313835)(-0.906110491071429,0.550401045961596)(-0.905970982142857,0.54895803791733)(-0.905831473214286,0.547515724718001)(-0.905691964285714,0.546074108899228)(-0.905552455357143,0.544633192995264)(-0.905412946428571,0.543192979539)(-0.9052734375,0.541753471061972)(-0.905133928571429,0.54031467009433)(-0.904994419642857,0.538876579164852)(-0.904854910714286,0.537439200800932)(-0.904715401785714,0.536002537528584)(-0.904575892857143,0.534566591872424)(-0.904436383928571,0.533131366355668)(-0.904296875,0.531696863500142)(-0.904157366071429,0.530263085826257)(-0.904017857142857,0.528830035853013)(-0.903878348214286,0.527397716097994)(-0.903738839285714,0.525966129077367)(-0.903599330357143,0.524535277305868)(-0.903459821428571,0.5231051632968)(-0.9033203125,0.521675789562042)(-0.903180803571429,0.520247158612019)(-0.903041294642857,0.518819272955716)(-0.902901785714286,0.517392135100665)(-0.902762276785714,0.515965747552949)(-0.902622767857143,0.514540112817183)(-0.902483258928571,0.513115233396517)(-0.90234375,0.511691111792641)(-0.902204241071429,0.510267750505759)(-0.902064732142857,0.5088451520346)(-0.901925223214286,0.507423318876406)(-0.901785714285714,0.506002253526934)(-0.901646205357143,0.504581958480441)(-0.901506696428571,0.503162436229686)(-0.9013671875,0.50174368926593)(-0.901227678571429,0.500325720078919)(-0.901088169642857,0.498908531156886)(-0.900948660714286,0.497492124986544)(-0.900809151785714,0.496076504053089)(-0.900669642857143,0.494661670840182)(-0.900530133928571,0.493247627829949)(-0.900390625,0.491834377502991)(-0.900251116071429,0.490421922338353)(-0.900111607142857,0.489010264813536)(-0.899972098214286,0.48759940740449)(-0.899832589285714,0.486189352585609)(-0.899693080357143,0.484780102829722)(-0.899553571428571,0.483371660608088)(-0.8994140625,0.481964028390407)(-0.899274553571429,0.48055720864479)(-0.899135044642857,0.479151203837771)(-0.898995535714286,0.477746016434295)(-0.898856026785714,0.476341648897724)(-0.898716517857143,0.474938103689816)(-0.898577008928571,0.473535383270729)(-0.8984375,0.472133490099025)(-0.898297991071429,0.470732426631647)(-0.898158482142857,0.469332195323923)(-0.898018973214286,0.467932798629565)(-0.897879464285714,0.466534239000662)(-0.897739955357143,0.465136518887669)(-0.897600446428571,0.463739640739406)(-0.8974609375,0.462343607003066)(-0.897321428571429,0.460948420124186)(-0.897181919642857,0.459554082546658)(-0.897042410714286,0.458160596712723)(-0.896902901785714,0.456767965062965)(-0.896763392857143,0.455376190036302)(-0.896623883928571,0.453985274069984)(-0.896484375,0.452595219599598)(-0.896344866071429,0.451206029059046)(-0.896205357142857,0.449817704880549)(-0.896065848214286,0.448430249494643)(-0.895926339285714,0.447043665330175)(-0.895786830357143,0.445657954814295)(-0.895647321428571,0.444273120372449)(-0.8955078125,0.442889164428387)(-0.895368303571429,0.441506089404141)(-0.895228794642857,0.440123897720031)(-0.895089285714286,0.438742591794656)(-0.894949776785714,0.437362174044897)(-0.894810267857143,0.435982646885899)(-0.894670758928571,0.434604012731076)(-0.89453125,0.43322627399211)(-0.894391741071429,0.431849433078933)(-0.894252232142857,0.43047349239973)(-0.894112723214286,0.429098454360936)(-0.893973214285714,0.427724321367232)(-0.893833705357143,0.426351095821531)(-0.893694196428571,0.424978780124982)(-0.8935546875,0.423607376676971)(-0.893415178571429,0.422236887875098)(-0.893275669642857,0.420867316115186)(-0.893136160714286,0.419498663791275)(-0.892996651785714,0.418130933295615)(-0.892857142857143,0.416764127018662)(-0.892717633928571,0.415398247349068)(-0.892578125,0.414033296673694)(-0.892438616071429,0.41266927737758)(-0.892299107142857,0.411306191843961)(-0.892159598214286,0.409944042454249)(-0.892020089285714,0.40858283158804)(-0.891880580357143,0.407222561623099)(-0.891741071428571,0.405863234935359)(-0.8916015625,0.404504853898925)(-0.891462053571429,0.403147420886053)(-0.891322544642857,0.401790938267156)(-0.891183035714286,0.400435408410797)(-0.891043526785714,0.399080833683689)(-0.890904017857143,0.39772721645068)(-0.890764508928571,0.396374559074756)(-0.890625,0.395022863917041)(-0.890485491071429,0.393672133336778)(-0.890345982142857,0.392322369691337)(-0.890206473214286,0.390973575336202)(-0.890066964285714,0.389625752624978)(-0.889927455357143,0.388278903909373)(-0.889787946428571,0.386933031539196)(-0.8896484375,0.385588137862368)(-0.889508928571429,0.384244225224895)(-0.889369419642857,0.382901295970876)(-0.889229910714286,0.381559352442495)(-0.889090401785714,0.380218396980022)(-0.888950892857143,0.378878431921798)(-0.888811383928571,0.377539459604238)(-0.888671875,0.376201482361832)(-0.888532366071429,0.374864502527125)(-0.888392857142857,0.373528522430721)(-0.888253348214286,0.372193544401281)(-0.888113839285714,0.370859570765518)(-0.887974330357143,0.369526603848185)(-0.887834821428571,0.368194645972076)(-0.8876953125,0.366863699458028)(-0.887555803571429,0.365533766624903)(-0.887416294642857,0.36420484978959)(-0.887276785714286,0.362876951267004)(-0.887137276785714,0.361550073370076)(-0.886997767857143,0.360224218409752)(-0.886858258928571,0.358899388694983)(-0.88671875,0.357575586532733)(-0.886579241071429,0.356252814227958)(-0.886439732142857,0.354931074083613)(-0.886300223214286,0.353610368400642)(-0.886160714285714,0.35229069947798)(-0.886021205357143,0.35097206961254)(-0.885881696428571,0.349654481099211)(-0.8857421875,0.348337936230865)(-0.885602678571429,0.347022437298333)(-0.885463169642857,0.345707986590414)(-0.885323660714286,0.344394586393864)(-0.885184151785714,0.3430822389934)(-0.885044642857143,0.341770946671686)(-0.884905133928571,0.340460711709329)(-0.884765625,0.33915153638489)(-0.884626116071429,0.337843422974856)(-0.884486607142857,0.336536373753652)(-0.884347098214286,0.335230390993628)(-0.884207589285714,0.333925476965067)(-0.884068080357143,0.332621633936163)(-0.883928571428571,0.331318864173026) 
};
\addplot [
color=blue,
solid,
forget plot
]
coordinates{
 (-0.883928571428571,0.331318864173026)(-0.8837890625,0.330017169939686)(-0.883649553571429,0.328716553498073)(-0.883510044642857,0.327417017108017)(-0.883370535714286,0.32611856302725)(-0.883231026785714,0.324821193511399)(-0.883091517857143,0.323524910813975)(-0.882952008928571,0.322229717186374)(-0.8828125,0.320935614877882)(-0.882672991071429,0.31964260613565)(-0.882533482142857,0.318350693204704)(-0.882393973214286,0.317059878327937)(-0.882254464285714,0.315770163746109)(-0.882114955357143,0.314481551697832)(-0.881975446428571,0.313194044419573)(-0.8818359375,0.311907644145658)(-0.881696428571429,0.310622353108248)(-0.881556919642857,0.309338173537348)(-0.881417410714286,0.3080551076608)(-0.881277901785714,0.306773157704282)(-0.881138392857143,0.305492325891294)(-0.880998883928571,0.304212614443161)(-0.880859375,0.302934025579036)(-0.880719866071429,0.301656561515876)(-0.880580357142857,0.300380224468455)(-0.880440848214286,0.299105016649348)(-0.880301339285714,0.29783094026894)(-0.880161830357143,0.296557997535409)(-0.880022321428571,0.295286190654724)(-0.8798828125,0.294015521830653)(-0.879743303571429,0.29274599326474)(-0.879603794642857,0.291477607156312)(-0.879464285714286,0.290210365702472)(-0.879324776785714,0.2889442710981)(-0.879185267857143,0.287679325535837)(-0.879045758928571,0.286415531206089)(-0.87890625,0.285152890297029)(-0.878766741071429,0.283891404994577)(-0.878627232142857,0.282631077482405)(-0.878487723214286,0.281371909941932)(-0.878348214285714,0.280113904552322)(-0.878208705357143,0.278857063490476)(-0.878069196428571,0.277601388931024)(-0.8779296875,0.276346883046337)(-0.877790178571429,0.275093548006499)(-0.877650669642857,0.273841385979323)(-0.877511160714286,0.272590399130334)(-0.877371651785714,0.271340589622776)(-0.877232142857143,0.270091959617595)(-0.877092633928571,0.268844511273443)(-0.876953125,0.267598246746678)(-0.876813616071429,0.266353168191348)(-0.876674107142857,0.265109277759193)(-0.876534598214286,0.26386657759964)(-0.876395089285714,0.262625069859805)(-0.876255580357143,0.261384756684478)(-0.876116071428571,0.260145640216122)(-0.8759765625,0.258907722594881)(-0.875837053571429,0.257671005958555)(-0.875697544642857,0.256435492442611)(-0.875558035714286,0.255201184180173)(-0.875418526785714,0.253968083302023)(-0.875279017857143,0.252736191936589)(-0.875139508928571,0.251505512209942)(-0.875,0.250276046245808)(-0.874860491071429,0.249047796165536)(-0.874720982142857,0.247820764088115)(-0.874581473214286,0.246594952130163)(-0.874441964285714,0.245370362405924)(-0.874302455357143,0.244146997027262)(-0.874162946428571,0.242924858103655)(-0.8740234375,0.241703947742204)(-0.873883928571429,0.240484268047608)(-0.873744419642857,0.239265821122177)(-0.873604910714286,0.238048609065816)(-0.873465401785714,0.236832633976035)(-0.873325892857143,0.23561789794793)(-0.873186383928571,0.234404403074184)(-0.873046875,0.233192151445074)(-0.872907366071429,0.231981145148448)(-0.872767857142857,0.230771386269733)(-0.872628348214286,0.229562876891927)(-0.872488839285714,0.2283556190956)(-0.872349330357143,0.227149614958884)(-0.872209821428571,0.225944866557466)(-0.8720703125,0.2247413759646)(-0.871930803571429,0.223539145251083)(-0.871791294642857,0.222338176485262)(-0.871651785714286,0.221138471733028)(-0.871512276785714,0.219940033057815)(-0.871372767857143,0.218742862520587)(-0.871233258928571,0.217546962179843)(-0.87109375,0.216352334091613)(-0.870954241071429,0.215158980309447)(-0.870814732142857,0.213966902884415)(-0.870675223214286,0.212776103865102)(-0.870535714285714,0.211586585297608)(-0.870396205357143,0.21039834922554)(-0.870256696428571,0.209211397690004)(-0.8701171875,0.208025732729616)(-0.869977678571429,0.206841356380479)(-0.869838169642857,0.205658270676191)(-0.869698660714286,0.204476477647838)(-0.869559151785714,0.203295979323991)(-0.869419642857143,0.202116777730701)(-0.869280133928571,0.200938874891491)(-0.869140625,0.199762272827365)(-0.869001116071429,0.198586973556788)(-0.868861607142857,0.197412979095691)(-0.868722098214286,0.196240291457465)(-0.868582589285714,0.195068912652962)(-0.868443080357143,0.193898844690481)(-0.868303571428571,0.192730089575769)(-0.8681640625,0.191562649312028)(-0.868024553571429,0.19039652589989)(-0.867885044642857,0.189231721337428)(-0.867745535714286,0.188068237620147)(-0.867606026785714,0.186906076740984)(-0.867466517857143,0.1857452406903)(-0.867327008928571,0.184585731455873)(-0.8671875,0.183427551022909)(-0.867047991071429,0.182270701374018)(-0.866908482142857,0.181115184489226)(-0.866768973214286,0.179961002345959)(-0.866629464285714,0.178808156919052)(-0.866489955357143,0.177656650180734)(-0.866350446428571,0.176506484100628)(-0.8662109375,0.175357660645754)(-0.866071428571429,0.174210181780512)(-0.865931919642857,0.173064049466687)(-0.865792410714286,0.171919265663444)(-0.865652901785714,0.170775832327324)(-0.865513392857143,0.169633751412238)(-0.865373883928571,0.168493024869463)(-0.865234375,0.167353654647648)(-0.865094866071429,0.166215642692795)(-0.864955357142857,0.165078990948262)(-0.864815848214286,0.163943701354763)(-0.864676339285714,0.162809775850362)(-0.864536830357143,0.161677216370463)(-0.864397321428571,0.160546024847813)(-0.8642578125,0.159416203212503)(-0.864118303571429,0.158287753391951)(-0.863978794642857,0.157160677310904)(-0.863839285714286,0.15603497689144)(-0.863699776785714,0.154910654052957)(-0.863560267857143,0.153787710712174)(-0.863420758928571,0.15266614878312)(-0.86328125,0.151545970177144)(-0.863141741071429,0.150427176802895)(-0.863002232142857,0.149309770566329)(-0.862862723214286,0.148193753370701)(-0.862723214285714,0.147079127116565)(-0.862583705357143,0.145965893701765)(-0.862444196428571,0.144854055021435)(-0.8623046875,0.143743612967997)(-0.862165178571429,0.142634569431153)(-0.862025669642857,0.141526926297881)(-0.861886160714286,0.140420685452434)(-0.861746651785714,0.13931584877634)(-0.861607142857143,0.138212418148391)(-0.861467633928571,0.137110395444637)(-0.861328125,0.136009782538402)(-0.861188616071429,0.134910581300252)(-0.861049107142857,0.133812793598013)(-0.860909598214286,0.132716421296755)(-0.860770089285714,0.1316214662588)(-0.860630580357143,0.130527930343706)(-0.860491071428571,0.129435815408267)(-0.8603515625,0.128345123306522)(-0.860212053571429,0.127255855889731)(-0.860072544642857,0.126168015006384)(-0.859933035714286,0.125081602502192)(-0.859793526785714,0.123996620220093)(-0.859654017857143,0.122913070000234)(-0.859514508928571,0.121830953679977)(-0.859375,0.120750273093899)(-0.859235491071429,0.119671030073773)(-0.859095982142857,0.118593226448582)(-0.858956473214286,0.117516864044501)(-0.858816964285714,0.116441944684906)(-0.858677455357143,0.115368470190361)(-0.858537946428571,0.114296442378617)(-0.8583984375,0.113225863064615)(-0.858258928571429,0.112156734060471)(-0.858119419642857,0.111089057175479)(-0.857979910714286,0.110022834216108)(-0.857840401785714,0.108958066986)(-0.857700892857143,0.107894757285958)(-0.857561383928571,0.106832906913951)(-0.857421875,0.105772517665111)(-0.857282366071429,0.104713591331721)(-0.857142857142857,0.103656129703221)(-0.857003348214286,0.102600134566195)(-0.856863839285714,0.101545607704379)(-0.856724330357143,0.100492550898647)(-0.856584821428571,0.0994409659270119)(-0.8564453125,0.0983908545646274)(-0.856305803571429,0.0973422185837729)(-0.856166294642857,0.0962950597538587)(-0.856026785714286,0.0952493798414187)(-0.855887276785714,0.0942051806101121)(-0.855747767857143,0.0931624638207128)(-0.855608258928571,0.0921212312311082)(-0.85546875,0.0910814845963042)(-0.855329241071429,0.0900432256684075)(-0.855189732142857,0.0890064561966316)(-0.855050223214286,0.0879711779272903)(-0.854910714285714,0.086937392603798)(-0.854771205357143,0.0859051019666603)(-0.854631696428571,0.0848743077534725)(-0.8544921875,0.0838450116989247)(-0.854352678571429,0.0828172155347837)(-0.854213169642857,0.0817909209898998)(-0.854073660714286,0.0807661297901995)(-0.853934151785714,0.0797428436586866)(-0.853794642857143,0.0787210643154325)(-0.853655133928571,0.0777007934775744)(-0.853515625,0.0766820328593213)(-0.853376116071429,0.075664784171935)(-0.853236607142857,0.0746490491237376)(-0.853097098214286,0.0736348294201038)(-0.852957589285714,0.0726221267634626)(-0.852818080357143,0.0716109428532867)(-0.852678571428571,0.0706012793860922)(-0.8525390625,0.0695931380554428)(-0.852399553571429,0.0685865205519324)(-0.852260044642857,0.0675814285631919)(-0.852120535714286,0.0665778637738818)(-0.851981026785714,0.0655758278656935)(-0.851841517857143,0.0645753225173396)(-0.851702008928571,0.0635763494045525)(-0.8515625,0.0625789102000895)(-0.851422991071429,0.0615830065737156)(-0.851283482142857,0.0605886401922093)(-0.851143973214286,0.0595958127193562)(-0.851004464285714,0.05860452581595)(-0.850864955357143,0.0576147811397826)(-0.850725446428571,0.0566265803456434)(-0.8505859375,0.0556399250853238)(-0.850446428571429,0.0546548170076001)(-0.850306919642857,0.0536712577582398)(-0.850167410714286,0.0526892489799948)(-0.850027901785714,0.0517087923126028)(-0.849888392857143,0.0507298893927771)(-0.849748883928571,0.049752541854206)(-0.849609375,0.0487767513275574)(-0.849469866071429,0.0478025194404621)(-0.849330357142857,0.0468298478175194)(-0.849190848214286,0.045858738080291)(-0.849051339285714,0.0448891918473018)(-0.848911830357143,0.0439212107340308)(-0.848772321428571,0.0429547963529091)(-0.8486328125,0.041989950313326)(-0.848493303571429,0.041026674221611)(-0.848353794642857,0.0400649696810407)(-0.848214285714286,0.0391048382918318)(-0.848074776785714,0.0381462816511422)(-0.847935267857143,0.0371893013530622)(-0.847795758928571,0.0362338989886126)(-0.84765625,0.0352800761457498)(-0.847516741071429,0.0343278344093495)(-0.847377232142857,0.0333771753612123)(-0.847237723214286,0.0324281005800573)(-0.847098214285714,0.0314806116415234)(-0.846958705357143,0.0305347101181599)(-0.846819196428571,0.0295903975794253)(-0.8466796875,0.0286476755916923)(-0.846540178571429,0.0277065457182313)(-0.846400669642857,0.0267670095192158)(-0.846261160714286,0.0258290685517174)(-0.846121651785714,0.024892724369705)(-0.845982142857143,0.0239579785240375)(-0.845842633928571,0.0230248325624612)(-0.845703125,0.0220932880296159)(-0.845563616071429,0.021163346467017)(-0.845424107142857,0.0202350094130632)(-0.845284598214286,0.019308278403029)(-0.845145089285714,0.0183831549690658)(-0.845005580357143,0.0174596406401939)(-0.844866071428571,0.0165377369423001)(-0.8447265625,0.0156174453981431)(-0.844587053571429,0.0146987675273378)(-0.844447544642857,0.01378170484636)(-0.844308035714286,0.0128662588685413)(-0.844168526785714,0.0119524311040696)(-0.844029017857143,0.0110402230599805)(-0.843889508928571,0.0101296362401557)(-0.84375,0.0092206721453284)(-0.843610491071429,0.00831333227306663)(-0.843470982142857,0.0074076181177799)(-0.843331473214286,0.00650353117071223)(-0.843191964285714,0.00560107291994383)(-0.843052455357143,0.0047002448503819)(-0.842912946428571,0.00380104844375984)(-0.8427734375,0.00290348517864181)(-0.842633928571429,0.00200755653040696)(-0.842494419642857,0.00111326397125511)(-0.842354910714286,0.000220608970200931)(-0.842215401785714,-0.000670407006925222)(-0.842075892857143,-0.00155978249748578)(-0.841936383928571,-0.0024475160420373)(-0.841796875,-0.00333360618432543)(-0.841657366071429,-0.00421805147130083)(-0.841517857142857,-0.00510085045311359)(-0.841378348214286,-0.00598200168311858)(-0.841238839285714,-0.00686150371787486)(-0.841099330357143,-0.00773935511715429)(-0.840959821428571,-0.00861555444394235)(-0.8408203125,-0.00949010026443353)(-0.840680803571429,-0.010362991148047)(-0.840541294642857,-0.0112342256674206)(-0.840401785714286,-0.0121038023984169)(-0.840262276785714,-0.0129717199201224)(-0.840122767857143,-0.0138379768148551)(-0.839983258928571,-0.0147025716681669)(-0.83984375,-0.0155655030688377)(-0.839704241071429,-0.0164267696088914)(-0.839564732142857,-0.0172863698835903)(-0.839425223214286,-0.0181443024914401)(-0.839285714285714,-0.0190005660341899)(-0.839146205357143,-0.0198551591168399)(-0.839006696428571,-0.0207080803476426)(-0.8388671875,-0.0215593283380981)(-0.838727678571429,-0.0224089017029696)(-0.838588169642857,-0.0232567990602771)(-0.838448660714286,-0.0241030190313039)(-0.838309151785714,-0.0249475602405946)(-0.838169642857143,-0.0257904213159647)(-0.838030133928571,-0.0266316008884999)(-0.837890625,-0.0274710975925534)(-0.837751116071429,-0.0283089100657591)(-0.837611607142857,-0.0291450369490273)(-0.837472098214286,-0.0299794768865493)(-0.837332589285714,-0.0308122285257976)(-0.837193080357143,-0.0316432905175329)(-0.837053571428571,-0.0324726615158055)(-0.8369140625,-0.033300340177951)(-0.836774553571429,-0.0341263251646047)(-0.836635044642857,-0.0349506151396961)(-0.836495535714286,-0.0357732087704549)(-0.836356026785714,-0.0365941047274093)(-0.836216517857143,-0.0374133016843945)(-0.836077008928571,-0.0382307983185535)(-0.8359375,-0.0390465933103326)(-0.835797991071429,-0.039860685343496)(-0.835658482142857,-0.0406730731051201)(-0.835518973214286,-0.0414837552855997)(-0.835379464285714,-0.0422927305786457)(-0.835239955357143,-0.0430999976812945)(-0.835100446428571,-0.0439055552939075)(-0.8349609375,-0.0447094021201678)(-0.834821428571429,-0.0455115368670941)(-0.834681919642857,-0.0463119582450349)(-0.834542410714286,-0.0471106649676746)(-0.834402901785714,-0.0479076557520319)(-0.834263392857143,-0.0487029293184682)(-0.834123883928571,-0.0494964843906878)(-0.833984375,-0.0502883196957339)(-0.833844866071429,-0.0510784339640028)(-0.833705357142857,-0.0518668259292383)(-0.833565848214286,-0.0526534943285372)(-0.833426339285714,-0.0534384379023483)(-0.833286830357143,-0.0542216553944799)(-0.833147321428571,-0.0550031455521007)(-0.8330078125,-0.0557829071257359)(-0.832868303571429,-0.0565609388692802)(-0.832728794642857,-0.0573372395399935)(-0.832589285714286,-0.0581118078985056)(-0.832449776785714,-0.0588846427088151)(-0.832310267857143,-0.0596557427382975)(-0.832170758928571,-0.0604251067577053)(-0.83203125,-0.0611927335411643)(-0.831891741071429,-0.0619586218661872)(-0.831752232142857,-0.062722770513668)(-0.831612723214286,-0.0634851782678877)(-0.831473214285714,-0.064245843916513)(-0.831333705357143,-0.0650047662506037)(-0.831194196428571,-0.0657619440646134)(-0.8310546875,-0.0665173761563859)(-0.830915178571429,-0.0672710613271678)(-0.830775669642857,-0.0680229983816039)(-0.830636160714286,-0.0687731861277422)(-0.830496651785714,-0.0695216233770329)(-0.830357142857143,-0.0702683089443358)(-0.830217633928571,-0.0710132416479204)(-0.830078125,-0.0717564203094629)(-0.829938616071429,-0.0724978437540584)(-0.829799107142857,-0.0732375108102165)(-0.829659598214286,-0.0739754203098659)(-0.829520089285714,-0.0747115710883537)(-0.829380580357143,-0.0754459619844523)(-0.829241071428571,-0.0761785918403601)(-0.8291015625,-0.0769094595016983)(-0.828962053571429,-0.0776385638175221)(-0.828822544642857,-0.0783659036403179)(-0.828683035714286,-0.0790914778260063)(-0.828543526785714,-0.0798152852339424)(-0.828404017857143,-0.0805373247269223)(-0.828264508928571,-0.0812575951711836)(-0.828125,-0.0819760954364018)(-0.827985491071429,-0.0826928243957032)(-0.827845982142857,-0.0834077809256595)(-0.827706473214286,-0.0841209639062928)(-0.827566964285714,-0.0848323722210749)(-0.827427455357143,-0.0855420047569337)(-0.827287946428571,-0.0862498604042543)(-0.8271484375,-0.086955938056875)(-0.827008928571429,-0.0876602366120994)(-0.826869419642857,-0.0883627549706927)(-0.826729910714286,-0.0890634920368847)(-0.826590401785714,-0.0897624467183705)(-0.826450892857143,-0.0904596179263162)(-0.826311383928571,-0.0911550045753601)(-0.826171875,-0.0918486055836084)(-0.826032366071429,-0.092540419872648)(-0.825892857142857,-0.0932304463675415)(-0.825753348214286,-0.0939186839968316)(-0.825613839285714,-0.0946051316925405)(-0.825474330357143,-0.0952897883901765)(-0.825334821428571,-0.0959726530287344)(-0.8251953125,-0.0966537245506919)(-0.825055803571429,-0.0973330019020214)(-0.824916294642857,-0.098010484032186)(-0.824776785714286,-0.0986861698941432)(-0.824637276785714,-0.0993600584443444)(-0.824497767857143,-0.100032148642741)(-0.824358258928571,-0.100702439452786)(-0.82421875,-0.10137092984143)(-0.824079241071429,-0.10203761877913)(-0.823939732142857,-0.102702505239851)(-0.823800223214286,-0.103365588201065)(-0.823660714285714,-0.104026866643751)(-0.823521205357143,-0.104686339552405)(-0.823381696428571,-0.105344005915035)(-0.8232421875,-0.105999864723162)(-0.823102678571429,-0.106653914971829)(-0.822963169642857,-0.107306155659599)(-0.822823660714286,-0.107956585788555)(-0.822684151785714,-0.108605204364304)(-0.822544642857143,-0.109252010395979)(-0.822405133928571,-0.109897002896242)(-0.822265625,-0.110540180881283)(-0.822126116071429,-0.111181543370823)(-0.821986607142857,-0.111821089388119)(-0.821847098214286,-0.112458817959962)(-0.821707589285714,-0.113094728116679)(-0.821568080357143,-0.113728818892138)(-0.821428571428571,-0.114361089323749)(-0.8212890625,-0.11499153845246)(-0.821149553571429,-0.115620165322768)(-0.821010044642857,-0.116246968982716)(-0.820870535714286,-0.116871948483896)(-0.820731026785714,-0.117495102881448)(-0.820591517857143,-0.118116431234065)(-0.820452008928571,-0.118735932603998)(-0.8203125,-0.119353606057045)(-0.820172991071429,-0.11996945066257)(-0.820033482142857,-0.120583465493493)(-0.819893973214286,-0.121195649626296)(-0.819754464285714,-0.121806002141023)(-0.819614955357143,-0.122414522121283)(-0.819475446428571,-0.123021208654255)(-0.8193359375,-0.123626060830679)(-0.819196428571429,-0.124229077744871)(-0.819056919642857,-0.12483025849472)(-0.818917410714286,-0.125429602181684)(-0.818777901785714,-0.126027107910799)(-0.818638392857143,-0.126622774790678)(-0.818498883928571,-0.127216601933514)(-0.818359375,-0.127808588455077)(-0.818219866071429,-0.128398733474722)(-0.818080357142857,-0.12898703611539)(-0.817940848214286,-0.129573495503604)(-0.817801339285714,-0.130158110769475)(-0.817661830357143,-0.130740881046707)(-0.817522321428571,-0.131321805472591)(-0.8173828125,-0.131900883188009)(-0.817243303571429,-0.132478113337443)(-0.817103794642857,-0.133053495068966)(-0.816964285714286,-0.133627027534251)(-0.816824776785714,-0.134198709888569)(-0.816685267857143,-0.134768541290792)(-0.816545758928571,-0.135336520903396)(-0.81640625,-0.135902647892457)(-0.816266741071429,-0.136466921427662)(-0.816127232142857,-0.137029340682301)(-0.815987723214286,-0.137589904833276)(-0.815848214285714,-0.138148613061098)(-0.815708705357143,-0.13870546454989)(-0.815569196428571,-0.13926045848739)(-0.8154296875,-0.139813594064949)(-0.815290178571429,-0.140364870477537)(-0.815150669642857,-0.140914286923741)(-0.815011160714286,-0.141461842605771)(-0.814871651785714,-0.142007536729454)(-0.814732142857143,-0.142551368504243)(-0.814592633928571,-0.143093337143216)(-0.814453125,-0.143633441863074)(-0.814313616071429,-0.14417168188415)(-0.814174107142857,-0.144708056430404)(-0.814034598214286,-0.145242564729426)(-0.813895089285714,-0.145775206012438)(-0.813755580357143,-0.146305979514299)(-0.813616071428571,-0.146834884473502)(-0.8134765625,-0.147361920132172)(-0.813337053571429,-0.147887085736078)(-0.813197544642857,-0.148410380534626)(-0.813058035714286,-0.148931803780864)(-0.812918526785714,-0.149451354731482)(-0.812779017857143,-0.149969032646814)(-0.812639508928571,-0.150484836790842)(-0.8125,-0.150998766431188)(-0.812360491071429,-0.15151082083913)(-0.812220982142857,-0.152020999289591)(-0.812081473214286,-0.152529301061147)(-0.811941964285714,-0.153035725436027)(-0.811802455357143,-0.153540271700112)(-0.811662946428571,-0.154042939142941)(-0.8115234375,-0.154543727057706)(-0.811383928571429,-0.15504263474126)(-0.811244419642857,-0.155539661494115)(-0.811104910714286,-0.156034806620443)(-0.810965401785714,-0.156528069428079)(-0.810825892857143,-0.15701944922852)(-0.810686383928571,-0.157508945336932)(-0.810546875,-0.15799655707214)(-0.810407366071429,-0.158482283756642)(-0.810267857142857,-0.158966124716605)(-0.810128348214286,-0.159448079281864)(-0.809988839285714,-0.159928146785924)(-0.809849330357143,-0.160406326565968)(-0.809709821428571,-0.160882617962848)(-0.8095703125,-0.161357020321093)(-0.809430803571429,-0.161829532988908)(-0.809291294642857,-0.162300155318178)(-0.809151785714286,-0.162768886664466)(-0.809012276785714,-0.163235726387014)(-0.808872767857143,-0.163700673848746)(-0.808733258928571,-0.164163728416272)(-0.80859375,-0.16462488945988)(-0.808454241071429,-0.16508415635355)(-0.808314732142857,-0.165541528474944)(-0.808175223214286,-0.165997005205414)(-0.808035714285714,-0.166450585929999)(-0.807896205357143,-0.16690227003743)(-0.807756696428571,-0.16735205692013)(-0.8076171875,-0.16779994597421)(-0.807477678571429,-0.16824593659948)(-0.807338169642857,-0.168690028199442)(-0.807198660714286,-0.169132220181296)(-0.807059151785714,-0.169572511955936)(-0.806919642857143,-0.170010902937958)(-0.806780133928571,-0.170447392545655)(-0.806640625,-0.17088198020102)(-0.806501116071429,-0.17131466532975)(-0.806361607142857,-0.171745447361245)(-0.806222098214286,-0.172174325728607)(-0.806082589285714,-0.172601299868642)(-0.805943080357143,-0.173026369221867)(-0.805803571428571,-0.173449533232502)(-0.8056640625,-0.173870791348475)(-0.805524553571429,-0.174290143021427)(-0.805385044642857,-0.174707587706707)(-0.805245535714286,-0.175123124863376)(-0.805106026785714,-0.175536753954206)(-0.804966517857143,-0.175948474445686)(-0.804827008928571,-0.176358285808017)(-0.8046875,-0.176766187515116)(-0.804547991071429,-0.177172179044618)(-0.804408482142857,-0.177576259877874)(-0.804268973214286,-0.177978429499955)(-0.804129464285714,-0.178378687399652)(-0.803989955357143,-0.178777033069476)(-0.803850446428571,-0.17917346600566)(-0.8037109375,-0.179567985708157)(-0.803571428571429,-0.17996059168065)(-0.803431919642857,-0.18035128343054)(-0.803292410714286,-0.180740060468958)(-0.803152901785714,-0.181126922310759)(-0.803013392857143,-0.181511868474527)(-0.802873883928571,-0.181894898482575)(-0.802734375,-0.182276011860943)(-0.802594866071429,-0.182655208139402)(-0.802455357142857,-0.183032486851456)(-0.802315848214286,-0.18340784753434)(-0.802176339285714,-0.183781289729021)(-0.802036830357143,-0.184152812980202)(-0.801897321428571,-0.184522416836319)(-0.8017578125,-0.184890100849543)(-0.801618303571429,-0.185255864575784)(-0.801478794642857,-0.185619707574689)(-0.801339285714286,-0.185981629409642)(-0.801199776785714,-0.186341629647765)(-0.801060267857143,-0.186699707859923)(-0.800920758928571,-0.187055863620721)(-0.80078125,-0.187410096508504)(-0.800641741071429,-0.18776240610536)(-0.800502232142857,-0.188112791997121)(-0.800362723214286,-0.188461253773363)(-0.800223214285714,-0.188807791027406)(-0.800083705357143,-0.189152403356316)(-0.799944196428571,-0.189495090360908)(-0.7998046875,-0.189835851645738)(-0.799665178571429,-0.190174686819117)(-0.799525669642857,-0.1905115954931)(-0.799386160714286,-0.190846577283493)(-0.799246651785714,-0.191179631809853)(-0.799107142857143,-0.191510758695487)(-0.798967633928571,-0.191839957567454)(-0.798828125,-0.192167228056566)(-0.798688616071429,-0.192492569797386)(-0.798549107142857,-0.192815982428233)(-0.798409598214286,-0.193137465591182)(-0.798270089285714,-0.193457018932058)(-0.798130580357143,-0.193774642100448)(-0.797991071428571,-0.194090334749693)(-0.7978515625,-0.194404096536889)(-0.797712053571429,-0.194715927122894)(-0.797572544642857,-0.195025826172322)(-0.797433035714286,-0.195333793353547)(-0.797293526785714,-0.195639828338704)(-0.797154017857143,-0.195943930803688)(-0.797014508928571,-0.196246100428154)(-0.796875,-0.19654633689552)(-0.796735491071429,-0.196844639892966)(-0.796595982142857,-0.197141009111435)(-0.796456473214286,-0.197435444245635)(-0.796316964285714,-0.197727944994036)(-0.796177455357143,-0.198018511058875)(-0.796037946428571,-0.198307142146153)(-0.7958984375,-0.198593837965637)(-0.795758928571429,-0.198878598230862)(-0.795619419642857,-0.199161422659129)(-0.795479910714286,-0.199442310971506)(-0.795340401785714,-0.19972126289283)(-0.795200892857143,-0.199998278151708)(-0.795061383928571,-0.200273356480514)(-0.794921875,-0.200546497615393)(-0.794782366071429,-0.20081770129626)(-0.794642857142857,-0.201086967266801)(-0.794503348214286,-0.201354295274474)(-0.794363839285714,-0.201619685070507)(-0.794224330357143,-0.201883136409902)(-0.794084821428571,-0.202144649051434)(-0.7939453125,-0.202404222757648)(-0.793805803571429,-0.202661857294867)(-0.793666294642857,-0.202917552433185)(-0.793526785714286,-0.203171307946473)(-0.793387276785714,-0.203423123612376)(-0.793247767857143,-0.203672999212313)(-0.793108258928571,-0.203920934531481)(-0.79296875,-0.204166929358852)(-0.792829241071429,-0.204410983487176)(-0.792689732142857,-0.20465309671298)(-0.792550223214286,-0.204893268836566)(-0.792410714285714,-0.205131499662018)(-0.792271205357143,-0.205367788997194)(-0.792131696428571,-0.205602136653736)(-0.7919921875,-0.205834542447058)(-0.791852678571429,-0.206065006196359)(-0.791713169642857,-0.206293527724617)(-0.791573660714286,-0.206520106858588)(-0.791434151785714,-0.20674474342881)(-0.791294642857143,-0.206967437269601)(-0.791155133928571,-0.20718818821906)(-0.791015625,-0.207406996119069)(-0.790876116071429,-0.207623860815289)(-0.790736607142857,-0.207838782157166)(-0.790597098214286,-0.208051759997925)(-0.790457589285714,-0.208262794194577)(-0.790318080357143,-0.208471884607914)(-0.790178571428571,-0.208679031102511)(-0.7900390625,-0.208884233546726)(-0.789899553571429,-0.209087491812703)(-0.789760044642857,-0.209288805776367)(-0.789620535714286,-0.209488175317429)(-0.789481026785714,-0.209685600319384)(-0.789341517857143,-0.209881080669512)(-0.789202008928571,-0.210074616258877)(-0.7890625,-0.210266206982327)(-0.788922991071429,-0.210455852738498)(-0.788783482142857,-0.210643553429809)(-0.788643973214286,-0.210829308962467)(-0.788504464285714,-0.211013119246462)(-0.788364955357143,-0.211194984195572)(-0.788225446428571,-0.211374903727362)(-0.7880859375,-0.211552877763179)(-0.787946428571429,-0.211728906228162)(-0.787806919642857,-0.211902989051232)(-0.787667410714286,-0.2120751261651)(-0.787527901785714,-0.212245317506263)(-0.787388392857143,-0.212413563015003)(-0.787248883928571,-0.212579862635392)(-0.787109375,-0.212744216315288)(-0.786969866071429,-0.212906624006335)(-0.786830357142857,-0.213067085663968)(-0.786690848214286,-0.213225601247405)(-0.786551339285714,-0.213382170719655)(-0.786411830357143,-0.213536794047513)(-0.786272321428571,-0.213689471201563)(-0.7861328125,-0.213840202156175)(-0.785993303571429,-0.213988986889509)(-0.785853794642857,-0.214135825383511)(-0.785714285714286,-0.214280717623915)(-0.785574776785714,-0.214423663600245)(-0.785435267857143,-0.214564663305811)(-0.785295758928571,-0.214703716737712)(-0.78515625,-0.214840823896834)(-0.785016741071429,-0.214975984787851)(-0.784877232142857,-0.215109199419226)(-0.784737723214286,-0.21524046780321)(-0.784598214285714,-0.21536978995584)(-0.784458705357143,-0.215497165896944)(-0.784319196428571,-0.215622595650135)(-0.7841796875,-0.215746079242814)(-0.784040178571429,-0.215867616706173)(-0.783900669642857,-0.215987208075188)(-0.783761160714286,-0.216104853388624)(-0.783621651785714,-0.216220552689034)(-0.783482142857143,-0.216334306022757)(-0.783342633928571,-0.216446113439922)(-0.783203125,-0.216555974994443)(-0.783063616071429,-0.216663890744022)(-0.782924107142857,-0.216769860750148)(-0.782784598214286,-0.216873885078096)(-0.782645089285714,-0.216975963796931)(-0.782505580357143,-0.2170760969795)(-0.782366071428571,-0.21717428470244)(-0.7822265625,-0.217270527046173)(-0.782087053571429,-0.217364824094909)(-0.781947544642857,-0.217457175936641)(-0.781808035714286,-0.217547582663151)(-0.781668526785714,-0.217636044370004)(-0.781529017857143,-0.217722561156552)(-0.781389508928571,-0.217807133125934)(-0.78125,-0.21788976038507)(-0.781110491071429,-0.217970443044669)(-0.780970982142857,-0.218049181219222)(-0.780831473214286,-0.218125975027006)(-0.780691964285714,-0.218200824590082)(-0.780552455357143,-0.218273730034294)(-0.780412946428571,-0.218344691489272)(-0.7802734375,-0.218413709088426)(-0.780133928571429,-0.218480782968954)(-0.779994419642857,-0.218545913271833)(-0.779854910714286,-0.218609100141824)(-0.779715401785714,-0.218670343727471)(-0.779575892857143,-0.2187296441811)(-0.779436383928571,-0.218787001658818)(-0.779296875,-0.218842416320515)(-0.779157366071429,-0.218895888329861)(-0.779017857142857,-0.218947417854308)(-0.778878348214286,-0.218997005065087)(-0.778738839285714,-0.21904465013721)(-0.778599330357143,-0.219090353249471)(-0.778459821428571,-0.219134114584441)(-0.7783203125,-0.21917593432847)(-0.778180803571429,-0.21921581267169)(-0.778041294642857,-0.219253749808007)(-0.777901785714286,-0.219289745935109)(-0.777762276785714,-0.21932380125446)(-0.777622767857143,-0.219355915971302)(-0.777483258928571,-0.219386090294652)(-0.77734375,-0.219414324437307)(-0.777204241071429,-0.219440618615837)(-0.777064732142857,-0.219464973050589)(-0.776925223214286,-0.219487387965686)(-0.776785714285714,-0.219507863589024)(-0.776646205357143,-0.219526400152275)(-0.776506696428571,-0.219542997890884)(-0.7763671875,-0.21955765704407)(-0.776227678571429,-0.219570377854825)(-0.776088169642857,-0.219581160569913)(-0.775948660714286,-0.219590005439871)(-0.775809151785714,-0.219596912719006)(-0.775669642857143,-0.219601882665399)(-0.775530133928571,-0.219604915540898)(-0.775390625,-0.219606011611123)(-0.775251116071429,-0.219605171145464)(-0.775111607142857,-0.219602394417079)(-0.774972098214286,-0.219597681702894)(-0.774832589285714,-0.219591033283606)(-0.774693080357143,-0.219582449443675)(-0.774553571428571,-0.219571930471331)(-0.7744140625,-0.219559476658569)(-0.774274553571429,-0.219545088301151)(-0.774135044642857,-0.219528765698602)(-0.773995535714286,-0.219510509154214)(-0.773856026785714,-0.219490318975041)(-0.773716517857143,-0.219468195471902)(-0.773577008928571,-0.219444138959377)(-0.7734375,-0.219418149755809)(-0.773297991071429,-0.219390228183304)(-0.773158482142857,-0.219360374567727)(-0.773018973214286,-0.219328589238703)(-0.772879464285714,-0.219294872529619)(-0.772739955357143,-0.219259224777618)(-0.772600446428571,-0.219221646323604)(-0.7724609375,-0.219182137512237)(-0.772321428571429,-0.219140698691935)(-0.772181919642857,-0.219097330214872)(-0.772042410714286,-0.219052032436976)(-0.771902901785714,-0.219004805717933)(-0.771763392857143,-0.218955650421182)(-0.771623883928571,-0.218904566913913)(-0.771484375,-0.218851555567073)(-0.771344866071429,-0.218796616755359)(-0.771205357142857,-0.21873975085722)(-0.771065848214286,-0.218680958254853)(-0.770926339285714,-0.218620239334209)(-0.770786830357143,-0.218557594484986)(-0.770647321428571,-0.21849302410063)(-0.7705078125,-0.218426528578335)(-0.770368303571429,-0.218358108319043)(-0.770228794642857,-0.218287763727441)(-0.770089285714286,-0.218215495211961)(-0.769949776785714,-0.21814130318478)(-0.769810267857143,-0.218065188061819)(-0.769670758928571,-0.21798715026274)(-0.76953125,-0.21790719021095)(-0.769391741071429,-0.217825308333596)(-0.769252232142857,-0.217741505061563)(-0.769112723214286,-0.21765578082948)(-0.768973214285714,-0.217568136075711)(-0.768833705357143,-0.21747857124236)(-0.768694196428571,-0.217387086775267)(-0.7685546875,-0.217293683124008)(-0.768415178571429,-0.217198360741896)(-0.768275669642857,-0.217101120085975)(-0.768136160714286,-0.217001961617026)(-0.767996651785714,-0.216900885799561)(-0.767857142857143,-0.216797893101823)(-0.767717633928571,-0.216692983995787)(-0.767578125,-0.216586158957157)(-0.767438616071429,-0.216477418465368)(-0.767299107142857,-0.216366763003579)(-0.767159598214286,-0.21625419305868)(-0.767020089285714,-0.216139709121285)(-0.766880580357143,-0.216023311685734)(-0.766741071428571,-0.21590500125009)(-0.7666015625,-0.215784778316141)(-0.766462053571429,-0.215662643389395)(-0.766322544642857,-0.215538596979085)(-0.766183035714286,-0.215412639598159)(-0.766043526785714,-0.215284771763289)(-0.765904017857143,-0.215154993994863)(-0.765764508928571,-0.215023306816987)(-0.765625,-0.214889710757482)(-0.765485491071429,-0.214754206347885)(-0.765345982142857,-0.214616794123448)(-0.765206473214286,-0.214477474623135)(-0.765066964285714,-0.214336248389623)(-0.764927455357143,-0.214193115969299)(-0.764787946428571,-0.21404807791226)(-0.7646484375,-0.213901134772314)(-0.764508928571429,-0.213752287106973)(-0.764369419642857,-0.213601535477459)(-0.764229910714286,-0.213448880448699)(-0.764090401785714,-0.213294322589323)(-0.763950892857143,-0.213137862471666)(-0.763811383928571,-0.212979500671765)(-0.763671875,-0.212819237769357)(-0.763532366071429,-0.212657074347882)(-0.763392857142857,-0.212493010994475)(-0.763253348214286,-0.212327048299972)(-0.763113839285714,-0.212159186858904)(-0.762974330357143,-0.211989427269498)(-0.762834821428571,-0.211817770133674)(-0.7626953125,-0.211644216057048)(-0.762555803571429,-0.211468765648927)(-0.762416294642857,-0.211291419522306)(-0.762276785714286,-0.211112178293872)(-0.762137276785714,-0.210931042584002)(-0.761997767857143,-0.210748013016756)(-0.761858258928571,-0.210563090219883)(-0.76171875,-0.210376274824817)(-0.761579241071429,-0.210187567466673)(-0.761439732142857,-0.20999696878425)(-0.761300223214286,-0.209804479420029)(-0.761160714285714,-0.209610100020169)(-0.761021205357143,-0.209413831234507)(-0.760881696428571,-0.209215673716559)(-0.7607421875,-0.209015628123517)(-0.760602678571429,-0.208813695116247)(-0.760463169642857,-0.208609875359288)(-0.760323660714286,-0.208404169520851)(-0.760184151785714,-0.20819657827282)(-0.760044642857143,-0.207987102290746)(-0.759905133928571,-0.207775742253849)(-0.759765625,-0.207562498845017)(-0.759626116071429,-0.207347372750802)(-0.759486607142857,-0.207130364661421)(-0.759347098214286,-0.206911475270754)(-0.759207589285714,-0.206690705276341)(-0.759068080357143,-0.206468055379385)(-0.758928571428571,-0.206243526284745)(-0.7587890625,-0.20601711870094)(-0.758649553571429,-0.205788833340143)(-0.758510044642857,-0.205558670918182)(-0.758370535714286,-0.205326632154539)(-0.758231026785714,-0.205092717772348)(-0.758091517857143,-0.204856928498392)(-0.757952008928571,-0.204619265063104)(-0.7578125,-0.204379728200567)(-0.757672991071429,-0.204138318648505)(-0.757533482142857,-0.203895037148291)(-0.757393973214286,-0.203649884444941)(-0.757254464285714,-0.20340286128711)(-0.757114955357143,-0.203153968427096)(-0.756975446428571,-0.202903206620834)(-0.7568359375,-0.2026505766279)(-0.756696428571429,-0.202396079211503)(-0.756556919642857,-0.202139715138486)(-0.756417410714286,-0.201881485179328)(-0.756277901785714,-0.201621390108137)(-0.756138392857143,-0.201359430702651)(-0.755998883928571,-0.201095607744238)(-0.755859375,-0.200829922017892)(-0.755719866071429,-0.200562374312234)(-0.755580357142857,-0.200292965419505)(-0.755440848214286,-0.200021696135574)(-0.755301339285714,-0.199748567259927)(-0.755161830357143,-0.199473579595669)(-0.755022321428571,-0.199196733949524)(-0.7548828125,-0.198918031131833)(-0.754743303571429,-0.198637471956551)(-0.754603794642857,-0.198355057241244)(-0.754464285714286,-0.198070787807091)(-0.754324776785714,-0.197784664478883)(-0.754185267857143,-0.197496688085014)(-0.754045758928571,-0.197206859457488)(-0.75390625,-0.196915179431916)(-0.753766741071429,-0.196621648847507)(-0.753627232142857,-0.196326268547075)(-0.753487723214286,-0.196029039377035)(-0.753348214285714,-0.195729962187397)(-0.753208705357143,-0.195429037831771)(-0.753069196428571,-0.19512626716736)(-0.7529296875,-0.194821651054961)(-0.752790178571429,-0.194515190358964)(-0.752650669642857,-0.194206885947347)(-0.752511160714286,-0.193896738691678)(-0.752371651785714,-0.193584749467111)(-0.752232142857143,-0.193270919152385)(-0.752092633928571,-0.192955248629821)(-0.751953125,-0.192637738785325)(-0.751813616071429,-0.19231839050838)(-0.751674107142857,-0.191997204692046)(-0.751534598214286,-0.191674182232963)(-0.751395089285714,-0.191349324031344)(-0.751255580357143,-0.191022630990973)(-0.751116071428571,-0.190694104019207)(-0.7509765625,-0.190363744026973)(-0.750837053571429,-0.190031551928764)(-0.750697544642857,-0.18969752864264)(-0.750558035714286,-0.189361675090223)(-0.750418526785714,-0.1890239921967)(-0.750279017857143,-0.188684480890815)(-0.750139508928571,-0.188343142104874)(-0.75,-0.187999976774739)(-0.749860491071429,-0.187654985839825)(-0.749720982142857,-0.187308170243101)(-0.749581473214286,-0.186959530931089)(-0.749441964285714,-0.186609068853859)(-0.749302455357143,-0.186256784965028)(-0.749162946428571,-0.185902680221758)(-0.7490234375,-0.18554675558476)(-0.748883928571429,-0.185189012018281)(-0.748744419642857,-0.184829450490111)(-0.748604910714286,-0.184468071971578)(-0.748465401785714,-0.184104877437545)(-0.748325892857143,-0.183739867866412)(-0.748186383928571,-0.183373044240109)(-0.748046875,-0.183004407544099)(-0.747907366071429,-0.182633958767372)(-0.747767857142857,-0.182261698902444)(-0.747628348214286,-0.181887628945359)(-0.747488839285714,-0.181511749895682)(-0.747349330357143,-0.181134062756499)(-0.747209821428571,-0.180754568534413)(-0.7470703125,-0.180373268239549)(-0.746930803571429,-0.179990162885543)(-0.746791294642857,-0.179605253489546)(-0.746651785714286,-0.17921854107222)(-0.746512276785714,-0.178830026657735)(-0.746372767857143,-0.178439711273769)(-0.746233258928571,-0.178047595951505)(-0.74609375,-0.17765368172563)(-0.745954241071429,-0.177257969634332)(-0.745814732142857,-0.176860460719297)(-0.745675223214286,-0.176461156025708)(-0.745535714285714,-0.176060056602246)(-0.745396205357143,-0.17565716350108)(-0.745256696428571,-0.175252477777874)(-0.7451171875,-0.174846000491781)(-0.744977678571429,-0.174437732705439)(-0.744838169642857,-0.174027675484971)(-0.744698660714286,-0.173615829899983)(-0.744559151785714,-0.173202197023563)(-0.744419642857143,-0.172786777932275)(-0.744280133928571,-0.172369573706161)(-0.744140625,-0.171950585428738)(-0.744001116071429,-0.171529814186995)(-0.743861607142857,-0.17110726107139)(-0.743722098214286,-0.170682927175849)(-0.743582589285714,-0.170256813597765)(-0.743443080357143,-0.169828921437994)(-0.743303571428571,-0.169399251800854)(-0.7431640625,-0.168967805794123)(-0.743024553571429,-0.168534584529035)(-0.742885044642857,-0.16809958912028)(-0.742745535714286,-0.167662820686)(-0.742606026785714,-0.16722428034779)(-0.742466517857143,-0.166783969230691)(-0.742327008928571,-0.166341888463191)(-0.7421875,-0.165898039177224)(-0.742047991071429,-0.165452422508165)(-0.741908482142857,-0.165005039594828)(-0.741768973214286,-0.164555891579465)(-0.741629464285714,-0.164104979607764)(-0.741489955357143,-0.163652304828846)(-0.741350446428571,-0.163197868395261)(-0.7412109375,-0.162741671462992)(-0.741071428571429,-0.162283715191443)(-0.740931919642857,-0.161824000743447)(-0.740792410714286,-0.161362529285255)(-0.740652901785714,-0.160899301986541)(-0.740513392857143,-0.160434320020394)(-0.740373883928571,-0.159967584563318)(-0.740234375,-0.159499096795232)(-0.740094866071429,-0.159028857899463)(-0.739955357142857,-0.158556869062748)(-0.739815848214286,-0.158083131475228)(-0.739676339285714,-0.15760764633045)(-0.739536830357143,-0.15713041482536)(-0.739397321428571,-0.156651438160302)(-0.7392578125,-0.156170717539022)(-0.739118303571429,-0.155688254168655)(-0.738978794642857,-0.155204049259729)(-0.738839285714286,-0.154718104026161)(-0.738699776785714,-0.154230419685258)(-0.738560267857143,-0.153740997457709)(-0.738420758928571,-0.153249838567584)(-0.73828125,-0.15275694424234)(-0.738141741071429,-0.152262315712803)(-0.738002232142857,-0.15176595421318)(-0.737862723214286,-0.151267860981047)(-0.737723214285714,-0.150768037257355)(-0.737583705357143,-0.150266484286419)(-0.737444196428571,-0.14976320331592)(-0.7373046875,-0.149258195596905)(-0.737165178571429,-0.148751462383779)(-0.737025669642857,-0.148243004934306)(-0.736886160714286,-0.147732824509607)(-0.736746651785714,-0.147220922374154)(-0.736607142857143,-0.146707299795772)(-0.736467633928571,-0.146191958045632)(-0.736328125,-0.145674898398256)(-0.736188616071429,-0.145156122131505)(-0.736049107142857,-0.14463563052658)(-0.735909598214286,-0.144113424868026)(-0.735770089285714,-0.143589506443719)(-0.735630580357143,-0.14306387654487)(-0.735491071428571,-0.142536536466021)(-0.7353515625,-0.142007487505045)(-0.735212053571429,-0.141476730963138)(-0.735072544642857,-0.140944268144818)(-0.734933035714286,-0.140410100357927)(-0.734793526785714,-0.139874228913626)(-0.734654017857143,-0.139336655126388)(-0.734514508928571,-0.138797380314002)(-0.734375,-0.138256405797568)(-0.734235491071429,-0.137713732901492)(-0.734095982142857,-0.137169362953487)(-0.733956473214286,-0.136623297284568)(-0.733816964285714,-0.136075537229052)(-0.733677455357143,-0.135526084124552)(-0.733537946428571,-0.134974939311976)(-0.7333984375,-0.134422104135526)(-0.733258928571429,-0.133867579942694)(-0.733119419642857,-0.133311368084258)(-0.732979910714286,-0.13275346991428)(-0.732840401785714,-0.132193886790106)(-0.732700892857143,-0.131632620072361)(-0.732561383928571,-0.131069671124944)(-0.732421875,-0.130505041315034)(-0.732282366071429,-0.129938732013076)(-0.732142857142857,-0.129370744592785)(-0.732003348214286,-0.128801080431143)(-0.731863839285714,-0.128229740908395)(-0.731724330357143,-0.127656727408047)(-0.731584821428571,-0.127082041316862)(-0.7314453125,-0.12650568402486)(-0.731305803571429,-0.125927656925313)(-0.731166294642857,-0.125347961414741)(-0.731026785714286,-0.124766598892913)(-0.730887276785714,-0.124183570762844)(-0.730747767857143,-0.123598878430788)(-0.730608258928571,-0.123012523306236)(-0.73046875,-0.122424506801923)(-0.730329241071429,-0.121834830333811)(-0.730189732142857,-0.121243495321093)(-0.730050223214286,-0.120650503186193)(-0.729910714285714,-0.120055855354758)(-0.729771205357143,-0.119459553255658)(-0.729631696428571,-0.118861598320982)(-0.7294921875,-0.118261991986038)(-0.729352678571429,-0.117660735689347)(-0.729213169642857,-0.117057830872639)(-0.729073660714286,-0.116453278980855)(-0.728934151785714,-0.115847081462143)(-0.728794642857143,-0.11523923976785)(-0.728655133928571,-0.114629755352524)(-0.728515625,-0.114018629673914)(-0.728376116071429,-0.113405864192959)(-0.728236607142857,-0.11279146037379)(-0.728097098214286,-0.112175419683729)(-0.727957589285714,-0.111557743593283)(-0.727818080357143,-0.11093843357614)(-0.727678571428571,-0.110317491109168)(-0.7275390625,-0.109694917672417)(-0.727399553571429,-0.109070714749107)(-0.727260044642857,-0.108444883825629)(-0.727120535714286,-0.107817426391544)(-0.726981026785714,-0.10718834393958)(-0.726841517857143,-0.106557637965626)(-0.726702008928571,-0.105925309968728)(-0.7265625,-0.105291361451097)(-0.726422991071429,-0.10465579391809)(-0.726283482142857,-0.104018608878218)(-0.726143973214286,-0.103379807843141)(-0.726004464285714,-0.102739392327663)(-0.725864955357143,-0.102097363849731)(-0.725725446428571,-0.101453723930428)(-0.7255859375,-0.10080847409398)(-0.725446428571429,-0.100161615867741)(-0.725306919642857,-0.0995131507821957)(-0.725167410714286,-0.0988630803709581)(-0.725027901785714,-0.0982114061707664)(-0.724888392857143,-0.0975581297214788)(-0.724748883928571,-0.0969032525660711)(-0.724609375,-0.0962467762506398)(-0.724469866071429,-0.0955887023243881)(-0.724330357142857,-0.0949290323396302)(-0.724190848214286,-0.0942677678517877)(-0.724051339285714,-0.0936049104193852)(-0.723911830357143,-0.092940461604047)(-0.723772321428571,-0.092274422970493)(-0.7236328125,-0.0916067960865433)(-0.723493303571429,-0.090937582523103)(-0.723353794642857,-0.0902667838541671)(-0.723214285714286,-0.0895944016568166)(-0.723074776785714,-0.0889204375112145)(-0.722935267857143,-0.0882448930006015)(-0.722795758928571,-0.0875677697112929)(-0.72265625,-0.0868890692326822)(-0.722516741071429,-0.0862087931572274)(-0.722377232142857,-0.0855269430804538)(-0.722237723214286,-0.0848435206009516)(-0.722098214285714,-0.0841585273203709)(-0.721958705357143,-0.0834719648434185)(-0.721819196428571,-0.0827838347778534)(-0.7216796875,-0.0820941387344917)(-0.721540178571429,-0.0814028783271913)(-0.721400669642857,-0.0807100551728562)(-0.721261160714286,-0.0800156708914332)(-0.721121651785714,-0.0793197271059074)(-0.720982142857143,-0.0786222254422977)(-0.720842633928571,-0.0779231675296542)(-0.720703125,-0.0772225550000614)(-0.720563616071429,-0.0765203894886236)(-0.720424107142857,-0.0758166726334693)(-0.720284598214286,-0.0751114060757473)(-0.720145089285714,-0.0744045914596224)(-0.720005580357143,-0.0736962304322717)(-0.719866071428571,-0.07298632464388)(-0.7197265625,-0.0722748757476449)(-0.719587053571429,-0.0715618853997615)(-0.719447544642857,-0.0708473552594262)(-0.719308035714286,-0.0701312869888333)(-0.719168526785714,-0.069413682253171)(-0.719029017857143,-0.0686945427206165)(-0.718889508928571,-0.0679738700623328)(-0.71875,-0.0672516659524729)(-0.718610491071429,-0.0665279320681642)(-0.718470982142857,-0.065802670089513)(-0.718331473214286,-0.065075881699601)(-0.718191964285714,-0.0643475685844802)(-0.718052455357143,-0.0636177324331697)(-0.717912946428571,-0.0628863749376508)(-0.7177734375,-0.0621534977928722)(-0.717633928571429,-0.0614191026967338)(-0.717494419642857,-0.0606831913500918)(-0.717354910714286,-0.0599457654567541)(-0.717215401785714,-0.0592068267234768)(-0.717075892857143,-0.0584663768599588)(-0.716936383928571,-0.0577244175788388)(-0.716796875,-0.0569809505956994)(-0.716657366071429,-0.0562359776290514)(-0.716517857142857,-0.0554895004003384)(-0.716378348214286,-0.0547415206339322)(-0.716238839285714,-0.0539920400571298)(-0.716099330357143,-0.0532410604001473)(-0.715959821428571,-0.0524885833961174)(-0.7158203125,-0.051734610781093)(-0.715680803571429,-0.0509791442940317)(-0.715541294642857,-0.0502221856768004)(-0.715401785714286,-0.0494637366741706)(-0.715262276785714,-0.0487037990338152)(-0.715122767857143,-0.0479423745063026)(-0.714983258928571,-0.0471794648450942)(-0.71484375,-0.0464150718065476)(-0.714704241071429,-0.0456491971499016)(-0.714564732142857,-0.0448818426372798)(-0.714425223214286,-0.0441130100336874)(-0.714285714285714,-0.0433427011070064)(-0.714146205357143,-0.0425709176279908)(-0.714006696428571,-0.041797661370263)(-0.7138671875,-0.0410229341103182)(-0.713727678571429,-0.0402467376275083)(-0.713588169642857,-0.0394690737040462)(-0.713448660714286,-0.0386899441250017)(-0.713309151785714,-0.0379093506782977)(-0.713169642857143,-0.0371272951547047)(-0.713030133928571,-0.0363437793478374)(-0.712890625,-0.0355588050541588)(-0.712751116071429,-0.0347723740729642)(-0.712611607142857,-0.0339844882063857)(-0.712472098214286,-0.033195149259388)(-0.712332589285714,-0.032404359039764)(-0.712193080357143,-0.0316121193581301)(-0.712053571428571,-0.0308184320279219)(-0.7119140625,-0.0300232988653992)(-0.711774553571429,-0.0292267216896293)(-0.711635044642857,-0.0284287023224913)(-0.711495535714286,-0.0276292425886727)(-0.711356026785714,-0.0268283443156642)(-0.711216517857143,-0.026026009333755)(-0.711077008928571,-0.0252222394760292)(-0.7109375,-0.0244170365783697)(-0.710797991071429,-0.0236104024794425)(-0.710658482142857,-0.0228023390207002)(-0.710518973214286,-0.0219928480463787)(-0.710379464285714,-0.0211819314034923)(-0.710239955357143,-0.0203695909418288)(-0.710100446428571,-0.0195558285139456)(-0.7099609375,-0.0187406459751739)(-0.709821428571429,-0.0179240451836028)(-0.709681919642857,-0.0171060280000828)(-0.709542410714286,-0.0162865962882226)(-0.709402901785714,-0.0154657519143837)(-0.709263392857143,-0.0146434967476762)(-0.709123883928571,-0.0138198326599538)(-0.708984375,-0.0129947615258197)(-0.708844866071429,-0.0121682852226084)(-0.708705357142857,-0.0113404056303911)(-0.708565848214286,-0.0105111246319712)(-0.708426339285714,-0.00968044411288005)(-0.708286830357143,-0.00884836596137106)(-0.708147321428571,-0.00801489206841655)(-0.7080078125,-0.00718002432771203)(-0.707868303571429,-0.006343764635659)(-0.707728794642857,-0.00550611489136976)(-0.707589285714286,-0.00466707699666324)(-0.707449776785714,-0.0038266528560601)(-0.707310267857143,-0.00298484437677782)(-0.707170758928571,-0.00214165346872641)(-0.70703125,-0.0012970820445134)(-0.706891741071429,-0.000451132019426259)(-0.706752232142857,0.000396194688562557)(-0.706612723214286,0.00124489615880019)(-0.706473214285714,0.00209497046795792)(-0.706333705357143,0.00294641569003651)(-0.706194196428571,0.00379922989637016)(-0.7060546875,0.00465341115562168)(-0.705915178571429,0.00550895753380032)(-0.705775669642857,0.00636586709425646)(-0.705636160714286,0.0072241378976865)(-0.705496651785714,0.00808376800213706)(-0.705357142857143,0.00894475546301077)(-0.705217633928571,0.00980709833307003)(-0.705078125,0.0106707946624321)(-0.704938616071429,0.0115358424985875)(-0.704799107142857,0.0124022398863942)(-0.704659598214286,0.0132699848680825)(-0.704520089285714,0.0141390754832598)(-0.704380580357143,0.0150095097689158)(-0.704241071428571,0.0158812857594269)(-0.7041015625,0.0167544014865513)(-0.703962053571429,0.0176288549794461)(-0.703822544642857,0.0185046442646638)(-0.703683035714286,0.0193817673661553)(-0.703543526785714,0.0202602223052756)(-0.703404017857143,0.021140007100789)(-0.703264508928571,0.0220211197688732)(-0.703125,0.0229035583231145)(-0.702985491071429,0.0237873207745255)(-0.702845982142857,0.0246724051315407)(-0.702706473214286,0.0255588094000202)(-0.702566964285714,0.0264465315832555)(-0.702427455357143,0.0273355696819743)(-0.702287946428571,0.0282259216943448)(-0.7021484375,0.0291175856159708)(-0.702008928571429,0.03001055943991)(-0.701869419642857,0.0309048411566691)(-0.701729910714286,0.0318004287542076)(-0.701590401785714,0.0326973202179437)(-0.701450892857143,0.0335955135307591)(-0.701311383928571,0.0344950066730034)(-0.701171875,0.0353957976224895)(-0.701032366071429,0.0362978843545112)(-0.700892857142857,0.0372012648418391)(-0.700753348214286,0.0381059370547238)(-0.700613839285714,0.0390118989609025)(-0.700474330357143,0.0399191485256034)(-0.700334821428571,0.0408276837115502)(-0.7001953125,0.0417375024789572)(-0.700055803571429,0.0426486027855479)(-0.699916294642857,0.0435609825865497)(-0.699776785714286,0.0444746398346986)(-0.699637276785714,0.045389572480244)(-0.699497767857143,0.0463057784709546)(-0.699358258928571,0.0472232557521227)(-0.69921875,0.0481420022665586)(-0.699079241071429,0.0490620159546102)(-0.698939732142857,0.0499832947541569)(-0.698800223214286,0.0509058366006145)(-0.698660714285714,0.0518296394269411)(-0.698521205357143,0.0527547011636412)(-0.698381696428571,0.0536810197387716)(-0.6982421875,0.054608593077935)(-0.698102678571429,0.0555374191042995)(-0.697963169642857,0.0564674957385939)(-0.697823660714286,0.0573988208991107)(-0.697684151785714,0.0583313925017132)(-0.697544642857143,0.0592652084598395)(-0.697405133928571,0.0602002666845083)(-0.697265625,0.0611365650843126)(-0.697126116071429,0.0620741015654394)(-0.696986607142857,0.0630128740316638)(-0.696847098214286,0.0639528803843545)(-0.696707589285714,0.0648941185224786)(-0.696568080357143,0.065836586342607)(-0.696428571428571,0.0667802817389193)(-0.6962890625,0.0677252026031979)(-0.696149553571429,0.0686713468248485)(-0.696010044642857,0.0696187122908932)(-0.695870535714286,0.0705672968859764)(-0.695731026785714,0.0715170984923692)(-0.695591517857143,0.072468114989976)(-0.695452008928571,0.0734203442563381)(-0.6953125,0.074373784166629)(-0.695172991071429,0.0753284325936737)(-0.695033482142857,0.0762842874079429)(-0.694893973214286,0.0772413464775583)(-0.694754464285714,0.0781996076682976)(-0.694614955357143,0.0791590688436001)(-0.694475446428571,0.080119727864572)(-0.6943359375,0.0810815825899798)(-0.694196428571429,0.082044630876271)(-0.694056919642857,0.083008870577568)(-0.693917410714286,0.0839742995456731)(-0.693777901785714,0.084940915630074)(-0.693638392857143,0.0859087166779492)(-0.693498883928571,0.0868777005341728)(-0.693359375,0.0878478650413093)(-0.693219866071429,0.0888192080396328)(-0.693080357142857,0.0897917273671222)(-0.692940848214286,0.0907654208594652)(-0.692801339285714,0.0917402863500645)(-0.692661830357143,0.0927163216700428)(-0.692522321428571,0.0936935246482485)(-0.6923828125,0.0946718931112491)(-0.692243303571429,0.0956514248833519)(-0.692103794642857,0.0966321177865976)(-0.691964285714286,0.0976139696407665)(-0.691824776785714,0.0985969782633824)(-0.691685267857143,0.0995811414697197)(-0.691545758928571,0.100566457072808)(-0.69140625,0.101552922883424)(-0.691266741071429,0.102540536710116)(-0.691127232142857,0.103529296359196)(-0.690987723214286,0.104519199634743)(-0.690848214285714,0.10551024433861)(-0.690708705357143,0.106502428270433)(-0.690569196428571,0.107495749227628)(-0.6904296875,0.108490205005395)(-0.690290178571429,0.109485793396728)(-0.690150669642857,0.110482512192421)(-0.690011160714286,0.11148035918106)(-0.689871651785714,0.11247933214904)(-0.689732142857143,0.113479428880564)(-0.689592633928571,0.114480647157649)(-0.689453125,0.115482984760123)(-0.689313616071429,0.116486439465643)(-0.689174107142857,0.117491009049689)(-0.689034598214286,0.118496691285571)(-0.688895089285714,0.119503483944431)(-0.688755580357143,0.120511384795255)(-0.688616071428571,0.121520391604871)(-0.6884765625,0.122530502137948)(-0.688337053571429,0.123541714157014)(-0.688197544642857,0.124554025422451)(-0.688058035714286,0.125567433692503)(-0.687918526785714,0.126581936723274)(-0.687779017857143,0.127597532268744)(-0.687639508928571,0.128614218080765)(-0.6875,0.12963199190906)(-0.687360491071429,0.130650851501242)(-0.687220982142857,0.131670794602811)(-0.687081473214286,0.132691818957154)(-0.686941964285714,0.133713922305555)(-0.686802455357143,0.134737102387199)(-0.686662946428571,0.135761356939178)(-0.6865234375,0.136786683696484)(-0.686383928571429,0.137813080392031)(-0.686244419642857,0.138840544756647)(-0.686104910714286,0.139869074519083)(-0.685965401785714,0.140898667406014)(-0.685825892857143,0.141929321142049)(-0.685686383928571,0.142961033449734)(-0.685546875,0.143993802049547)(-0.685407366071429,0.145027624659916)(-0.685267857142857,0.146062498997218)(-0.685128348214286,0.147098422775782)(-0.684988839285714,0.148135393707892)(-0.684849330357143,0.149173409503797)(-0.684709821428571,0.150212467871715)(-0.6845703125,0.151252566517825)(-0.684430803571429,0.152293703146291)(-0.684291294642857,0.153335875459254)(-0.684151785714286,0.154379081156838)(-0.684012276785714,0.155423317937155)(-0.683872767857143,0.156468583496315)(-0.683733258928571,0.157514875528422)(-0.68359375,0.158562191725579)(-0.683454241071429,0.159610529777902)(-0.683314732142857,0.160659887373516)(-0.683175223214286,0.161710262198561)(-0.683035714285714,0.162761651937195)(-0.682896205357143,0.163814054271606)(-0.682756696428571,0.16486746688201)(-0.6826171875,0.16592188744665)(-0.682477678571429,0.166977313641814)(-0.682338169642857,0.168033743141833)(-0.682198660714286,0.169091173619083)(-0.682059151785714,0.17014960274399)(-0.681919642857143,0.17120902818504)(-0.681780133928571,0.172269447608781)(-0.681640625,0.173330858679819)(-0.681501116071429,0.174393259060837)(-0.681361607142857,0.175456646412591)(-0.681222098214286,0.176521018393915)(-0.681082589285714,0.177586372661727)(-0.680943080357143,0.178652706871034)(-0.680803571428571,0.179720018674936)(-0.6806640625,0.180788305724627)(-0.680524553571429,0.181857565669405)(-0.680385044642857,0.182927796156677)(-0.680245535714286,0.183998994831959)(-0.680106026785714,0.18507115933888)(-0.679966517857143,0.186144287319194)(-0.679827008928571,0.187218376412779)(-0.6796875,0.188293424257637)(-0.679547991071429,0.189369428489909)(-0.679408482142857,0.190446386743875)(-0.679268973214286,0.191524296651957)(-0.679129464285714,0.192603155844722)(-0.678989955357143,0.193682961950895)(-0.678850446428571,0.194763712597356)(-0.6787109375,0.195845405409141)(-0.678571428571429,0.19692803800946)(-0.678431919642857,0.198011608019692)(-0.678292410714286,0.199096113059388)(-0.678152901785714,0.200181550746283)(-0.678013392857143,0.201267918696295)(-0.677873883928571,0.202355214523536)(-0.677734375,0.203443435840301)(-0.677594866071429,0.204532580257094)(-0.677455357142857,0.20562264538262)(-0.677315848214286,0.206713628823791)(-0.677176339285714,0.207805528185731)(-0.677036830357143,0.208898341071783)(-0.676897321428571,0.209992065083513)(-0.6767578125,0.211086697820709)(-0.676618303571429,0.212182236881393)(-0.676478794642857,0.213278679861827)(-0.676339285714286,0.214376024356508)(-0.676199776785714,0.21547426795818)(-0.676060267857143,0.216573408257839)(-0.675920758928571,0.217673442844736)(-0.67578125,0.218774369306376)(-0.675641741071429,0.219876185228533)(-0.675502232142857,0.220978888195252)(-0.675362723214286,0.222082475788846)(-0.675223214285714,0.223186945589909)(-0.675083705357143,0.224292295177317)(-0.674944196428571,0.22539852212824)(-0.6748046875,0.226505624018127)(-0.674665178571429,0.227613598420735)(-0.674525669642857,0.228722442908122)(-0.674386160714286,0.229832155050649)(-0.674246651785714,0.230942732416988)(-0.674107142857143,0.232054172574132)(-0.673967633928571,0.233166473087392)(-0.673828125,0.234279631520399)(-0.673688616071429,0.235393645435122)(-0.673549107142857,0.236508512391865)(-0.673409598214286,0.237624229949266)(-0.673270089285714,0.23874079566431)(-0.673130580357143,0.239858207092334)(-0.672991071428571,0.240976461787029)(-0.6728515625,0.242095557300437)(-0.672712053571429,0.243215491182972)(-0.672572544642857,0.244336260983414)(-0.672433035714286,0.245457864248914)(-0.672293526785714,0.246580298525003)(-0.672154017857143,0.247703561355595)(-0.672014508928571,0.248827650282992)(-0.671875,0.249952562847881)(-0.671735491071429,0.251078296589355)(-0.671595982142857,0.252204849044906)(-0.671456473214286,0.253332217750431)(-0.671316964285714,0.254460400240239)(-0.671177455357143,0.255589394047055)(-0.671037946428571,0.25671919670203)(-0.6708984375,0.257849805734729)(-0.670758928571429,0.258981218673157)(-0.670619419642857,0.260113433043755)(-0.670479910714286,0.261246446371398)(-0.670340401785714,0.26238025617941)(-0.670200892857143,0.263514859989565)(-0.670061383928571,0.264650255322094)(-0.669921875,0.265786439695677)(-0.669782366071429,0.266923410627469)(-0.669642857142857,0.268061165633091)(-0.669503348214286,0.269199702226639)(-0.669363839285714,0.270339017920683)(-0.669224330357143,0.271479110226283)(-0.669084821428571,0.272619976652986)(-0.6689453125,0.273761614708826)(-0.668805803571429,0.274904021900343)(-0.668666294642857,0.276047195732579)(-0.668526785714286,0.277191133709079)(-0.668387276785714,0.278335833331905)(-0.668247767857143,0.279481292101636)(-0.668108258928571,0.280627507517376)(-0.66796875,0.281774477076747)(-0.667829241071429,0.282922198275913)(-0.667689732142857,0.284070668609572)(-0.667550223214286,0.285219885570964)(-0.667410714285714,0.286369846651874)(-0.667271205357143,0.28752054934264)(-0.667131696428571,0.288671991132163)(-0.6669921875,0.28982416950789)(-0.666852678571429,0.290977081955849)(-0.666713169642857,0.292130725960635)(-0.666573660714286,0.293285099005416)(-0.666434151785714,0.294440198571944)(-0.666294642857143,0.295596022140556)(-0.666155133928571,0.296752567190185)(-0.666015625,0.297909831198347)(-0.665876116071429,0.299067811641171)(-0.665736607142857,0.300226505993389)(-0.665597098214286,0.30138591172834)(-0.665457589285714,0.30254602631798)(-0.665318080357143,0.303706847232886)(-0.665178571428571,0.304868371942264)(-0.6650390625,0.30603059791394)(-0.664899553571429,0.307193522614384)(-0.664760044642857,0.308357143508705)(-0.664620535714286,0.309521458060655)(-0.664481026785714,0.310686463732634)(-0.664341517857143,0.311852157985702)(-0.664202008928571,0.313018538279579)(-0.6640625,0.314185602072641)(-0.663922991071429,0.315353346821942)(-0.663783482142857,0.31652176998321)(-0.663643973214286,0.31769086901085)(-0.663504464285714,0.318860641357951)(-0.663364955357143,0.320031084476296)(-0.663225446428571,0.32120219581636)(-0.6630859375,0.322373972827313)(-0.662946428571429,0.323546412957036)(-0.662806919642857,0.324719513652118)(-0.662667410714286,0.32589327235786)(-0.662527901785714,0.327067686518283)(-0.662388392857143,0.328242753576134)(-0.662248883928571,0.32941847097289)(-0.662109375,0.330594836148756)(-0.661969866071429,0.331771846542683)(-0.661830357142857,0.332949499592365)(-0.661690848214286,0.334127792734244)(-0.661551339285714,0.335306723403513)(-0.661411830357143,0.336486289034131)(-0.661272321428571,0.33766648705882)(-0.6611328125,0.338847314909063)(-0.660993303571429,0.340028770015126)(-0.660853794642857,0.341210849806051)(-0.660714285714286,0.342393551709666)(-0.660574776785714,0.343576873152583)(-0.660435267857143,0.344760811560216)(-0.660295758928571,0.345945364356774)(-0.66015625,0.347130528965267)(-0.660016741071429,0.348316302807519)(-0.659877232142857,0.349502683304168)(-0.659737723214286,0.350689667874669)(-0.659598214285714,0.351877253937302)(-0.659458705357143,0.353065438909178)(-0.659319196428571,0.354254220206245)(-0.6591796875,0.355443595243278)(-0.659040178571429,0.356633561433909)(-0.658900669642857,0.357824116190617)(-0.658761160714286,0.359015256924733)(-0.658621651785714,0.360206981046447)(-0.658482142857143,0.361399285964816)(-0.658342633928571,0.362592169087769)(-0.658203125,0.363785627822099)(-0.658063616071429,0.364979659573491)(-0.657924107142857,0.366174261746508)(-0.657784598214286,0.367369431744606)(-0.657645089285714,0.368565166970133)(-0.657505580357143,0.369761464824338)(-0.657366071428571,0.370958322707379)(-0.6572265625,0.372155738018315)(-0.657087053571429,0.373353708155127)(-0.656947544642857,0.374552230514716)(-0.656808035714286,0.375751302492906)(-0.656668526785714,0.376950921484452)(-0.656529017857143,0.378151084883044)(-0.656389508928571,0.379351790081316)(-0.65625,0.380553034470839)(-0.656110491071429,0.381754815442142)(-0.655970982142857,0.38295713038471)(-0.655831473214286,0.384159976686983)(-0.655691964285714,0.385363351736372)(-0.655552455357143,0.386567252919257)(-0.655412946428571,0.387771677620997)(-0.6552734375,0.388976623225923)(-0.655133928571429,0.390182087117361)(-0.654994419642857,0.391388066677627)(-0.654854910714286,0.392594559288032)(-0.654715401785714,0.393801562328886)(-0.654575892857143,0.39500907317951)(-0.654436383928571,0.396217089218237)(-0.654296875,0.397425607822409)(-0.654157366071429,0.398634626368399)(-0.654017857142857,0.399844142231602)(-0.653878348214286,0.401054152786447)(-0.653738839285714,0.402264655406399)(-0.653599330357143,0.403475647463966)(-0.653459821428571,0.404687126330707)(-0.6533203125,0.405899089377224)(-0.653180803571429,0.407111533973184)(-0.653041294642857,0.408324457487317)(-0.652901785714286,0.409537857287419)(-0.652762276785714,0.410751730740358)(-0.652622767857143,0.411966075212082)(-0.652483258928571,0.413180888067624)(-0.65234375,0.414396166671098)(-0.652204241071429,0.415611908385718)(-0.652064732142857,0.416828110573798)(-0.651925223214286,0.418044770596752)(-0.651785714285714,0.419261885815101)(-0.651646205357143,0.420479453588485)(-0.651506696428571,0.421697471275665)(-0.6513671875,0.422915936234516)(-0.651227678571429,0.424134845822051)(-0.651088169642857,0.425354197394418)(-0.650948660714286,0.4265739883069)(-0.650809151785714,0.427794215913928)(-0.650669642857143,0.429014877569083)(-0.650530133928571,0.430235970625102)(-0.650390625,0.431457492433877)(-0.650251116071429,0.432679440346472)(-0.650111607142857,0.433901811713119)(-0.649972098214286,0.435124603883226)(-0.649832589285714,0.436347814205382)(-0.649693080357143,0.437571440027362)(-0.649553571428571,0.438795478696136)(-0.6494140625,0.440019927557861)(-0.649274553571429,0.441244783957904)(-0.649135044642857,0.442470045240838)(-0.648995535714286,0.443695708750446)(-0.648856026785714,0.444921771829727)(-0.648716517857143,0.446148231820906)(-0.648577008928571,0.447375086065435)(-0.6484375,0.448602331903992)(-0.648297991071429,0.449829966676501)(-0.648158482142857,0.451057987722127)(-0.648018973214286,0.452286392379281)(-0.647879464285714,0.453515177985627)(-0.647739955357143,0.454744341878092)(-0.647600446428571,0.455973881392866)(-0.6474609375,0.457203793865399)(-0.647321428571429,0.458434076630427)(-0.647181919642857,0.459664727021959)(-0.647042410714286,0.46089574237329)(-0.646902901785714,0.462127120017004)(-0.646763392857143,0.463358857284981)(-0.646623883928571,0.464590951508403)(-0.646484375,0.465823400017749)(-0.646344866071429,0.467056200142817)(-0.646205357142857,0.468289349212719)(-0.646065848214286,0.469522844555886)(-0.645926339285714,0.470756683500074)(-0.645786830357143,0.471990863372374)(-0.645647321428571,0.473225381499214)(-0.6455078125,0.474460235206355)(-0.645368303571429,0.475695421818914)(-0.645228794642857,0.476930938661358)(-0.645089285714286,0.47816678305751)(-0.644949776785714,0.479402952330555)(-0.644810267857143,0.480639443803046)(-0.644670758928571,0.481876254796915)(-0.64453125,0.483113382633458)(-0.644391741071429,0.484350824633367)(-0.644252232142857,0.485588578116719)(-0.644112723214286,0.486826640402984)(-0.643973214285714,0.488065008811031)(-0.643833705357143,0.489303680659132)(-0.643694196428571,0.490542653264976)(-0.6435546875,0.491781923945652)(-0.643415178571429,0.493021490017681)(-0.643275669642857,0.494261348797006)(-0.643136160714286,0.495501497599)(-0.642996651785714,0.49674193373847)(-0.642857142857143,0.497982654529665)(-0.642717633928571,0.499223657286285)(-0.642578125,0.500464939321469)(-0.642438616071429,0.501706497947822)(-0.642299107142857,0.50294833047741)(-0.642159598214286,0.504190434221764)(-0.642020089285714,0.505432806491887)(-0.641880580357143,0.506675444598259)(-0.641741071428571,0.507918345850847)(-0.6416015625,0.509161507559097)(-0.641462053571429,0.510404927031956)(-0.641322544642857,0.511648601577867)(-0.641183035714286,0.512892528504776)(-0.641043526785714,0.514136705120138)(-0.640904017857143,0.515381128730921)(-0.640764508928571,0.516625796643617)(-0.640625,0.517870706164234)(-0.640485491071429,0.519115854598317)(-0.640345982142857,0.520361239250944)(-0.640206473214286,0.521606857426732)(-0.640066964285714,0.522852706429844)(-0.639927455357143,0.524098783563996)(-0.639787946428571,0.525345086132459)(-0.6396484375,0.52659161143806)(-0.639508928571429,0.5278383567832)(-0.639369419642857,0.529085319469848)(-0.639229910714286,0.530332496799551)(-0.639090401785714,0.531579886073437)(-0.638950892857143,0.532827484592222)(-0.638811383928571,0.53407528965622)(-0.638671875,0.535323298565331)(-0.638532366071429,0.536571508619068)(-0.638392857142857,0.537819917116551)(-0.638253348214286,0.539068521356513)(-0.638113839285714,0.540317318637304)(-0.637974330357143,0.5415663062569)(-0.637834821428571,0.542815481512911)(-0.6376953125,0.54406484170257)(-0.637555803571429,0.545314384122762)(-0.637416294642857,0.546564106070012)(-0.637276785714286,0.547814004840497)(-0.637137276785714,0.549064077730047)(-0.636997767857143,0.550314322034157)(-0.636858258928571,0.55156473504799)(-0.63671875,0.55281531406637)(-0.636579241071429,0.554066056383809)(-0.636439732142857,0.555316959294499)(-0.636300223214286,0.556568020092316)(-0.636160714285714,0.55781923607083)(-0.636021205357143,0.55907060452331)(-0.635881696428571,0.560322122742732)(-0.6357421875,0.561573788021769)(-0.635602678571429,0.562825597652819)(-0.635463169642857,0.564077548927995)(-0.635323660714286,0.565329639139135)(-0.635184151785714,0.566581865577804)(-0.635044642857143,0.567834225535306)(-0.634905133928571,0.569086716302685)(-0.634765625,0.570339335170722)(-0.634626116071429,0.57159207942996)(-0.634486607142857,0.572844946370692)(-0.634347098214286,0.574097933282973)(-0.634207589285714,0.575351037456625)(-0.634068080357143,0.576604256181242)(-0.633928571428571,0.577857586746197)(-0.6337890625,0.579111026440637)(-0.633649553571429,0.580364572553507)(-0.633510044642857,0.581618222373539)(-0.633370535714286,0.582871973189266)(-0.633231026785714,0.584125822289021)(-0.633091517857143,0.585379766960949)(-0.632952008928571,0.586633804493011)(-0.6328125,0.587887932172978)(-0.632672991071429,0.589142147288453)(-0.632533482142857,0.590396447126871)(-0.632393973214286,0.591650828975495)(-0.632254464285714,0.592905290121433)(-0.632114955357143,0.594159827851638)(-0.631975446428571,0.595414439452916)(-0.6318359375,0.59666912221192)(-0.631696428571429,0.597923873415176)(-0.631556919642857,0.599178690349072)(-0.631417410714286,0.600433570299868)(-0.631277901785714,0.601688510553699)(-0.631138392857143,0.602943508396587)(-0.630998883928571,0.604198561114443)(-0.630859375,0.605453665993063)(-0.630719866071429,0.606708820318148)(-0.630580357142857,0.607964021375303)(-0.630440848214286,0.60921926645004)(-0.630301339285714,0.610474552827784)(-0.630161830357143,0.611729877793884)(-0.630022321428571,0.612985238633613)(-0.6298828125,0.614240632632167)(-0.629743303571429,0.615496057074685)(-0.629603794642857,0.616751509246248)(-0.629464285714286,0.618006986431876)(-0.629324776785714,0.619262485916545)(-0.629185267857143,0.620518004985187)(-0.629045758928571,0.621773540922699)(-0.62890625,0.623029091013935)(-0.628766741071429,0.624284652543732)(-0.628627232142857,0.6255402227969)(-0.628487723214286,0.626795799058235)(-0.628348214285714,0.628051378612516)(-0.628208705357143,0.62930695874452)(-0.628069196428571,0.630562536739027)(-0.6279296875,0.631818109880806)(-0.627790178571429,0.633073675454652)(-0.627650669642857,0.634329230745366)(-0.627511160714286,0.635584773037772)(-0.627371651785714,0.636840299616718)(-0.627232142857143,0.638095807767082)(-0.627092633928571,0.639351294773782)(-0.626953125,0.640606757921769)(-0.626813616071429,0.641862194496049)(-0.626674107142857,0.643117601781675)(-0.626534598214286,0.644372977063758)(-0.626395089285714,0.645628317627471)(-0.626255580357143,0.646883620758056)(-0.626116071428571,0.648138883740829)(-0.6259765625,0.649394103861176)(-0.625837053571429,0.650649278404577)(-0.625697544642857,0.651904404656595)(-0.625558035714286,0.653159479902888)(-0.625418526785714,0.654414501429212)(-0.625279017857143,0.655669466521431)(-0.625139508928571,0.656924372465519)(-0.625,0.658179216547557)(-0.624860491071429,0.659433996053755)(-0.624720982142857,0.660688708270446)(-0.624581473214286,0.661943350484095)(-0.624441964285714,0.6631979199813)(-0.624302455357143,0.664452414048804)(-0.624162946428571,0.665706829973499)(-0.6240234375,0.666961165042418)(-0.623883928571429,0.668215416542763)(-0.623744419642857,0.669469581761896)(-0.623604910714286,0.670723657987343)(-0.623465401785714,0.671977642506806)(-0.623325892857143,0.673231532608166)(-0.623186383928571,0.674485325579491)(-0.623046875,0.675739018709026)(-0.622907366071429,0.676992609285222)(-0.622767857142857,0.678246094596728)(-0.622628348214286,0.679499471932393)(-0.622488839285714,0.68075273858128)(-0.622349330357143,0.682005891832665)(-0.622209821428571,0.683258928976052)(-0.6220703125,0.684511847301157)(-0.621930803571429,0.685764644097938)(-0.621791294642857,0.687017316656589)(-0.621651785714286,0.688269862267543)(-0.621512276785714,0.689522278221478)(-0.621372767857143,0.69077456180933)(-0.621233258928571,0.692026710322292)(-0.62109375,0.693278721051811)(-0.620954241071429,0.694530591289612)(-0.620814732142857,0.695782318327691)(-0.620675223214286,0.697033899458321)(-0.620535714285714,0.69828533197406)(-0.620396205357143,0.699536613167755)(-0.620256696428571,0.70078774033255)(-0.6201171875,0.702038710761882)(-0.619977678571429,0.7032895217495)(-0.619838169642857,0.70454017058946)(-0.619698660714286,0.705790654576136)(-0.619559151785714,0.707040971004219)(-0.619419642857143,0.708291117168729)(-0.619280133928571,0.70954109036502)(-0.619140625,0.710790887888773)(-0.619001116071429,0.712040507036021)(-0.618861607142857,0.713289945103141)(-0.618722098214286,0.714539199386861)(-0.618582589285714,0.715788267184266)(-0.618443080357143,0.717037145792808)(-0.618303571428571,0.718285832510306)(-0.6181640625,0.719534324634946)(-0.618024553571429,0.720782619465301)(-0.617885044642857,0.722030714300325)(-0.617745535714286,0.723278606439361)(-0.617606026785714,0.724526293182145)(-0.617466517857143,0.725773771828817)(-0.617327008928571,0.727021039679921)(-0.6171875,0.728268094036404)(-0.617047991071429,0.729514932199637)(-0.616908482142857,0.73076155147141)(-0.616768973214286,0.732007949153937)(-0.616629464285714,0.733254122549864)(-0.616489955357143,0.734500068962274)(-0.616350446428571,0.735745785694695)(-0.6162109375,0.736991270051092)(-0.616071428571429,0.738236519335893)(-0.615931919642857,0.739481530853979)(-0.615792410714286,0.740726301910694)(-0.615652901785714,0.741970829811849)(-0.615513392857143,0.74321511186373)(-0.615373883928571,0.744459145373104)(-0.615234375,0.745702927647213)(-0.615094866071429,0.746946455993797)(-0.614955357142857,0.748189727721086)(-0.614815848214286,0.749432740137811)(-0.614676339285714,0.750675490553205)(-0.614536830357143,0.751917976277014)(-0.614397321428571,0.753160194619502)(-0.6142578125,0.754402142891442)(-0.614118303571429,0.755643818404143)(-0.613978794642857,0.756885218469444)(-0.613839285714286,0.758126340399717)(-0.613699776785714,0.759367181507875)(-0.613560267857143,0.760607739107379)(-0.613420758928571,0.761848010512246)(-0.61328125,0.763087993037039)(-0.613141741071429,0.764327683996893)(-0.613002232142857,0.765567080707508)(-0.612862723214286,0.766806180485156)(-0.612723214285714,0.768044980646685)(-0.612583705357143,0.76928347850953)(-0.612444196428571,0.770521671391715)(-0.6123046875,0.77175955661185)(-0.612165178571429,0.77299713148915)(-0.612025669642857,0.774234393343435)(-0.611886160714286,0.775471339495131)(-0.611746651785714,0.776707967265279)(-0.611607142857143,0.777944273975541)(-0.611467633928571,0.779180256948206)(-0.611328125,0.780415913506183)(-0.611188616071429,0.781651240973028)(-0.611049107142857,0.782886236672933)(-0.610909598214286,0.784120897930735)(-0.610770089285714,0.785355222071922)(-0.610630580357143,0.78658920642264)(-0.610491071428571,0.787822848309696)(-0.6103515625,0.789056145060559)(-0.610212053571429,0.790289094003374)(-0.610072544642857,0.791521692466965)(-0.609933035714286,0.792753937780833)(-0.609793526785714,0.793985827275169)(-0.609654017857143,0.795217358280856)(-0.609514508928571,0.796448528129479)(-0.609375,0.797679334153315)(-0.609235491071429,0.798909773685359)(-0.609095982142857,0.800139844059317)(-0.608956473214286,0.801369542609612)(-0.608816964285714,0.802598866671391)(-0.608677455357143,0.80382781358053)(-0.608537946428571,0.805056380673643)(-0.6083984375,0.806284565288073)(-0.608258928571429,0.807512364761918)(-0.608119419642857,0.808739776434021)(-0.607979910714286,0.80996679764398)(-0.607840401785714,0.811193425732153)(-0.607700892857143,0.812419658039662)(-0.607561383928571,0.813645491908407)(-0.607421875,0.814870924681047)(-0.607282366071429,0.816095953701038)(-0.607142857142857,0.817320576312616)(-0.607003348214286,0.818544789860805)(-0.606863839285714,0.819768591691428)(-0.606724330357143,0.820991979151109)(-0.606584821428571,0.822214949587283)(-0.6064453125,0.823437500348185)(-0.606305803571429,0.824659628782877)(-0.606166294642857,0.825881332241241)(-0.606026785714286,0.827102608073983)(-0.605887276785714,0.828323453632645)(-0.605747767857143,0.829543866269603)(-0.605608258928571,0.830763843338081)(-0.60546875,0.831983382192142)(-0.605329241071429,0.833202480186709)(-0.605189732142857,0.834421134677561)(-0.605050223214286,0.835639343021341)(-0.604910714285714,0.836857102575558)(-0.604771205357143,0.838074410698596)(-0.604631696428571,0.839291264749722)(-0.6044921875,0.840507662089075)(-0.604352678571429,0.841723600077694)(-0.604213169642857,0.842939076077508)(-0.604073660714286,0.844154087451345)(-0.603934151785714,0.845368631562937)(-0.603794642857143,0.846582705776926)(-0.603655133928571,0.847796307458873)(-0.603515625,0.849009433975248)(-0.603376116071429,0.850222082693454)(-0.603236607142857,0.851434250981824)(-0.603097098214286,0.852645936209622)(-0.602957589285714,0.853857135747055)(-0.602818080357143,0.855067846965274)(-0.602678571428571,0.856278067236384)(-0.6025390625,0.857487793933437)(-0.602399553571429,0.858697024430454)(-0.602260044642857,0.859905756102419)(-0.602120535714286,0.861113986325287)(-0.601981026785714,0.862321712475986)(-0.601841517857143,0.86352893193243)(-0.601702008928571,0.86473564207352)(-0.6015625,0.865941840279138)(-0.601422991071429,0.867147523930172)(-0.601283482142857,0.868352690408512)(-0.601143973214286,0.869557337097048)(-0.601004464285714,0.870761461379687)(-0.600864955357143,0.871965060641349)(-0.600725446428571,0.873168132267983)(-0.6005859375,0.874370673646553)(-0.600446428571429,0.875572682165063)(-0.600306919642857,0.876774155212556)(-0.600167410714286,0.877975090179112)(-0.600027901785714,0.879175484455859)(-0.599888392857143,0.880375335434979)(-0.599748883928571,0.881574640509716)(-0.599609375,0.882773397074363)(-0.599469866071429,0.883971602524293)(-0.599330357142857,0.885169254255949)(-0.599190848214286,0.886366349666848)(-0.599051339285714,0.887562886155592)(-0.598911830357143,0.888758861121871)(-0.598772321428571,0.889954271966472)(-0.5986328125,0.891149116091268)(-0.598493303571429,0.892343390899247)(-0.598353794642857,0.893537093794501)(-0.598214285714286,0.894730222182233)(-0.598074776785714,0.895922773468766)(-0.597935267857143,0.897114745061548)(-0.597795758928571,0.898306134369154)(-0.59765625,0.899496938801288)(-0.597516741071429,0.900687155768797)(-0.597377232142857,0.901876782683672)(-0.597237723214286,0.903065816959049)(-0.597098214285714,0.90425425600922)(-0.596958705357143,0.905442097249635)(-0.596819196428571,0.90662933809691)(-0.5966796875,0.907815975968819)(-0.596540178571429,0.909002008284323)(-0.596400669642857,0.910187432463555)(-0.596261160714286,0.911372245927832)(-0.596121651785714,0.912556446099659)(-0.595982142857143,0.913740030402736)(-0.595842633928571,0.914922996261966)(-0.595703125,0.916105341103443)(-0.595563616071429,0.917287062354482)(-0.595424107142857,0.918468157443607)(-0.595284598214286,0.919648623800562)(-0.595145089285714,0.920828458856311)(-0.595005580357143,0.922007660043051)(-0.594866071428571,0.923186224794214)(-0.5947265625,0.924364150544461)(-0.594587053571429,0.925541434729708)(-0.594447544642857,0.926718074787114)(-0.594308035714286,0.927894068155094)(-0.594168526785714,0.929069412273317)(-0.594029017857143,0.930244104582722)(-0.593889508928571,0.931418142525516)(-0.59375,0.93259152354517)(-0.593610491071429,0.933764245086446)(-0.593470982142857,0.934936304595382)(-0.593331473214286,0.936107699519309)(-0.593191964285714,0.937278427306846)(-0.593052455357143,0.938448485407915)(-0.592912946428571,0.939617871273743)(-0.5927734375,0.940786582356858)(-0.592633928571429,0.941954616111106)(-0.592494419642857,0.943121969991654)(-0.592354910714286,0.944288641454987)(-0.592215401785714,0.94545462795892)(-0.592075892857143,0.946619926962601)(-0.591936383928571,0.947784535926522)(-0.591796875,0.948948452312505)(-0.591657366071429,0.950111673583731)(-0.591517857142857,0.95127419720473)(-0.591378348214286,0.95243602064139)(-0.591238839285714,0.953597141360962)(-0.591099330357143,0.954757556832065)(-0.590959821428571,0.955917264524692)(-0.5908203125,0.957076261910207)(-0.590680803571429,0.958234546461363)(-0.590541294642857,0.959392115652301)(-0.590401785714286,0.96054896695855)(-0.590262276785714,0.961705097857037)(-0.590122767857143,0.962860505826094)(-0.589983258928571,0.96401518834546)(-0.58984375,0.965169142896279)(-0.589704241071429,0.966322366961119)(-0.589564732142857,0.967474858023969)(-0.589425223214286,0.968626613570243)(-0.589285714285714,0.969777631086785)(-0.589146205357143,0.970927908061879)(-0.589006696428571,0.972077441985251)(-0.5888671875,0.973226230348064)(-0.588727678571429,0.974374270642942)(-0.588588169642857,0.975521560363964)(-0.588448660714286,0.976668097006664)(-0.588309151785714,0.977813878068047)(-0.588169642857143,0.978958901046586)(-0.588030133928571,0.980103163442234)(-0.587890625,0.981246662756414)(-0.587751116071429,0.982389396492044)(-0.587611607142857,0.983531362153529)(-0.587472098214286,0.984672557246768)(-0.587332589285714,0.98581297927916)(-0.587193080357143,0.98695262575961)(-0.587053571428571,0.988091494198533)(-0.5869140625,0.989229582107851)(-0.586774553571429,0.990366887001015)(-0.586635044642857,0.991503406392995)(-0.586495535714286,0.99263913780029)(-0.586356026785714,0.993774078740933)(-0.586216517857143,0.994908226734496)(-0.586077008928571,0.996041579302097)(-0.5859375,0.997174133966394)(-0.585797991071429,0.998305888251603)(-0.585658482142857,0.999436839683502)(-0.585518973214286,1.00056698578942)(-0.585379464285714,1.00169632409827)(-0.585239955357143,1.00282485214053)(-0.585100446428571,1.00395256744824)(-0.5849609375,1.00507946755504)(-0.584821428571429,1.00620554999615)(-0.584681919642857,1.00733081230839)(-0.584542410714286,1.00845525203016)(-0.584402901785714,1.00957886670148)(-0.584263392857143,1.01070165386395)(-0.584123883928571,1.01182361106081)(-0.583984375,1.0129447358369)(-0.583844866071429,1.01406502573867)(-0.583705357142857,1.01518447831423)(-0.583565848214286,1.01630309111327)(-0.583426339285714,1.01742086168715)(-0.583286830357143,1.01853778758887)(-0.583147321428571,1.01965386637305)(-0.5830078125,1.02076909559598)(-0.582868303571429,1.0218834728156)(-0.582728794642857,1.02299699559149)(-0.582589285714286,1.02410966148492)(-0.582449776785714,1.02522146805881)(-0.582310267857143,1.02633241287776)(-0.582170758928571,1.02744249350804)(-0.58203125,1.02855170751761)(-0.581891741071429,1.02966005247611)(-0.581752232142857,1.03076752595488)(-0.581612723214286,1.03187412552695)(-0.581473214285714,1.03297984876705)(-0.581333705357143,1.03408469325162)(-0.581194196428571,1.0351886565588)(-0.5810546875,1.03629173626845)(-0.580915178571429,1.03739392996217)(-0.580775669642857,1.03849523522325)(-0.580636160714286,1.03959564963674)(-0.580496651785714,1.04069517078939)(-0.580357142857143,1.04179379626973)(-0.580217633928571,1.042891523668)(-0.580078125,1.0439883505762)(-0.579938616071429,1.04508427458808)(-0.579799107142857,1.04617929329915)(-0.579659598214286,1.04727340430669)(-0.579520089285714,1.04836660520972)(-0.579380580357143,1.04945889360907)(-0.579241071428571,1.05055026710731)(-0.5791015625,1.05164072330881)(-0.578962053571429,1.05273025981973)(-0.578822544642857,1.053818874248)(-0.578683035714286,1.05490656420336)(-0.578543526785714,1.05599332729734)(-0.578404017857143,1.05707916114328)(-0.578264508928571,1.05816406335634)(-0.578125,1.05924803155346)(-0.577985491071429,1.06033106335343)(-0.577845982142857,1.06141315637686)(-0.577706473214286,1.06249430824616)(-0.577566964285714,1.0635745165856)(-0.577427455357143,1.06465377902128)(-0.577287946428571,1.06573209318113)(-0.5771484375,1.06680945669493)(-0.577008928571429,1.06788586719432)(-0.576869419642857,1.06896132231279)(-0.576729910714286,1.07003581968568)(-0.576590401785714,1.0711093569502)(-0.576450892857143,1.07218193174544)(-0.576311383928571,1.07325354171234)(-0.576171875,1.07432418449374)(-0.576032366071429,1.07539385773436)(-0.575892857142857,1.07646255908079)(-0.575753348214286,1.07753028618152)(-0.575613839285714,1.07859703668694)(-0.575474330357143,1.07966280824935)(-0.575334821428571,1.08072759852293)(-0.5751953125,1.08179140516379)(-0.575055803571429,1.08285422582995)(-0.574916294642857,1.08391605818134)(-0.574776785714286,1.08497689987983)(-0.574637276785714,1.08603674858921)(-0.574497767857143,1.08709560197519)(-0.574358258928571,1.08815345770544)(-0.57421875,1.08921031344956)(-0.574079241071429,1.09026616687908)(-0.573939732142857,1.09132101566752)(-0.573800223214286,1.09237485749032)(-0.573660714285714,1.0934276900249)(-0.573521205357143,1.09447951095063)(-0.573381696428571,1.09553031794887)(-0.5732421875,1.09658010870291)(-0.573102678571429,1.09762888089808)(-0.572963169642857,1.09867663222164)(-0.572823660714286,1.09972336036286)(-0.572684151785714,1.100769063013)(-0.572544642857143,1.10181373786532)(-0.572405133928571,1.10285738261506)(-0.572265625,1.10389999495949)(-0.572126116071429,1.10494157259787)(-0.571986607142857,1.1059821132315)(-0.571847098214286,1.10702161456365)(-0.571707589285714,1.10806007429967)(-0.571568080357143,1.1090974901469)(-0.571428571428572,1.11013385981473)(-0.5712890625,1.11116918101456)(-0.571149553571429,1.11220345145987)(-0.571010044642857,1.11323666886615)(-0.570870535714286,1.11426883095096)(-0.570731026785714,1.1152999354339)(-0.570591517857143,1.11632998003664)(-0.570452008928572,1.11735896248292)(-0.5703125,1.11838688049852)(-0.570172991071429,1.11941373181131)(-0.570033482142857,1.12043951415124)(-0.569893973214286,1.12146422525033)(-0.569754464285714,1.12248786284269)(-0.569614955357143,1.12351042466452)(-0.569475446428572,1.12453190845411)(-0.5693359375,1.12555231195185)(-0.569196428571429,1.12657163290023)(-0.569056919642857,1.12758986904386)(-0.568917410714286,1.12860701812944)(-0.568777901785714,1.1296230779058)(-0.568638392857143,1.13063804612389)(-0.568498883928572,1.13165192053678)(-0.568359375,1.13266469889968)(-0.568219866071429,1.1336763789699)(-0.568080357142857,1.13468695850693)(-0.567940848214286,1.13569643527237)(-0.567801339285714,1.13670480702999)(-0.567661830357143,1.1377120715457)(-0.567522321428572,1.13871822658756)(-0.5673828125,1.13972326992579)(-0.567243303571429,1.14072719933279)(-0.567103794642857,1.1417300125831)(-0.566964285714286,1.14273170745347)(-0.566824776785714,1.14373228172279)(-0.566685267857143,1.14473173317214)(-0.566545758928572,1.1457300595848)(-0.56640625,1.14672725874624)(-0.566266741071429,1.1477233284441)(-0.566127232142857,1.14871826646824)(-0.565987723214286,1.14971207061071)(-0.565848214285714,1.15070473866578)(-0.565708705357143,1.15169626842992)(-0.565569196428572,1.15268665770182)(-0.5654296875,1.15367590428239)(-0.565290178571429,1.15466400597476)(-0.565150669642857,1.15565096058429)(-0.565011160714286,1.15663676591857)(-0.564871651785714,1.15762141978744)(-0.564732142857143,1.15860492000297)(-0.564592633928572,1.15958726437947)(-0.564453125,1.16056845073351)(-0.564313616071429,1.16154847688391)(-0.564174107142857,1.16252734065175)(-0.564034598214286,1.16350503986037)(-0.563895089285714,1.16448157233537)(-0.563755580357143,1.16545693590464)(-0.563616071428572,1.16643112839832)(-0.5634765625,1.16740414764886)(-0.563337053571429,1.16837599149095)(-0.563197544642857,1.16934665776162)(-0.563058035714286,1.17031614430014)(-0.562918526785714,1.17128444894812)(-0.562779017857143,1.17225156954945)(-0.562639508928572,1.17321750395032)(-0.5625,1.17418224999923)(-0.562360491071429,1.17514580554701)(-0.562220982142857,1.17610816844678)(-0.562081473214286,1.17706933655401)(-0.561941964285714,1.17802930772647)(-0.561802455357143,1.17898807982428)(-0.561662946428572,1.17994565070987)(-0.5615234375,1.18090201824804)(-0.561383928571429,1.18185718030591)(-0.561244419642857,1.18281113475294)(-0.561104910714286,1.18376387946096)(-0.560965401785714,1.18471541230414)(-0.560825892857143,1.18566573115902)(-0.560686383928572,1.1866148339045)(-0.560546875,1.18756271842183)(-0.560407366071429,1.18850938259465)(-0.560267857142857,1.18945482430898)(-0.560128348214286,1.1903990414532)(-0.559988839285714,1.19134203191808)(-0.559849330357143,1.19228379359678)(-0.559709821428572,1.19322432438485)(-0.5595703125,1.19416362218025)(-0.559430803571429,1.19510168488332)(-0.559291294642857,1.19603851039681)(-0.559151785714286,1.19697409662587)(-0.559012276785714,1.19790844147808)(-0.558872767857143,1.19884154286342)(-0.558733258928572,1.1997733986943)(-0.55859375,1.20070400688556)(-0.558454241071429,1.20163336535445)(-0.558314732142857,1.20256147202066)(-0.558175223214286,1.20348832480632)(-0.558035714285714,1.20441392163599)(-0.557896205357143,1.2053382604367)(-0.557756696428572,1.2062613391379)(-0.5576171875,1.2071831556715)(-0.557477678571429,1.20810370797188)(-0.557338169642857,1.20902299397586)(-0.557198660714286,1.20994101162275)(-0.557059151785714,1.2108577588543)(-0.556919642857143,1.21177323361475)(-0.556780133928572,1.21268743385081)(-0.556640625,1.21360035751169)(-0.556501116071429,1.21451200254904)(-0.556361607142857,1.21542236691705)(-0.556222098214286,1.21633144857237)(-0.556082589285714,1.21723924547416)(-0.555943080357143,1.21814575558407)(-0.555803571428572,1.21905097686626)(-0.5556640625,1.21995490728741)(-0.555524553571429,1.22085754481669)(-0.555385044642857,1.2217588874258)(-0.555245535714286,1.22265893308896)(-0.555106026785714,1.22355767978291)(-0.554966517857143,1.22445512548692)(-0.554827008928572,1.22535126818279)(-0.5546875,1.22624610585487)(-0.554547991071429,1.22713963649003)(-0.554408482142857,1.22803185807769)(-0.554268973214286,1.22892276860983)(-0.554129464285714,1.22981236608096)(-0.553989955357143,1.23070064848816)(-0.553850446428572,1.23158761383106)(-0.5537109375,1.23247326011188)(-0.553571428571429,1.23335758533537)(-0.553431919642857,1.23424058750888)(-0.553292410714286,1.23512226464231)(-0.553152901785714,1.23600261474816)(-0.553013392857143,1.23688163584151)(-0.552873883928572,1.23775932594002)(-0.552734375,1.23863568306393)(-0.552594866071429,1.2395107052361)(-0.552455357142857,1.24038439048196)(-0.552315848214286,1.24125673682958)(-0.552176339285714,1.24212774230959)(-0.552036830357143,1.24299740495526)(-0.551897321428572,1.24386572280248)(-0.5517578125,1.24473269388972)(-0.551618303571429,1.2455983162581)(-0.551478794642857,1.24646258795138)(-0.551339285714286,1.2473255070159)(-0.551199776785714,1.24818707150067)(-0.551060267857143,1.24904727945734)(-0.550920758928572,1.24990612894017)(-0.55078125,1.25076361800609)(-0.550641741071429,1.25161974471467)(-0.550502232142857,1.25247450712812)(-0.550362723214286,1.25332790331133)(-0.550223214285714,1.25417993133183)(-0.550083705357143,1.25503058925981)(-0.549944196428572,1.25587987516814)(-0.5498046875,1.25672778713236)(-0.549665178571429,1.25757432323067)(-0.549525669642857,1.25841948154395)(-0.549386160714286,1.25926326015579)(-0.549246651785714,1.26010565715243)(-0.549107142857143,1.2609466706228)(-0.548967633928572,1.26178629865856)(-0.548828125,1.26262453935403)(-0.548688616071429,1.26346139080623)(-0.548549107142857,1.26429685111492)(-0.548409598214286,1.26513091838252)(-0.548270089285714,1.26596359071419)(-0.548130580357143,1.26679486621781)(-0.547991071428572,1.26762474300395)(-0.5478515625,1.26845321918594)(-0.547712053571429,1.2692802928798)(-0.547572544642857,1.2701059622043)(-0.547433035714286,1.27093022528095)(-0.547293526785714,1.27175308023397)(-0.547154017857143,1.27257452519034)(-0.547014508928572,1.27339455827978)(-0.546875,1.27421317763477)(-0.546735491071429,1.27503038139052)(-0.546595982142857,1.275846167685)(-0.546456473214286,1.27666053465896)(-0.546316964285714,1.27747348045588)(-0.546177455357143,1.27828500322203)(-0.546037946428572,1.27909510110644)(-0.5458984375,1.27990377226092)(-0.545758928571429,1.28071101484005)(-0.545619419642857,1.28151682700118)(-0.545479910714286,1.28232120690448)(-0.545340401785714,1.28312415271286)(-0.545200892857143,1.28392566259206)(-0.545061383928572,1.2847257347106)(-0.544921875,1.28552436723979)(-0.544782366071429,1.28632155835376)(-0.544642857142857,1.28711730622942)(-0.544503348214286,1.28791160904652)(-0.544363839285714,1.28870446498759)(-0.544224330357143,1.28949587223801)(-0.544084821428572,1.29028582898595)(-0.5439453125,1.29107433342243)(-0.543805803571429,1.29186138374126)(-0.543666294642857,1.29264697813912)(-0.543526785714286,1.2934311148155)(-0.543387276785714,1.29421379197273)(-0.543247767857143,1.29499500781598)(-0.543108258928572,1.29577476055328)(-0.54296875,1.29655304839549)(-0.542829241071429,1.29732986955632)(-0.542689732142857,1.29810522225234)(-0.542550223214286,1.29887910470299)(-0.542410714285714,1.29965151513056)(-0.542271205357143,1.30042245176019)(-0.542131696428572,1.30119191281992)(-0.5419921875,1.30195989654063)(-0.541852678571429,1.30272640115611)(-0.541713169642857,1.30349142490299)(-0.541573660714286,1.30425496602081)(-0.541434151785714,1.305017022752)(-0.541294642857143,1.30577759334184)(-0.541155133928572,1.30653667603855)(-0.541015625,1.30729426909322)(-0.540876116071429,1.30805037075984)(-0.540736607142857,1.30880497929531)(-0.540597098214286,1.30955809295943)(-0.540457589285714,1.31030971001492)(-0.540318080357143,1.31105982872739)(-0.540178571428572,1.31180844736539)(-0.5400390625,1.31255556420038)(-0.539899553571429,1.31330117750674)(-0.539760044642857,1.31404528556178)(-0.539620535714286,1.31478788664575)(-0.539481026785714,1.31552897904182)(-0.539341517857143,1.31626856103608)(-0.539202008928572,1.3170066309176)(-0.5390625,1.31774318697836)(-0.538922991071429,1.3184782275133)(-0.538783482142857,1.31921175082031)(-0.538643973214286,1.31994375520023)(-0.538504464285714,1.32067423895685)(-0.538364955357143,1.32140320039694)(-0.538225446428572,1.3221306378302)(-0.5380859375,1.32285654956933)(-0.537946428571429,1.32358093392998)(-0.537806919642857,1.32430378923077)(-0.537667410714286,1.32502511379331)(-0.537527901785714,1.32574490594219)(-0.537388392857143,1.32646316400496)(-0.537248883928572,1.32717988631218)(-0.537109375,1.32789507119738)(-0.536969866071429,1.32860871699711)(-0.536830357142857,1.32932082205088)(-0.536690848214286,1.33003138470122)(-0.536551339285714,1.33074040329366)(-0.536411830357143,1.33144787617673)(-0.536272321428572,1.33215380170198)(-0.5361328125,1.33285817822396)(-0.535993303571429,1.33356100410023)(-0.535853794642857,1.33426227769138)(-0.535714285714286,1.33496199736102)(-0.535574776785714,1.33566016147578)(-0.535435267857143,1.33635676840532)(-0.535295758928572,1.33705181652233)(-0.53515625,1.33774530420252)(-0.535016741071429,1.33843722982467)(-0.534877232142857,1.33912759177058)(-0.534737723214286,1.33981638842508)(-0.534598214285714,1.34050361817607)(-0.534458705357143,1.34118927941448)(-0.534319196428572,1.34187337053431)(-0.5341796875,1.34255588993262)(-0.534040178571429,1.3432368360095)(-0.533900669642857,1.34391620716813)(-0.533761160714286,1.34459400181474)(-0.533621651785714,1.34527021835864)(-0.533482142857143,1.3459448552122)(-0.533342633928572,1.34661791079086)(-0.533203125,1.34728938351317)(-0.533063616071429,1.34795927180072)(-0.532924107142857,1.3486275740782)(-0.532784598214286,1.3492942887734)(-0.532645089285714,1.34995941431718)(-0.532505580357143,1.3506229491435)(-0.532366071428572,1.35128489168943)(-0.5322265625,1.35194524039511)(-0.532087053571429,1.3526039937038)(-0.531947544642857,1.35326115006188)(-0.531808035714286,1.3539167079188)(-0.531668526785714,1.35457066572715)(-0.531529017857143,1.35522302194263)(-0.531389508928572,1.35587377502405)(-0.53125,1.35652292343335)(-0.531110491071429,1.35717046563557)(-0.530970982142857,1.35781640009891)(-0.530831473214286,1.35846072529468)(-0.530691964285714,1.35910343969732)(-0.530552455357143,1.35974454178441)(-0.530412946428572,1.36038403003667)(-0.5302734375,1.36102190293795)(-0.530133928571429,1.36165815897526)(-0.529994419642857,1.36229279663875)(-0.529854910714286,1.36292581442172)(-0.529715401785714,1.36355721082061)(-0.529575892857143,1.36418698433503)(-0.529436383928572,1.36481513346775)(-0.529296875,1.36544165672471)(-0.529157366071429,1.36606655261498)(-0.529017857142857,1.36668981965083)(-0.528878348214286,1.36731145634768)(-0.528738839285714,1.36793146122414)(-0.528599330357143,1.36854983280199)(-0.528459821428572,1.36916656960619)(-0.5283203125,1.36978167016487)(-0.528180803571429,1.37039513300935)(-0.528041294642857,1.37100695667416)(-0.527901785714286,1.37161713969698)(-0.527762276785714,1.37222568061871)(-0.527622767857143,1.37283257798344)(-0.527483258928572,1.37343783033847)(-0.52734375,1.37404143623428)(-0.527204241071429,1.37464339422456)(-0.527064732142857,1.37524370286623)(-0.526925223214286,1.37584236071939)(-0.526785714285714,1.37643936634736)(-0.526646205357143,1.37703471831669)(-0.526506696428572,1.37762841519714)(-0.5263671875,1.37822045556169)(-0.526227678571429,1.37881083798655)(-0.526088169642857,1.37939956105114)(-0.525948660714286,1.37998662333812)(-0.525809151785714,1.38057202343339)(-0.525669642857143,1.38115575992608)(-0.525530133928572,1.38173783140856)(-0.525390625,1.38231823647643)(-0.525251116071429,1.38289697372854)(-0.525111607142857,1.38347404176699)(-0.524972098214286,1.38404943919712)(-0.524832589285714,1.38462316462753)(-0.524693080357143,1.38519521667007)(-0.524553571428572,1.38576559393985)(-0.5244140625,1.38633429505523)(-0.524274553571429,1.38690131863785)(-0.524135044642857,1.38746666331259)(-0.523995535714286,1.38803032770763)(-0.523856026785714,1.38859231045439)(-0.523716517857143,1.38915261018757)(-0.523577008928572,1.38971122554517)(-0.5234375,1.39026815516843)(-0.523297991071429,1.3908233977019)(-0.523158482142857,1.3913769517934)(-0.523018973214286,1.39192881609404)(-0.522879464285714,1.39247898925822)(-0.522739955357143,1.39302746994363)(-0.522600446428572,1.39357425681125)(-0.5224609375,1.39411934852536)(-0.522321428571429,1.39466274375355)(-0.522181919642857,1.39520444116669)(-0.522042410714286,1.39574443943897)(-0.521902901785714,1.39628273724788)(-0.521763392857143,1.39681933327423)(-0.521623883928572,1.39735422620212)(-0.521484375,1.397887414719)(-0.521344866071429,1.3984188975156)(-0.521205357142857,1.398948673286)(-0.521065848214286,1.39947674072758)(-0.520926339285714,1.40000309854105)(-0.520786830357143,1.40052774543046)(-0.520647321428572,1.40105068010317)(-0.5205078125,1.4015719012699)(-0.520368303571429,1.40209140764467)(-0.520228794642857,1.40260919794487)(-0.520089285714286,1.4031252708912)(-0.519949776785714,1.40363962520774)(-0.519810267857143,1.40415225962187)(-0.519670758928572,1.40466317286436)(-0.51953125,1.4051723636693)(-0.519391741071429,1.40567983077415)(-0.519252232142857,1.4061855729197)(-0.519112723214286,1.40668958885014)(-0.518973214285714,1.40719187731297)(-0.518833705357143,1.40769243705909)(-0.518694196428572,1.40819126684276)(-0.5185546875,1.40868836542158)(-0.518415178571429,1.40918373155654)(-0.518275669642857,1.40967736401201)(-0.518136160714286,1.41016926155571)(-0.517996651785714,1.41065942295877)(-0.517857142857143,1.41114784699566)(-0.517717633928572,1.41163453244425)(-0.517578125,1.41211947808581)(-0.517438616071429,1.41260268270497)(-0.517299107142857,1.41308414508977)(-0.517159598214286,1.41356386403161)(-0.517020089285714,1.41404183832531)(-0.516880580357143,1.41451806676909)(-0.516741071428572,1.41499254816455)(-0.5166015625,1.41546528131669)(-0.516462053571429,1.41593626503393)(-0.516322544642857,1.41640549812807)(-0.516183035714286,1.41687297941435)(-0.516043526785714,1.41733870771139)(-0.515904017857143,1.41780268184123)(-0.515764508928572,1.41826490062933)(-0.515625,1.41872536290457)(-0.515485491071429,1.41918406749923)(-0.515345982142857,1.41964101324903)(-0.515206473214286,1.4200961989931)(-0.515066964285714,1.42054962357401)(-0.514927455357143,1.42100128583774)(-0.514787946428572,1.42145118463372)(-0.5146484375,1.42189931881479)(-0.514508928571429,1.42234568723724)(-0.514369419642857,1.42279028876079)(-0.514229910714286,1.42323312224861)(-0.514090401785714,1.42367418656729)(-0.513950892857143,1.42411348058689)(-0.513811383928572,1.42455100318088)(-0.513671875,1.42498675322622)(-0.513532366071429,1.42542072960329)(-0.513392857142857,1.42585293119592)(-0.513253348214286,1.42628335689143)(-0.513113839285714,1.42671200558054)(-0.512974330357143,1.42713887615749)(-0.512834821428572,1.42756396751993)(-0.5126953125,1.42798727856899)(-0.512555803571429,1.42840880820929)(-0.512416294642857,1.42882855534888)(-0.512276785714286,1.4292465188993)(-0.512137276785714,1.42966269777555)(-0.511997767857143,1.43007709089611)(-0.511858258928572,1.43048969718294)(-0.51171875,1.43090051556147)(-0.511579241071429,1.43130954496062)(-0.511439732142857,1.43171678431277)(-0.511300223214286,1.4321222325538)(-0.511160714285714,1.43252588862309)(-0.511021205357143,1.43292775146347)(-0.510881696428572,1.4333278200213)(-0.5107421875,1.43372609324641)(-0.510602678571429,1.43412257009213)(-0.510463169642857,1.43451724951528)(-0.510323660714286,1.43491013047618)(-0.510184151785714,1.43530121193866)(-0.510044642857143,1.43569049287005)(-0.509905133928572,1.43607797224117)(-0.509765625,1.43646364902636)(-0.509626116071429,1.43684752220346)(-0.509486607142857,1.43722959075384)(-0.509347098214286,1.43760985366236)(-0.509207589285714,1.43798830991739)(-0.509068080357143,1.43836495851085)(-0.508928571428572,1.43873979843814)(-0.5087890625,1.4391128286982)(-0.508649553571429,1.43948404829348)(-0.508510044642857,1.43985345622998)(-0.508370535714286,1.4402210515172)(-0.508231026785714,1.44058683316818)(-0.508091517857143,1.44095080019947)(-0.507952008928572,1.44131295163119)(-0.5078125,1.44167328648696)(-0.507672991071429,1.44203180379395)(-0.507533482142857,1.44238850258286)(-0.507393973214286,1.44274338188794)(-0.507254464285714,1.44309644074697)(-0.507114955357143,1.44344767820128)(-0.506975446428572,1.44379709329575)(-0.5068359375,1.44414468507879)(-0.506696428571429,1.44449045260237)(-0.506556919642857,1.44483439492202)(-0.506417410714286,1.44517651109679)(-0.506277901785714,1.44551680018933)(-0.506138392857143,1.44585526126581)(-0.505998883928572,1.44619189339597)(-0.505859375,1.4465266956531)(-0.505719866071429,1.44685966711408)(-0.505580357142857,1.44719080685932)(-0.505440848214286,1.44752011397281)(-0.505301339285714,1.44784758754211)(-0.505161830357143,1.44817322665835)(-0.505022321428572,1.44849703041621)(-0.5048828125,1.44881899791396)(-0.504743303571429,1.44913912825345)(-0.504603794642857,1.44945742054008)(-0.504464285714286,1.44977387388286)(-0.504324776785714,1.45008848739436)(-0.504185267857143,1.45040126019073)(-0.504045758928572,1.4507121913917)(-0.50390625,1.4510212801206)(-0.503766741071429,1.45132852550433)(-0.503627232142857,1.45163392667338)(-0.503487723214286,1.45193748276185)(-0.503348214285714,1.45223919290739)(-0.503208705357143,1.45253905625129)(-0.503069196428572,1.4528370719384)(-0.5029296875,1.45313323911717)(-0.502790178571429,1.45342755693967)(-0.502650669642857,1.45372002456155)(-0.502511160714286,1.45401064114206)(-0.502371651785714,1.45429940584406)(-0.502232142857143,1.45458631783402)(-0.502092633928572,1.45487137628199)(-0.501953125,1.45515458036167)(-0.501813616071429,1.45543592925032)(-0.501674107142857,1.45571542212885)(-0.501534598214286,1.45599305818177)(-0.501395089285714,1.45626883659719)(-0.501255580357143,1.45654275656686)(-0.501116071428572,1.45681481728613)(-0.5009765625,1.45708501795396)(-0.500837053571429,1.45735335777296)(-0.500697544642857,1.45761983594934)(-0.500558035714286,1.45788445169294)(-0.500418526785714,1.45814720421721)(-0.500279017857143,1.45840809273925)(-0.500139508928572,1.45866711647978)(-0.5,1.45892427466314)(-0.499860491071429,1.45917956651731)(-0.499720982142857,1.45943299127389)(-0.499581473214286,1.45968454816814)(-0.499441964285714,1.45993423643892)(-0.499302455357143,1.46018205532877)(-0.499162946428572,1.46042800408382)(-0.4990234375,1.46067208195388)(-0.498883928571429,1.46091428819238)(-0.498744419642857,1.46115462205639)(-0.498604910714286,1.46139308280664)(-0.498465401785714,1.4616296697075)(-0.498325892857143,1.46186438202698)(-0.498186383928572,1.46209721903674)(-0.498046875,1.4623281800121)(-0.497907366071429,1.46255726423202)(-0.497767857142857,1.46278447097912)(-0.497628348214286,1.46300979953967)(-0.497488839285714,1.46323324920359)(-0.497349330357143,1.46345481926447)(-0.497209821428572,1.46367450901954)(-0.4970703125,1.46389231776972)(-0.496930803571429,1.46410824481956)(-0.496791294642857,1.46432228947729)(-0.496651785714286,1.4645344510548)(-0.496512276785714,1.46474472886762)(-0.496372767857143,1.464953122235)(-0.496233258928572,1.46515963047981)(-0.49609375,1.46536425292861)(-0.495954241071429,1.46556698891163)(-0.495814732142857,1.46576783776276)(-0.495675223214286,1.46596679881957)(-0.495535714285714,1.46616387142331)(-0.495396205357143,1.4663590549189)(-0.495256696428572,1.46655234865494)(-0.4951171875,1.46674375198369)(-0.494977678571429,1.46693326426112)(-0.494838169642857,1.46712088484686)(-0.494698660714286,1.46730661310423)(-0.494559151785714,1.46749044840022)(-0.494419642857143,1.46767239010552)(-0.494280133928572,1.4678524375945)(-0.494140625,1.46803059024521)(-0.494001116071429,1.4682068474394)(-0.493861607142857,1.4683812085625)(-0.493722098214286,1.46855367300363)(-0.493582589285714,1.4687242401556)(-0.493443080357143,1.46889290941494)(-0.493303571428572,1.46905968018182)(-0.4931640625,1.46922455186016)(-0.493024553571429,1.46938752385754)(-0.492885044642857,1.46954859558525)(-0.492745535714286,1.46970776645829)(-0.492606026785714,1.46986503589533)(-0.492466517857143,1.47002040331876)(-0.492327008928572,1.47017386815468)(-0.4921875,1.47032542983288)(-0.492047991071429,1.47047508778686)(-0.491908482142857,1.47062284145381)(-0.491768973214286,1.47076869027464)(-0.491629464285714,1.47091263369398)(-0.491489955357143,1.47105467116014)(-0.491350446428572,1.47119480212516)(-0.4912109375,1.47133302604478)(-0.491071428571429,1.47146934237847)(-0.490931919642857,1.47160375058938)(-0.490792410714286,1.4717362501444)(-0.490652901785714,1.47186684051412)(-0.490513392857143,1.47199552117287)(-0.490373883928572,1.47212229159866)(-0.490234375,1.47224715127325)(-0.490094866071429,1.47237009968211)(-0.489955357142857,1.47249113631441)(-0.489815848214286,1.47261026066307)(-0.489676339285714,1.47272747222471)(-0.489536830357143,1.47284277049969)(-0.489397321428572,1.47295615499208)(-0.4892578125,1.47306762520969)(-0.489118303571429,1.47317718066403)(-0.488978794642857,1.47328482087036)(-0.488839285714286,1.47339054534766)(-0.488699776785714,1.47349435361865)(-0.488560267857143,1.47359624520974)(-0.488420758928572,1.47369621965113)(-0.48828125,1.4737942764767)(-0.488141741071429,1.47389041522409)(-0.488002232142857,1.47398463543466)(-0.487862723214286,1.47407693665351)(-0.487723214285714,1.47416731842948)(-0.487583705357143,1.47425578031513)(-0.487444196428572,1.47434232186678)(-0.4873046875,1.47442694264446)(-0.487165178571429,1.47450964221196)(-0.487025669642857,1.4745904201368)(-0.486886160714286,1.47466927599024)(-0.486746651785714,1.47474620934728)(-0.486607142857143,1.47482121978667)(-0.486467633928572,1.47489430689089)(-0.486328125,1.47496547024618)(-0.486188616071429,1.4750347094425)(-0.486049107142857,1.47510202407359)(-0.485909598214286,1.47516741373689)(-0.485770089285714,1.47523087803363)(-0.485630580357143,1.47529241656875)(-0.485491071428572,1.47535202895098)(-0.4853515625,1.47540971479276)(-0.485212053571429,1.47546547371029)(-0.485072544642857,1.47551930532354)(-0.484933035714286,1.47557120925621)(-0.484793526785714,1.47562118513576)(-0.484654017857143,1.4756692325934)(-0.484514508928572,1.47571535126409)(-0.484375,1.47575954078655)(-0.484235491071429,1.47580180080325)(-0.484095982142857,1.47584213096043)(-0.483956473214286,1.47588053090806)(-0.483816964285714,1.47591700029988)(-0.483677455357143,1.4759515387934)(-0.483537946428572,1.47598414604987)(-0.4833984375,1.4760148217343)(-0.483258928571429,1.47604356551547)(-0.483119419642857,1.47607037706591)(-0.482979910714286,1.47609525606193)(-0.482840401785714,1.47611820218356)(-0.482700892857143,1.47613921511464)(-0.482561383928572,1.47615829454273)(-0.482421875,1.4761754401592)(-0.482282366071429,1.47619065165913)(-0.482142857142857,1.47620392874141)(-0.482003348214286,1.47621527110868)(-0.481863839285714,1.47622467846732)(-0.481724330357143,1.47623215052752)(-0.481584821428572,1.4762376870032)(-0.4814453125,1.47624128761208)(-0.481305803571429,1.4762429520756)(-0.481166294642857,1.47624268011903)(-0.481026785714286,1.47624047147136)(-0.480887276785714,1.47623632586536)(-0.480747767857143,1.4762302430376)(-0.480608258928572,1.47622222272837)(-0.48046875,1.47621226468177)(-0.480329241071429,1.47620036864565)(-0.480189732142857,1.47618653437166)(-0.480050223214286,1.47617076161518)(-0.479910714285714,1.47615305013539)(-0.479771205357143,1.47613339969526)(-0.479631696428572,1.47611181006148)(-0.4794921875,1.47608828100457)(-0.479352678571429,1.4760628122988)(-0.479213169642857,1.4760354037222)(-0.479073660714286,1.47600605505661)(-0.478934151785714,1.47597476608762)(-0.478794642857143,1.47594153660461)(-0.478655133928572,1.47590636640072)(-0.478515625,1.47586925527289)(-0.478376116071429,1.47583020302182)(-0.478236607142857,1.47578920945199)(-0.478097098214286,1.47574627437167)(-0.477957589285714,1.47570139759289)(-0.477818080357143,1.47565457893149)(-0.477678571428572,1.47560581820705)(-0.4775390625,1.47555511524295)(-0.477399553571429,1.47550246986636)(-0.477260044642857,1.47544788190822)(-0.477120535714286,1.47539135120324)(-0.476981026785714,1.47533287758993)(-0.476841517857143,1.47527246091057)(-0.476702008928572,1.47521010101122)(-0.4765625,1.47514579774174)(-0.476422991071429,1.47507955095574)(-0.476283482142857,1.47501136051065)(-0.476143973214286,1.47494122626766)(-0.476004464285714,1.47486914809174)(-0.475864955357143,1.47479512585166)(-0.475725446428572,1.47471915941996)(-0.4755859375,1.47464124867297)(-0.475446428571429,1.47456139349081)(-0.475306919642857,1.47447959375736)(-0.475167410714286,1.47439584936032)(-0.475027901785714,1.47431016019115)(-0.474888392857143,1.4742225261451)(-0.474748883928572,1.47413294712121)(-0.474609375,1.47404142302231)(-0.474469866071429,1.47394795375499)(-0.474330357142857,1.47385253922966)(-0.474190848214286,1.47375517936049)(-0.474051339285714,1.47365587406546)(-0.473911830357143,1.47355462326631)(-0.473772321428572,1.47345142688858)(-0.4736328125,1.4733462848616)(-0.473493303571429,1.47323919711849)(-0.473353794642857,1.47313016359613)(-0.473214285714286,1.47301918423523)(-0.473074776785714,1.47290625898024)(-0.472935267857143,1.47279138777944)(-0.472795758928572,1.47267457058486)(-0.47265625,1.47255580735236)(-0.472516741071429,1.47243509804154)(-0.472377232142857,1.47231244261582)(-0.472237723214286,1.4721878410424)(-0.472098214285714,1.47206129329227)(-0.471958705357143,1.4719327993402)(-0.471819196428572,1.47180235916474)(-0.4716796875,1.47166997274826)(-0.471540178571429,1.47153564007688)(-0.471400669642857,1.47139936114054)(-0.471261160714286,1.47126113593295)(-0.471121651785714,1.4711209644516)(-0.470982142857143,1.4709788466978)(-0.470842633928572,1.47083478267661)(-0.470703125,1.4706887723969)(-0.470563616071429,1.47054081587132)(-0.470424107142857,1.47039091311632)(-0.470284598214286,1.47023906415213)(-0.470145089285714,1.47008526900275)(-0.470005580357143,1.46992952769601)(-0.469866071428572,1.46977184026349)(-0.4697265625,1.46961220674056)(-0.469587053571429,1.4694506271664)(-0.469447544642857,1.46928710158397)(-0.469308035714286,1.46912163004001)(-0.469168526785714,1.46895421258504)(-0.469029017857143,1.46878484927338)(-0.468889508928572,1.46861354016315)(-0.46875,1.46844028531622)(-0.468610491071429,1.46826508479829)(-0.468470982142857,1.46808793867881)(-0.468331473214286,1.46790884703104)(-0.468191964285714,1.46772780993202)(-0.468052455357143,1.46754482746257)(-0.467912946428572,1.4673598997073)(-0.4677734375,1.46717302675461)(-0.467633928571429,1.46698420869669)(-0.467494419642857,1.4667934456295)(-0.467354910714286,1.4666007376528)(-0.467215401785714,1.46640608487013)(-0.467075892857143,1.46620948738881)(-0.466936383928572,1.46601094531995)(-0.466796875,1.46581045877845)(-0.466657366071429,1.46560802788299)(-0.466517857142857,1.46540365275602)(-0.466378348214286,1.46519733352381)(-0.466238839285714,1.46498907031637)(-0.466099330357143,1.46477886326753)(-0.465959821428572,1.46456671251487)(-0.4658203125,1.46435261819979)(-0.465680803571429,1.46413658046744)(-0.465541294642857,1.46391859946678)(-0.465401785714286,1.46369867535052)(-0.465262276785714,1.46347680827519)(-0.465122767857143,1.46325299840107)(-0.464983258928572,1.46302724589222)(-0.46484375,1.46279955091652)(-0.464704241071429,1.46256991364559)(-0.464564732142857,1.46233833425485)(-0.464425223214286,1.46210481292348)(-0.464285714285714,1.46186934983447)(-0.464146205357143,1.46163194517457)(-0.464006696428572,1.4613925991343)(-0.4638671875,1.46115131190799)(-0.463727678571429,1.46090808369371)(-0.463588169642857,1.46066291469334)(-0.463448660714286,1.46041580511252)(-0.463309151785714,1.46016675516066)(-0.463169642857143,1.45991576505097)(-0.463030133928572,1.45966283500042)(-0.462890625,1.45940796522975)(-0.462751116071429,1.45915115596349)(-0.462611607142857,1.45889240742994)(-0.462472098214286,1.45863171986117)(-0.462332589285714,1.45836909349304)(-0.462193080357143,1.45810452856515)(-0.462053571428572,1.4578380253209)(-0.4619140625,1.45756958400745)(-0.461774553571429,1.45729920487576)(-0.461635044642857,1.45702688818052)(-0.461495535714286,1.45675263418021)(-0.461356026785714,1.45647644313709)(-0.461216517857143,1.45619831531717)(-0.461077008928572,1.45591825099026)(-0.4609375,1.45563625042989)(-0.460797991071429,1.45535231391341)(-0.460658482142857,1.45506644172192)(-0.460518973214286,1.45477863414026)(-0.460379464285714,1.45448889145707)(-0.460239955357143,1.45419721396476)(-0.460100446428572,1.45390360195947)(-0.4599609375,1.45360805574114)(-0.459821428571429,1.45331057561346)(-0.459681919642857,1.45301116188388)(-0.459542410714286,1.45270981486362)(-0.459402901785714,1.45240653486766)(-0.459263392857143,1.45210132221475)(-0.459123883928572,1.4517941772274)(-0.458984375,1.45148510023186)(-0.458844866071429,1.45117409155816)(-0.458705357142857,1.45086115154009)(-0.458565848214286,1.45054628051519)(-0.458426339285714,1.45022947882476)(-0.458286830357143,1.44991074681388)(-0.458147321428572,1.44959008483134)(-0.4580078125,1.44926749322973)(-0.457868303571429,1.44894297236538)(-0.457728794642857,1.44861652259836)(-0.457589285714286,1.44828814429252)(-0.457449776785714,1.44795783781545)(-0.457310267857143,1.44762560353849)(-0.457170758928572,1.44729144183675)(-0.45703125,1.44695535308905)(-0.456891741071429,1.44661733767802)(-0.456752232142857,1.44627739598998)(-0.456612723214286,1.44593552841505)(-0.456473214285714,1.44559173534706)(-0.456333705357143,1.4452460171836)(-0.456194196428572,1.44489837432603)(-0.4560546875,1.44454880717942)(-0.455915178571429,1.4441973161526)(-0.455775669642857,1.44384390165815)(-0.455636160714286,1.44348856411239)(-0.455496651785714,1.44313130393538)(-0.455357142857143,1.44277212155092)(-0.455217633928572,1.44241101738656)(-0.455078125,1.44204799187358)(-0.454938616071429,1.44168304544701)(-0.454799107142857,1.44131617854561)(-0.454659598214286,1.44094739161189)(-0.454520089285714,1.44057668509207)(-0.454380580357143,1.44020405943613)(-0.454241071428572,1.43982951509779)(-0.4541015625,1.43945305253448)(-0.453962053571429,1.43907467220738)(-0.453822544642857,1.4386943745814)(-0.453683035714286,1.43831216012518)(-0.453543526785714,1.4379280293111)(-0.453404017857143,1.43754198261525)(-0.453264508928572,1.43715402051747)(-0.453125,1.43676414350131)(-0.452985491071429,1.43637235205405)(-0.452845982142857,1.43597864666671)(-0.452706473214286,1.43558302783403)(-0.452566964285714,1.43518549605446)(-0.452427455357143,1.43478605183019)(-0.452287946428572,1.43438469566711)(-0.4521484375,1.43398142807487)(-0.452008928571429,1.4335762495668)(-0.451869419642857,1.43316916065997)(-0.451729910714286,1.43276016187517)(-0.451590401785714,1.43234925373689)(-0.451450892857143,1.43193643677336)(-0.451311383928572,1.4315217115165)(-0.451171875,1.43110507850196)(-0.451032366071429,1.43068653826911)(-0.450892857142857,1.430266091361)(-0.450753348214286,1.42984373832443)(-0.450613839285714,1.42941947970989)(-0.450474330357143,1.42899331607157)(-0.450334821428572,1.42856524796739)(-0.4501953125,1.42813527595896)(-0.450055803571429,1.4277034006116)(-0.449916294642857,1.42726962249434)(-0.449776785714286,1.42683394217991)(-0.449637276785714,1.42639636024473)(-0.449497767857143,1.42595687726893)(-0.449358258928572,1.42551549383636)(-0.44921875,1.42507221053453)(-0.449079241071429,1.42462702795469)(-0.448939732142857,1.42417994669174)(-0.448800223214286,1.42373096734432)(-0.448660714285714,1.42328009051474)(-0.448521205357143,1.422827316809)(-0.448381696428572,1.4223726468368)(-0.4482421875,1.42191608121153)(-0.448102678571429,1.42145762055027)(-0.447963169642857,1.4209972654738)(-0.447823660714286,1.42053501660655)(-0.447684151785714,1.42007087457668)(-0.447544642857143,1.41960484001601)(-0.447405133928572,1.41913691356005)(-0.447265625,1.41866709584798)(-0.447126116071429,1.41819538752268)(-0.446986607142857,1.4177217892307)(-0.446847098214286,1.41724630162227)(-0.446707589285714,1.4167689253513)(-0.446568080357143,1.41628966107536)(-0.446428571428572,1.41580850945572)(-0.4462890625,1.4153254711573)(-0.446149553571429,1.41484054684871)(-0.446010044642857,1.41435373720222)(-0.445870535714286,1.41386504289377)(-0.445731026785714,1.41337446460298)(-0.445591517857143,1.41288200301311)(-0.445452008928572,1.41238765881112)(-0.4453125,1.4118914326876)(-0.445172991071429,1.41139332533684)(-0.445033482142857,1.41089333745675)(-0.444893973214286,1.41039146974893)(-0.444754464285714,1.40988772291863)(-0.444614955357143,1.40938209767476)(-0.444475446428572,1.40887459472988)(-0.4443359375,1.4083652148002)(-0.444196428571429,1.40785395860559)(-0.444056919642857,1.40734082686959)(-0.443917410714286,1.40682582031935)(-0.443777901785714,1.40630893968571)(-0.443638392857143,1.40579018570312)(-0.443498883928572,1.40526955910971)(-0.443359375,1.40474706064724)(-0.443219866071429,1.4042226910611)(-0.443080357142857,1.40369645110034)(-0.442940848214286,1.40316834151764)(-0.442801339285714,1.40263836306934)(-0.442661830357143,1.40210651651538)(-0.442522321428572,1.40157280261937)(-0.4423828125,1.40103722214853)(-0.442243303571429,1.40049977587373)(-0.442103794642857,1.39996046456947)(-0.441964285714286,1.39941928901387)(-0.441824776785714,1.39887624998868)(-0.441685267857143,1.39833134827929)(-0.441545758928572,1.39778458467469)(-0.44140625,1.39723595996752)(-0.441266741071429,1.39668547495403)(-0.441127232142857,1.39613313043408)(-0.440987723214286,1.39557892721119)(-0.440848214285714,1.39502286609244)(-0.440708705357143,1.39446494788856)(-0.440569196428572,1.39390517341391)(-0.4404296875,1.39334354348641)(-0.440290178571429,1.39278005892765)(-0.440150669642857,1.39221472056278)(-0.440011160714286,1.3916475292206)(-0.439871651785714,1.39107848573349)(-0.439732142857143,1.39050759093744)(-0.439592633928572,1.38993484567205)(-0.439453125,1.38936025078051)(-0.439313616071429,1.38878380710962)(-0.439174107142857,1.38820551550977)(-0.439034598214286,1.38762537683496)(-0.438895089285714,1.38704339194277)(-0.438755580357143,1.38645956169437)(-0.438616071428572,1.38587388695455)(-0.4384765625,1.38528636859167)(-0.438337053571429,1.38469700747766)(-0.438197544642857,1.38410580448807)(-0.438058035714286,1.38351276050202)(-0.437918526785714,1.38291787640221)(-0.437779017857143,1.38232115307492)(-0.437639508928572,1.38172259141003)(-0.4375,1.38112219230096)(-0.437360491071429,1.38051995664475)(-0.437220982142857,1.37991588534198)(-0.437081473214286,1.37930997929682)(-0.436941964285714,1.378702239417)(-0.436802455357143,1.37809266661384)(-0.436662946428572,1.3774812618022)(-0.4365234375,1.37686802590052)(-0.436383928571429,1.37625295983081)(-0.436244419642857,1.37563606451862)(-0.436104910714286,1.3750173408931)(-0.435965401785714,1.3743967898869)(-0.435825892857143,1.37377441243629)(-0.435686383928572,1.37315020948105)(-0.435546875,1.37252418196452)(-0.435407366071429,1.37189633083361)(-0.435267857142857,1.37126665703877)(-0.435128348214286,1.37063516153398)(-0.434988839285714,1.3700018452768)(-0.434849330357143,1.3693667092283)(-0.434709821428572,1.36872975435311)(-0.4345703125,1.3680909816194)(-0.434430803571429,1.36745039199886)(-0.434291294642857,1.36680798646675)(-0.434151785714286,1.36616376600183)(-0.434012276785714,1.36551773158642)(-0.433872767857143,1.36486988420634)(-0.433733258928572,1.36422022485097)(-0.43359375,1.3635687545132)(-0.433454241071429,1.36291547418944)(-0.433314732142857,1.36226038487963)(-0.433175223214286,1.36160348758724)(-0.433035714285714,1.36094478331925)(-0.432896205357143,1.36028427308615)(-0.432756696428572,1.35962195790194)(-0.4326171875,1.35895783878417)(-0.432477678571429,1.35829191675385)(-0.432338169642857,1.35762419283554)(-0.432198660714286,1.35695466805729)(-0.432059151785714,1.35628334345064)(-0.431919642857143,1.35561022005067)(-0.431780133928572,1.35493529889593)(-0.431640625,1.35425858102848)(-0.431501116071429,1.35358006749387)(-0.431361607142857,1.35289975934116)(-0.431222098214286,1.35221765762289)(-0.431082589285714,1.3515337633951)(-0.430943080357143,1.35084807771729)(-0.430803571428572,1.3501606016525)(-0.4306640625,1.3494713362672)(-0.430524553571429,1.34878028263138)(-0.430385044642857,1.3480874418185)(-0.430245535714286,1.34739281490548)(-0.430106026785714,1.34669640297274)(-0.429966517857143,1.34599820710417)(-0.429827008928572,1.34529822838712)(-0.4296875,1.34459646791243)(-0.429547991071429,1.34389292677438)(-0.429408482142857,1.34318760607074)(-0.429268973214286,1.34248050690273)(-0.429129464285714,1.34177163037504)(-0.428989955357143,1.34106097759581)(-0.428850446428572,1.34034854967665)(-0.4287109375,1.33963434773261)(-0.428571428571429,1.33891837288221)(-0.428431919642857,1.3382006262474)(-0.428292410714286,1.3374811089536)(-0.428152901785714,1.33675982212965)(-0.428013392857143,1.33603676690786)(-0.427873883928572,1.33531194442396)(-0.427734375,1.33458535581715)(-0.427594866071429,1.33385700223003)(-0.427455357142857,1.33312688480866)(-0.427315848214286,1.33239500470252)(-0.427176339285714,1.33166136306453)(-0.427036830357143,1.33092596105105)(-0.426897321428572,1.33018879982183)(-0.4267578125,1.32944988054007)(-0.426618303571429,1.32870920437239)(-0.426478794642857,1.32796677248882)(-0.426339285714286,1.32722258606283)(-0.426199776785714,1.32647664627127)(-0.426060267857143,1.32572895429442)(-0.425920758928572,1.32497951131599)(-0.42578125,1.32422831852305)(-0.425641741071429,1.32347537710613)(-0.425502232142857,1.32272068825913)(-0.425362723214286,1.32196425317936)(-0.425223214285714,1.32120607306752)(-0.425083705357143,1.32044614912771)(-0.424944196428572,1.31968448256744)(-0.4248046875,1.31892107459759)(-0.424665178571429,1.31815592643245)(-0.424525669642857,1.31738903928967)(-0.424386160714286,1.31662041439031)(-0.424246651785714,1.3158500529588)(-0.424107142857143,1.31507795622296)(-0.423967633928572,1.31430412541396)(-0.423828125,1.31352856176637)(-0.423688616071429,1.31275126651813)(-0.423549107142857,1.31197224091056)(-0.423409598214286,1.31119148618831)(-0.423270089285714,1.31040900359944)(-0.423130580357143,1.30962479439534)(-0.422991071428572,1.30883885983078)(-0.4228515625,1.30805120116387)(-0.422712053571429,1.30726181965611)(-0.422572544642857,1.30647071657231)(-0.422433035714286,1.30567789318065)(-0.422293526785714,1.30488335075267)(-0.422154017857143,1.30408709056325)(-0.422014508928572,1.30328911389058)(-0.421875,1.30248942201625)(-0.421735491071429,1.30168801622514)(-0.421595982142857,1.30088489780548)(-0.421456473214286,1.30008006804885)(-0.421316964285714,1.29927352825013)(-0.421177455357143,1.29846527970756)(-0.421037946428572,1.29765532372267)(-0.4208984375,1.29684366160036)(-0.420758928571429,1.2960302946488)(-0.420619419642857,1.29521522417952)(-0.420479910714286,1.29439845150734)(-0.420340401785714,1.2935799779504)(-0.420200892857143,1.29275980483015)(-0.420061383928572,1.29193793347136)(-0.419921875,1.29111436520207)(-0.419782366071429,1.29028910135368)(-0.419642857142857,1.28946214326082)(-0.419503348214286,1.28863349226149)(-0.419363839285714,1.28780314969692)(-0.419224330357143,1.28697111691168)(-0.419084821428572,1.28613739525361)(-0.4189453125,1.28530198607382)(-0.418805803571429,1.28446489072674)(-0.418666294642857,1.28362611057006)(-0.418526785714286,1.28278564696476)(-0.418387276785714,1.28194350127507)(-0.418247767857143,1.28109967486854)(-0.418108258928572,1.28025416911594)(-0.41796875,1.27940698539134)(-0.417829241071429,1.27855812507208)(-0.417689732142857,1.27770758953875)(-0.417550223214286,1.27685538017519)(-0.417410714285714,1.27600149836852)(-0.417271205357143,1.2751459455091)(-0.417131696428572,1.27428872299055)(-0.4169921875,1.27342983220973)(-0.416852678571429,1.27256927456677)(-0.416713169642857,1.27170705146502)(-0.416573660714286,1.27084316431108)(-0.416434151785714,1.26997761451478)(-0.416294642857143,1.26911040348921)(-0.416155133928572,1.26824153265067)(-0.416015625,1.26737100341869)(-0.415876116071429,1.26649881721605)(-0.415736607142857,1.26562497546873)(-0.415597098214286,1.26474947960594)(-0.415457589285714,1.26387233106013)(-0.415318080357143,1.26299353126693)(-0.415178571428572,1.26211308166521)(-0.4150390625,1.26123098369704)(-0.414899553571429,1.26034723880771)(-0.414760044642857,1.2594618484457)(-0.414620535714286,1.2585748140627)(-0.414481026785714,1.2576861371136)(-0.414341517857143,1.25679581905649)(-0.414202008928572,1.25590386135264)(-0.4140625,1.25501026546653)(-0.413922991071429,1.25411503286582)(-0.413783482142857,1.25321816502135)(-0.413643973214286,1.25231966340715)(-0.413504464285714,1.25141952950043)(-0.413364955357143,1.25051776478158)(-0.413225446428572,1.24961437073414)(-0.4130859375,1.24870934884485)(-0.412946428571429,1.24780270060361)(-0.412806919642857,1.24689442750349)(-0.412667410714286,1.24598453104069)(-0.412527901785714,1.24507301271463)(-0.412388392857143,1.24415987402782)(-0.412248883928572,1.24324511648598)(-0.412109375,1.24232874159794)(-0.411969866071429,1.2414107508757)(-0.411830357142857,1.24049114583439)(-0.411690848214286,1.23956992799231)(-0.411551339285714,1.23864709887086)(-0.411411830357143,1.23772265999461)(-0.411272321428572,1.23679661289123)(-0.4111328125,1.23586895909156)(-0.410993303571429,1.23493970012954)(-0.410853794642857,1.23400883754223)(-0.410714285714286,1.23307637286983)(-0.410574776785714,1.23214230765564)(-0.410435267857143,1.2312066434461)(-0.410295758928572,1.23026938179073)(-0.41015625,1.22933052424218)(-0.410016741071429,1.22839007235621)(-0.409877232142857,1.22744802769166)(-0.409737723214286,1.2265043918105)(-0.409598214285714,1.22555916627776)(-0.409458705357143,1.22461235266161)(-0.409319196428572,1.22366395253327)(-0.4091796875,1.22271396746706)(-0.409040178571429,1.2217623990404)(-0.408900669642857,1.22080924883378)(-0.408761160714286,1.21985451843077)(-0.408621651785714,1.218898209418)(-0.408482142857143,1.2179403233852)(-0.408342633928572,1.21698086192514)(-0.408203125,1.2160198266337)(-0.408063616071429,1.21505721910977)(-0.407924107142857,1.21409304095533)(-0.407784598214286,1.21312729377542)(-0.407645089285714,1.21215997917813)(-0.407505580357143,1.21119109877458)(-0.407366071428572,1.21022065417897)(-0.4072265625,1.20924864700851)(-0.407087053571429,1.20827507888349)(-0.406947544642857,1.2072999514272)(-0.406808035714286,1.20632326626599)(-0.406668526785714,1.20534502502924)(-0.406529017857143,1.20436522934935)(-0.406389508928572,1.20338388086174)(-0.40625,1.20240098120487)(-0.406110491071429,1.2014165320202)(-0.405970982142857,1.20043053495222)(-0.405831473214286,1.19944299164843)(-0.405691964285714,1.19845390375933)(-0.405552455357143,1.19746327293845)(-0.405412946428572,1.19647110084228)(-0.4052734375,1.19547738913036)(-0.405133928571429,1.19448213946519)(-0.404994419642857,1.19348535351227)(-0.404854910714286,1.19248703294012)(-0.404715401785714,1.1914871794202)(-0.404575892857143,1.19048579462698)(-0.404436383928572,1.18948288023791)(-0.404296875,1.18847843793341)(-0.404157366071429,1.18747246939689)(-0.404017857142857,1.1864649763147)(-0.403878348214286,1.18545596037618)(-0.403738839285714,1.18444542327364)(-0.403599330357143,1.18343336670233)(-0.403459821428572,1.18241979236046)(-0.4033203125,1.18140470194921)(-0.403180803571429,1.1803880971727)(-0.403041294642857,1.179369979738)(-0.402901785714286,1.17835035135512)(-0.402762276785714,1.17732921373701)(-0.402622767857143,1.17630656859957)(-0.402483258928572,1.17528241766161)(-0.40234375,1.1742567626449)(-0.402204241071429,1.17322960527412)(-0.402064732142857,1.17220094727688)(-0.401925223214286,1.17117079038369)(-0.401785714285714,1.17013913632802)(-0.401646205357143,1.16910598684621)(-0.401506696428572,1.16807134367754)(-0.4013671875,1.16703520856417)(-0.401227678571429,1.1659975832512)(-0.401088169642857,1.1649584694866)(-0.400948660714286,1.16391786902125)(-0.400809151785714,1.16287578360892)(-0.400669642857143,1.16183221500627)(-0.400530133928572,1.16078716497285)(-0.400390625,1.15974063527109)(-0.400251116071429,1.1586926276663)(-0.400111607142857,1.15764314392667)(-0.399972098214286,1.15659218582325)(-0.399832589285714,1.15553975512998)(-0.399693080357143,1.15448585362366)(-0.399553571428572,1.15343048308394)(-0.3994140625,1.15237364529333)(-0.399274553571429,1.15131534203722)(-0.399135044642857,1.15025557510382)(-0.398995535714286,1.14919434628422)(-0.398856026785714,1.14813165737232)(-0.398716517857143,1.1470675101649)(-0.398577008928572,1.14600190646155)(-0.3984375,1.1449348480647)(-0.398297991071429,1.14386633677963)(-0.398158482142857,1.14279637441442)(-0.398018973214286,1.14172496277998)(-0.397879464285714,1.14065210369007)(-0.397739955357143,1.13957779896123)(-0.397600446428572,1.13850205041283)(-0.3974609375,1.13742485986705)(-0.397321428571429,1.13634622914887)(-0.397181919642857,1.13526616008609)(-0.397042410714286,1.13418465450929)(-0.396902901785714,1.13310171425186)(-0.396763392857143,1.13201734114996)(-0.396623883928572,1.13093153704257)(-0.396484375,1.12984430377143)(-0.396344866071429,1.12875564318108)(-0.396205357142857,1.12766555711882)(-0.396065848214286,1.12657404743474)(-0.395926339285714,1.12548111598169)(-0.395786830357143,1.1243867646153)(-0.395647321428572,1.12329099519394)(-0.3955078125,1.12219380957876)(-0.395368303571429,1.12109520963367)(-0.395228794642857,1.11999519722531)(-0.395089285714286,1.1188937742231)(-0.394949776785714,1.11779094249917)(-0.394810267857143,1.11668670392842)(-0.394670758928572,1.11558106038848)(-0.39453125,1.1144740137597)(-0.394391741071429,1.11336556592519)(-0.394252232142857,1.11225571877076)(-0.394112723214286,1.11114447418496)(-0.393973214285714,1.11003183405904)(-0.393833705357143,1.108917800287)(-0.393694196428572,1.10780237476551)(-0.3935546875,1.10668555939399)(-0.393415178571429,1.10556735607454)(-0.393275669642857,1.10444776671196)(-0.393136160714286,1.10332679321376)(-0.392996651785714,1.10220443749015)(-0.392857142857143,1.101080701454)(-0.392717633928572,1.09995558702089)(-0.392578125,1.09882909610909)(-0.392438616071429,1.09770123063953)(-0.392299107142857,1.09657199253581)(-0.392159598214286,1.09544138372424)(-0.392020089285714,1.09430940613375)(-0.391880580357143,1.09317606169596)(-0.391741071428572,1.09204135234515)(-0.3916015625,1.09090528001825)(-0.391462053571429,1.08976784665485)(-0.391322544642857,1.08862905419719)(-0.391183035714286,1.08748890459013)(-0.391043526785714,1.0863473997812)(-0.390904017857143,1.08520454172056)(-0.390764508928572,1.08406033236101)(-0.390625,1.08291477365795)(-0.390485491071429,1.08176786756945)(-0.390345982142857,1.08061961605617)(-0.390206473214286,1.07947002108141)(-0.390066964285714,1.07831908461106)(-0.389927455357143,1.07716680861364)(-0.389787946428572,1.07601319506027)(-0.3896484375,1.07485824592468)(-0.389508928571429,1.07370196318319)(-0.389369419642857,1.07254434881473)(-0.389229910714286,1.0713854048008)(-0.389090401785714,1.07022513312551)(-0.388950892857143,1.06906353577553)(-0.388811383928572,1.06790061474014)(-0.388671875,1.06673637201117)(-0.388532366071429,1.06557080958303)(-0.388392857142857,1.06440392945271)(-0.388253348214286,1.06323573361976)(-0.388113839285714,1.06206622408627)(-0.387974330357143,1.06089540285691)(-0.387834821428572,1.0597232719389)(-0.3876953125,1.058549833342)(-0.387555803571429,1.05737508907853)(-0.387416294642857,1.05619904116333)(-0.387276785714286,1.05502169161379)(-0.387137276785714,1.05384304244984)(-0.386997767857143,1.05266309569392)(-0.386858258928572,1.05148185337102)(-0.38671875,1.05029931750862)(-0.386579241071429,1.04911549013676)(-0.386439732142857,1.04793037328796)(-0.386300223214286,1.04674396899725)(-0.386160714285714,1.04555627930218)(-0.386021205357143,1.0443673062428)(-0.385881696428572,1.04317705186165)(-0.3857421875,1.04198551820377)(-0.385602678571429,1.0407927073167)(-0.385463169642857,1.03959862125043)(-0.385323660714286,1.03840326205748)(-0.385184151785714,1.03720663179281)(-0.385044642857143,1.03600873251387)(-0.384905133928572,1.03480956628058)(-0.384765625,1.03360913515531)(-0.384626116071429,1.03240744120293)(-0.384486607142857,1.03120448649072)(-0.384347098214286,1.03000027308845)(-0.384207589285714,1.02879480306833)(-0.384068080357143,1.02758807850499)(-0.383928571428572,1.02638010147555)(-0.3837890625,1.02517087405952)(-0.383649553571429,1.02396039833888)(-0.383510044642857,1.02274867639802)(-0.383370535714286,1.02153571032376)(-0.383231026785714,1.02032150220533)(-0.383091517857143,1.01910605413441)(-0.382952008928572,1.01788936820506)(-0.3828125,1.01667144651377)(-0.382672991071429,1.01545229115941)(-0.382533482142857,1.01423190424328)(-0.382393973214286,1.01301028786906)(-0.382254464285714,1.01178744414284)(-0.382114955357143,1.01056337517306)(-0.381975446428572,1.0093380830706)(-0.3818359375,1.00811156994867)(-0.381696428571429,1.00688383792289)(-0.381556919642857,1.00565488911123)(-0.381417410714286,1.00442472563404)(-0.381277901785714,1.00319334961405)(-0.381138392857143,1.00196076317631)(-0.380998883928572,1.00072696844825)(-0.380859375,0.999491967559666)(-0.380719866071429,0.998255762642673)(-0.380580357142857,0.997018355831744)(-0.380440848214286,0.99577974926369)(-0.380301339285714,0.994539945077655)(-0.380161830357143,0.993298945415114)(-0.380022321428572,0.992056752419872)(-0.3798828125,0.990813368238049)(-0.379743303571429,0.989568795018095)(-0.379603794642857,0.988323034910765)(-0.379464285714286,0.987076090069126)(-0.379324776785714,0.985827962648552)(-0.379185267857143,0.984578654806718)(-0.379045758928572,0.983328168703596)(-0.37890625,0.982076506501451)(-0.378766741071429,0.980823670364842)(-0.378627232142857,0.979569662460605)(-0.378487723214286,0.97831448495786)(-0.378348214285714,0.977058140028006)(-0.378208705357143,0.975800629844709)(-0.378069196428572,0.974541956583907)(-0.3779296875,0.973282122423798)(-0.377790178571429,0.972021129544845)(-0.377650669642857,0.970758980129761)(-0.377511160714286,0.969495676363513)(-0.377371651785714,0.968231220433312)(-0.377232142857143,0.966965614528613)(-0.377092633928572,0.96569886084111)(-0.376953125,0.964430961564728)(-0.376813616071429,0.963161918895626)(-0.376674107142857,0.961891735032185)(-0.376534598214286,0.960620412175008)(-0.376395089285714,0.959347952526914)(-0.376255580357143,0.958074358292937)(-0.376116071428572,0.956799631680316)(-0.3759765625,0.955523774898494)(-0.375837053571429,0.954246790159119)(-0.375697544642857,0.952968679676028)(-0.375558035714286,0.951689445665252)(-0.375418526785714,0.950409090345008)(-0.375279017857143,0.949127615935695)(-0.375139508928572,0.94784502465989)(-0.375,0.946561318742344)(-0.374860491071429,0.94527650040998)(-0.374720982142857,0.943990571891882)(-0.374581473214286,0.942703535419297)(-0.374441964285714,0.941415393225628)(-0.374302455357143,0.940126147546428)(-0.374162946428572,0.938835800619402)(-0.3740234375,0.937544354684392)(-0.373883928571429,0.936251811983388)(-0.373744419642857,0.934958174760506)(-0.373604910714286,0.933663445261997)(-0.373465401785714,0.932367625736236)(-0.373325892857143,0.93107071843372)(-0.373186383928572,0.929772725607063)(-0.373046875,0.928473649510991)(-0.372907366071429,0.927173492402341)(-0.372767857142857,0.925872256540052)(-0.372628348214286,0.924569944185162)(-0.372488839285714,0.923266557600805)(-0.372349330357143,0.921962099052206)(-0.372209821428572,0.920656570806675)(-0.3720703125,0.919349975133603)(-0.371930803571429,0.918042314304465)(-0.371791294642857,0.916733590592802)(-0.371651785714286,0.915423806274226)(-0.371512276785714,0.914112963626414)(-0.371372767857143,0.912801064929101)(-0.371233258928572,0.91148811246408)(-0.37109375,0.91017410851519)(-0.370954241071429,0.908859055368324)(-0.370814732142857,0.907542955311412)(-0.370675223214286,0.906225810634422)(-0.370535714285714,0.904907623629355)(-0.370396205357143,0.903588396590241)(-0.370256696428572,0.902268131813136)(-0.3701171875,0.900946831596111)(-0.369977678571429,0.89962449823926)(-0.369838169642857,0.898301134044681)(-0.369698660714286,0.89697674131648)(-0.369559151785714,0.895651322360766)(-0.369419642857143,0.894324879485644)(-0.369280133928572,0.892997415001213)(-0.369140625,0.891668931219558)(-0.369001116071429,0.890339430454752)(-0.368861607142857,0.889008915022845)(-0.368722098214286,0.887677387241861)(-0.368582589285714,0.886344849431794)(-0.368443080357143,0.885011303914606)(-0.368303571428572,0.883676753014219)(-0.3681640625,0.88234119905651)(-0.368024553571429,0.881004644369313)(-0.367885044642857,0.879667091282406)(-0.367745535714286,0.878328542127511)(-0.367606026785714,0.876988999238288)(-0.367466517857143,0.875648464950332)(-0.367327008928572,0.874306941601167)(-0.3671875,0.87296443153024)(-0.367047991071429,0.871620937078925)(-0.366908482142857,0.870276460590505)(-0.366768973214286,0.868931004410177)(-0.366629464285714,0.867584570885045)(-0.366489955357143,0.866237162364114)(-0.366350446428572,0.864888781198287)(-0.3662109375,0.86353942974036)(-0.366071428571429,0.862189110345021)(-0.365931919642857,0.860837825368836)(-0.365792410714286,0.859485577170253)(-0.365652901785714,0.858132368109596)(-0.365513392857143,0.856778200549057)(-0.365373883928572,0.855423076852694)(-0.365234375,0.854066999386424)(-0.365094866071429,0.852709970518027)(-0.364955357142857,0.851351992617128)(-0.364815848214286,0.849993068055202)(-0.364676339285714,0.848633199205564)(-0.364536830357143,0.84727238844337)(-0.364397321428572,0.845910638145607)(-0.3642578125,0.844547950691089)(-0.364118303571429,0.843184328460462)(-0.363978794642857,0.841819773836182)(-0.363839285714286,0.840454289202524)(-0.363699776785714,0.839087876945572)(-0.363560267857143,0.837720539453216)(-0.363420758928572,0.836352279115147)(-0.36328125,0.834983098322849)(-0.363141741071429,0.833612999469605)(-0.363002232142857,0.832241984950477)(-0.362862723214286,0.830870057162313)(-0.362723214285714,0.829497218503736)(-0.362583705357143,0.828123471375145)(-0.362444196428572,0.826748818178705)(-0.3623046875,0.825373261318342)(-0.362165178571429,0.823996803199749)(-0.362025669642857,0.822619446230365)(-0.361886160714286,0.821241192819381)(-0.361746651785714,0.819862045377734)(-0.361607142857143,0.8184820063181)(-0.361467633928572,0.817101078054888)(-0.361328125,0.81571926300424)(-0.361188616071429,0.814336563584028)(-0.361049107142857,0.812952982213838)(-0.360909598214286,0.811568521314977)(-0.360770089285714,0.810183183310461)(-0.360630580357143,0.808796970625014)(-0.360491071428572,0.807409885685065)(-0.3603515625,0.806021930918735)(-0.360212053571429,0.804633108755845)(-0.360072544642857,0.803243421627901)(-0.359933035714286,0.801852871968089)(-0.359793526785714,0.800461462211279)(-0.359654017857143,0.799069194794012)(-0.359514508928572,0.797676072154498)(-0.359375,0.796282096732613)(-0.359235491071429,0.794887270969894)(-0.359095982142857,0.79349159730953)(-0.358956473214286,0.792095078196362)(-0.358816964285714,0.790697716076876)(-0.358677455357143,0.789299513399198)(-0.358537946428572,0.78790047261309)(-0.3583984375,0.786500596169946)(-0.358258928571429,0.785099886522789)(-0.358119419642857,0.783698346126258)(-0.357979910714286,0.782295977436614)(-0.357840401785714,0.780892782911726)(-0.357700892857143,0.779488765011073)(-0.357561383928572,0.778083926195735)(-0.357421875,0.77667826892839)(-0.357282366071429,0.775271795673311)(-0.357142857142857,0.773864508896357)(-0.357003348214286,0.772456411064971)(-0.356863839285714,0.771047504648174)(-0.356724330357143,0.769637792116561)(-0.356584821428572,0.768227275942295)(-0.3564453125,0.766815958599104)(-0.356305803571429,0.765403842562278)(-0.356166294642857,0.763990930308658)(-0.356026785714286,0.762577224316634)(-0.355887276785714,0.761162727066143)(-0.355747767857143,0.759747441038662)(-0.355608258928572,0.758331368717202)(-0.35546875,0.756914512586303)(-0.355329241071429,0.755496875132037)(-0.355189732142857,0.754078458841989)(-0.355050223214286,0.752659266205263)(-0.354910714285714,0.751239299712474)(-0.354771205357143,0.749818561855744)(-0.354631696428572,0.748397055128694)(-0.3544921875,0.746974782026441)(-0.354352678571429,0.7455517450456)(-0.354213169642857,0.744127946684266)(-0.354073660714286,0.742703389442017)(-0.353934151785714,0.741278075819909)(-0.353794642857143,0.73985200832047)(-0.353655133928572,0.738425189447696)(-0.353515625,0.736997621707042)(-0.353376116071429,0.735569307605427)(-0.353236607142857,0.734140249651218)(-0.353097098214286,0.732710450354229)(-0.352957589285714,0.731279912225718)(-0.352818080357143,0.729848637778383)(-0.352678571428572,0.728416629526352)(-0.3525390625,0.726983889985181)(-0.352399553571429,0.725550421671856)(-0.352260044642857,0.724116227104772)(-0.352120535714286,0.722681308803743)(-0.351981026785714,0.721245669289991)(-0.351841517857143,0.71980931108614)(-0.351702008928572,0.718372236716214)(-0.3515625,0.716934448705628)(-0.351422991071429,0.715495949581193)(-0.351283482142857,0.714056741871099)(-0.351143973214286,0.712616828104913)(-0.351004464285714,0.711176210813582)(-0.350864955357143,0.709734892529416)(-0.350725446428572,0.708292875786094)(-0.3505859375,0.706850163118651)(-0.350446428571429,0.705406757063482)(-0.350306919642857,0.703962660158326)(-0.350167410714286,0.702517874942268)(-0.350027901785714,0.701072403955734)(-0.349888392857143,0.699626249740482)(-0.349748883928572,0.698179414839603)(-0.349609375,0.696731901797508)(-0.349469866071429,0.695283713159937)(-0.349330357142857,0.693834851473935)(-0.349190848214286,0.692385319287863)(-0.349051339285714,0.690935119151383)(-0.348911830357143,0.68948425361546)(-0.348772321428572,0.688032725232353)(-0.3486328125,0.686580536555608)(-0.348493303571429,0.685127690140065)(-0.348353794642857,0.683674188541835)(-0.348214285714286,0.682220034318308)(-0.348074776785714,0.680765230028144)(-0.347935267857143,0.679309778231267)(-0.347795758928572,0.677853681488863)(-0.34765625,0.676396942363369)(-0.347516741071429,0.674939563418482)(-0.347377232142857,0.673481547219134)(-0.347237723214286,0.672022896331502)(-0.347098214285714,0.670563613322997)(-0.346958705357143,0.66910370076226)(-0.346819196428572,0.66764316121916)(-0.3466796875,0.66618199726478)(-0.346540178571429,0.664720211471428)(-0.346400669642857,0.663257806412615)(-0.346261160714286,0.661794784663058)(-0.346121651785714,0.660331148798676)(-0.345982142857143,0.658866901396583)(-0.345842633928572,0.657402045035082)(-0.345703125,0.655936582293659)(-0.345563616071429,0.65447051575299)(-0.345424107142857,0.653003847994915)(-0.345284598214286,0.651536581602449)(-0.345145089285714,0.650068719159771)(-0.345005580357143,0.648600263252223)(-0.344866071428572,0.647131216466297)(-0.3447265625,0.645661581389636)(-0.344587053571429,0.644191360611036)(-0.344447544642857,0.642720556720422)(-0.344308035714286,0.64124917230886)(-0.344168526785714,0.639777209968543)(-0.344029017857143,0.638304672292792)(-0.343889508928572,0.636831561876045)(-0.34375,0.635357881313853)(-0.343610491071429,0.633883633202884)(-0.343470982142857,0.632408820140904)(-0.343331473214286,0.63093344472678)(-0.343191964285714,0.629457509560475)(-0.343052455357143,0.627981017243038)(-0.342912946428572,0.626503970376608)(-0.3427734375,0.625026371564395)(-0.342633928571429,0.623548223410694)(-0.342494419642857,0.622069528520861)(-0.342354910714286,0.620590289501319)(-0.342215401785714,0.619110508959549)(-0.342075892857143,0.617630189504087)(-0.341936383928572,0.616149333744516)(-0.341796875,0.614667944291464)(-0.341657366071429,0.613186023756602)(-0.341517857142857,0.611703574752627)(-0.341378348214286,0.610220599893269)(-0.341238839285714,0.60873710179328)(-0.341099330357143,0.607253083068431)(-0.340959821428572,0.605768546335507)(-0.3408203125,0.604283494212298)(-0.340680803571429,0.602797929317603)(-0.340541294642857,0.601311854271214)(-0.340401785714286,0.599825271693917)(-0.340262276785714,0.598338184207487)(-0.340122767857143,0.596850594434679)(-0.339983258928572,0.595362504999229)(-0.33984375,0.593873918525841)(-0.339704241071429,0.592384837640194)(-0.339564732142857,0.590895264968923)(-0.339425223214286,0.58940520313962)(-0.339285714285714,0.587914654780832)(-0.339146205357143,0.586423622522052)(-0.339006696428572,0.584932108993713)(-0.3388671875,0.583440116827185)(-0.338727678571429,0.581947648654775)(-0.338588169642857,0.580454707109709)(-0.338448660714286,0.578961294826139)(-0.338309151785714,0.57746741443913)(-0.338169642857143,0.575973068584661)(-0.338030133928572,0.574478259899615)(-0.337890625,0.572982991021774)(-0.337751116071429,0.571487264589823)(-0.337611607142857,0.56999108324333)(-0.337472098214286,0.568494449622752)(-0.337332589285714,0.566997366369423)(-0.337193080357143,0.565499836125557)(-0.337053571428572,0.564001861534233)(-0.3369140625,0.562503445239396)(-0.336774553571429,0.561004589885857)(-0.336635044642857,0.559505298119273)(-0.336495535714286,0.558005572586153)(-0.336356026785714,0.556505415933853)(-0.336216517857143,0.555004830810565)(-0.336077008928572,0.553503819865315)(-0.3359375,0.552002385747958)(-0.335797991071429,0.550500531109177)(-0.335658482142857,0.548998258600467)(-0.335518973214286,0.54749557087414)(-0.335379464285714,0.545992470583316)(-0.335239955357143,0.544488960381915)(-0.335100446428572,0.542985042924658)(-0.3349609375,0.541480720867055)(-0.334821428571429,0.539975996865411)(-0.334681919642857,0.538470873576807)(-0.334542410714286,0.5369653536591)(-0.334402901785714,0.535459439770923)(-0.334263392857143,0.533953134571674)(-0.334123883928572,0.532446440721513)(-0.333984375,0.530939360881354)(-0.333844866071429,0.529431897712867)(-0.333705357142857,0.527924053878466)(-0.333565848214286,0.526415832041304)(-0.333426339285714,0.524907234865271)(-0.333286830357143,0.523398265014988)(-0.333147321428572,0.521888925155801)(-0.3330078125,0.520379217953774)(-0.332868303571429,0.518869146075691)(-0.332728794642857,0.517358712189042)(-0.332589285714286,0.515847918962022)(-0.332449776785714,0.514336769063525)(-0.332310267857143,0.512825265163141)(-0.332170758928572,0.511313409931147)(-0.33203125,0.509801206038503)(-0.331891741071429,0.508288656156854)(-0.331752232142857,0.50677576295851)(-0.331612723214286,0.505262529116454)(-0.331473214285714,0.503748957304331)(-0.331333705357143,0.502235050196442)(-0.331194196428572,0.500720810467744)(-0.3310546875,0.499206240793836)(-0.330915178571429,0.497691343850968)(-0.330775669642857,0.49617612231602)(-0.330636160714286,0.494660578866503)(-0.330496651785714,0.49314471618056)(-0.330357142857143,0.49162853693695)(-0.330217633928572,0.490112043815051)(-0.330078125,0.488595239494849)(-0.329938616071429,0.487078126656943)(-0.329799107142857,0.485560707982524)(-0.329659598214286,0.484042986153383)(-0.329520089285714,0.482524963851899)(-0.329380580357143,0.481006643761036)(-0.329241071428572,0.479488028564339)(-0.3291015625,0.477969120945923)(-0.328962053571429,0.476449923590481)(-0.328822544642857,0.474930439183262)(-0.328683035714286,0.473410670410075)(-0.328543526785714,0.471890619957286)(-0.328404017857143,0.470370290511804)(-0.328264508928572,0.468849684761086)(-0.328125,0.467328805393121)(-0.327985491071429,0.46580765509644)(-0.327845982142857,0.464286236560091)(-0.327706473214286,0.46276455247365)(-0.327566964285714,0.461242605527208)(-0.327427455357143,0.459720398411368)(-0.327287946428572,0.458197933817239)(-0.3271484375,0.456675214436429)(-0.327008928571429,0.455152242961049)(-0.326869419642857,0.453629022083694)(-0.326729910714286,0.452105554497445)(-0.326590401785714,0.450581842895866)(-0.326450892857143,0.449057889972994)(-0.326311383928572,0.447533698423336)(-0.326171875,0.446009270941861)(-0.326032366071429,0.444484610224004)(-0.325892857142857,0.442959718965648) 
};
\addplot [
color=blue,
solid,
forget plot
]
coordinates{
 (-0.325892857142857,0.442959718965648)(-0.325753348214286,0.441434599863126)(-0.325613839285714,0.439909255613215)(-0.325474330357143,0.43838368891313)(-0.325334821428572,0.436857902460518)(-0.3251953125,0.435331898953453)(-0.325055803571429,0.433805681090438)(-0.324916294642857,0.432279251570386)(-0.324776785714286,0.430752613092624)(-0.324637276785714,0.429225768356886)(-0.324497767857143,0.427698720063306)(-0.324358258928572,0.426171470912418)(-0.32421875,0.42464402360514)(-0.324079241071429,0.423116380842785)(-0.323939732142857,0.42158854532704)(-0.323800223214286,0.420060519759969)(-0.323660714285714,0.418532306844005)(-0.323521205357143,0.417003909281945)(-0.323381696428572,0.415475329776949)(-0.3232421875,0.413946571032524)(-0.323102678571429,0.412417635752536)(-0.322963169642857,0.410888526641187)(-0.322823660714286,0.40935924640302)(-0.322684151785714,0.40782979774291)(-0.322544642857143,0.40630018336606)(-0.322405133928572,0.404770405977995)(-0.322265625,0.403240468284558)(-0.322126116071429,0.401710372991906)(-0.321986607142857,0.400180122806501)(-0.321847098214286,0.398649720435105)(-0.321707589285714,0.397119168584777)(-0.321568080357143,0.395588469962869)(-0.321428571428572,0.394057627277016)(-0.3212890625,0.392526643235131)(-0.321149553571429,0.390995520545413)(-0.321010044642857,0.389464261916318)(-0.320870535714286,0.387932870056575)(-0.320731026785714,0.386401347675167)(-0.320591517857143,0.384869697481333)(-0.320452008928572,0.383337922184564)(-0.3203125,0.381806024494586)(-0.320172991071429,0.380274007121375)(-0.320033482142857,0.378741872775132)(-0.319893973214286,0.377209624166285)(-0.319754464285714,0.37567726400549)(-0.319614955357143,0.374144795003614)(-0.319475446428572,0.37261221987174)(-0.3193359375,0.371079541321153)(-0.319196428571429,0.369546762063349)(-0.319056919642857,0.36801388481001)(-0.318917410714286,0.366480912273012)(-0.318777901785714,0.364947847164418)(-0.318638392857143,0.363414692196468)(-0.318498883928572,0.36188145008158)(-0.318359375,0.360348123532338)(-0.318219866071429,0.358814715261497)(-0.318080357142857,0.357281227981964)(-0.317940848214286,0.355747664406804)(-0.317801339285714,0.354214027249228)(-0.317661830357143,0.352680319222592)(-0.317522321428572,0.351146543040388)(-0.3173828125,0.34961270141624)(-0.317243303571429,0.348078797063907)(-0.317103794642857,0.34654483269726)(-0.316964285714286,0.345010811030292)(-0.316824776785714,0.343476734777107)(-0.316685267857143,0.341942606651914)(-0.316545758928572,0.340408429369024)(-0.31640625,0.338874205642842)(-0.316266741071429,0.337339938187869)(-0.316127232142857,0.335805629718685)(-0.315987723214286,0.33427128294995)(-0.315848214285714,0.332736900596403)(-0.315708705357143,0.331202485372848)(-0.315569196428572,0.329668039994155)(-0.3154296875,0.328133567175249)(-0.315290178571429,0.326599069631118)(-0.315150669642857,0.325064550076788)(-0.315011160714286,0.323530011227333)(-0.314871651785714,0.321995455797862)(-0.314732142857143,0.320460886503518)(-0.314592633928572,0.31892630605947)(-0.314453125,0.317391717180908)(-0.314313616071429,0.315857122583043)(-0.314174107142857,0.314322524981091)(-0.314034598214286,0.312787927090278)(-0.313895089285714,0.311253331625828)(-0.313755580357143,0.309718741302962)(-0.313616071428572,0.30818415883689)(-0.3134765625,0.306649586942805)(-0.313337053571429,0.305115028335885)(-0.313197544642857,0.303580485731278)(-0.313058035714286,0.302045961844101)(-0.312918526785714,0.300511459389435)(-0.312779017857143,0.298976981082319)(-0.312639508928572,0.297442529637745)(-0.3125,0.295908107770652)(-0.312360491071429,0.294373718195926)(-0.312220982142857,0.292839363628385)(-0.312081473214286,0.291305046782781)(-0.311941964285714,0.28977077037379)(-0.311802455357143,0.288236537116014)(-0.311662946428572,0.286702349723967)(-0.3115234375,0.285168210912072)(-0.311383928571429,0.283634123394667)(-0.311244419642857,0.282100089885981)(-0.311104910714286,0.280566113100141)(-0.310965401785714,0.279032195751162)(-0.310825892857143,0.277498340552945)(-0.310686383928572,0.27596455021927)(-0.310546875,0.274430827463788)(-0.310407366071429,0.272897175000025)(-0.310267857142857,0.271363595541364)(-0.310128348214286,0.269830091801049)(-0.309988839285714,0.268296666492174)(-0.309849330357143,0.266763322327683)(-0.309709821428572,0.26523006202036)(-0.3095703125,0.263696888282825)(-0.309430803571429,0.262163803827536)(-0.309291294642857,0.26063081136677)(-0.309151785714286,0.259097913612628)(-0.309012276785714,0.257565113277026)(-0.308872767857143,0.256032413071691)(-0.308733258928572,0.254499815708153)(-0.30859375,0.252967323897742)(-0.308454241071429,0.25143494035159)(-0.308314732142857,0.249902667780609)(-0.308175223214286,0.248370508895498)(-0.308035714285714,0.246838466406735)(-0.307896205357143,0.245306543024573)(-0.307756696428572,0.243774741459031)(-0.3076171875,0.242243064419889)(-0.307477678571429,0.240711514616692)(-0.307338169642857,0.239180094758731)(-0.307198660714286,0.237648807555045)(-0.307059151785714,0.236117655714416)(-0.306919642857143,0.234586641945362)(-0.306780133928572,0.233055768956133)(-0.306640625,0.231525039454701)(-0.306501116071429,0.229994456148768)(-0.306361607142857,0.228464021745742)(-0.306222098214286,0.226933738952746)(-0.306082589285714,0.225403610476607)(-0.305943080357143,0.223873639023849)(-0.305803571428572,0.222343827300696)(-0.3056640625,0.220814178013053)(-0.305524553571429,0.219284693866519)(-0.305385044642857,0.217755377566365)(-0.305245535714286,0.216226231817536)(-0.305106026785714,0.214697259324646)(-0.304966517857143,0.213168462791971)(-0.304827008928572,0.211639844923446)(-0.3046875,0.210111408422654)(-0.304547991071429,0.208583155992833)(-0.304408482142857,0.207055090336857)(-0.304268973214286,0.205527214157237)(-0.304129464285714,0.203999530156114)(-0.303989955357143,0.20247204103526)(-0.303850446428572,0.200944749496062)(-0.3037109375,0.199417658239523)(-0.303571428571429,0.197890769966264)(-0.303431919642857,0.196364087376502)(-0.303292410714286,0.194837613170056)(-0.303152901785714,0.193311350046342)(-0.303013392857143,0.191785300704361)(-0.302873883928572,0.190259467842702)(-0.302734375,0.188733854159529)(-0.302594866071429,0.187208462352584)(-0.302455357142857,0.185683295119174)(-0.302315848214286,0.184158355156168)(-0.302176339285714,0.182633645159996)(-0.302036830357143,0.181109167826637)(-0.301897321428572,0.179584925851619)(-0.3017578125,0.17806092193001)(-0.301618303571429,0.17653715875642)(-0.301478794642857,0.175013639024985)(-0.301339285714286,0.173490365429369)(-0.301199776785714,0.171967340662757)(-0.301060267857143,0.17044456741785)(-0.300920758928572,0.168922048386858)(-0.30078125,0.167399786261495)(-0.300641741071429,0.165877783732984)(-0.300502232142857,0.164356043492031)(-0.300362723214286,0.162834568228838)(-0.300223214285714,0.16131336063309)(-0.300083705357143,0.159792423393951)(-0.299944196428572,0.158271759200058)(-0.2998046875,0.156751370739515)(-0.299665178571429,0.155231260699897)(-0.299525669642857,0.153711431768229)(-0.299386160714286,0.152191886630992)(-0.299246651785714,0.150672627974114)(-0.299107142857143,0.149153658482966)(-0.298967633928572,0.147634980842356)(-0.298828125,0.146116597736521)(-0.298688616071429,0.144598511849132)(-0.298549107142857,0.143080725863276)(-0.298409598214286,0.141563242461458)(-0.298270089285714,0.140046064325592)(-0.298130580357143,0.138529194137)(-0.297991071428572,0.137012634576405)(-0.2978515625,0.135496388323921)(-0.297712053571429,0.133980458059062)(-0.297572544642857,0.132464846460717)(-0.297433035714286,0.130949556207158)(-0.297293526785714,0.129434589976035)(-0.297154017857143,0.127919950444361)(-0.297014508928572,0.12640564028852)(-0.296875,0.124891662184246)(-0.296735491071429,0.12337801880664)(-0.296595982142857,0.12186471283014)(-0.296456473214286,0.120351746928531)(-0.296316964285714,0.118839123774937)(-0.296177455357143,0.117326846041813)(-0.296037946428572,0.115814916400945)(-0.2958984375,0.114303337523436)(-0.295758928571429,0.112792112079714)(-0.295619419642857,0.111281242739514)(-0.295479910714286,0.109770732171877)(-0.295340401785714,0.108260583045149)(-0.295200892857143,0.106750798026971)(-0.295061383928572,0.105241379784274)(-0.294921875,0.103732330983276)(-0.294782366071429,0.102223654289481)(-0.294642857142857,0.100715352367661)(-0.294503348214286,0.0992074278818625)(-0.294363839285714,0.097699883495397)(-0.294224330357143,0.096192721870836)(-0.294084821428572,0.0946859456700061)(-0.2939453125,0.0931795575539813)(-0.293805803571429,0.0916735601830876)(-0.293666294642857,0.0901679562168831)(-0.293526785714286,0.0886627483141631)(-0.293387276785714,0.0871579391329514)(-0.293247767857143,0.0856535313304963)(-0.293108258928572,0.0841495275632647)(-0.29296875,0.0826459304869347)(-0.292829241071429,0.0811427427564001)(-0.292689732142857,0.0796399670257509)(-0.292550223214286,0.0781376059482776)(-0.292410714285714,0.0766356621764634)(-0.292271205357143,0.0751341383619798)(-0.292131696428572,0.0736330371556806)(-0.2919921875,0.0721323612075947)(-0.291852678571429,0.0706321131669307)(-0.291713169642857,0.0691322956820571)(-0.291573660714286,0.0676329114005066)(-0.291434151785714,0.0661339629689692)(-0.291294642857143,0.0646354530332863)(-0.291155133928572,0.063137384238446)(-0.291015625,0.0616397592285753)(-0.290876116071429,0.0601425806469448)(-0.290736607142857,0.0586458511359489)(-0.290597098214286,0.0571495733371105)(-0.290457589285714,0.0556537498910734)(-0.290318080357143,0.0541583834375969)(-0.290178571428572,0.0526634766155509)(-0.2900390625,0.0511690320629086)(-0.289899553571429,0.0496750524167504)(-0.289760044642857,0.0481815403132448)(-0.289620535714286,0.0466884983876527)(-0.289481026785714,0.0451959292743198)(-0.289341517857143,0.0437038356066719)(-0.289202008928572,0.042212220017209)(-0.2890625,0.0407210851374987)(-0.288922991071429,0.0392304335981798)(-0.288783482142857,0.0377402680289435)(-0.288643973214286,0.0362505910585373)(-0.288504464285714,0.0347614053147579)(-0.288364955357143,0.033272713424446)(-0.288225446428572,0.0317845180134811)(-0.2880859375,0.0302968217067738)(-0.287946428571429,0.0288096271282708)(-0.287806919642857,0.0273229369009348)(-0.287667410714286,0.0258367536467498)(-0.287527901785714,0.0243510799867124)(-0.287388392857143,0.0228659185408279)(-0.287248883928572,0.0213812719281045)(-0.287109375,0.0198971427665461)(-0.286969866071429,0.0184135336731566)(-0.286830357142857,0.0169304472639205)(-0.286690848214286,0.0154478861538075)(-0.286551339285714,0.0139658529567648)(-0.286411830357143,0.0124843502857123)(-0.286272321428572,0.0110033807525372)(-0.2861328125,0.0095229469680867)(-0.285993303571429,0.00804305154217261)(-0.285853794642857,0.00656369708355153)(-0.285714285714286,0.00508488619992981)(-0.285574776785714,0.00360662149795571)(-0.285435267857143,0.00212890558321455)(-0.285295758928572,0.000651741060223476)(-0.28515625,-0.000824869467575817)(-0.285016741071429,-0.00230092339781274)(-0.284877232142857,-0.00377641812920726)(-0.284737723214286,-0.0052513510615651)(-0.284598214285714,-0.00672571959578555)(-0.284458705357143,-0.00819952113386635)(-0.284319196428572,-0.00967275307890875)(-0.2841796875,-0.0111454128351249)(-0.284040178571429,-0.0126174978078336)(-0.283900669642857,-0.0140890054034793)(-0.283761160714286,-0.015559933029628)(-0.283621651785714,-0.0170302780949744)(-0.283482142857143,-0.0185000380093469)(-0.283342633928572,-0.0199692101837128)(-0.283203125,-0.0214377920301858)(-0.283063616071429,-0.0229057809620214)(-0.282924107142857,-0.0243731743936361)(-0.282784598214286,-0.0258399697406032)(-0.282645089285714,-0.0273061644196598)(-0.282505580357143,-0.0287717558487121)(-0.282366071428572,-0.0302367414468407)(-0.2822265625,-0.0317011186343072)(-0.282087053571429,-0.0331648848325508)(-0.281947544642857,-0.0346280374642063)(-0.281808035714286,-0.0360905739531009)(-0.281668526785714,-0.0375524917242606)(-0.281529017857143,-0.0390137882039157)(-0.281389508928572,-0.0404744608195058)(-0.28125,-0.0419345069996869)(-0.281110491071429,-0.0433939241743273)(-0.280970982142857,-0.0448527097745265)(-0.280831473214286,-0.0463108612326109)(-0.280691964285714,-0.0477683759821411)(-0.280552455357143,-0.0492252514579169)(-0.280412946428572,-0.0506814850959822)(-0.2802734375,-0.0521370743336327)(-0.280133928571429,-0.0535920166094109)(-0.279994419642857,-0.0550463093631259)(-0.279854910714286,-0.0564999500358483)(-0.279715401785714,-0.0579529360699179)(-0.279575892857143,-0.0594052649089485)(-0.279436383928572,-0.0608569339978334)(-0.279296875,-0.0623079407827517)(-0.279157366071429,-0.063758282711165)(-0.279017857142857,-0.0652079572318356)(-0.278878348214286,-0.0666569617948227)(-0.278738839285714,-0.0681052938514891)(-0.278599330357143,-0.0695529508545066)(-0.278459821428572,-0.0709999302578611)(-0.2783203125,-0.0724462295168591)(-0.278180803571429,-0.0738918460881244)(-0.278041294642857,-0.0753367774296162)(-0.277901785714286,-0.076781021000625)(-0.277762276785714,-0.07822457426178)(-0.277622767857143,-0.0796674346750538)(-0.277483258928572,-0.0811095997037676)(-0.27734375,-0.0825510668125984)(-0.277204241071429,-0.0839918334675743)(-0.277064732142857,-0.0854318971360938)(-0.276925223214286,-0.0868712552869214)(-0.276785714285714,-0.0883099053901946)(-0.276646205357143,-0.0897478449174289)(-0.276506696428572,-0.0911850713415229)(-0.2763671875,-0.0926215821367652)(-0.276227678571429,-0.0940573747788307)(-0.276088169642857,-0.0954924467447985)(-0.275948660714286,-0.0969267955131482)(-0.275809151785714,-0.098360418563767)(-0.275669642857143,-0.0997933133779546)(-0.275530133928572,-0.101225477438428)(-0.275390625,-0.102656908229328)(-0.275251116071429,-0.104087603236217)(-0.275111607142857,-0.105517559946095)(-0.274972098214286,-0.106946775847399)(-0.274832589285714,-0.108375248430004)(-0.274693080357143,-0.109802975185235)(-0.274553571428572,-0.111229953605867)(-0.2744140625,-0.112656181186134)(-0.274274553571429,-0.114081655421725)(-0.274135044642857,-0.115506373809802)(-0.273995535714286,-0.116930333848995)(-0.273856026785714,-0.11835353303941)(-0.273716517857143,-0.119775968882636)(-0.273577008928572,-0.121197638881744)(-0.2734375,-0.1226185405413)(-0.273297991071429,-0.12403867136736)(-0.273158482142857,-0.125458028867483)(-0.273018973214286,-0.126876610550736)(-0.272879464285714,-0.128294413927691)(-0.272739955357143,-0.129711436510438)(-0.272600446428572,-0.131127675812585)(-0.2724609375,-0.132543129349269)(-0.272321428571429,-0.133957794637146)(-0.272181919642857,-0.135371669194415)(-0.272042410714286,-0.136784750540813)(-0.271902901785714,-0.138197036197618)(-0.271763392857143,-0.139608523687659)(-0.271623883928572,-0.141019210535317)(-0.271484375,-0.142429094266534)(-0.271344866071429,-0.143838172408808)(-0.271205357142857,-0.145246442491213)(-0.271065848214286,-0.146653902044391)(-0.270926339285714,-0.148060548600563)(-0.270786830357143,-0.149466379693534)(-0.270647321428572,-0.150871392858692)(-0.2705078125,-0.152275585633023)(-0.270368303571429,-0.153678955555102)(-0.270228794642857,-0.15508150016511)(-0.270089285714286,-0.156483217004835)(-0.269949776785714,-0.157884103617674)(-0.269810267857143,-0.159284157548641)(-0.269670758928572,-0.16068337634437)(-0.26953125,-0.162081757553124)(-0.269391741071429,-0.163479298724788)(-0.269252232142857,-0.164875997410888)(-0.269112723214286,-0.166271851164589)(-0.268973214285714,-0.167666857540701)(-0.268833705357143,-0.169061014095681)(-0.268694196428572,-0.170454318387641)(-0.2685546875,-0.171846767976356)(-0.268415178571429,-0.173238360423254)(-0.268275669642857,-0.174629093291441)(-0.268136160714286,-0.176018964145692)(-0.267996651785714,-0.17740797055246)(-0.267857142857143,-0.178796110079882)(-0.267717633928572,-0.18018338029778)(-0.267578125,-0.181569778777672)(-0.267438616071429,-0.182955303092765)(-0.267299107142857,-0.184339950817973)(-0.267159598214286,-0.185723719529915)(-0.267020089285714,-0.187106606806923)(-0.266880580357143,-0.18848861022904)(-0.266741071428572,-0.189869727378032)(-0.2666015625,-0.191249955837392)(-0.266462053571429,-0.192629293192336)(-0.266322544642857,-0.19400773702982)(-0.266183035714286,-0.195385284938538)(-0.266043526785714,-0.196761934508927)(-0.265904017857143,-0.198137683333172)(-0.265764508928572,-0.199512529005212)(-0.265625,-0.200886469120746)(-0.265485491071429,-0.202259501277228)(-0.265345982142857,-0.203631623073887)(-0.265206473214286,-0.20500283211172)(-0.265066964285714,-0.206373125993502)(-0.264927455357143,-0.207742502323789)(-0.264787946428572,-0.209110958708921)(-0.2646484375,-0.210478492757032)(-0.264508928571429,-0.211845102078046)(-0.264369419642857,-0.21321078428369)(-0.264229910714286,-0.214575536987496)(-0.264090401785714,-0.215939357804805)(-0.263950892857143,-0.217302244352769)(-0.263811383928572,-0.218664194250362)(-0.263671875,-0.22002520511838)(-0.263532366071429,-0.221385274579442)(-0.263392857142857,-0.222744400258005)(-0.263253348214286,-0.224102579780361)(-0.263113839285714,-0.225459810774644)(-0.262974330357143,-0.226816090870832)(-0.262834821428572,-0.228171417700756)(-0.2626953125,-0.229525788898104)(-0.262555803571429,-0.230879202098416)(-0.262416294642857,-0.232231654939105)(-0.262276785714286,-0.23358314505945)(-0.262137276785714,-0.234933670100603)(-0.261997767857143,-0.236283227705596)(-0.261858258928572,-0.237631815519343)(-0.26171875,-0.238979431188647)(-0.261579241071429,-0.240326072362197)(-0.261439732142857,-0.241671736690586)(-0.261300223214286,-0.243016421826305)(-0.261160714285714,-0.244360125423751)(-0.261021205357143,-0.245702845139231)(-0.260881696428572,-0.247044578630969)(-0.2607421875,-0.248385323559107)(-0.260602678571429,-0.249725077585708)(-0.260463169642857,-0.251063838374768)(-0.260323660714286,-0.252401603592215)(-0.260184151785714,-0.253738370905913)(-0.260044642857143,-0.25507413798567)(-0.259905133928572,-0.256408902503241)(-0.259765625,-0.257742662132333)(-0.259626116071429,-0.259075414548602)(-0.259486607142857,-0.260407157429672)(-0.259347098214286,-0.26173788845513)(-0.259207589285714,-0.263067605306531)(-0.259068080357143,-0.264396305667405)(-0.258928571428572,-0.265723987223259)(-0.2587890625,-0.267050647661586)(-0.258649553571429,-0.268376284671861)(-0.258510044642857,-0.269700895945555)(-0.258370535714286,-0.271024479176135)(-0.258231026785714,-0.272347032059069)(-0.258091517857143,-0.273668552291827)(-0.257952008928572,-0.274989037573893)(-0.2578125,-0.276308485606766)(-0.257672991071429,-0.277626894093957)(-0.257533482142857,-0.278944260741007)(-0.257393973214286,-0.280260583255481)(-0.257254464285714,-0.281575859346979)(-0.257114955357143,-0.282890086727136)(-0.256975446428572,-0.284203263109628)(-0.2568359375,-0.285515386210179)(-0.256696428571429,-0.286826453746557)(-0.256556919642857,-0.288136463438589)(-0.256417410714286,-0.289445413008161)(-0.256277901785714,-0.29075330017922)(-0.256138392857143,-0.292060122677784)(-0.255998883928572,-0.293365878231941)(-0.255859375,-0.294670564571856)(-0.255719866071429,-0.295974179429773)(-0.255580357142857,-0.297276720540023)(-0.255440848214286,-0.298578185639028)(-0.255301339285714,-0.299878572465303)(-0.255161830357143,-0.301177878759462)(-0.255022321428572,-0.302476102264221)(-0.2548828125,-0.303773240724407)(-0.254743303571429,-0.305069291886952)(-0.254603794642857,-0.30636425350091)(-0.254464285714286,-0.307658123317453)(-0.254324776785714,-0.30895089908988)(-0.254185267857143,-0.310242578573616)(-0.254045758928572,-0.311533159526224)(-0.25390625,-0.312822639707403)(-0.253766741071429,-0.31411101687899)(-0.253627232142857,-0.315398288804976)(-0.253487723214286,-0.3166844532515)(-0.253348214285714,-0.317969507986855)(-0.253208705357143,-0.319253450781496)(-0.253069196428572,-0.320536279408041)(-0.2529296875,-0.32181799164128)(-0.252790178571429,-0.323098585258167)(-0.252650669642857,-0.324378058037842)(-0.252511160714286,-0.325656407761623)(-0.252371651785714,-0.326933632213013)(-0.252232142857143,-0.328209729177708)(-0.252092633928572,-0.329484696443594)(-0.251953125,-0.330758531800762)(-0.251813616071429,-0.332031233041498)(-0.251674107142857,-0.333302797960302)(-0.251534598214286,-0.334573224353881)(-0.251395089285714,-0.335842510021162)(-0.251255580357143,-0.337110652763289)(-0.251116071428572,-0.338377650383632)(-0.2509765625,-0.339643500687792)(-0.250837053571429,-0.340908201483595)(-0.250697544642857,-0.342171750581112)(-0.250558035714286,-0.343434145792654)(-0.250418526785714,-0.344695384932775)(-0.250279017857143,-0.345955465818283)(-0.250139508928572,-0.347214386268236)(-0.25,-0.348472144103957)(-0.249860491071429,-0.349728737149021)(-0.249720982142857,-0.35098416322928)(-0.249581473214286,-0.352238420172854)(-0.249441964285714,-0.353491505810137)(-0.249302455357143,-0.354743417973804)(-0.249162946428572,-0.355994154498815)(-0.2490234375,-0.357243713222417)(-0.248883928571429,-0.358492091984147)(-0.248744419642857,-0.359739288625841)(-0.248604910714286,-0.360985300991637)(-0.248465401785714,-0.362230126927975)(-0.248325892857143,-0.363473764283606)(-0.248186383928572,-0.364716210909594)(-0.248046875,-0.365957464659322)(-0.247907366071429,-0.367197523388488)(-0.247767857142857,-0.368436384955125)(-0.247628348214286,-0.36967404721959)(-0.247488839285714,-0.370910508044577)(-0.247349330357143,-0.372145765295116)(-0.247209821428572,-0.373379816838582)(-0.2470703125,-0.374612660544696)(-0.246930803571429,-0.375844294285527)(-0.246791294642857,-0.377074715935501)(-0.246651785714286,-0.378303923371405)(-0.246512276785714,-0.379531914472387)(-0.246372767857143,-0.380758687119961)(-0.246233258928572,-0.381984239198016)(-0.24609375,-0.383208568592816)(-0.245954241071429,-0.384431673193)(-0.245814732142857,-0.385653550889595)(-0.245675223214286,-0.386874199576016)(-0.245535714285714,-0.388093617148068)(-0.245396205357143,-0.389311801503954)(-0.245256696428572,-0.390528750544275)(-0.2451171875,-0.391744462172042)(-0.244977678571429,-0.392958934292663)(-0.244838169642857,-0.394172164813969)(-0.244698660714286,-0.395384151646204)(-0.244559151785714,-0.396594892702032)(-0.244419642857143,-0.397804385896541)(-0.244280133928572,-0.39901262914725)(-0.244140625,-0.400219620374111)(-0.244001116071429,-0.401425357499506)(-0.243861607142857,-0.402629838448267)(-0.243722098214286,-0.403833061147666)(-0.243582589285714,-0.405035023527424)(-0.243443080357143,-0.406235723519716)(-0.243303571428572,-0.407435159059175)(-0.2431640625,-0.408633328082894)(-0.243024553571429,-0.409830228530427)(-0.242885044642857,-0.411025858343803)(-0.242745535714286,-0.412220215467522)(-0.242606026785714,-0.413413297848559)(-0.242466517857143,-0.414605103436373)(-0.242327008928572,-0.415795630182906)(-0.2421875,-0.416984876042591)(-0.242047991071429,-0.41817283897235)(-0.241908482142857,-0.419359516931605)(-0.241768973214286,-0.42054490788228)(-0.241629464285714,-0.4217290097888)(-0.241489955357143,-0.422911820618103)(-0.241350446428572,-0.424093338339637)(-0.2412109375,-0.42527356092537)(-0.241071428571429,-0.426452486349784)(-0.240931919642857,-0.427630112589891)(-0.240792410714286,-0.428806437625232)(-0.240652901785714,-0.429981459437878)(-0.240513392857143,-0.431155176012438)(-0.240373883928572,-0.432327585336061)(-0.240234375,-0.433498685398441)(-0.240094866071429,-0.434668474191815)(-0.239955357142857,-0.43583694971098)(-0.239815848214286,-0.437004109953284)(-0.239676339285714,-0.438169952918637)(-0.239536830357143,-0.439334476609511)(-0.239397321428572,-0.440497679030947)(-0.2392578125,-0.441659558190559)(-0.239118303571429,-0.442820112098531)(-0.238978794642857,-0.443979338767632)(-0.238839285714286,-0.445137236213211)(-0.238699776785714,-0.446293802453207)(-0.238560267857143,-0.447449035508147)(-0.238420758928572,-0.448602933401153)(-0.23828125,-0.449755494157949)(-0.238141741071429,-0.450906715806855)(-0.238002232142857,-0.452056596378803)(-0.237862723214286,-0.453205133907332)(-0.237723214285714,-0.454352326428598)(-0.237583705357143,-0.455498171981372)(-0.237444196428572,-0.456642668607047)(-0.2373046875,-0.457785814349644)(-0.237165178571429,-0.458927607255807)(-0.237025669642857,-0.460068045374819)(-0.236886160714286,-0.461207126758597)(-0.236746651785714,-0.462344849461701)(-0.236607142857143,-0.463481211541332)(-0.236467633928572,-0.464616211057342)(-0.236328125,-0.465749846072235)(-0.236188616071429,-0.466882114651166)(-0.236049107142857,-0.468013014861955)(-0.235909598214286,-0.469142544775082)(-0.235770089285714,-0.470270702463696)(-0.235630580357143,-0.471397486003616)(-0.235491071428572,-0.472522893473335)(-0.2353515625,-0.473646922954026)(-0.235212053571429,-0.474769572529539)(-0.235072544642857,-0.475890840286416)(-0.234933035714286,-0.477010724313883)(-0.234793526785714,-0.478129222703864)(-0.234654017857143,-0.479246333550977)(-0.234514508928572,-0.48036205495254)(-0.234375,-0.481476385008579)(-0.234235491071429,-0.482589321821821)(-0.234095982142857,-0.483700863497711)(-0.233956473214286,-0.484811008144407)(-0.233816964285714,-0.485919753872787)(-0.233677455357143,-0.487027098796449)(-0.233537946428572,-0.48813304103172)(-0.2333984375,-0.489237578697658)(-0.233258928571429,-0.49034070991605)(-0.233119419642857,-0.491442432811424)(-0.232979910714286,-0.492542745511048)(-0.232840401785714,-0.493641646144936)(-0.232700892857143,-0.494739132845848)(-0.232561383928572,-0.495835203749298)(-0.232421875,-0.496929856993555)(-0.232282366071429,-0.498023090719643)(-0.232142857142857,-0.499114903071356)(-0.232003348214286,-0.500205292195249)(-0.231863839285714,-0.501294256240649)(-0.231724330357143,-0.502381793359658)(-0.231584821428572,-0.503467901707152)(-0.2314453125,-0.504552579440791)(-0.231305803571429,-0.505635824721015)(-0.231166294642857,-0.506717635711056)(-0.231026785714286,-0.507798010576936)(-0.230887276785714,-0.508876947487473)(-0.230747767857143,-0.509954444614283)(-0.230608258928572,-0.511030500131783)(-0.23046875,-0.5121051122172)(-0.230329241071429,-0.513178279050563)(-0.230189732142857,-0.514249998814719)(-0.230050223214286,-0.515320269695332)(-0.229910714285714,-0.516389089880884)(-0.229771205357143,-0.51745645756268)(-0.229631696428572,-0.518522370934854)(-0.2294921875,-0.51958682819437)(-0.229352678571429,-0.520649827541022)(-0.229213169642857,-0.521711367177448)(-0.229073660714286,-0.522771445309122)(-0.228934151785714,-0.523830060144365)(-0.228794642857143,-0.524887209894344)(-0.228655133928572,-0.525942892773081)(-0.228515625,-0.526997106997452)(-0.228376116071429,-0.528049850787184)(-0.228236607142857,-0.529101122364877)(-0.228097098214286,-0.530150919955989)(-0.227957589285714,-0.53119924178885)(-0.227818080357143,-0.532246086094661)(-0.227678571428572,-0.533291451107499)(-0.2275390625,-0.534335335064321)(-0.227399553571429,-0.535377736204964)(-0.227260044642857,-0.536418652772152)(-0.227120535714286,-0.537458083011501)(-0.226981026785714,-0.538496025171516)(-0.226841517857143,-0.539532477503601)(-0.226702008928572,-0.540567438262058)(-0.2265625,-0.541600905704095)(-0.226422991071429,-0.54263287808982)(-0.226283482142857,-0.543663353682256)(-0.226143973214286,-0.544692330747339)(-0.226004464285714,-0.545719807553919)(-0.225864955357143,-0.546745782373769)(-0.225725446428572,-0.547770253481583)(-0.2255859375,-0.548793219154984)(-0.225446428571429,-0.54981467767452)(-0.225306919642857,-0.550834627323677)(-0.225167410714286,-0.551853066388877)(-0.225027901785714,-0.55286999315948)(-0.224888392857143,-0.553885405927792)(-0.224748883928572,-0.554899302989063)(-0.224609375,-0.555911682641497)(-0.224469866071429,-0.556922543186246)(-0.224330357142857,-0.55793188292742)(-0.224190848214286,-0.558939700172093)(-0.224051339285714,-0.559945993230298)(-0.223911830357143,-0.560950760415036)(-0.223772321428572,-0.561954000042277)(-0.2236328125,-0.562955710430966)(-0.223493303571429,-0.563955889903021)(-0.223353794642857,-0.564954536783343)(-0.223214285714286,-0.565951649399813)(-0.223074776785714,-0.5669472260833)(-0.222935267857143,-0.567941265167662)(-0.222795758928572,-0.568933764989749)(-0.22265625,-0.569924723889409)(-0.222516741071429,-0.570914140209484)(-0.222377232142857,-0.571902012295823)(-0.222237723214286,-0.572888338497279)(-0.222098214285714,-0.573873117165714)(-0.221958705357143,-0.574856346656001)(-0.221819196428572,-0.57583802532603)(-0.2216796875,-0.576818151536709)(-0.221540178571429,-0.577796723651963)(-0.221400669642857,-0.578773740038747)(-0.221261160714286,-0.579749199067043)(-0.221121651785714,-0.580723099109862)(-0.220982142857143,-0.581695438543252)(-0.220842633928572,-0.582666215746296)(-0.220703125,-0.583635429101119)(-0.220563616071429,-0.584603076992887)(-0.220424107142857,-0.585569157809816)(-0.220284598214286,-0.586533669943171)(-0.220145089285714,-0.58749661178727)(-0.220005580357143,-0.588457981739486)(-0.219866071428572,-0.589417778200254)(-0.2197265625,-0.590375999573071)(-0.219587053571429,-0.591332644264494)(-0.219447544642857,-0.592287710684157)(-0.219308035714286,-0.593241197244759)(-0.219168526785714,-0.594193102362079)(-0.219029017857143,-0.595143424454971)(-0.218889508928572,-0.59609216194537)(-0.21875,-0.597039313258297)(-0.218610491071429,-0.597984876821856)(-0.218470982142857,-0.598928851067247)(-0.218331473214286,-0.599871234428758)(-0.218191964285714,-0.600812025343777)(-0.218052455357143,-0.601751222252789)(-0.217912946428572,-0.602688823599382)(-0.2177734375,-0.603624827830253)(-0.217633928571429,-0.604559233395199)(-0.217494419642857,-0.605492038747136)(-0.217354910714286,-0.606423242342091)(-0.217215401785714,-0.607352842639211)(-0.217075892857143,-0.608280838100761)(-0.216936383928572,-0.60920722719213)(-0.216796875,-0.610132008381835)(-0.216657366071429,-0.61105518014152)(-0.216517857142857,-0.611976740945962)(-0.216378348214286,-0.612896689273075)(-0.216238839285714,-0.61381502360391)(-0.216099330357143,-0.61473174242266)(-0.215959821428572,-0.615646844216661)(-0.2158203125,-0.616560327476398)(-0.215680803571429,-0.617472190695504)(-0.215541294642857,-0.618382432370765)(-0.215401785714286,-0.619291051002126)(-0.215262276785714,-0.620198045092689)(-0.215122767857143,-0.621103413148716)(-0.214983258928572,-0.622007153679637)(-0.21484375,-0.622909265198049)(-0.214704241071429,-0.623809746219715)(-0.214564732142857,-0.624708595263578)(-0.214425223214286,-0.625605810851753)(-0.214285714285714,-0.626501391509535)(-0.214146205357143,-0.627395335765403)(-0.214006696428572,-0.628287642151018)(-0.2138671875,-0.629178309201232)(-0.213727678571429,-0.630067335454082)(-0.213588169642857,-0.630954719450805)(-0.213448660714286,-0.63184045973583)(-0.213309151785714,-0.632724554856786)(-0.213169642857143,-0.633607003364507)(-0.213030133928572,-0.634487803813026)(-0.212890625,-0.635366954759591)(-0.212751116071429,-0.636244454764651)(-0.212611607142857,-0.637120302391876)(-0.212472098214286,-0.637994496208148)(-0.212332589285714,-0.638867034783571)(-0.212193080357143,-0.639737916691467)(-0.212053571428572,-0.640607140508385)(-0.2119140625,-0.641474704814101)(-0.211774553571429,-0.642340608191617)(-0.211635044642857,-0.643204849227171)(-0.211495535714286,-0.644067426510238)(-0.211356026785714,-0.644928338633527)(-0.211216517857143,-0.645787584192991)(-0.211077008928572,-0.646645161787826)(-0.2109375,-0.647501070020473)(-0.210797991071429,-0.648355307496622)(-0.210658482142857,-0.649207872825216)(-0.210518973214286,-0.650058764618453)(-0.210379464285714,-0.650907981491786)(-0.210239955357143,-0.651755522063929)(-0.210100446428572,-0.65260138495686)(-0.2099609375,-0.65344556879582)(-0.209821428571429,-0.654288072209318)(-0.209681919642857,-0.655128893829135)(-0.209542410714286,-0.655968032290324)(-0.209402901785714,-0.656805486231214)(-0.209263392857143,-0.657641254293415)(-0.209123883928572,-0.658475335121813)(-0.208984375,-0.659307727364585)(-0.208844866071429,-0.660138429673186)(-0.208705357142857,-0.660967440702366)(-0.208565848214286,-0.661794759110166)(-0.208426339285714,-0.66262038355792)(-0.208286830357143,-0.663444312710259)(-0.208147321428572,-0.664266545235115)(-0.2080078125,-0.66508707980372)(-0.207868303571429,-0.665905915090611)(-0.207728794642857,-0.666723049773633)(-0.207589285714286,-0.667538482533942)(-0.207449776785714,-0.668352212056004)(-0.207310267857143,-0.669164237027602)(-0.207170758928572,-0.669974556139835)(-0.20703125,-0.670783168087125)(-0.206891741071429,-0.67159007156721)(-0.206752232142857,-0.67239526528116)(-0.206612723214286,-0.67319874793337)(-0.206473214285714,-0.674000518231566)(-0.206333705357143,-0.674800574886804)(-0.206194196428572,-0.67559891661348)(-0.2060546875,-0.676395542129324)(-0.205915178571429,-0.677190450155405)(-0.205775669642857,-0.677983639416139)(-0.205636160714286,-0.678775108639285)(-0.205496651785714,-0.67956485655595)(-0.205357142857143,-0.68035288190059)(-0.205217633928572,-0.681139183411014)(-0.205078125,-0.68192375982839)(-0.204938616071429,-0.682706609897236)(-0.204799107142857,-0.683487732365435)(-0.204659598214286,-0.684267125984232)(-0.204520089285714,-0.685044789508236)(-0.204380580357143,-0.685820721695424)(-0.204241071428572,-0.686594921307141)(-0.2041015625,-0.687367387108107)(-0.203962053571429,-0.688138117866412)(-0.203822544642857,-0.688907112353527)(-0.203683035714286,-0.689674369344301)(-0.203543526785714,-0.690439887616964)(-0.203404017857143,-0.69120366595313)(-0.203264508928572,-0.6919657031378)(-0.203125,-0.692725997959366)(-0.202985491071429,-0.693484549209604)(-0.202845982142857,-0.694241355683691)(-0.202706473214286,-0.694996416180197)(-0.202566964285714,-0.69574972950109)(-0.202427455357143,-0.696501294451737)(-0.202287946428572,-0.697251109840912)(-0.2021484375,-0.697999174480791)(-0.202008928571429,-0.698745487186955)(-0.201869419642857,-0.699490046778401)(-0.201729910714286,-0.700232852077533)(-0.201590401785714,-0.700973901910172)(-0.201450892857143,-0.701713195105555)(-0.201311383928572,-0.702450730496336)(-0.201171875,-0.703186506918594)(-0.201032366071429,-0.703920523211827)(-0.200892857142857,-0.704652778218962)(-0.200753348214286,-0.705383270786353)(-0.200613839285714,-0.706111999763783)(-0.200474330357143,-0.70683896400447)(-0.200334821428572,-0.707564162365064)(-0.2001953125,-0.708287593705654)(-0.200055803571429,-0.709009256889765)(-0.199916294642857,-0.709729150784368)(-0.199776785714286,-0.710447274259873)(-0.199637276785714,-0.711163626190138)(-0.199497767857143,-0.71187820545247)(-0.199358258928572,-0.712591010927623)(-0.19921875,-0.713302041499807)(-0.199079241071429,-0.714011296056682)(-0.198939732142857,-0.714718773489368)(-0.198800223214286,-0.715424472692443)(-0.198660714285714,-0.716128392563946)(-0.198521205357143,-0.716830532005377)(-0.198381696428572,-0.717530889921705)(-0.1982421875,-0.718229465221365)(-0.198102678571429,-0.718926256816258)(-0.197963169642857,-0.719621263621761)(-0.197823660714286,-0.720314484556722)(-0.197684151785714,-0.721005918543467)(-0.197544642857143,-0.721695564507799)(-0.197405133928572,-0.722383421378999)(-0.197265625,-0.723069488089834)(-0.197126116071429,-0.723753763576551)(-0.196986607142857,-0.724436246778886)(-0.196847098214286,-0.725116936640062)(-0.196707589285714,-0.725795832106793)(-0.196568080357143,-0.726472932129284)(-0.196428571428572,-0.727148235661238)(-0.1962890625,-0.727821741659851)(-0.196149553571429,-0.728493449085817)(-0.196010044642857,-0.729163356903333)(-0.195870535714286,-0.729831464080098)(-0.195731026785714,-0.730497769587314)(-0.195591517857143,-0.731162272399692)(-0.195452008928572,-0.731824971495449)(-0.1953125,-0.732485865856316)(-0.195172991071429,-0.73314495446753)(-0.195033482142857,-0.73380223631785)(-0.194893973214286,-0.734457710399547)(-0.194754464285714,-0.735111375708412)(-0.194614955357143,-0.735763231243755)(-0.194475446428572,-0.73641327600841)(-0.1943359375,-0.737061509008736)(-0.194196428571429,-0.737707929254613)(-0.194056919642857,-0.738352535759456)(-0.193917410714286,-0.738995327540204)(-0.193777901785714,-0.739636303617334)(-0.193638392857143,-0.740275463014851)(-0.193498883928572,-0.740912804760299)(-0.193359375,-0.741548327884761)(-0.193219866071429,-0.742182031422855)(-0.193080357142857,-0.742813914412744)(-0.192940848214286,-0.743443975896133)(-0.192801339285714,-0.744072214918273)(-0.192661830357143,-0.744698630527961)(-0.192522321428572,-0.745323221777543)(-0.1923828125,-0.745945987722917)(-0.192243303571429,-0.746566927423532)(-0.192103794642857,-0.747186039942391)(-0.191964285714286,-0.747803324346055)(-0.191824776785714,-0.748418779704642)(-0.191685267857143,-0.749032405091829)(-0.191545758928572,-0.749644199584858)(-0.19140625,-0.75025416226453)(-0.191266741071429,-0.750862292215214)(-0.191127232142857,-0.751468588524844)(-0.190987723214286,-0.752073050284925)(-0.190848214285714,-0.752675676590531)(-0.190708705357143,-0.753276466540309)(-0.190569196428572,-0.75387541923648)(-0.1904296875,-0.754472533784842)(-0.190290178571429,-0.755067809294766)(-0.190150669642857,-0.755661244879207)(-0.190011160714286,-0.7562528396547)(-0.189871651785714,-0.756842592741362)(-0.189732142857143,-0.757430503262894)(-0.189592633928572,-0.758016570346584)(-0.189453125,-0.758600793123308)(-0.189313616071429,-0.75918317072753)(-0.189174107142857,-0.759763702297308)(-0.189034598214286,-0.760342386974289)(-0.188895089285714,-0.760919223903719)(-0.188755580357143,-0.761494212234436)(-0.188616071428572,-0.762067351118879)(-0.1884765625,-0.762638639713086)(-0.188337053571429,-0.763208077176694)(-0.188197544642857,-0.763775662672945)(-0.188058035714286,-0.764341395368684)(-0.187918526785714,-0.764905274434365)(-0.187779017857143,-0.765467299044045)(-0.187639508928572,-0.766027468375395)(-0.1875,-0.766585781609694)(-0.187360491071429,-0.767142237931833)(-0.187220982142857,-0.767696836530319)(-0.187081473214286,-0.768249576597274)(-0.186941964285714,-0.768800457328437)(-0.186802455357143,-0.769349477923166)(-0.186662946428572,-0.769896637584438)(-0.1865234375,-0.770441935518856)(-0.186383928571429,-0.77098537093664)(-0.186244419642857,-0.77152694305164)(-0.186104910714286,-0.772066651081331)(-0.185965401785714,-0.772604494246816)(-0.185825892857143,-0.773140471772828)(-0.185686383928572,-0.773674582887729)(-0.185546875,-0.774206826823517)(-0.185407366071429,-0.774737202815821)(-0.185267857142857,-0.775265710103907)(-0.185128348214286,-0.775792347930677)(-0.184988839285714,-0.776317115542673)(-0.184849330357143,-0.776840012190075)(-0.184709821428572,-0.777361037126707)(-0.1845703125,-0.777880189610035)(-0.184430803571429,-0.778397468901167)(-0.184291294642857,-0.778912874264858)(-0.184151785714286,-0.779426404969512)(-0.184012276785714,-0.779938060287179)(-0.183872767857143,-0.780447839493562)(-0.183733258928572,-0.780955741868012)(-0.18359375,-0.781461766693536)(-0.183454241071429,-0.781965913256792)(-0.183314732142857,-0.782468180848097)(-0.183175223214286,-0.782968568761422)(-0.183035714285714,-0.783467076294398)(-0.182896205357143,-0.783963702748315)(-0.182756696428572,-0.784458447428126)(-0.1826171875,-0.784951309642445)(-0.182477678571429,-0.785442288703547)(-0.182338169642857,-0.785931383927378)(-0.182198660714286,-0.786418594633545)(-0.182059151785714,-0.786903920145327)(-0.181919642857143,-0.787387359789669)(-0.181780133928572,-0.787868912897189)(-0.181640625,-0.788348578802174)(-0.181501116071429,-0.788826356842586)(-0.181361607142857,-0.78930224636006)(-0.181222098214286,-0.789776246699908)(-0.181082589285714,-0.790248357211118)(-0.180943080357143,-0.790718577246357)(-0.180803571428572,-0.791186906161969)(-0.1806640625,-0.791653343317982)(-0.180524553571429,-0.792117888078103)(-0.180385044642857,-0.792580539809723)(-0.180245535714286,-0.793041297883919)(-0.180106026785714,-0.79350016167545)(-0.179966517857143,-0.793957130562767)(-0.179827008928572,-0.794412203928003)(-0.1796875,-0.794865381156987)(-0.179547991071429,-0.79531666163923)(-0.179408482142857,-0.795766044767942)(-0.179268973214286,-0.796213529940023)(-0.179129464285714,-0.796659116556065)(-0.178989955357143,-0.797102804020359)(-0.178850446428572,-0.79754459174089)(-0.1787109375,-0.79798447912934)(-0.178571428571429,-0.798422465601091)(-0.178431919642857,-0.798858550575223)(-0.178292410714286,-0.799292733474518)(-0.178152901785714,-0.799725013725461)(-0.178013392857143,-0.800155390758237)(-0.177873883928572,-0.800583864006739)(-0.177734375,-0.801010432908562)(-0.177594866071429,-0.801435096905009)(-0.177455357142857,-0.80185785544109)(-0.177315848214286,-0.802278707965524)(-0.177176339285714,-0.802697653930739)(-0.177036830357143,-0.803114692792875)(-0.176897321428572,-0.803529824011781)(-0.1767578125,-0.803943047051023)(-0.176618303571429,-0.804354361377876)(-0.176478794642857,-0.804763766463333)(-0.176339285714286,-0.805171261782102)(-0.176199776785714,-0.805576846812609)(-0.176060267857143,-0.805980521036997)(-0.175920758928572,-0.806382283941127)(-0.17578125,-0.806782135014583)(-0.175641741071429,-0.807180073750665)(-0.175502232142857,-0.807576099646399)(-0.175362723214286,-0.807970212202534)(-0.175223214285714,-0.80836241092354)(-0.175083705357143,-0.808752695317614)(-0.174944196428572,-0.809141064896679)(-0.1748046875,-0.809527519176384)(-0.174665178571429,-0.809912057676105)(-0.174525669642857,-0.810294679918946)(-0.174386160714286,-0.810675385431745)(-0.174246651785714,-0.811054173745065)(-0.174107142857143,-0.811431044393204)(-0.173967633928572,-0.811805996914191)(-0.173828125,-0.812179030849789)(-0.173688616071429,-0.812550145745492)(-0.173549107142857,-0.812919341150534)(-0.173409598214286,-0.81328661661788)(-0.173270089285714,-0.813651971704235)(-0.173130580357143,-0.814015405970042)(-0.172991071428572,-0.814376918979478)(-0.1728515625,-0.814736510300465)(-0.172712053571429,-0.815094179504661)(-0.172572544642857,-0.815449926167466)(-0.172433035714286,-0.815803749868024)(-0.172293526785714,-0.816155650189217)(-0.172154017857143,-0.816505626717676)(-0.172014508928572,-0.816853679043772)(-0.171875,-0.817199806761623)(-0.171735491071429,-0.817544009469091)(-0.171595982142857,-0.817886286767787)(-0.171456473214286,-0.818226638263068)(-0.171316964285714,-0.818565063564039)(-0.171177455357143,-0.818901562283555)(-0.171037946428572,-0.81923613403822)(-0.1708984375,-0.819568778448388)(-0.170758928571429,-0.819899495138164)(-0.170619419642857,-0.820228283735406)(-0.170479910714286,-0.820555143871723)(-0.170340401785714,-0.82088007518248)(-0.170200892857143,-0.821203077306792)(-0.170061383928572,-0.821524149887533)(-0.169921875,-0.821843292571329)(-0.169782366071429,-0.822160505008565)(-0.169642857142857,-0.82247578685338)(-0.169503348214286,-0.822789137763672)(-0.169363839285714,-0.823100557401097)(-0.169224330357143,-0.82341004543107)(-0.169084821428572,-0.823717601522766)(-0.1689453125,-0.824023225349117)(-0.168805803571429,-0.82432691658682)(-0.168666294642857,-0.82462867491633)(-0.168526785714286,-0.824928500021865)(-0.168387276785714,-0.825226391591407)(-0.168247767857143,-0.825522349316699)(-0.168108258928572,-0.82581637289325)(-0.16796875,-0.826108462020331)(-0.167829241071429,-0.82639861640098)(-0.167689732142857,-0.826686835741999)(-0.167550223214286,-0.826973119753957)(-0.167410714285714,-0.82725746815119)(-0.167271205357143,-0.8275398806518)(-0.167131696428572,-0.827820356977658)(-0.1669921875,-0.828098896854403)(-0.166852678571429,-0.828375500011442)(-0.166713169642857,-0.828650166181953)(-0.166573660714286,-0.828922895102883)(-0.166434151785714,-0.829193686514948)(-0.166294642857143,-0.829462540162638)(-0.166155133928572,-0.829729455794211)(-0.166015625,-0.829994433161701)(-0.165876116071429,-0.83025747202091)(-0.165736607142857,-0.830518572131416)(-0.165597098214286,-0.830777733256569)(-0.165457589285714,-0.831034955163493)(-0.165318080357143,-0.831290237623087)(-0.165178571428572,-0.831543580410023)(-0.1650390625,-0.83179498330275)(-0.164899553571429,-0.83204444608349)(-0.164760044642857,-0.832291968538245)(-0.164620535714286,-0.83253755045679)(-0.164481026785714,-0.832781191632677)(-0.164341517857143,-0.833022891863238)(-0.164202008928572,-0.83326265094958)(-0.1640625,-0.833500468696589)(-0.163922991071429,-0.833736344912927)(-0.163783482142857,-0.83397027941104)(-0.163643973214286,-0.834202272007148)(-0.163504464285714,-0.834432322521253)(-0.163364955357143,-0.834660430777137)(-0.163225446428572,-0.83488659660236)(-0.1630859375,-0.835110819828266)(-0.162946428571429,-0.835333100289975)(-0.162806919642857,-0.835553437826393)(-0.162667410714286,-0.835771832280205)(-0.162527901785714,-0.835988283497878)(-0.162388392857143,-0.836202791329661)(-0.162248883928572,-0.836415355629586)(-0.162109375,-0.836625976255468)(-0.161969866071429,-0.836834653068902)(-0.161830357142857,-0.83704138593527)(-0.161690848214286,-0.837246174723736)(-0.161551339285714,-0.837449019307246)(-0.161411830357143,-0.837649919562533)(-0.161272321428572,-0.837848875370112)(-0.1611328125,-0.838045886614283)(-0.160993303571429,-0.838240953183131)(-0.160853794642857,-0.838434074968526)(-0.160714285714286,-0.838625251866122)(-0.160574776785714,-0.838814483775359)(-0.160435267857143,-0.839001770599463)(-0.160295758928572,-0.839187112245444)(-0.16015625,-0.839370508624101)(-0.160016741071429,-0.839551959650016)(-0.159877232142857,-0.839731465241557)(-0.159737723214286,-0.839909025320881)(-0.159598214285714,-0.84008463981393)(-0.159458705357143,-0.840258308650432)(-0.159319196428572,-0.840430031763903)(-0.1591796875,-0.840599809091645)(-0.159040178571429,-0.840767640574747)(-0.158900669642857,-0.840933526158087)(-0.158761160714286,-0.841097465790327)(-0.158621651785714,-0.84125945942392)(-0.158482142857143,-0.841419507015103)(-0.158342633928572,-0.841577608523903)(-0.158203125,-0.841733763914134)(-0.158063616071429,-0.841887973153396)(-0.157924107142857,-0.84204023621308)(-0.157784598214286,-0.842190553068362)(-0.157645089285714,-0.842338923698207)(-0.157505580357143,-0.842485348085368)(-0.157366071428572,-0.842629826216384)(-0.1572265625,-0.842772358081586)(-0.157087053571429,-0.84291294367509)(-0.156947544642857,-0.843051582994799)(-0.156808035714286,-0.843188276042406)(-0.156668526785714,-0.843323022823392)(-0.156529017857143,-0.843455823347026)(-0.156389508928572,-0.843586677626363)(-0.15625,-0.843715585678249)(-0.156110491071429,-0.843842547523315)(-0.155970982142857,-0.84396756318598)(-0.155831473214286,-0.844090632694455)(-0.155691964285714,-0.844211756080732)(-0.155552455357143,-0.844330933380597)(-0.155412946428572,-0.84444816463362)(-0.1552734375,-0.84456344988316)(-0.155133928571429,-0.844676789176362)(-0.154994419642857,-0.844788182564159)(-0.154854910714286,-0.844897630101273)(-0.154715401785714,-0.84500513184621)(-0.154575892857143,-0.845110687861266)(-0.154436383928572,-0.845214298212523)(-0.154296875,-0.845315962969848)(-0.154157366071429,-0.845415682206897)(-0.154017857142857,-0.845513456001111)(-0.153878348214286,-0.845609284433719)(-0.153738839285714,-0.845703167589735)(-0.153599330357143,-0.845795105557959)(-0.153459821428572,-0.845885098430976)(-0.1533203125,-0.84597314630516)(-0.153180803571429,-0.846059249280667)(-0.153041294642857,-0.846143407461439)(-0.152901785714286,-0.846225620955205)(-0.152762276785714,-0.846305889873477)(-0.152622767857143,-0.846384214331553)(-0.152483258928572,-0.846460594448513)(-0.15234375,-0.846535030347225)(-0.152204241071429,-0.846607522154338)(-0.152064732142857,-0.846678070000285)(-0.151925223214286,-0.846746674019284)(-0.151785714285714,-0.846813334349335)(-0.151646205357143,-0.846878051132222)(-0.151506696428572,-0.84694082451351)(-0.1513671875,-0.847001654642548)(-0.151227678571429,-0.847060541672467)(-0.151088169642857,-0.847117485760178)(-0.150948660714286,-0.847172487066375)(-0.150809151785714,-0.847225545755534)(-0.150669642857143,-0.84727666199591)(-0.150530133928572,-0.84732583595954)(-0.150390625,-0.84737306782224)(-0.150251116071429,-0.847418357763606)(-0.150111607142857,-0.847461705967015)(-0.149972098214286,-0.847503112619622)(-0.149832589285714,-0.84754257791236)(-0.149693080357143,-0.847580102039942)(-0.149553571428572,-0.847615685200859)(-0.1494140625,-0.847649327597377)(-0.149274553571429,-0.847681029435543)(-0.149135044642857,-0.847710790925179)(-0.148995535714286,-0.847738612279883)(-0.148856026785714,-0.847764493717029)(-0.148716517857143,-0.847788435457769)(-0.148577008928572,-0.847810437727026)(-0.1484375,-0.847830500753503)(-0.148297991071429,-0.847848624769672)(-0.148158482142857,-0.847864810011782)(-0.148018973214286,-0.847879056719855)(-0.147879464285714,-0.847891365137686)(-0.147739955357143,-0.847901735512841)(-0.147600446428572,-0.847910168096659)(-0.1474609375,-0.847916663144252)(-0.147321428571429,-0.847921220914499)(-0.147181919642857,-0.847923841670054)(-0.147042410714286,-0.847924525677338)(-0.146902901785714,-0.847923273206542)(-0.146763392857143,-0.847920084531627)(-0.146623883928572,-0.847914959930322)(-0.146484375,-0.847907899684121)(-0.146344866071429,-0.84789890407829)(-0.146205357142857,-0.84788797340186)(-0.146065848214286,-0.847875107947625)(-0.145926339285714,-0.84786030801215)(-0.145786830357143,-0.84784357389576)(-0.145647321428572,-0.847824905902549)(-0.1455078125,-0.847804304340371)(-0.145368303571429,-0.847781769520845)(-0.145228794642857,-0.847757301759353)(-0.145089285714286,-0.847730901375038)(-0.144949776785714,-0.847702568690806)(-0.144810267857143,-0.847672304033322)(-0.144670758928572,-0.847640107733011)(-0.14453125,-0.84760598012406)(-0.144391741071429,-0.847569921544413)(-0.144252232142857,-0.847531932335771)(-0.144112723214286,-0.847492012843596)(-0.143973214285714,-0.847450163417103)(-0.143833705357143,-0.847406384409266)(-0.143694196428572,-0.847360676176814)(-0.1435546875,-0.847313039080229)(-0.143415178571429,-0.847263473483749)(-0.143275669642857,-0.847211979755365)(-0.143136160714286,-0.84715855826682)(-0.142996651785714,-0.84710320939361)(-0.142857142857143,-0.847045933514981)(-0.142717633928572,-0.846986731013931)(-0.142578125,-0.846925602277206)(-0.142438616071429,-0.846862547695302)(-0.142299107142857,-0.846797567662463)(-0.142159598214286,-0.846730662576682)(-0.142020089285714,-0.846661832839695)(-0.141880580357143,-0.846591078856986)(-0.141741071428572,-0.846518401037787)(-0.1416015625,-0.846443799795068)(-0.141462053571429,-0.846367275545549)(-0.141322544642857,-0.846288828709688)(-0.141183035714286,-0.846208459711687)(-0.141043526785714,-0.846126168979488)(-0.140904017857143,-0.846041956944776)(-0.140764508928572,-0.84595582404297)(-0.140625,-0.845867770713234)(-0.140485491071429,-0.845777797398464)(-0.140345982142857,-0.845685904545297)(-0.140206473214286,-0.845592092604103)(-0.140066964285714,-0.845496362028989)(-0.139927455357143,-0.845398713277796)(-0.139787946428572,-0.845299146812096)(-0.1396484375,-0.845197663097197)(-0.139508928571429,-0.845094262602135)(-0.139369419642857,-0.84498894579968)(-0.139229910714286,-0.844881713166329)(-0.139090401785714,-0.844772565182309)(-0.138950892857143,-0.844661502331573)(-0.138811383928572,-0.844548525101804)(-0.138671875,-0.844433633984409)(-0.138532366071429,-0.844316829474519)(-0.138392857142857,-0.844198112070992)(-0.138253348214286,-0.844077482276405)(-0.138113839285714,-0.843954940597061)(-0.137974330357143,-0.843830487542981)(-0.137834821428572,-0.843704123627908)(-0.1376953125,-0.843575849369303)(-0.137555803571429,-0.843445665288346)(-0.137416294642857,-0.843313571909932)(-0.137276785714286,-0.843179569762675)(-0.137137276785714,-0.843043659378901)(-0.136997767857143,-0.842905841294653)(-0.136858258928572,-0.842766116049685)(-0.13671875,-0.842624484187462)(-0.136579241071429,-0.842480946255161)(-0.136439732142857,-0.84233550280367)(-0.136300223214286,-0.842188154387584)(-0.136160714285714,-0.842038901565205)(-0.136021205357143,-0.841887744898544)(-0.135881696428572,-0.841734684953314)(-0.1357421875,-0.841579722298935)(-0.135602678571429,-0.84142285750853)(-0.135463169642857,-0.841264091158922)(-0.135323660714286,-0.841103423830637)(-0.135184151785714,-0.840940856107901)(-0.135044642857143,-0.840776388578637)(-0.134905133928572,-0.840610021834466)(-0.134765625,-0.840441756470707)(-0.134626116071429,-0.840271593086372)(-0.134486607142857,-0.840099532284169)(-0.134347098214286,-0.839925574670497)(-0.134207589285714,-0.839749720855448)(-0.134068080357143,-0.839571971452805)(-0.133928571428572,-0.839392327080039)(-0.1337890625,-0.83921078835831)(-0.133649553571429,-0.839027355912464)(-0.133510044642857,-0.838842030371034)(-0.133370535714286,-0.838654812366237)(-0.133231026785714,-0.838465702533972)(-0.133091517857143,-0.838274701513821)(-0.132952008928572,-0.838081809949049)(-0.1328125,-0.837887028486596)(-0.132672991071429,-0.837690357777084)(-0.132533482142857,-0.83749179847481)(-0.132393973214286,-0.837291351237748)(-0.132254464285714,-0.837089016727545)(-0.132114955357143,-0.836884795609522)(-0.131975446428572,-0.836678688552672)(-0.1318359375,-0.836470696229658)(-0.131696428571429,-0.836260819316814)(-0.131556919642857,-0.836049058494139)(-0.131417410714286,-0.8358354144453)(-0.131277901785714,-0.835619887857631)(-0.131138392857143,-0.835402479422127)(-0.130998883928572,-0.835183189833447)(-0.130859375,-0.83496201978991)(-0.130719866071429,-0.834738969993499)(-0.130580357142857,-0.834514041149851)(-0.130440848214286,-0.834287233968261)(-0.130301339285714,-0.834058549161681)(-0.130161830357143,-0.833827987446717)(-0.130022321428572,-0.833595549543628)(-0.1298828125,-0.833361236176324)(-0.129743303571429,-0.833125048072365)(-0.129603794642857,-0.832886985962961)(-0.129464285714286,-0.832647050582969)(-0.129324776785714,-0.832405242670891)(-0.129185267857143,-0.832161562968874)(-0.129045758928572,-0.831916012222707)(-0.12890625,-0.831668591181823)(-0.128766741071429,-0.831419300599294)(-0.128627232142857,-0.831168141231829)(-0.128487723214286,-0.830915113839776)(-0.128348214285714,-0.830660219187118)(-0.128208705357143,-0.830403458041472)(-0.128069196428572,-0.83014483117409)(-0.1279296875,-0.82988433935985)(-0.127790178571429,-0.829621983377266)(-0.127650669642857,-0.829357764008475)(-0.127511160714286,-0.829091682039245)(-0.127371651785714,-0.828823738258965)(-0.127232142857143,-0.82855393346065)(-0.127092633928572,-0.828282268440937)(-0.126953125,-0.828008744000083)(-0.126813616071429,-0.827733360941963)(-0.126674107142857,-0.827456120074072)(-0.126534598214286,-0.827177022207519)(-0.126395089285714,-0.826896068157026)(-0.126255580357143,-0.826613258740931)(-0.126116071428572,-0.82632859478118)(-0.1259765625,-0.826042077103329)(-0.125837053571429,-0.825753706536543)(-0.125697544642857,-0.825463483913594)(-0.125558035714286,-0.825171410070855)(-0.125418526785714,-0.824877485848306)(-0.125279017857143,-0.824581712089525)(-0.125139508928572,-0.824284089641693)(-0.125,-0.823984619355586)(-0.124860491071429,-0.823683302085579)(-0.124720982142857,-0.823380138689639)(-0.124581473214286,-0.823075130029329)(-0.124441964285714,-0.822768276969801)(-0.124302455357143,-0.822459580379797)(-0.124162946428572,-0.82214904113165)(-0.1240234375,-0.821836660101274)(-0.123883928571429,-0.821522438168173)(-0.123744419642857,-0.82120637621543)(-0.123604910714286,-0.820888475129712)(-0.123465401785714,-0.820568735801263)(-0.123325892857143,-0.820247159123907)(-0.123186383928572,-0.819923745995041)(-0.123046875,-0.819598497315639)(-0.122907366071429,-0.819271413990247)(-0.122767857142857,-0.818942496926982)(-0.122628348214286,-0.818611747037527)(-0.122488839285714,-0.818279165237136)(-0.122349330357143,-0.817944752444626)(-0.122209821428572,-0.81760850958238)(-0.1220703125,-0.817270437576339)(-0.121930803571429,-0.816930537356008)(-0.121791294642857,-0.816588809854447)(-0.121651785714286,-0.816245256008275)(-0.121512276785714,-0.815899876757665)(-0.121372767857143,-0.815552673046341)(-0.121233258928572,-0.815203645821579)(-0.12109375,-0.814852796034204)(-0.120954241071429,-0.814500124638589)(-0.120814732142857,-0.814145632592652)(-0.120675223214286,-0.813789320857853)(-0.120535714285714,-0.813431190399196)(-0.120396205357143,-0.813071242185222)(-0.120256696428572,-0.812709477188013)(-0.1201171875,-0.812345896383184)(-0.119977678571429,-0.811980500749886)(-0.119838169642857,-0.811613291270801)(-0.119698660714286,-0.811244268932143)(-0.119559151785714,-0.810873434723652)(-0.119419642857143,-0.810500789638596)(-0.119280133928572,-0.810126334673766)(-0.119140625,-0.809750070829477)(-0.119001116071429,-0.809371999109564)(-0.118861607142857,-0.808992120521382)(-0.118722098214286,-0.808610436075799)(-0.118582589285714,-0.808226946787202)(-0.118443080357143,-0.807841653673488)(-0.118303571428572,-0.807454557756064)(-0.1181640625,-0.807065660059849)(-0.118024553571429,-0.806674961613267)(-0.117885044642857,-0.806282463448246)(-0.117745535714286,-0.805888166600218)(-0.117606026785714,-0.805492072108115)(-0.117466517857143,-0.805094181014367)(-0.117327008928572,-0.804694494364903)(-0.1171875,-0.804293013209144)(-0.117047991071429,-0.803889738600006)(-0.116908482142857,-0.803484671593894)(-0.116768973214286,-0.803077813250702)(-0.116629464285714,-0.802669164633809)(-0.116489955357143,-0.802258726810081)(-0.116350446428572,-0.801846500849865)(-0.1162109375,-0.801432487826987)(-0.116071428571429,-0.801016688818753)(-0.115931919642857,-0.800599104905945)(-0.115792410714286,-0.800179737172818)(-0.115652901785714,-0.799758586707098)(-0.115513392857143,-0.799335654599982)(-0.115373883928572,-0.798910941946135)(-0.115234375,-0.798484449843685)(-0.115094866071429,-0.798056179394227)(-0.114955357142857,-0.797626131702814)(-0.114815848214286,-0.79719430787796)(-0.114676339285714,-0.796760709031634)(-0.114536830357143,-0.796325336279261)(-0.114397321428572,-0.795888190739719)(-0.1142578125,-0.795449273535333)(-0.114118303571429,-0.795008585791881)(-0.113978794642857,-0.794566128638584)(-0.113839285714286,-0.794121903208107)(-0.113699776785714,-0.793675910636558)(-0.113560267857143,-0.793228152063481)(-0.113420758928572,-0.792778628631862)(-0.11328125,-0.792327341488116)(-0.113141741071429,-0.791874291782097)(-0.113002232142857,-0.791419480667085)(-0.112862723214286,-0.79096290929979)(-0.112723214285714,-0.790504578840347)(-0.112583705357143,-0.790044490452316)(-0.112444196428572,-0.789582645302676)(-0.1123046875,-0.789119044561827)(-0.112165178571429,-0.788653689403587)(-0.112025669642857,-0.788186581005186)(-0.111886160714286,-0.787717720547268)(-0.111746651785714,-0.787247109213885)(-0.111607142857143,-0.7867747481925)(-0.111467633928572,-0.786300638673978)(-0.111328125,-0.785824781852589)(-0.111188616071429,-0.785347178926004)(-0.111049107142857,-0.78486783109529)(-0.110909598214286,-0.784386739564914)(-0.110770089285714,-0.783903905542732)(-0.110630580357143,-0.783419330239995)(-0.110491071428572,-0.782933014871342)(-0.1103515625,-0.782444960654797)(-0.110212053571429,-0.781955168811771)(-0.110072544642857,-0.781463640567056)(-0.109933035714286,-0.780970377148823)(-0.109793526785714,-0.780475379788619)(-0.109654017857143,-0.779978649721367)(-0.109514508928572,-0.779480188185364)(-0.109375,-0.778979996422273)(-0.109235491071429,-0.778478075677129)(-0.109095982142857,-0.777974427198328)(-0.108956473214286,-0.777469052237631)(-0.108816964285714,-0.776961952050159)(-0.108677455357143,-0.77645312789439)(-0.108537946428572,-0.775942581032158)(-0.1083984375,-0.775430312728648)(-0.108258928571429,-0.774916324252399)(-0.108119419642857,-0.774400616875296)(-0.107979910714286,-0.773883191872568)(-0.107840401785714,-0.773364050522788)(-0.107700892857143,-0.772843194107871)(-0.107561383928572,-0.772320623913068)(-0.107421875,-0.771796341226966)(-0.107282366071429,-0.771270347341487)(-0.107142857142857,-0.77074264355188)(-0.107003348214286,-0.770213231156724)(-0.106863839285714,-0.769682111457924)(-0.106724330357143,-0.769149285760705)(-0.106584821428572,-0.768614755373615)(-0.1064453125,-0.76807852160852)(-0.106305803571429,-0.767540585780599)(-0.106166294642857,-0.767000949208347)(-0.106026785714286,-0.766459613213565)(-0.105887276785714,-0.765916579121364)(-0.105747767857143,-0.765371848260161)(-0.105608258928572,-0.764825421961672)(-0.10546875,-0.764277301560916)(-0.105329241071429,-0.763727488396207)(-0.105189732142857,-0.763175983809157)(-0.105050223214286,-0.762622789144664)(-0.104910714285714,-0.762067905750922)(-0.104771205357143,-0.761511334979406)(-0.104631696428572,-0.760953078184879)(-0.1044921875,-0.760393136725384)(-0.104352678571429,-0.759831511962243)(-0.104213169642857,-0.759268205260054)(-0.104073660714286,-0.758703217986689)(-0.103934151785714,-0.75813655151329)(-0.103794642857143,-0.757568207214268)(-0.103655133928572,-0.756998186467299)(-0.103515625,-0.756426490653322)(-0.103376116071429,-0.755853121156536)(-0.103236607142857,-0.755278079364399)(-0.103097098214286,-0.75470136666762)(-0.102957589285714,-0.754122984460165)(-0.102818080357143,-0.753542934139244)(-0.102678571428572,-0.752961217105318)(-0.1025390625,-0.752377834762089)(-0.102399553571429,-0.751792788516503)(-0.102260044642857,-0.751206079778742)(-0.102120535714286,-0.750617709962224)(-0.101981026785714,-0.750027680483601)(-0.101841517857143,-0.749435992762755)(-0.101702008928572,-0.748842648222793)(-0.1015625,-0.748247648290048)(-0.101422991071429,-0.747650994394079)(-0.101283482142857,-0.747052687967658)(-0.101143973214286,-0.746452730446775)(-0.101004464285714,-0.745851123270636)(-0.100864955357143,-0.745247867881655)(-0.100725446428572,-0.744642965725455)(-0.1005859375,-0.744036418250862)(-0.100446428571429,-0.743428226909909)(-0.100306919642857,-0.742818393157824)(-0.100167410714286,-0.742206918453033)(-0.100027901785714,-0.741593804257156)(-0.099888392857143,-0.740979052035003)(-0.0997488839285716,-0.740362663254573)(-0.099609375,-0.73974463938705)(-0.0994698660714286,-0.739124981906801)(-0.0993303571428572,-0.73850369229137)(-0.0991908482142858,-0.737880772021482)(-0.0990513392857144,-0.73725622258103)(-0.098911830357143,-0.736630045457083)(-0.0987723214285716,-0.736002242139874)(-0.0986328125,-0.735372814122803)(-0.0984933035714286,-0.734741762902433)(-0.0983537946428572,-0.734109089978484)(-0.0982142857142858,-0.733474796853834)(-0.0980747767857144,-0.732838885034513)(-0.097935267857143,-0.732201356029703)(-0.0977957589285716,-0.731562211351732)(-0.09765625,-0.730921452516072)(-0.0975167410714286,-0.730279081041339)(-0.0973772321428572,-0.729635098449287)(-0.0972377232142858,-0.728989506264804)(-0.0970982142857144,-0.728342306015911)(-0.096958705357143,-0.727693499233761)(-0.0968191964285716,-0.727043087452632)(-0.0966796875,-0.726391072209924)(-0.0965401785714286,-0.725737455046163)(-0.0964006696428572,-0.725082237504987)(-0.0962611607142858,-0.724425421133152)(-0.0961216517857144,-0.723767007480526)(-0.095982142857143,-0.723106998100083)(-0.0958426339285716,-0.722445394547906)(-0.095703125,-0.721782198383178)(-0.0955636160714286,-0.721117411168185)(-0.0954241071428572,-0.720451034468305)(-0.0952845982142858,-0.719783069852014)(-0.0951450892857144,-0.719113518890876)(-0.095005580357143,-0.718442383159543)(-0.0948660714285716,-0.717769664235751)(-0.0947265625,-0.717095363700317)(-0.0945870535714286,-0.71641948313714)(-0.0944475446428572,-0.715742024133189)(-0.0943080357142858,-0.715062988278507)(-0.0941685267857144,-0.714382377166208)(-0.094029017857143,-0.713700192392469)(-0.0938895089285716,-0.713016435556531)(-0.09375,-0.712331108260693)(-0.0936104910714286,-0.711644212110315)(-0.0934709821428572,-0.710955748713807)(-0.0933314732142858,-0.710265719682628)(-0.0931919642857144,-0.709574126631288)(-0.093052455357143,-0.708880971177338)(-0.0929129464285716,-0.708186254941371)(-0.0927734375,-0.707489979547017)(-0.0926339285714286,-0.706792146620942)(-0.0924944196428572,-0.706092757792843)(-0.0923549107142858,-0.705391814695443)(-0.0922154017857144,-0.704689318964494)(-0.092075892857143,-0.703985272238765)(-0.0919363839285716,-0.703279676160049)(-0.091796875,-0.702572532373148)(-0.0916573660714286,-0.701863842525884)(-0.0915178571428572,-0.701153608269083)(-0.0913783482142858,-0.700441831256576)(-0.0912388392857144,-0.699728513145201)(-0.091099330357143,-0.699013655594791)(-0.0909598214285716,-0.698297260268178)(-0.0908203125,-0.697579328831185)(-0.0906808035714286,-0.696859862952627)(-0.0905412946428572,-0.696138864304305)(-0.0904017857142858,-0.695416334561)(-0.0902622767857144,-0.694692275400477)(-0.090122767857143,-0.693966688503475)(-0.0899832589285716,-0.693239575553708)(-0.08984375,-0.692510938237857)(-0.0897042410714286,-0.691780778245577)(-0.0895647321428572,-0.691049097269478)(-0.0894252232142858,-0.690315897005136)(-0.0892857142857144,-0.689581179151081)(-0.089146205357143,-0.688844945408798)(-0.0890066964285716,-0.688107197482723)(-0.0888671875,-0.687367937080235)(-0.0887276785714286,-0.686627165911663)(-0.0885881696428572,-0.685884885690272)(-0.0884486607142858,-0.685141098132266)(-0.0883091517857144,-0.68439580495678)(-0.088169642857143,-0.683649007885881)(-0.0880301339285716,-0.682900708644566)(-0.087890625,-0.682150908960748)(-0.0877511160714286,-0.68139961056527)(-0.0876116071428572,-0.680646815191885)(-0.0874720982142858,-0.679892524577262)(-0.0873325892857144,-0.679136740460978)(-0.087193080357143,-0.67837946458552)(-0.0870535714285716,-0.677620698696276)(-0.0869140625,-0.676860444541533)(-0.0867745535714286,-0.676098703872481)(-0.0866350446428572,-0.675335478443194)(-0.0864955357142858,-0.674570770010643)(-0.0863560267857144,-0.673804580334682)(-0.086216517857143,-0.673036911178047)(-0.0860770089285716,-0.672267764306356)(-0.0859375,-0.671497141488101)(-0.0857979910714286,-0.67072504449465)(-0.0856584821428572,-0.669951475100236)(-0.0855189732142858,-0.66917643508196)(-0.0853794642857144,-0.668399926219783)(-0.085239955357143,-0.667621950296529)(-0.0851004464285716,-0.666842509097873)(-0.0849609375,-0.666061604412342)(-0.0848214285714286,-0.665279238031317)(-0.0846819196428572,-0.664495411749017)(-0.0845424107142858,-0.663710127362507)(-0.0844029017857144,-0.662923386671686)(-0.084263392857143,-0.66213519147929)(-0.0841238839285716,-0.661345543590885)(-0.083984375,-0.660554444814864)(-0.0838448660714286,-0.659761896962447)(-0.0837053571428572,-0.65896790184767)(-0.0835658482142858,-0.658172461287387)(-0.0834263392857144,-0.657375577101266)(-0.083286830357143,-0.656577251111785)(-0.0831473214285716,-0.655777485144227)(-0.0830078125,-0.654976281026677)(-0.0828683035714286,-0.654173640590022)(-0.0827287946428572,-0.653369565667943)(-0.0825892857142858,-0.652564058096911)(-0.0824497767857144,-0.651757119716187)(-0.082310267857143,-0.650948752367818)(-0.0821707589285716,-0.650138957896629)(-0.08203125,-0.649327738150224)(-0.0818917410714286,-0.648515094978983)(-0.0817522321428572,-0.647701030236055)(-0.0816127232142858,-0.646885545777354)(-0.0814732142857144,-0.646068643461559)(-0.081333705357143,-0.645250325150108)(-0.0811941964285716,-0.644430592707196)(-0.0810546875,-0.643609447999767)(-0.0809151785714286,-0.64278689289752)(-0.0807756696428572,-0.641962929272891)(-0.0806361607142858,-0.641137559001064)(-0.0804966517857144,-0.640310783959957)(-0.080357142857143,-0.639482606030224)(-0.0802176339285716,-0.638653027095247)(-0.080078125,-0.637822049041137)(-0.0799386160714286,-0.636989673756728)(-0.0797991071428572,-0.636155903133573)(-0.0796595982142858,-0.635320739065939)(-0.0795200892857144,-0.634484183450808)(-0.079380580357143,-0.633646238187869)(-0.0792410714285716,-0.632806905179514)(-0.0791015625,-0.631966186330837)(-0.0789620535714286,-0.631124083549632)(-0.0788225446428572,-0.630280598746383)(-0.0786830357142858,-0.629435733834265)(-0.0785435267857144,-0.628589490729139)(-0.078404017857143,-0.627741871349548)(-0.0782645089285716,-0.626892877616715)(-0.078125,-0.626042511454534)(-0.0779854910714286,-0.625190774789576)(-0.0778459821428572,-0.624337669551076)(-0.0777064732142858,-0.623483197670932)(-0.0775669642857144,-0.622627361083704)(-0.077427455357143,-0.621770161726605)(-0.0772879464285716,-0.620911601539504)(-0.0771484375,-0.620051682464914)(-0.0770089285714286,-0.619190406447999)(-0.0768694196428572,-0.618327775436559)(-0.0767299107142858,-0.617463791381033)(-0.0765904017857144,-0.616598456234493)(-0.076450892857143,-0.61573177195264)(-0.0763113839285716,-0.614863740493803)(-0.076171875,-0.613994363818928)(-0.0760323660714286,-0.613123643891587)(-0.0758928571428572,-0.612251582677961)(-0.0757533482142858,-0.61137818214684)(-0.0756138392857144,-0.610503444269626)(-0.075474330357143,-0.609627371020319)(-0.0753348214285716,-0.60874996437552)(-0.0751953125,-0.607871226314425)(-0.0750558035714286,-0.606991158818824)(-0.0749162946428572,-0.60610976387309)(-0.0747767857142858,-0.605227043464181)(-0.0746372767857144,-0.604342999581636)(-0.074497767857143,-0.603457634217569)(-0.0743582589285716,-0.602570949366666)(-0.07421875,-0.601682947026179)(-0.0740792410714286,-0.600793629195929)(-0.0739397321428572,-0.599902997878294)(-0.0738002232142858,-0.599011055078209)(-0.0736607142857144,-0.598117802803161)(-0.073521205357143,-0.597223243063186)(-0.0733816964285716,-0.596327377870865)(-0.0732421875,-0.595430209241318)(-0.0731026785714286,-0.594531739192206)(-0.0729631696428572,-0.593631969743719)(-0.0728236607142858,-0.592730902918578)(-0.0726841517857144,-0.591828540742027)(-0.072544642857143,-0.590924885241834)(-0.0724051339285716,-0.590019938448281)(-0.072265625,-0.589113702394164)(-0.0721261160714286,-0.588206179114793)(-0.0719866071428572,-0.587297370647976)(-0.0718470982142858,-0.586387279034027)(-0.0717075892857144,-0.585475906315755)(-0.071568080357143,-0.584563254538465)(-0.0714285714285716,-0.583649325749948)(-0.0712890625,-0.582734122000483)(-0.0711495535714286,-0.58181764534283)(-0.0710100446428572,-0.580899897832226)(-0.0708705357142858,-0.579980881526382)(-0.0707310267857144,-0.579060598485477)(-0.070591517857143,-0.578139050772156)(-0.0704520089285716,-0.577216240451527)(-0.0703125,-0.576292169591152)(-0.0701729910714286,-0.575366840261052)(-0.0700334821428572,-0.574440254533691)(-0.0698939732142858,-0.573512414483982)(-0.0697544642857144,-0.572583322189278)(-0.069614955357143,-0.57165297972937)(-0.0694754464285716,-0.570721389186482)(-0.0693359375,-0.569788552645264)(-0.0691964285714286,-0.568854472192799)(-0.0690569196428572,-0.567919149918582)(-0.0689174107142858,-0.56698258791453)(-0.0687779017857144,-0.566044788274972)(-0.068638392857143,-0.565105753096645)(-0.0684988839285716,-0.564165484478691)(-0.068359375,-0.563223984522651)(-0.0682198660714286,-0.562281255332469)(-0.0680803571428572,-0.561337299014473)(-0.0679408482142858,-0.560392117677384)(-0.0678013392857144,-0.559445713432306)(-0.067661830357143,-0.558498088392724)(-0.0675223214285716,-0.557549244674497)(-0.0673828125,-0.556599184395857)(-0.0672433035714286,-0.555647909677407)(-0.0671037946428572,-0.554695422642108)(-0.0669642857142858,-0.553741725415285)(-0.0668247767857144,-0.552786820124615)(-0.066685267857143,-0.551830708900127)(-0.0665457589285716,-0.5508733938742)(-0.06640625,-0.549914877181551)(-0.0662667410714286,-0.548955160959241)(-0.0661272321428572,-0.547994247346663)(-0.0659877232142858,-0.54703213848554)(-0.0658482142857144,-0.546068836519921)(-0.065708705357143,-0.545104343596179)(-0.0655691964285716,-0.544138661863003)(-0.0654296875,-0.543171793471395)(-0.0652901785714286,-0.542203740574673)(-0.0651506696428572,-0.541234505328452)(-0.0650111607142858,-0.540264089890653)(-0.0648716517857144,-0.539292496421493)(-0.064732142857143,-0.53831972708348)(-0.0645926339285716,-0.537345784041413)(-0.064453125,-0.536370669462372)(-0.0643136160714286,-0.535394385515724)(-0.0641741071428572,-0.534416934373105)(-0.0640345982142858,-0.533438318208425)(-0.0638950892857144,-0.532458539197862)(-0.063755580357143,-0.531477599519857)(-0.0636160714285716,-0.53049550135511)(-0.0634765625,-0.529512246886573)(-0.0633370535714286,-0.528527838299454)(-0.0631975446428572,-0.527542277781203)(-0.0630580357142858,-0.526555567521514)(-0.0629185267857144,-0.525567709712317)(-0.062779017857143,-0.524578706547776)(-0.0626395089285716,-0.523588560224286)(-0.0625,-0.522597272940462)(-0.0623604910714286,-0.521604846897148)(-0.0622209821428572,-0.520611284297396)(-0.0620814732142858,-0.519616587346475)(-0.0619419642857144,-0.51862075825186)(-0.061802455357143,-0.517623799223228)(-0.0616629464285716,-0.516625712472459)(-0.0615234375,-0.515626500213623)(-0.0613839285714286,-0.514626164662986)(-0.0612444196428572,-0.513624708038996)(-0.0611049107142858,-0.512622132562285)(-0.0609654017857144,-0.511618440455662)(-0.060825892857143,-0.510613633944109)(-0.0606863839285716,-0.509607715254776)(-0.060546875,-0.508600686616979)(-0.0604073660714286,-0.507592550262197)(-0.0602678571428572,-0.506583308424059)(-0.0601283482142858,-0.505572963338349)(-0.0599888392857144,-0.504561517242998)(-0.059849330357143,-0.50354897237808)(-0.0597098214285716,-0.502535330985806)(-0.0595703125,-0.501520595310523)(-0.0594308035714286,-0.500504767598707)(-0.0592912946428572,-0.49948785009896)(-0.0591517857142858,-0.498469845062004)(-0.0590122767857144,-0.497450754740678)(-0.058872767857143,-0.496430581389934)(-0.0587332589285716,-0.49540932726683)(-0.05859375,-0.494386994630527)(-0.0584542410714286,-0.493363585742292)(-0.0583147321428572,-0.492339102865478)(-0.0581752232142858,-0.491313548265532)(-0.0580357142857144,-0.490286924209988)(-0.057896205357143,-0.489259232968458)(-0.0577566964285716,-0.488230476812635)(-0.0576171875,-0.48720065801628)(-0.0574776785714286,-0.486169778855228)(-0.0573381696428572,-0.485137841607374)(-0.0571986607142858,-0.484104848552672)(-0.0570591517857144,-0.483070801973133)(-0.056919642857143,-0.482035704152816)(-0.0567801339285716,-0.480999557377828)(-0.056640625,-0.479962363936316)(-0.0565011160714286,-0.478924126118467)(-0.0563616071428572,-0.477884846216498)(-0.0562220982142858,-0.476844526524654)(-0.0560825892857144,-0.475803169339205)(-0.055943080357143,-0.47476077695844)(-0.0558035714285716,-0.473717351682662)(-0.0556640625,-0.472672895814183)(-0.0555245535714286,-0.471627411657327)(-0.0553850446428572,-0.470580901518413)(-0.0552455357142858,-0.469533367705758)(-0.0551060267857144,-0.468484812529673)(-0.054966517857143,-0.467435238302455)(-0.0548270089285716,-0.466384647338387)(-0.0546875,-0.465333041953727)(-0.0545479910714286,-0.464280424466712)(-0.0544084821428572,-0.463226797197546)(-0.0542689732142858,-0.462172162468398)(-0.0541294642857144,-0.461116522603399)(-0.053989955357143,-0.460059879928635)(-0.0538504464285716,-0.459002236772145)(-0.0537109375,-0.457943595463913)(-0.0535714285714286,-0.456883958335871)(-0.0534319196428572,-0.455823327721883)(-0.0532924107142858,-0.454761705957751)(-0.0531529017857144,-0.453699095381202)(-0.053013392857143,-0.452635498331891)(-0.0528738839285716,-0.451570917151392)(-0.052734375,-0.450505354183191)(-0.0525948660714286,-0.449438811772692)(-0.0524553571428572,-0.4483712922672)(-0.0523158482142858,-0.447302798015921)(-0.0521763392857144,-0.446233331369962)(-0.052036830357143,-0.44516289468232)(-0.0518973214285716,-0.444091490307879)(-0.0517578125,-0.443019120603408)(-0.0516183035714286,-0.441945787927557)(-0.0514787946428572,-0.440871494640846)(-0.0513392857142858,-0.439796243105667)(-0.0511997767857144,-0.438720035686276)(-0.051060267857143,-0.437642874748791)(-0.0509207589285716,-0.436564762661183)(-0.05078125,-0.435485701793276)(-0.0506417410714286,-0.434405694516742)(-0.0505022321428572,-0.433324743205093)(-0.0503627232142858,-0.432242850233678)(-0.0502232142857144,-0.43116001797968)(-0.050083705357143,-0.43007624882211)(-0.0499441964285716,-0.428991545141801)(-0.0498046875,-0.427905909321406)(-0.0496651785714286,-0.426819343745393)(-0.0495256696428572,-0.425731850800039)(-0.0493861607142858,-0.424643432873425)(-0.0492466517857144,-0.423554092355433)(-0.049107142857143,-0.42246383163774)(-0.0489676339285716,-0.421372653113815)(-0.048828125,-0.420280559178911)(-0.0486886160714286,-0.419187552230067)(-0.0485491071428572,-0.418093634666096)(-0.0484095982142858,-0.416998808887582)(-0.0482700892857144,-0.41590307729688)(-0.048130580357143,-0.414806442298105)(-0.0479910714285716,-0.413708906297132)(-0.0478515625,-0.412610471701588)(-0.0477120535714286,-0.411511140920852)(-0.0475725446428572,-0.410410916366045)(-0.0474330357142858,-0.409309800450028)(-0.0472935267857144,-0.408207795587395)(-0.047154017857143,-0.407104904194473)(-0.0470145089285716,-0.406001128689312)(-0.046875,-0.404896471491684)(-0.0467354910714286,-0.403790935023079)(-0.0465959821428572,-0.402684521706695)(-0.0464564732142858,-0.401577233967438)(-0.0463169642857144,-0.400469074231915)(-0.046177455357143,-0.399360044928431)(-0.0460379464285716,-0.398250148486984)(-0.0458984375,-0.397139387339257)(-0.0457589285714286,-0.396027763918621)(-0.0456194196428572,-0.39491528066012)(-0.0454799107142858,-0.393801940000474)(-0.0453404017857144,-0.392687744378072)(-0.045200892857143,-0.391572696232965)(-0.0450613839285716,-0.390456798006866)(-0.044921875,-0.389340052143138)(-0.0447823660714286,-0.388222461086801)(-0.0446428571428572,-0.387104027284514)(-0.0445033482142858,-0.385984753184578)(-0.0443638392857144,-0.384864641236929)(-0.044224330357143,-0.383743693893135)(-0.0440848214285716,-0.382621913606388)(-0.0439453125,-0.3814993028315)(-0.0438058035714286,-0.380375864024907)(-0.0436662946428572,-0.379251599644647)(-0.0435267857142858,-0.378126512150371)(-0.0433872767857144,-0.377000604003328)(-0.043247767857143,-0.375873877666368)(-0.0431082589285716,-0.37474633560393)(-0.04296875,-0.373617980282041)(-0.0428292410714286,-0.372488814168316)(-0.0426897321428572,-0.371358839731941)(-0.0425502232142858,-0.370228059443679)(-0.0424107142857144,-0.36909647577586)(-0.042271205357143,-0.367964091202379)(-0.0421316964285716,-0.366830908198689)(-0.0419921875,-0.365696929241795)(-0.0418526785714286,-0.364562156810256)(-0.0417131696428572,-0.363426593384171)(-0.0415736607142858,-0.362290241445181)(-0.0414341517857144,-0.361153103476459)(-0.041294642857143,-0.360015181962711)(-0.0411551339285716,-0.358876479390164)(-0.041015625,-0.357736998246568)(-0.0408761160714286,-0.356596741021191)(-0.0407366071428572,-0.355455710204806)(-0.0405970982142858,-0.354313908289694)(-0.0404575892857144,-0.353171337769638)(-0.040318080357143,-0.352028001139913)(-0.0401785714285716,-0.35088390089729)(-0.0400390625,-0.349739039540022)(-0.0398995535714286,-0.348593419567847)(-0.0397600446428572,-0.347447043481978)(-0.0396205357142858,-0.346299913785098)(-0.0394810267857144,-0.345152032981358)(-0.039341517857143,-0.344003403576372)(-0.0392020089285716,-0.34285402807721)(-0.0390625,-0.34170390899239)(-0.0389229910714286,-0.340553048831887)(-0.0387834821428572,-0.33940145010711)(-0.0386439732142858,-0.338249115330908)(-0.0385044642857144,-0.337096047017562)(-0.038364955357143,-0.335942247682783)(-0.0382254464285716,-0.334787719843701)(-0.0380859375,-0.333632466018865)(-0.0379464285714286,-0.332476488728241)(-0.0378069196428572,-0.331319790493198)(-0.0376674107142858,-0.330162373836511)(-0.0375279017857144,-0.329004241282352)(-0.037388392857143,-0.327845395356286)(-0.0372488839285716,-0.326685838585267)(-0.037109375,-0.325525573497631)(-0.0369698660714286,-0.324364602623098)(-0.0368303571428572,-0.323202928492756)(-0.0366908482142858,-0.322040553639063)(-0.0365513392857144,-0.320877480595842)(-0.036411830357143,-0.319713711898273)(-0.0362723214285716,-0.318549250082892)(-0.0361328125,-0.317384097687581)(-0.0359933035714286,-0.316218257251572)(-0.0358537946428572,-0.31505173131543)(-0.0357142857142858,-0.313884522421057)(-0.0355747767857144,-0.312716633111684)(-0.035435267857143,-0.311548065931864)(-0.0352957589285716,-0.310378823427474)(-0.03515625,-0.309208908145698)(-0.0350167410714286,-0.30803832263504)(-0.0348772321428572,-0.3068670694453)(-0.0347377232142858,-0.305695151127579)(-0.0345982142857144,-0.304522570234275)(-0.034458705357143,-0.303349329319072)(-0.0343191964285716,-0.302175430936943)(-0.0341796875,-0.301000877644135)(-0.0340401785714286,-0.299825671998177)(-0.0339006696428572,-0.298649816557861)(-0.0337611607142858,-0.297473313883247)(-0.0336216517857144,-0.296296166535653)(-0.033482142857143,-0.295118377077654)(-0.0333426339285716,-0.293939948073072)(-0.033203125,-0.292760882086974)(-0.0330636160714286,-0.291581181685672)(-0.0329241071428572,-0.290400849436705)(-0.0327845982142858,-0.289219887908847)(-0.0326450892857144,-0.288038299672094)(-0.032505580357143,-0.286856087297663)(-0.0323660714285716,-0.285673253357986)(-0.0322265625,-0.284489800426702)(-0.0320870535714286,-0.283305731078661)(-0.0319475446428572,-0.282121047889906)(-0.0318080357142858,-0.280935753437679)(-0.0316685267857144,-0.279749850300409)(-0.031529017857143,-0.278563341057711)(-0.0313895089285716,-0.277376228290379)(-0.03125,-0.276188514580381)(-0.0311104910714286,-0.275000202510859)(-0.0309709821428572,-0.273811294666114)(-0.0308314732142858,-0.27262179363161)(-0.0306919642857144,-0.271431701993962)(-0.030552455357143,-0.270241022340939)(-0.0304129464285716,-0.269049757261451)(-0.0302734375,-0.267857909345548)(-0.0301339285714286,-0.266665481184417)(-0.0299944196428572,-0.265472475370371)(-0.0298549107142858,-0.264278894496848)(-0.0297154017857144,-0.263084741158408)(-0.029575892857143,-0.26189001795072)(-0.0294363839285716,-0.260694727470567)(-0.029296875,-0.259498872315831)(-0.0291573660714286,-0.258302455085501)(-0.0290178571428572,-0.257105478379652)(-0.0288783482142858,-0.255907944799453)(-0.0287388392857144,-0.254709856947152)(-0.028599330357143,-0.253511217426081)(-0.0284598214285716,-0.252312028840642)(-0.0283203125,-0.251112293796305)(-0.0281808035714286,-0.24991201489961)(-0.0280412946428572,-0.248711194758149)(-0.0279017857142858,-0.247509835980568)(-0.0277622767857144,-0.246307941176564)(-0.027622767857143,-0.245105512956875)(-0.0274832589285716,-0.24390255393328)(-0.02734375,-0.242699066718585)(-0.0272042410714286,-0.241495053926633)(-0.0270647321428572,-0.240290518172284)(-0.0269252232142858,-0.239085462071415)(-0.0267857142857144,-0.23787988824092)(-0.026646205357143,-0.236673799298697)(-0.0265066964285716,-0.235467197863647)(-0.0263671875,-0.234260086555669)(-0.0262276785714286,-0.233052467995659)(-0.0260881696428572,-0.231844344805492)(-0.0259486607142858,-0.23063571960803)(-0.0258091517857144,-0.229426595027112)(-0.025669642857143,-0.228216973687547)(-0.0255301339285716,-0.227006858215112)(-0.025390625,-0.225796251236544)(-0.0252511160714286,-0.224585155379542)(-0.0251116071428572,-0.22337357327275)(-0.0249720982142858,-0.222161507545763)(-0.0248325892857144,-0.220948960829113)(-0.024693080357143,-0.219735935754272)(-0.0245535714285716,-0.218522434953642)(-0.0244140625,-0.217308461060546)(-0.0242745535714286,-0.216094016709239)(-0.0241350446428572,-0.214879104534881)(-0.0239955357142858,-0.213663727173547)(-0.0238560267857144,-0.212447887262216)(-0.023716517857143,-0.211231587438768)(-0.0235770089285716,-0.210014830341979)(-0.0234375,-0.208797618611511)(-0.0232979910714286,-0.207579954887919)(-0.0231584821428572,-0.20636184181263)(-0.0230189732142858,-0.205143282027947)(-0.0228794642857144,-0.203924278177046)(-0.022739955357143,-0.202704832903962)(-0.0226004464285716,-0.201484948853595)(-0.0224609375,-0.200264628671691)(-0.0223214285714286,-0.199043875004855)(-0.0221819196428572,-0.19782269050053)(-0.0220424107142858,-0.196601077806995)(-0.0219029017857144,-0.195379039573367)(-0.021763392857143,-0.19415657844959)(-0.0216238839285716,-0.19293369708643)(-0.021484375,-0.19171039813547)(-0.0213448660714286,-0.19048668424911)(-0.0212053571428572,-0.189262558080554)(-0.0210658482142858,-0.18803802228381)(-0.0209263392857144,-0.18681307951368)(-0.020786830357143,-0.185587732425763)(-0.0206473214285716,-0.184361983676441)(-0.0205078125,-0.183135835922878)(-0.0203683035714286,-0.18190929182302)(-0.0202287946428572,-0.180682354035577)(-0.0200892857142858,-0.17945502522003)(-0.0199497767857144,-0.178227308036619)(-0.019810267857143,-0.17699920514634)(-0.0196707589285716,-0.17577071921094)(-0.01953125,-0.17454185289291)(-0.0193917410714286,-0.173312608855486)(-0.0192522321428572,-0.172082989762635)(-0.0191127232142858,-0.170852998279055)(-0.0189732142857144,-0.169622637070169)(-0.018833705357143,-0.168391908802118)(-0.0186941964285716,-0.16716081614176)(-0.0185546875,-0.165929361756658)(-0.0184151785714286,-0.164697548315087)(-0.0182756696428572,-0.163465378486013)(-0.0181361607142858,-0.162232854939099)(-0.0179966517857144,-0.160999980344695)(-0.017857142857143,-0.159766757373836)(-0.0177176339285716,-0.158533188698234)(-0.017578125,-0.157299276990272)(-0.0174386160714286,-0.156065024923006)(-0.0172991071428572,-0.15483043517015)(-0.0171595982142858,-0.153595510406075)(-0.0170200892857144,-0.152360253305806)(-0.016880580357143,-0.151124666545015)(-0.0167410714285716,-0.149888752800013)(-0.0166015625,-0.148652514747748)(-0.0164620535714286,-0.147415955065804)(-0.0163225446428572,-0.146179076432386)(-0.0161830357142858,-0.14494188152632)(-0.0160435267857144,-0.143704373027049)(-0.015904017857143,-0.142466553614627)(-0.0157645089285716,-0.14122842596971)(-0.015625,-0.139989992773556)(-0.0154854910714286,-0.13875125670802)(-0.0153459821428572,-0.137512220455544)(-0.0152064732142858,-0.136272886699152)(-0.0150669642857144,-0.135033258122452)(-0.014927455357143,-0.133793337409621)(-0.0147879464285716,-0.132553127245407)(-0.0146484375,-0.131312630315121)(-0.0145089285714286,-0.130071849304634)(-0.0143694196428572,-0.128830786900368)(-0.0142299107142858,-0.127589445789291)(-0.0140904017857144,-0.126347828658917)(-0.013950892857143,-0.125105938197294)(-0.0138113839285716,-0.123863777093004)(-0.013671875,-0.122621348035154)(-0.0135323660714286,-0.121378653713376)(-0.0133928571428572,-0.120135696817816)(-0.0132533482142858,-0.118892480039129)(-0.0131138392857144,-0.117649006068479)(-0.012974330357143,-0.11640527759753)(-0.0128348214285716,-0.115161297318439)(-0.0126953125,-0.113917067923854)(-0.0125558035714286,-0.112672592106912)(-0.0124162946428572,-0.111427872561225)(-0.0122767857142858,-0.110182911980879)(-0.0121372767857144,-0.108937713060433)(-0.011997767857143,-0.107692278494906)(-0.0118582589285716,-0.106446610979778)(-0.01171875,-0.10520071321098)(-0.0115792410714286,-0.103954587884895)(-0.0114397321428572,-0.102708237698346)(-0.0113002232142858,-0.101461665348594)(-0.0111607142857144,-0.100214873533333)(-0.011021205357143,-0.0989678649506838)(-0.0108816964285716,-0.0977206422991885)(-0.0107421875,-0.0964732082778047)(-0.0106026785714286,-0.0952255655859073)(-0.0104631696428572,-0.093977716923271)(-0.0103236607142858,-0.0927296649900736)(-0.0101841517857144,-0.0914814124868884)(-0.010044642857143,-0.0902329621146792)(-0.00990513392857162,-0.0889843165747947)(-0.009765625,-0.0877354785689617)(-0.0096261160714286,-0.0864864507992874)(-0.00948660714285721,-0.0852372359682425)(-0.00934709821428581,-0.0839878367786634)(-0.00920758928571441,-0.0827382559337458)(-0.00906808035714302,-0.0814884961370387)(-0.00892857142857162,-0.0802385600924394)(-0.0087890625,-0.0789884505041863)(-0.0086495535714286,-0.0777381700768614)(-0.00851004464285721,-0.0764877215153734)(-0.00837053571428581,-0.0752371075249601)(-0.00823102678571441,-0.0739863308111813)(-0.00809151785714302,-0.0727353940799136)(-0.00795200892857162,-0.0714843000373449)(-0.0078125,-0.0702330513899673)(-0.0076729910714286,-0.0689816508445799)(-0.00753348214285721,-0.0677301011082713)(-0.00739397321428581,-0.0664784048884226)(-0.00725446428571441,-0.0652265648926999)(-0.00711495535714302,-0.0639745838290491)(-0.00697544642857162,-0.0627224644056908)(-0.0068359375,-0.0614702093311128)(-0.0066964285714286,-0.0602178213140728)(-0.00655691964285721,-0.0589653030635815)(-0.00641741071428581,-0.0577126572889048)(-0.00627790178571441,-0.0564598866995571)(-0.00613839285714302,-0.0552069940052953)(-0.00599888392857162,-0.0539539819161144)(-0.005859375,-0.0527008531422396)(-0.0057198660714286,-0.0514476103941289)(-0.00558035714285721,-0.0501942563824567)(-0.00544084821428581,-0.0489407938181155)(-0.00530133928571441,-0.0476872254122092)(-0.00516183035714302,-0.0464335538760478)(-0.00502232142857162,-0.0451797819211419)(-0.0048828125,-0.0439259122591955)(-0.0047433035714286,-0.042671947602109)(-0.00460379464285721,-0.0414178906619614)(-0.00446428571428581,-0.0401637441510135)(-0.00432477678571441,-0.0389095107817003)(-0.00418526785714302,-0.0376551932666261)(-0.00404575892857162,-0.0364007943185585)(-0.00390625,-0.0351463166504222)(-0.0037667410714286,-0.0338917629753005)(-0.00362723214285721,-0.0326371360064191)(-0.00348772321428581,-0.0313824384571481)(-0.00334821428571441,-0.030127673040995)(-0.00320870535714302,-0.0288728424715996)(-0.00306919642857162,-0.0276179494627282)(-0.0029296875,-0.0263629967282667)(-0.0027901785714286,-0.0251079869822234)(-0.00265066964285721,-0.0238529229387114)(-0.00251116071428581,-0.0225978073119516)(-0.00237165178571441,-0.0213426428162652)(-0.00223214285714302,-0.0200874321660686)(-0.00209263392857162,-0.0188321780758678)(-0.001953125,-0.0175768832602517)(-0.0018136160714286,-0.0163215504338942)(-0.00167410714285721,-0.0150661823115373)(-0.00153459821428581,-0.0138107816079937)(-0.00139508928571441,-0.0125553510381393)(-0.00125558035714302,-0.0112998933169085)(-0.00111607142857162,-0.0100444111592882)(-0.0009765625,-0.00878890728031104)(-0.000837053571428603,-0.00753338439505793)(-0.000697544642857206,-0.0062778452186407)(-0.000558035714285809,-0.00502229246620495)(-0.000418526785714413,-0.00376672885292267)(-0.000279017857143016,-0.002511157093987)(-0.000139508928571619,-0.00125557990460692)(0,0)(0.000139508928571175,0.00125557990460292)(0.000279017857142794,0.002511157093985)(0.000418526785713969,0.00376672885291868)(0.000558035714285587,0.00502229246620295)(0.000697544642857206,0.0062778452186407)(0.000837053571428381,0.00753338439505593)(0.0009765625,0.00878890728031104)(0.00111607142857117,0.0100444111592842)(0.00125558035714279,0.0112998933169065)(0.00139508928571397,0.0125553510381353)(0.00153459821428559,0.0138107816079917)(0.00167410714285721,0.0150661823115373)(0.00181361607142838,0.0163215504338922)(0.001953125,0.0175768832602517)(0.00209263392857117,0.0188321780758638)(0.00223214285714279,0.0200874321660666)(0.00237165178571397,0.0213426428162612)(0.00251116071428559,0.0225978073119496)(0.00265066964285721,0.0238529229387114)(0.00279017857142838,0.0251079869822214)(0.0029296875,0.0263629967282667)(0.00306919642857117,0.0276179494627242)(0.00320870535714279,0.0288728424715976)(0.00334821428571397,0.0301276730409911)(0.00348772321428559,0.0313824384571461)(0.00362723214285721,0.0326371360064191)(0.00376674107142838,0.0338917629752985)(0.00390625,0.0351463166504222)(0.00404575892857117,0.0364007943185545)(0.00418526785714279,0.0376551932666241)(0.00432477678571397,0.0389095107816963)(0.00446428571428559,0.0401637441510115)(0.00460379464285721,0.0414178906619614)(0.00474330357142838,0.042671947602107)(0.0048828125,0.0439259122591955)(0.00502232142857117,0.0451797819211379)(0.00516183035714279,0.0464335538760458)(0.00530133928571397,0.0476872254122052)(0.00544084821428559,0.0489407938181135)(0.00558035714285721,0.0501942563824567)(0.00571986607142838,0.051447610394127)(0.005859375,0.0527008531422396)(0.00599888392857117,0.0539539819161105)(0.00613839285714279,0.0552069940052933)(0.00627790178571397,0.0564598866995531)(0.00641741071428559,0.0577126572889028)(0.00655691964285721,0.0589653030635815)(0.00669642857142838,0.0602178213140708)(0.0068359375,0.0614702093311128)(0.00697544642857117,0.0627224644056868)(0.00711495535714279,0.0639745838290471)(0.00725446428571397,0.0652265648926959)(0.00739397321428559,0.0664784048884206)(0.00753348214285721,0.0677301011082713)(0.00767299107142838,0.0689816508445779)(0.0078125,0.0702330513899673)(0.00795200892857117,0.0714843000373409)(0.00809151785714279,0.0727353940799116)(0.00823102678571397,0.0739863308111774)(0.00837053571428559,0.0752371075249581)(0.00851004464285721,0.0764877215153734)(0.00864955357142838,0.0777381700768594)(0.0087890625,0.0789884505041863)(0.00892857142857117,0.0802385600924355)(0.00906808035714279,0.0814884961370368)(0.00920758928571397,0.0827382559337419)(0.00934709821428559,0.0839878367786614)(0.00948660714285721,0.0852372359682425)(0.00962611607142838,0.0864864507992855)(0.009765625,0.0877354785689617)(0.00990513392857117,0.0889843165747908)(0.0100446428571428,0.0902329621146772)(0.010184151785714,0.0914814124868844)(0.0103236607142856,0.0927296649900716)(0.0104631696428572,0.093977716923271)(0.0106026785714284,0.0952255655859053)(0.0107421875,0.0964732082778047)(0.0108816964285712,0.0977206422991845)(0.0110212053571428,0.0989678649506818)(0.011160714285714,0.100214873533329)(0.0113002232142856,0.101461665348592)(0.0114397321428572,0.102708237698346)(0.0115792410714284,0.103954587884893)(0.01171875,0.10520071321098)(0.0118582589285712,0.106446610979774)(0.0119977678571428,0.107692278494904)(0.012137276785714,0.108937713060429)(0.0122767857142856,0.110182911980877)(0.0124162946428572,0.111427872561225)(0.0125558035714284,0.11267259210691)(0.0126953125,0.113917067923854)(0.0128348214285712,0.115161297318435)(0.0129743303571428,0.116405277597528)(0.013113839285714,0.117649006068475)(0.0132533482142856,0.118892480039127)(0.0133928571428572,0.120135696817816)(0.0135323660714284,0.121378653713374)(0.013671875,0.122621348035154)(0.0138113839285712,0.123863777093)(0.0139508928571428,0.125105938197292)(0.014090401785714,0.126347828658913)(0.0142299107142856,0.127589445789289)(0.0143694196428572,0.128830786900368)(0.0145089285714284,0.130071849304632)(0.0146484375,0.131312630315121)(0.0147879464285712,0.132553127245403)(0.0149274553571428,0.133793337409619)(0.015066964285714,0.135033258122448)(0.0152064732142856,0.13627288669915)(0.0153459821428572,0.137512220455544)(0.0154854910714284,0.138751256708019)(0.015625,0.139989992773556)(0.0157645089285712,0.141228425969706)(0.0159040178571428,0.142466553614625)(0.016043526785714,0.143704373027045)(0.0161830357142856,0.144941881526318)(0.0163225446428572,0.146179076432386)(0.0164620535714284,0.147415955065802)(0.0166015625,0.148652514747748)(0.0167410714285712,0.149888752800009)(0.0168805803571428,0.151124666545013)(0.017020089285714,0.152360253305803)(0.0171595982142856,0.153595510406073)(0.0172991071428572,0.15483043517015)(0.0174386160714284,0.156065024923004)(0.017578125,0.157299276990272)(0.0177176339285712,0.15853318869823)(0.0178571428571428,0.159766757373834)(0.017996651785714,0.160999980344691)(0.0181361607142856,0.162232854939097)(0.0182756696428572,0.163465378486013)(0.0184151785714284,0.164697548315085)(0.0185546875,0.165929361756658)(0.0186941964285712,0.167160816141756)(0.0188337053571428,0.168391908802116)(0.018973214285714,0.169622637070165)(0.0191127232142856,0.170852998279053)(0.0192522321428572,0.172082989762635)(0.0193917410714284,0.173312608855484)(0.01953125,0.17454185289291)(0.0196707589285712,0.175770719210936)(0.0198102678571428,0.176999205146338)(0.019949776785714,0.178227308036615)(0.0200892857142856,0.179455025220028)(0.0202287946428572,0.180682354035577)(0.0203683035714284,0.181909291823018)(0.0205078125,0.183135835922878)(0.0206473214285712,0.184361983676437)(0.0207868303571428,0.185587732425761)(0.020926339285714,0.186813079513676)(0.0210658482142856,0.188038022283808)(0.0212053571428572,0.189262558080554)(0.0213448660714284,0.190486684249108)(0.021484375,0.19171039813547)(0.0216238839285712,0.192933697086426)(0.0217633928571428,0.194156578449588)(0.021902901785714,0.195379039573364)(0.0220424107142856,0.196601077806993)(0.0221819196428572,0.19782269050053)(0.0223214285714284,0.199043875004854)(0.0224609375,0.200264628671691)(0.0226004464285712,0.201484948853591)(0.0227399553571428,0.20270483290396)(0.022879464285714,0.203924278177042)(0.0230189732142856,0.205143282027946)(0.0231584821428572,0.20636184181263)(0.0232979910714284,0.207579954887917)(0.0234375,0.208797618611511)(0.0235770089285712,0.210014830341975)(0.0237165178571428,0.211231587438766)(0.023856026785714,0.212447887262212)(0.0239955357142856,0.213663727173545)(0.0241350446428572,0.214879104534881)(0.0242745535714284,0.216094016709237)(0.0244140625,0.217308461060546)(0.0245535714285712,0.218522434953638)(0.0246930803571428,0.21973593575427)(0.024832589285714,0.220948960829109)(0.0249720982142856,0.222161507545761)(0.0251116071428572,0.22337357327275)(0.0252511160714284,0.22458515537954)(0.025390625,0.225796251236544)(0.0255301339285712,0.227006858215108)(0.0256696428571428,0.228216973687545)(0.025809151785714,0.229426595027108)(0.0259486607142856,0.230635719608028)(0.0260881696428572,0.231844344805492)(0.0262276785714284,0.233052467995657)(0.0263671875,0.234260086555669)(0.0265066964285712,0.235467197863643)(0.0266462053571428,0.236673799298695)(0.026785714285714,0.237879888240916)(0.0269252232142856,0.239085462071413)(0.0270647321428572,0.240290518172284)(0.0272042410714284,0.241495053926631)(0.02734375,0.242699066718585)(0.0274832589285712,0.243902553933276)(0.0276227678571428,0.245105512956874)(0.027762276785714,0.24630794117656)(0.0279017857142856,0.247509835980566)(0.0280412946428572,0.248711194758149)(0.0281808035714284,0.249912014899608)(0.0283203125,0.251112293796305)(0.0284598214285712,0.252312028840638)(0.0285993303571428,0.253511217426079)(0.028738839285714,0.254709856947148)(0.0288783482142856,0.255907944799451)(0.0290178571428572,0.257105478379652)(0.0291573660714284,0.258302455085499)(0.029296875,0.259498872315831)(0.0294363839285712,0.260694727470563)(0.0295758928571428,0.261890017950718)(0.029715401785714,0.263084741158404)(0.0298549107142856,0.264278894496847)(0.0299944196428572,0.265472475370371)(0.0301339285714284,0.266665481184415)(0.0302734375,0.267857909345548)(0.0304129464285712,0.269049757261448)(0.0305524553571428,0.270241022340937)(0.030691964285714,0.271431701993959)(0.0308314732142856,0.272621793631608)(0.0309709821428572,0.273811294666114)(0.0311104910714284,0.275000202510857)(0.03125,0.276188514580381)(0.0313895089285712,0.277376228290375)(0.0315290178571428,0.278563341057709)(0.031668526785714,0.279749850300405)(0.0318080357142856,0.280935753437677)(0.0319475446428572,0.282121047889906)(0.0320870535714284,0.283305731078659)(0.0322265625,0.284489800426702)(0.0323660714285712,0.285673253357982)(0.0325055803571428,0.286856087297661)(0.032645089285714,0.28803829967209)(0.0327845982142856,0.289219887908845)(0.0329241071428572,0.290400849436705)(0.0330636160714284,0.29158118168567)(0.033203125,0.292760882086974)(0.0333426339285712,0.293939948073068)(0.0334821428571428,0.295118377077652)(0.033621651785714,0.296296166535649)(0.0337611607142856,0.297473313883245)(0.0339006696428572,0.298649816557861)(0.0340401785714284,0.299825671998175)(0.0341796875,0.301000877644135)(0.0343191964285712,0.302175430936939)(0.0344587053571428,0.30334932931907)(0.034598214285714,0.304522570234271)(0.0347377232142856,0.305695151127577)(0.0348772321428572,0.3068670694453)(0.0350167410714284,0.308038322635038)(0.03515625,0.309208908145698)(0.0352957589285712,0.31037882342747)(0.0354352678571428,0.311548065931862)(0.035574776785714,0.31271663311168)(0.0357142857142856,0.313884522421055)(0.0358537946428572,0.31505173131543)(0.0359933035714284,0.31621825725157)(0.0361328125,0.317384097687581)(0.0362723214285712,0.318549250082888)(0.0364118303571428,0.319713711898271)(0.036551339285714,0.320877480595838)(0.0366908482142856,0.322040553639061)(0.0368303571428572,0.323202928492756)(0.0369698660714284,0.324364602623096)(0.037109375,0.325525573497631)(0.0372488839285712,0.326685838585263)(0.0373883928571428,0.327845395356284)(0.037527901785714,0.329004241282348)(0.0376674107142856,0.330162373836509)(0.0378069196428572,0.331319790493198)(0.0379464285714284,0.332476488728239)(0.0380859375,0.333632466018865)(0.0382254464285712,0.334787719843697)(0.0383649553571428,0.335942247682781)(0.038504464285714,0.337096047017559)(0.0386439732142856,0.338249115330906)(0.0387834821428572,0.33940145010711)(0.0389229910714284,0.340553048831885)(0.0390625,0.34170390899239)(0.0392020089285712,0.342854028077206)(0.0393415178571428,0.344003403576371)(0.039481026785714,0.345152032981355)(0.0396205357142856,0.346299913785096)(0.0397600446428572,0.347447043481978)(0.0398995535714284,0.348593419567845)(0.0400390625,0.349739039540022)(0.0401785714285712,0.350883900897287)(0.0403180803571428,0.352028001139911)(0.040457589285714,0.353171337769634)(0.0405970982142856,0.354313908289693)(0.0407366071428572,0.355455710204806)(0.0408761160714284,0.356596741021189)(0.041015625,0.357736998246568)(0.0411551339285712,0.358876479390161)(0.0412946428571428,0.360015181962709)(0.041434151785714,0.361153103476456)(0.0415736607142856,0.362290241445179)(0.0417131696428572,0.363426593384171)(0.0418526785714284,0.364562156810254)(0.0419921875,0.365696929241795)(0.0421316964285712,0.366830908198685)(0.0422712053571428,0.367964091202377)(0.042410714285714,0.369096475775856)(0.0425502232142856,0.370228059443677)(0.0426897321428572,0.371358839731941)(0.0428292410714284,0.372488814168314)(0.04296875,0.373617980282041)(0.0431082589285712,0.374746335603927)(0.0432477678571428,0.375873877666366)(0.043387276785714,0.377000604003325)(0.0435267857142856,0.378126512150369)(0.0436662946428572,0.379251599644647)(0.0438058035714284,0.380375864024905)(0.0439453125,0.3814993028315)(0.0440848214285712,0.382621913606384)(0.0442243303571428,0.383743693893133)(0.044363839285714,0.384864641236926)(0.0445033482142856,0.385984753184576)(0.0446428571428572,0.387104027284514)(0.0447823660714284,0.3882224610868)(0.044921875,0.389340052143138)(0.0450613839285712,0.390456798006863)(0.0452008928571428,0.391572696232964)(0.045340401785714,0.392687744378069)(0.0454799107142856,0.393801940000473)(0.0456194196428572,0.39491528066012)(0.0457589285714284,0.396027763918619)(0.0458984375,0.397139387339257)(0.0460379464285712,0.398250148486981)(0.0461774553571428,0.39936004492843)(0.046316964285714,0.400469074231911)(0.0464564732142856,0.401577233967436)(0.0465959821428572,0.402684521706695)(0.0467354910714284,0.403790935023077)(0.046875,0.404896471491684)(0.0470145089285712,0.406001128689309)(0.0471540178571428,0.407104904194471)(0.047293526785714,0.408207795587391)(0.0474330357142856,0.409309800450026)(0.0475725446428572,0.410410916366045)(0.0477120535714284,0.411511140920851)(0.0478515625,0.412610471701588)(0.0479910714285712,0.413708906297129)(0.0481305803571428,0.414806442298103)(0.048270089285714,0.415903077296877)(0.0484095982142856,0.416998808887581)(0.0485491071428572,0.418093634666096)(0.0486886160714284,0.419187552230066)(0.048828125,0.420280559178911)(0.0489676339285712,0.421372653113812)(0.0491071428571428,0.422463831637739)(0.049246651785714,0.42355409235543)(0.0493861607142856,0.424643432873424)(0.0495256696428572,0.425731850800039)(0.0496651785714284,0.426819343745392)(0.0498046875,0.427905909321406)(0.0499441964285712,0.428991545141798)(0.0500837053571428,0.430076248822108)(0.050223214285714,0.431160017979677)(0.0503627232142856,0.432242850233676)(0.0505022321428572,0.433324743205093)(0.0506417410714284,0.43440569451674)(0.05078125,0.435485701793276)(0.0509207589285712,0.43656476266118)(0.0510602678571428,0.437642874748789)(0.051199776785714,0.438720035686273)(0.0513392857142856,0.439796243105665)(0.0514787946428572,0.440871494640846)(0.0516183035714284,0.441945787927555)(0.0517578125,0.443019120603408)(0.0518973214285712,0.444091490307876)(0.0520368303571428,0.445162894682318)(0.052176339285714,0.446233331369959)(0.0523158482142856,0.44730279801592)(0.0524553571428572,0.4483712922672)(0.0525948660714284,0.449438811772691)(0.052734375,0.450505354183191)(0.0528738839285712,0.451570917151389)(0.0530133928571428,0.45263549833189)(0.053152901785714,0.453699095381199)(0.0532924107142856,0.454761705957749)(0.0534319196428572,0.455823327721883)(0.0535714285714284,0.456883958335869)(0.0537109375,0.457943595463913)(0.0538504464285712,0.459002236772142)(0.0539899553571428,0.460059879928633)(0.054129464285714,0.461116522603396)(0.0542689732142856,0.462172162468397)(0.0544084821428572,0.463226797197546)(0.0545479910714284,0.464280424466711)(0.0546875,0.465333041953727)(0.0548270089285712,0.466384647338384)(0.0549665178571428,0.467435238302454)(0.055106026785714,0.468484812529669)(0.0552455357142856,0.469533367705756)(0.0553850446428572,0.470580901518413)(0.0555245535714284,0.471627411657325)(0.0556640625,0.472672895814183)(0.0558035714285712,0.473717351682658)(0.0559430803571428,0.474760776958438)(0.056082589285714,0.475803169339202)(0.0562220982142856,0.476844526524652)(0.0563616071428572,0.477884846216498)(0.0565011160714284,0.478924126118465)(0.056640625,0.479962363936316)(0.0567801339285712,0.480999557377825)(0.0569196428571428,0.482035704152814)(0.057059151785714,0.483070801973129)(0.0571986607142856,0.48410484855267)(0.0573381696428572,0.485137841607374)(0.0574776785714284,0.486169778855226)(0.0576171875,0.48720065801628)(0.0577566964285712,0.488230476812631)(0.0578962053571428,0.489259232968457)(0.058035714285714,0.490286924209984)(0.0581752232142856,0.491313548265531)(0.0583147321428572,0.492339102865478)(0.0584542410714284,0.49336358574229)(0.05859375,0.494386994630527)(0.0587332589285712,0.495409327266826)(0.0588727678571428,0.496430581389932)(0.059012276785714,0.497450754740675)(0.0591517857142856,0.498469845062003)(0.0592912946428572,0.49948785009896)(0.0594308035714284,0.500504767598706)(0.0595703125,0.501520595310523)(0.0597098214285712,0.502535330985803)(0.0598493303571428,0.503548972378078)(0.059988839285714,0.504561517242995)(0.0601283482142856,0.505572963338347)(0.0602678571428572,0.506583308424059)(0.0604073660714284,0.507592550262195)(0.060546875,0.508600686616979)(0.0606863839285712,0.509607715254773)(0.0608258928571428,0.510613633944107)(0.060965401785714,0.511618440455659)(0.0611049107142856,0.512622132562284)(0.0612444196428572,0.513624708038996)(0.0613839285714284,0.514626164662984)(0.0615234375,0.515626500213623)(0.0616629464285712,0.516625712472456)(0.0618024553571428,0.517623799223227)(0.061941964285714,0.518620758251856)(0.0620814732142856,0.519616587346474)(0.0622209821428572,0.520611284297396)(0.0623604910714284,0.521604846897146)(0.0625,0.522597272940462)(0.0626395089285712,0.523588560224282)(0.0627790178571428,0.524578706547774)(0.062918526785714,0.525567709712313)(0.0630580357142856,0.526555567521512)(0.0631975446428572,0.527542277781203)(0.0633370535714284,0.528527838299452)(0.0634765625,0.529512246886573)(0.0636160714285712,0.530495501355107)(0.0637555803571428,0.531477599519856)(0.063895089285714,0.532458539197859)(0.0640345982142856,0.533438318208424)(0.0641741071428572,0.534416934373105)(0.0643136160714284,0.535394385515722)(0.064453125,0.536370669462372)(0.0645926339285712,0.537345784041409)(0.0647321428571428,0.538319727083478)(0.064871651785714,0.539292496421489)(0.0650111607142856,0.540264089890652)(0.0651506696428572,0.541234505328452)(0.0652901785714284,0.542203740574671)(0.0654296875,0.543171793471395)(0.0655691964285712,0.544138661863)(0.0657087053571428,0.545104343596177)(0.065848214285714,0.546068836519918)(0.0659877232142856,0.547032138485538)(0.0661272321428572,0.547994247346663)(0.0662667410714284,0.54895516095924)(0.06640625,0.549914877181551)(0.0665457589285712,0.550873393874197)(0.0666852678571428,0.551830708900126)(0.066824776785714,0.552786820124611)(0.0669642857142856,0.553741725415283)(0.0671037946428572,0.554695422642108)(0.0672433035714284,0.555647909677405)(0.0673828125,0.556599184395857)(0.0675223214285712,0.557549244674494)(0.0676618303571428,0.558498088392722)(0.067801339285714,0.559445713432303)(0.0679408482142856,0.560392117677382)(0.0680803571428572,0.561337299014473)(0.0682198660714284,0.562281255332467)(0.068359375,0.563223984522651)(0.0684988839285712,0.564165484478688)(0.0686383928571428,0.565105753096643)(0.068777901785714,0.566044788274969)(0.0689174107142856,0.566982587914529)(0.0690569196428572,0.567919149918582)(0.0691964285714284,0.568854472192797)(0.0693359375,0.569788552645264)(0.0694754464285712,0.570721389186479)(0.0696149553571428,0.571652979729369)(0.069754464285714,0.572583322189275)(0.0698939732142856,0.573512414483981)(0.0700334821428572,0.574440254533691)(0.0701729910714284,0.57536684026105)(0.0703125,0.576292169591152)(0.0704520089285712,0.577216240451524)(0.0705915178571428,0.578139050772155)(0.070731026785714,0.579060598485474)(0.0708705357142856,0.57998088152638)(0.0710100446428572,0.580899897832226)(0.0711495535714284,0.581817645342829)(0.0712890625,0.582734122000483)(0.0714285714285712,0.583649325749945)(0.0715680803571428,0.584563254538463)(0.071707589285714,0.585475906315752)(0.0718470982142856,0.586387279034025)(0.0719866071428572,0.587297370647976)(0.0721261160714284,0.588206179114791)(0.072265625,0.589113702394164)(0.0724051339285712,0.590019938448278)(0.0725446428571428,0.590924885241832)(0.072684151785714,0.591828540742024)(0.0728236607142856,0.592730902918577)(0.0729631696428572,0.593631969743719)(0.0731026785714284,0.594531739192205)(0.0732421875,0.595430209241318)(0.0733816964285712,0.596327377870862)(0.0735212053571428,0.597223243063185)(0.073660714285714,0.598117802803158)(0.0738002232142856,0.599011055078208)(0.0739397321428572,0.599902997878294)(0.0740792410714284,0.600793629195928)(0.07421875,0.601682947026179)(0.0743582589285712,0.602570949366663)(0.0744977678571428,0.603457634217568)(0.074637276785714,0.604342999581634)(0.0747767857142856,0.60522704346418)(0.0749162946428572,0.60610976387309)(0.0750558035714284,0.606991158818823)(0.0751953125,0.607871226314425)(0.0753348214285712,0.608749964375517)(0.0754743303571428,0.609627371020317)(0.075613839285714,0.610503444269623)(0.0757533482142856,0.611378182146839)(0.0758928571428572,0.612251582677961)(0.0760323660714284,0.613123643891586)(0.076171875,0.613994363818928)(0.0763113839285712,0.6148637404938)(0.0764508928571428,0.615731771952639)(0.076590401785714,0.61659845623449)(0.0767299107142856,0.617463791381032)(0.0768694196428572,0.618327775436559)(0.0770089285714284,0.619190406447998)(0.0771484375,0.620051682464914)(0.0772879464285712,0.620911601539501)(0.0774274553571428,0.621770161726604)(0.077566964285714,0.622627361083701)(0.0777064732142856,0.623483197670931)(0.0778459821428572,0.624337669551076)(0.0779854910714284,0.625190774789575)(0.078125,0.626042511454534)(0.0782645089285712,0.626892877616712)(0.0784040178571428,0.627741871349547)(0.078543526785714,0.628589490729136)(0.0786830357142856,0.629435733834264)(0.0788225446428572,0.630280598746383)(0.0789620535714284,0.631124083549631)(0.0791015625,0.631966186330837)(0.0792410714285712,0.632806905179511)(0.0793805803571428,0.633646238187867)(0.079520089285714,0.634484183450806)(0.0796595982142856,0.635320739065938)(0.0797991071428572,0.636155903133573)(0.0799386160714284,0.636989673756727)(0.080078125,0.637822049041137)(0.0802176339285712,0.638653027095245)(0.0803571428571428,0.639482606030223)(0.080496651785714,0.640310783959955)(0.0806361607142856,0.641137559001063)(0.0807756696428572,0.641962929272891)(0.0809151785714284,0.642786892897518)(0.0810546875,0.643609447999767)(0.0811941964285712,0.644430592707194)(0.0813337053571428,0.645250325150107)(0.081473214285714,0.646068643461556)(0.0816127232142856,0.646885545777352)(0.0817522321428572,0.647701030236055)(0.0818917410714284,0.648515094978982)(0.08203125,0.649327738150224)(0.0821707589285712,0.650138957896626)(0.0823102678571428,0.650948752367816)(0.082449776785714,0.651757119716185)(0.0825892857142856,0.65256405809691)(0.0827287946428572,0.653369565667943)(0.0828683035714284,0.654173640590021)(0.0830078125,0.654976281026677)(0.0831473214285712,0.655777485144224)(0.0832868303571428,0.656577251111784)(0.083426339285714,0.657375577101263)(0.0835658482142856,0.658172461287385)(0.0837053571428572,0.65896790184767)(0.0838448660714284,0.659761896962446)(0.083984375,0.660554444814864)(0.0841238839285712,0.661345543590882)(0.0842633928571428,0.662135191479288)(0.084402901785714,0.662923386671683)(0.0845424107142856,0.663710127362505)(0.0846819196428572,0.664495411749017)(0.0848214285714284,0.665279238031316)(0.0849609375,0.666061604412342)(0.0851004464285712,0.66684250909787)(0.0852399553571428,0.667621950296528)(0.085379464285714,0.668399926219781)(0.0855189732142856,0.669176435081958)(0.0856584821428572,0.669951475100236)(0.0857979910714284,0.670725044494649)(0.0859375,0.671497141488101)(0.0860770089285712,0.672267764306354)(0.0862165178571428,0.673036911178046)(0.086356026785714,0.673804580334679)(0.0864955357142856,0.674570770010642)(0.0866350446428572,0.675335478443194)(0.0867745535714284,0.676098703872479)(0.0869140625,0.676860444541533)(0.0870535714285712,0.677620698696273)(0.0871930803571428,0.678379464585518)(0.087332589285714,0.679136740460975)(0.0874720982142856,0.67989252457726)(0.0876116071428572,0.680646815191885)(0.0877511160714284,0.681399610565269)(0.087890625,0.682150908960748)(0.0880301339285712,0.682900708644563)(0.0881696428571428,0.68364900788588)(0.088309151785714,0.684395804956777)(0.0884486607142856,0.685141098132264)(0.0885881696428572,0.685884885690272)(0.0887276785714284,0.686627165911662)(0.0888671875,0.687367937080235)(0.0890066964285712,0.68810719748272)(0.0891462053571428,0.688844945408797)(0.089285714285714,0.689581179151079)(0.0894252232142856,0.690315897005135)(0.0895647321428572,0.691049097269478)(0.0897042410714284,0.691780778245576)(0.08984375,0.692510938237857)(0.0899832589285712,0.693239575553705)(0.0901227678571428,0.693966688503474)(0.090262276785714,0.694692275400475)(0.0904017857142856,0.695416334560999)(0.0905412946428572,0.696138864304305)(0.0906808035714284,0.696859862952626)(0.0908203125,0.697579328831185)(0.0909598214285712,0.698297260268176)(0.0910993303571428,0.69901365559479)(0.091238839285714,0.699728513145198)(0.0913783482142856,0.700441831256575)(0.0915178571428572,0.701153608269083)(0.0916573660714284,0.701863842525883)(0.091796875,0.702572532373148)(0.0919363839285712,0.703279676160046)(0.0920758928571428,0.703985272238764)(0.092215401785714,0.704689318964491)(0.0923549107142856,0.705391814695442)(0.0924944196428572,0.706092757792843)(0.0926339285714284,0.706792146620941)(0.0927734375,0.707489979547017)(0.0929129464285712,0.708186254941369)(0.0930524553571428,0.708880971177337)(0.093191964285714,0.709574126631286)(0.0933314732142856,0.710265719682627)(0.0934709821428572,0.710955748713807)(0.0936104910714284,0.711644212110314)(0.09375,0.712331108260693)(0.0938895089285712,0.713016435556528)(0.0940290178571428,0.713700192392468)(0.094168526785714,0.714382377166206)(0.0943080357142856,0.715062988278506)(0.0944475446428572,0.715742024133189)(0.0945870535714284,0.716419483137139)(0.0947265625,0.717095363700317)(0.0948660714285712,0.717769664235749)(0.0950055803571428,0.718442383159542)(0.095145089285714,0.719113518890874)(0.0952845982142856,0.719783069852013)(0.0954241071428572,0.720451034468305)(0.0955636160714284,0.721117411168184)(0.095703125,0.721782198383178)(0.0958426339285712,0.722445394547904)(0.0959821428571428,0.723106998100082)(0.096121651785714,0.723767007480524)(0.0962611607142856,0.724425421133151)(0.0964006696428572,0.725082237504987)(0.0965401785714284,0.725737455046161)(0.0966796875,0.726391072209924)(0.0968191964285712,0.72704308745263)(0.0969587053571428,0.72769349923376)(0.097098214285714,0.728342306015909)(0.0972377232142856,0.728989506264803)(0.0973772321428572,0.729635098449287)(0.0975167410714284,0.730279081041338)(0.09765625,0.730921452516072)(0.0977957589285712,0.73156221135173)(0.0979352678571428,0.732201356029702)(0.098074776785714,0.732838885034511)(0.0982142857142856,0.733474796853833)(0.0983537946428572,0.734109089978484)(0.0984933035714284,0.734741762902432)(0.0986328125,0.735372814122803)(0.0987723214285712,0.736002242139872)(0.0989118303571428,0.736630045457082)(0.099051339285714,0.737256222581028)(0.0991908482142856,0.737880772021481)(0.0993303571428572,0.73850369229137)(0.0994698660714284,0.7391249819068)(0.099609375,0.73974463938705)(0.0997488839285712,0.740362663254571)(0.0998883928571428,0.740979052035002)(0.100027901785714,0.741593804257154)(0.100167410714286,0.742206918453032)(0.100306919642857,0.742818393157824)(0.100446428571428,0.743428226909908)(0.1005859375,0.744036418250862)(0.100725446428571,0.744642965725453)(0.100864955357143,0.745247867881654)(0.101004464285714,0.745851123270634)(0.101143973214286,0.746452730446774)(0.101283482142857,0.747052687967658)(0.101422991071428,0.747650994394078)(0.1015625,0.748247648290048)(0.101702008928571,0.748842648222791)(0.101841517857143,0.749435992762754)(0.101981026785714,0.750027680483599)(0.102120535714286,0.750617709962224)(0.102260044642857,0.751206079778742)(0.102399553571428,0.751792788516502)(0.1025390625,0.752377834762089)(0.102678571428571,0.752961217105316)(0.102818080357143,0.753542934139243)(0.102957589285714,0.754122984460163)(0.103097098214286,0.754701366667619)(0.103236607142857,0.755278079364399)(0.103376116071428,0.755853121156535)(0.103515625,0.756426490653322)(0.103655133928571,0.756998186467298)(0.103794642857143,0.757568207214267)(0.103934151785714,0.758136551513288)(0.104073660714286,0.758703217986688)(0.104213169642857,0.759268205260054)(0.104352678571428,0.759831511962242)(0.1044921875,0.760393136725384)(0.104631696428571,0.760953078184878)(0.104771205357143,0.761511334979405)(0.104910714285714,0.76206790575092)(0.105050223214286,0.762622789144663)(0.105189732142857,0.763175983809157)(0.105329241071428,0.763727488396207)(0.10546875,0.764277301560916)(0.105608258928571,0.76482542196167)(0.105747767857143,0.76537184826016)(0.105887276785714,0.765916579121363)(0.106026785714286,0.766459613213564)(0.106166294642857,0.767000949208347)(0.106305803571428,0.767540585780598)(0.1064453125,0.76807852160852)(0.106584821428571,0.768614755373614)(0.106724330357143,0.769149285760704)(0.106863839285714,0.769682111457922)(0.107003348214286,0.770213231156723)(0.107142857142857,0.77074264355188)(0.107282366071428,0.771270347341486)(0.107421875,0.771796341226966)(0.107561383928571,0.772320623913067)(0.107700892857143,0.77284319410787)(0.107840401785714,0.773364050522786)(0.107979910714286,0.773883191872567)(0.108119419642857,0.774400616875296)(0.108258928571428,0.774916324252399)(0.1083984375,0.775430312728648)(0.108537946428571,0.775942581032156)(0.108677455357143,0.776453127894389)(0.108816964285714,0.776961952050158)(0.108956473214286,0.777469052237631)(0.109095982142857,0.777974427198328)(0.109235491071428,0.778478075677128)(0.109375,0.778979996422273)(0.109514508928571,0.779480188185362)(0.109654017857143,0.779978649721367)(0.109793526785714,0.780475379788617)(0.109933035714286,0.780970377148822)(0.110072544642857,0.781463640567056)(0.110212053571428,0.781955168811771)(0.1103515625,0.782444960654797)(0.110491071428571,0.78293301487134)(0.110630580357143,0.783419330239994)(0.110770089285714,0.783903905542731)(0.110909598214286,0.784386739564913)(0.111049107142857,0.78486783109529)(0.111188616071428,0.785347178926003)(0.111328125,0.785824781852589)(0.111467633928571,0.786300638673977)(0.111607142857143,0.786774748192499)(0.111746651785714,0.787247109213884)(0.111886160714286,0.787717720547267)(0.112025669642857,0.788186581005186)(0.112165178571428,0.788653689403586)(0.1123046875,0.789119044561827)(0.112444196428571,0.789582645302674)(0.112583705357143,0.790044490452315)(0.112723214285714,0.790504578840346)(0.112862723214286,0.790962909299789)(0.113002232142857,0.791419480667085)(0.113141741071428,0.791874291782096)(0.11328125,0.792327341488116)(0.113420758928571,0.79277862863186)(0.113560267857143,0.793228152063481)(0.113699776785714,0.793675910636556)(0.113839285714286,0.794121903208107)(0.113978794642857,0.794566128638584)(0.114118303571428,0.79500858579188)(0.1142578125,0.795449273535333)(0.114397321428571,0.795888190739717)(0.114536830357143,0.796325336279261)(0.114676339285714,0.796760709031633)(0.114815848214286,0.79719430787796)(0.114955357142857,0.797626131702814)(0.115094866071428,0.798056179394226)(0.115234375,0.798484449843685)(0.115373883928571,0.798910941946134)(0.115513392857143,0.799335654599981)(0.115652901785714,0.799758586707097)(0.115792410714286,0.800179737172817)(0.115931919642857,0.800599104905945)(0.116071428571428,0.801016688818753)(0.1162109375,0.801432487826987)(0.116350446428571,0.801846500849864)(0.116489955357143,0.802258726810081)(0.116629464285714,0.802669164633808)(0.116768973214286,0.803077813250701)(0.116908482142857,0.803484671593894)(0.117047991071428,0.803889738600006)(0.1171875,0.804293013209144)(0.117327008928571,0.804694494364902)(0.117466517857143,0.805094181014367)(0.117606026785714,0.805492072108114)(0.117745535714286,0.805888166600218)(0.117885044642857,0.806282463448246)(0.118024553571428,0.806674961613266)(0.1181640625,0.807065660059849)(0.118303571428571,0.807454557756063)(0.118443080357143,0.807841653673487)(0.118582589285714,0.808226946787201)(0.118722098214286,0.808610436075799)(0.118861607142857,0.808992120521382)(0.119001116071428,0.809371999109564)(0.119140625,0.809750070829477)(0.119280133928571,0.810126334673765)(0.119419642857143,0.810500789638595)(0.119559151785714,0.810873434723651)(0.119698660714286,0.811244268932142)(0.119838169642857,0.811613291270801)(0.119977678571428,0.811980500749885)(0.1201171875,0.812345896383184)(0.120256696428571,0.812709477188012)(0.120396205357143,0.813071242185222)(0.120535714285714,0.813431190399194)(0.120675223214286,0.813789320857852)(0.120814732142857,0.814145632592652)(0.120954241071428,0.814500124638588)(0.12109375,0.814852796034204)(0.121233258928571,0.815203645821578)(0.121372767857143,0.81555267304634)(0.121512276785714,0.815899876757664)(0.121651785714286,0.816245256008275)(0.121791294642857,0.816588809854447)(0.121930803571428,0.816930537356007)(0.1220703125,0.817270437576339)(0.122209821428571,0.817608509582379)(0.122349330357143,0.817944752444626)(0.122488839285714,0.818279165237135)(0.122628348214286,0.818611747037526)(0.122767857142857,0.818942496926982)(0.122907366071428,0.819271413990247)(0.123046875,0.819598497315639)(0.123186383928571,0.81992374599504)(0.123325892857143,0.820247159123906)(0.123465401785714,0.820568735801262)(0.123604910714286,0.820888475129712)(0.123744419642857,0.82120637621543)(0.123883928571428,0.821522438168172)(0.1240234375,0.821836660101274)(0.124162946428571,0.822149041131649)(0.124302455357143,0.822459580379797)(0.124441964285714,0.8227682769698)(0.124581473214286,0.823075130029328)(0.124720982142857,0.823380138689639)(0.124860491071428,0.823683302085578)(0.125,0.823984619355586)(0.125139508928571,0.824284089641692)(0.125279017857143,0.824581712089525)(0.125418526785714,0.824877485848305)(0.125558035714286,0.825171410070855)(0.125697544642857,0.825463483913594)(0.125837053571428,0.825753706536543)(0.1259765625,0.826042077103329)(0.126116071428571,0.826328594781179)(0.126255580357143,0.826613258740931)(0.126395089285714,0.826896068157026)(0.126534598214286,0.827177022207519)(0.126674107142857,0.827456120074072)(0.126813616071428,0.827733360941963)(0.126953125,0.828008744000083)(0.127092633928571,0.828282268440936)(0.127232142857143,0.82855393346065)(0.127371651785714,0.828823738258964)(0.127511160714286,0.829091682039244)(0.127650669642857,0.829357764008475)(0.127790178571428,0.829621983377265)(0.1279296875,0.82988433935985)(0.128069196428571,0.830144831174089)(0.128208705357143,0.830403458041472)(0.128348214285714,0.830660219187117)(0.128487723214286,0.830915113839775)(0.128627232142857,0.831168141231829)(0.128766741071428,0.831419300599293)(0.12890625,0.831668591181823)(0.129045758928571,0.831916012222707)(0.129185267857143,0.832161562968873)(0.129324776785714,0.83240524267089)(0.129464285714286,0.832647050582969)(0.129603794642857,0.832886985962961)(0.129743303571428,0.833125048072365)(0.1298828125,0.833361236176324)(0.130022321428571,0.833595549543627)(0.130161830357143,0.833827987446717)(0.130301339285714,0.83405854916168)(0.130440848214286,0.834287233968261)(0.130580357142857,0.834514041149851)(0.130719866071428,0.834738969993499)(0.130859375,0.83496201978991)(0.130998883928571,0.835183189833446)(0.131138392857143,0.835402479422126)(0.131277901785714,0.83561988785763)(0.131417410714286,0.8358354144453)(0.131556919642857,0.836049058494139)(0.131696428571428,0.836260819316814)(0.1318359375,0.836470696229658)(0.131975446428571,0.836678688552671)(0.132114955357143,0.836884795609522)(0.132254464285714,0.837089016727544)(0.132393973214286,0.837291351237747)(0.132533482142857,0.83749179847481)(0.132672991071428,0.837690357777084)(0.1328125,0.837887028486596)(0.132952008928571,0.838081809949048)(0.133091517857143,0.838274701513821)(0.133231026785714,0.838465702533971)(0.133370535714286,0.838654812366236)(0.133510044642857,0.838842030371034)(0.133649553571428,0.839027355912464)(0.1337890625,0.83921078835831)(0.133928571428571,0.839392327080039)(0.134068080357143,0.839571971452805)(0.134207589285714,0.839749720855448)(0.134347098214286,0.839925574670497)(0.134486607142857,0.840099532284169)(0.134626116071428,0.840271593086372)(0.134765625,0.840441756470707)(0.134905133928571,0.840610021834466)(0.135044642857143,0.840776388578637)(0.135184151785714,0.8409408561079)(0.135323660714286,0.841103423830637)(0.135463169642857,0.841264091158922)(0.135602678571428,0.841422857508529)(0.1357421875,0.841579722298935)(0.135881696428571,0.841734684953313)(0.136021205357143,0.841887744898544)(0.136160714285714,0.842038901565205)(0.136300223214286,0.842188154387584)(0.136439732142857,0.84233550280367)(0.136579241071428,0.842480946255161)(0.13671875,0.842624484187462)(0.136858258928571,0.842766116049684)(0.136997767857143,0.842905841294653)(0.137137276785714,0.843043659378901)(0.137276785714286,0.843179569762675)(0.137416294642857,0.843313571909932)(0.137555803571428,0.843445665288345)(0.1376953125,0.843575849369303)(0.137834821428571,0.843704123627908)(0.137974330357143,0.843830487542981)(0.138113839285714,0.843954940597061)(0.138253348214286,0.844077482276405)(0.138392857142857,0.844198112070992)(0.138532366071428,0.844316829474519)(0.138671875,0.844433633984409)(0.138811383928571,0.844548525101804)(0.138950892857143,0.844661502331573)(0.139090401785714,0.844772565182308)(0.139229910714286,0.844881713166329)(0.139369419642857,0.84498894579968)(0.139508928571428,0.845094262602135)(0.1396484375,0.845197663097197)(0.139787946428571,0.845299146812096)(0.139927455357143,0.845398713277796)(0.140066964285714,0.845496362028989)(0.140206473214286,0.845592092604103)(0.140345982142857,0.845685904545297)(0.140485491071428,0.845777797398464)(0.140625,0.845867770713234)(0.140764508928571,0.84595582404297)(0.140904017857143,0.846041956944775)(0.141043526785714,0.846126168979488)(0.141183035714286,0.846208459711687)(0.141322544642857,0.846288828709688)(0.141462053571428,0.846367275545549)(0.1416015625,0.846443799795068)(0.141741071428571,0.846518401037786)(0.141880580357143,0.846591078856986)(0.142020089285714,0.846661832839694)(0.142159598214286,0.846730662576681)(0.142299107142857,0.846797567662463)(0.142438616071428,0.846862547695302)(0.142578125,0.846925602277206)(0.142717633928571,0.846986731013931)(0.142857142857143,0.847045933514981)(0.142996651785714,0.84710320939361)(0.143136160714286,0.84715855826682)(0.143275669642857,0.847211979755365)(0.143415178571428,0.847263473483749)(0.1435546875,0.847313039080229)(0.143694196428571,0.847360676176814)(0.143833705357143,0.847406384409266)(0.143973214285714,0.847450163417103)(0.144112723214286,0.847492012843596)(0.144252232142857,0.847531932335771)(0.144391741071428,0.847569921544413)(0.14453125,0.84760598012406)(0.144670758928571,0.847640107733011)(0.144810267857143,0.847672304033322)(0.144949776785714,0.847702568690806)(0.145089285714286,0.847730901375038)(0.145228794642857,0.847757301759353)(0.145368303571428,0.847781769520845)(0.1455078125,0.847804304340371)(0.145647321428571,0.847824905902549)(0.145786830357143,0.84784357389576)(0.145926339285714,0.84786030801215)(0.146065848214286,0.847875107947625)(0.146205357142857,0.84788797340186)(0.146344866071428,0.84789890407829)(0.146484375,0.847907899684121)(0.146623883928571,0.847914959930322)(0.146763392857143,0.847920084531627)(0.146902901785714,0.847923273206542)(0.147042410714286,0.847924525677338)(0.147181919642857,0.847923841670054)(0.147321428571428,0.847921220914499)(0.1474609375,0.847916663144252)(0.147600446428571,0.847910168096659)(0.147739955357143,0.847901735512841)(0.147879464285714,0.847891365137686)(0.148018973214286,0.847879056719855)(0.148158482142857,0.847864810011782)(0.148297991071428,0.847848624769672)(0.1484375,0.847830500753503)(0.148577008928571,0.847810437727026)(0.148716517857143,0.847788435457769)(0.148856026785714,0.847764493717029)(0.148995535714286,0.847738612279883)(0.149135044642857,0.847710790925179)(0.149274553571428,0.847681029435543)(0.1494140625,0.847649327597377)(0.149553571428571,0.847615685200859)(0.149693080357143,0.847580102039942)(0.149832589285714,0.84754257791236)(0.149972098214286,0.847503112619622)(0.150111607142857,0.847461705967015)(0.150251116071428,0.847418357763606)(0.150390625,0.84737306782224)(0.150530133928571,0.84732583595954)(0.150669642857143,0.84727666199591)(0.150809151785714,0.847225545755534)(0.150948660714286,0.847172487066375)(0.151088169642857,0.847117485760178)(0.151227678571428,0.847060541672467)(0.1513671875,0.847001654642548)(0.151506696428571,0.84694082451351)(0.151646205357143,0.846878051132222)(0.151785714285714,0.846813334349335)(0.151925223214286,0.846746674019284)(0.152064732142857,0.846678070000285)(0.152204241071428,0.846607522154338)(0.15234375,0.846535030347225)(0.152483258928571,0.846460594448514)(0.152622767857143,0.846384214331553)(0.152762276785714,0.846305889873478)(0.152901785714286,0.846225620955205)(0.153041294642857,0.846143407461439)(0.153180803571428,0.846059249280667)(0.1533203125,0.84597314630516)(0.153459821428571,0.845885098430977)(0.153599330357143,0.845795105557959)(0.153738839285714,0.845703167589735)(0.153878348214286,0.84560928443372)(0.154017857142857,0.845513456001111)(0.154157366071428,0.845415682206897)(0.154296875,0.845315962969848)(0.154436383928571,0.845214298212523)(0.154575892857143,0.845110687861266)(0.154715401785714,0.845005131846211)(0.154854910714286,0.844897630101273)(0.154994419642857,0.844788182564159)(0.155133928571428,0.844676789176362)(0.1552734375,0.84456344988316)(0.155412946428571,0.844448164633621)(0.155552455357143,0.844330933380598)(0.155691964285714,0.844211756080733)(0.155831473214286,0.844090632694455)(0.155970982142857,0.84396756318598)(0.156110491071428,0.843842547523315)(0.15625,0.843715585678249)(0.156389508928571,0.843586677626364)(0.156529017857143,0.843455823347026)(0.156668526785714,0.843323022823393)(0.156808035714286,0.843188276042406)(0.156947544642857,0.843051582994799)(0.157087053571428,0.84291294367509)(0.1572265625,0.842772358081586)(0.157366071428571,0.842629826216385)(0.157505580357143,0.842485348085368)(0.157645089285714,0.842338923698208)(0.157784598214286,0.842190553068362)(0.157924107142857,0.84204023621308)(0.158063616071428,0.841887973153397)(0.158203125,0.841733763914134)(0.158342633928571,0.841577608523903)(0.158482142857143,0.841419507015103)(0.158621651785714,0.84125945942392)(0.158761160714286,0.841097465790327)(0.158900669642857,0.840933526158087)(0.159040178571428,0.840767640574747)(0.1591796875,0.840599809091645)(0.159319196428571,0.840430031763903)(0.159458705357143,0.840258308650432)(0.159598214285714,0.84008463981393)(0.159737723214286,0.839909025320881)(0.159877232142857,0.839731465241557)(0.160016741071428,0.839551959650016)(0.16015625,0.839370508624101)(0.160295758928571,0.839187112245445)(0.160435267857143,0.839001770599463)(0.160574776785714,0.838814483775359)(0.160714285714286,0.838625251866122)(0.160853794642857,0.838434074968526)(0.160993303571428,0.838240953183132)(0.1611328125,0.838045886614283)(0.161272321428571,0.837848875370112)(0.161411830357143,0.837649919562533)(0.161551339285714,0.837449019307247)(0.161690848214286,0.837246174723736)(0.161830357142857,0.83704138593527)(0.161969866071428,0.836834653068902)(0.162109375,0.836625976255468)(0.162248883928571,0.836415355629587)(0.162388392857143,0.836202791329662)(0.162527901785714,0.835988283497879)(0.162667410714286,0.835771832280206)(0.162806919642857,0.835553437826393)(0.162946428571428,0.835333100289976)(0.1630859375,0.835110819828266)(0.163225446428571,0.834886596602361)(0.163364955357143,0.834660430777137)(0.163504464285714,0.834432322521254)(0.163643973214286,0.834202272007149)(0.163783482142857,0.83397027941104)(0.163922991071428,0.833736344912928)(0.1640625,0.833500468696589)(0.164202008928571,0.833262650949581)(0.164341517857143,0.833022891863239)(0.164481026785714,0.832781191632678)(0.164620535714286,0.83253755045679)(0.164760044642857,0.832291968538245)(0.164899553571428,0.832044446083491)(0.1650390625,0.83179498330275)(0.165178571428571,0.831543580410024)(0.165318080357143,0.831290237623087)(0.165457589285714,0.831034955163494)(0.165597098214286,0.83077773325657)(0.165736607142857,0.830518572131416)(0.165876116071428,0.830257472020911)(0.166015625,0.829994433161701)(0.166155133928571,0.829729455794212)(0.166294642857143,0.829462540162638)(0.166434151785714,0.829193686514949)(0.166573660714286,0.828922895102883)(0.166713169642857,0.828650166181953)(0.166852678571428,0.828375500011443)(0.1669921875,0.828098896854403)(0.167131696428571,0.827820356977659)(0.167271205357143,0.8275398806518)(0.167410714285714,0.827257468151191)(0.167550223214286,0.826973119753957)(0.167689732142857,0.826686835741999)(0.167829241071428,0.82639861640098)(0.16796875,0.826108462020331)(0.168108258928571,0.825816372893251)(0.168247767857143,0.8255223493167)(0.168387276785714,0.825226391591408)(0.168526785714286,0.824928500021865)(0.168666294642857,0.82462867491633)(0.168805803571428,0.82432691658682)(0.1689453125,0.824023225349117)(0.169084821428571,0.823717601522767)(0.169224330357143,0.823410045431071)(0.169363839285714,0.823100557401098)(0.169503348214286,0.822789137763672)(0.169642857142857,0.82247578685338)(0.169782366071428,0.822160505008565)(0.169921875,0.821843292571329)(0.170061383928571,0.821524149887534)(0.170200892857143,0.821203077306793)(0.170340401785714,0.820880075182481)(0.170479910714286,0.820555143871724)(0.170619419642857,0.820228283735406)(0.170758928571428,0.819899495138165)(0.1708984375,0.819568778448388)(0.171037946428571,0.819236134038221)(0.171177455357143,0.818901562283556)(0.171316964285714,0.818565063564041)(0.171456473214286,0.818226638263069)(0.171595982142857,0.817886286767787)(0.171735491071428,0.817544009469091)(0.171875,0.817199806761623)(0.172014508928571,0.816853679043773)(0.172154017857143,0.816505626717677)(0.172293526785714,0.816155650189219)(0.172433035714286,0.815803749868024)(0.172572544642857,0.815449926167466)(0.172712053571428,0.815094179504662)(0.1728515625,0.814736510300465)(0.172991071428571,0.814376918979479)(0.173130580357143,0.814015405970042)(0.173270089285714,0.813651971704237)(0.173409598214286,0.813286616617881)(0.173549107142857,0.812919341150534)(0.173688616071428,0.812550145745493)(0.173828125,0.812179030849789)(0.173967633928571,0.811805996914192)(0.174107142857143,0.811431044393205)(0.174246651785714,0.811054173745066)(0.174386160714286,0.810675385431746)(0.174525669642857,0.810294679918946)(0.174665178571428,0.809912057676105)(0.1748046875,0.809527519176384)(0.174944196428571,0.80914106489668)(0.175083705357143,0.808752695317615)(0.175223214285714,0.808362410923541)(0.175362723214286,0.807970212202535)(0.175502232142857,0.807576099646399)(0.175641741071428,0.807180073750665)(0.17578125,0.806782135014583)(0.175920758928571,0.806382283941128)(0.176060267857143,0.805980521036998)(0.176199776785714,0.805576846812611)(0.176339285714286,0.805171261782103)(0.176478794642857,0.804763766463333)(0.176618303571428,0.804354361377876)(0.1767578125,0.803943047051023)(0.176897321428571,0.803529824011782)(0.177036830357143,0.803114692792875)(0.177176339285714,0.802697653930741)(0.177315848214286,0.802278707965525)(0.177455357142857,0.80185785544109)(0.177594866071428,0.80143509690501)(0.177734375,0.801010432908562)(0.177873883928571,0.80058386400674)(0.178013392857143,0.800155390758238)(0.178152901785714,0.799725013725462)(0.178292410714286,0.799292733474519)(0.178431919642857,0.798858550575223)(0.178571428571428,0.798422465601092)(0.1787109375,0.79798447912934)(0.178850446428571,0.797544591740891)(0.178989955357143,0.79710280402036)(0.179129464285714,0.796659116556067)(0.179268973214286,0.796213529940023)(0.179408482142857,0.795766044767942)(0.179547991071428,0.795316661639231)(0.1796875,0.794865381156987)(0.179827008928571,0.794412203928005)(0.179966517857143,0.793957130562768)(0.180106026785714,0.793500161675452)(0.180245535714286,0.793041297883919)(0.180385044642857,0.792580539809723)(0.180524553571428,0.792117888078103)(0.1806640625,0.791653343317982)(0.180803571428571,0.79118690616197)(0.180943080357143,0.790718577246357)(0.181082589285714,0.79024835721112)(0.181222098214286,0.789776246699909)(0.181361607142857,0.78930224636006)(0.181501116071428,0.788826356842587)(0.181640625,0.788348578802174)(0.181780133928571,0.78786891289719)(0.181919642857143,0.78738735978967)(0.182059151785714,0.786903920145329)(0.182198660714286,0.786418594633546)(0.182338169642857,0.785931383927378)(0.182477678571428,0.785442288703548)(0.1826171875,0.784951309642445)(0.182756696428571,0.784458447428128)(0.182896205357143,0.783963702748316)(0.183035714285714,0.783467076294399)(0.183175223214286,0.782968568761423)(0.183314732142857,0.782468180848097)(0.183454241071428,0.781965913256793)(0.18359375,0.781461766693536)(0.183733258928571,0.780955741868014)(0.183872767857143,0.780447839493563)(0.184012276785714,0.779938060287181)(0.184151785714286,0.779426404969513)(0.184291294642857,0.778912874264858)(0.184430803571428,0.778397468901167)(0.1845703125,0.777880189610035)(0.184709821428571,0.777361037126709)(0.184849330357143,0.776840012190076)(0.184988839285714,0.776317115542674)(0.185128348214286,0.775792347930678)(0.185267857142857,0.775265710103907)(0.185407366071428,0.774737202815822)(0.185546875,0.774206826823517)(0.185686383928571,0.773674582887731)(0.185825892857143,0.773140471772829)(0.185965401785714,0.772604494246818)(0.186104910714286,0.772066651081332)(0.186244419642857,0.77152694305164)(0.186383928571428,0.770985370936641)(0.1865234375,0.770441935518856)(0.186662946428571,0.76989663758444)(0.186802455357143,0.769349477923166)(0.186941964285714,0.768800457328439)(0.187081473214286,0.768249576597275)(0.187220982142857,0.767696836530319)(0.187360491071428,0.767142237931834)(0.1875,0.766585781609694)(0.187639508928571,0.766027468375396)(0.187779017857143,0.765467299044046)(0.187918526785714,0.764905274434366)(0.188058035714286,0.764341395368685)(0.188197544642857,0.763775662672945)(0.188337053571428,0.763208077176695)(0.1884765625,0.762638639713086)(0.188616071428571,0.762067351118881)(0.188755580357143,0.761494212234437)(0.188895089285714,0.76091922390372)(0.189034598214286,0.76034238697429)(0.189174107142857,0.759763702297308)(0.189313616071428,0.759183170727531)(0.189453125,0.758600793123308)(0.189592633928571,0.758016570346586)(0.189732142857143,0.757430503262895)(0.189871651785714,0.756842592741364)(0.190011160714286,0.756252839654701)(0.190150669642857,0.755661244879207)(0.190290178571428,0.755067809294767)(0.1904296875,0.754472533784842)(0.190569196428571,0.753875419236482)(0.190708705357143,0.75327646654031)(0.190848214285714,0.752675676590533)(0.190987723214286,0.752073050284926)(0.191127232142857,0.751468588524844)(0.191266741071428,0.750862292215215)(0.19140625,0.75025416226453)(0.191545758928571,0.74964419958486)(0.191685267857143,0.74903240509183)(0.191824776785714,0.748418779704644)(0.191964285714286,0.747803324346056)(0.192103794642857,0.747186039942391)(0.192243303571428,0.746566927423533)(0.1923828125,0.745945987722917)(0.192522321428571,0.745323221777545)(0.192661830357143,0.744698630527962)(0.192801339285714,0.744072214918275)(0.192940848214286,0.743443975896134)(0.193080357142857,0.742813914412744)(0.193219866071428,0.742182031422856)(0.193359375,0.741548327884761)(0.193498883928571,0.740912804760301)(0.193638392857143,0.740275463014852)(0.193777901785714,0.739636303617336)(0.193917410714286,0.738995327540205)(0.194056919642857,0.738352535759456)(0.194196428571428,0.737707929254614)(0.1943359375,0.737061509008736)(0.194475446428571,0.736413276008412)(0.194614955357143,0.735763231243756)(0.194754464285714,0.735111375708414)(0.194893973214286,0.734457710399548)(0.195033482142857,0.73380223631785)(0.195172991071428,0.733144954467531)(0.1953125,0.732485865856316)(0.195452008928571,0.731824971495451)(0.195591517857143,0.731162272399693)(0.195731026785714,0.730497769587316)(0.195870535714286,0.729831464080099)(0.196010044642857,0.729163356903333)(0.196149553571428,0.728493449085818)(0.1962890625,0.727821741659851)(0.196428571428571,0.72714823566124)(0.196568080357143,0.726472932129285)(0.196707589285714,0.725795832106795)(0.196847098214286,0.725116936640063)(0.196986607142857,0.724436246778886)(0.197126116071428,0.723753763576552)(0.197265625,0.723069488089834)(0.197405133928571,0.722383421379001)(0.197544642857143,0.7216955645078)(0.197684151785714,0.72100591854347)(0.197823660714286,0.720314484556724)(0.197963169642857,0.719621263621761)(0.198102678571428,0.718926256816259)(0.1982421875,0.718229465221365)(0.198381696428571,0.717530889921708)(0.198521205357143,0.716830532005378)(0.198660714285714,0.716128392563948)(0.198800223214286,0.715424472692444)(0.198939732142857,0.714718773489368)(0.199079241071428,0.714011296056683)(0.19921875,0.713302041499807)(0.199358258928571,0.712591010927625)(0.199497767857143,0.711878205452471)(0.199637276785714,0.711163626190141)(0.199776785714286,0.710447274259874)(0.199916294642857,0.709729150784368)(0.200055803571428,0.709009256889766)(0.2001953125,0.708287593705654)(0.200334821428571,0.707564162365066)(0.200474330357143,0.706838964004471)(0.200613839285714,0.706111999763786)(0.200753348214286,0.705383270786354)(0.200892857142857,0.704652778218962)(0.201032366071428,0.703920523211829)(0.201171875,0.703186506918594)(0.201311383928571,0.702450730496338)(0.201450892857143,0.701713195105556)(0.201590401785714,0.700973901910175)(0.201729910714286,0.700232852077534)(0.201869419642857,0.699490046778401)(0.202008928571428,0.698745487186957)(0.2021484375,0.697999174480791)(0.202287946428571,0.697251109840914)(0.202427455357143,0.696501294451739)(0.202566964285714,0.695749729501092)(0.202706473214286,0.694996416180198)(0.202845982142857,0.694241355683691)(0.202985491071428,0.693484549209606)(0.203125,0.692725997959366)(0.203264508928571,0.691965703137803)(0.203404017857143,0.691203665953131)(0.203543526785714,0.690439887616966)(0.203683035714286,0.689674369344302)(0.203822544642857,0.688907112353527)(0.203962053571428,0.688138117866413)(0.2041015625,0.687367387108107)(0.204241071428571,0.686594921307143)(0.204380580357143,0.685820721695425)(0.204520089285714,0.685044789508238)(0.204659598214286,0.684267125984233)(0.204799107142857,0.683487732365435)(0.204938616071428,0.682706609897237)(0.205078125,0.68192375982839)(0.205217633928571,0.681139183411017)(0.205357142857143,0.680352881900591)(0.205496651785714,0.679564856555952)(0.205636160714286,0.678775108639286)(0.205775669642857,0.677983639416139)(0.205915178571428,0.677190450155406)(0.2060546875,0.676395542129324)(0.206194196428571,0.675598916613482)(0.206333705357143,0.674800574886806)(0.206473214285714,0.674000518231568)(0.206612723214286,0.673198747933372)(0.206752232142857,0.67239526528116)(0.206891741071428,0.671590071567211)(0.20703125,0.670783168087125)(0.207170758928571,0.669974556139838)(0.207310267857143,0.669164237027604)(0.207449776785714,0.668352212056007)(0.207589285714286,0.667538482533943)(0.207728794642857,0.666723049773633)(0.207868303571428,0.665905915090612)(0.2080078125,0.66508707980372)(0.208147321428571,0.664266545235117)(0.208286830357143,0.663444312710261)(0.208426339285714,0.662620383557923)(0.208565848214286,0.661794759110168)(0.208705357142857,0.660967440702366)(0.208844866071428,0.660138429673187)(0.208984375,0.659307727364585)(0.209123883928571,0.658475335121816)(0.209263392857143,0.657641254293416)(0.209402901785714,0.656805486231217)(0.209542410714286,0.655968032290325)(0.209681919642857,0.655128893829135)(0.209821428571428,0.65428807220932)(0.2099609375,0.65344556879582)(0.210100446428571,0.652601384956862)(0.210239955357143,0.651755522063931)(0.210379464285714,0.650907981491789)(0.210518973214286,0.650058764618454)(0.210658482142857,0.649207872825216)(0.210797991071428,0.648355307496624)(0.2109375,0.647501070020473)(0.211077008928571,0.646645161787828)(0.211216517857143,0.645787584192993)(0.211356026785714,0.64492833863353)(0.211495535714286,0.644067426510239)(0.211635044642857,0.643204849227171)(0.211774553571428,0.642340608191618)(0.2119140625,0.641474704814101)(0.212053571428571,0.640607140508388)(0.212193080357143,0.639737916691469)(0.212332589285714,0.638867034783574)(0.212472098214286,0.63799449620815)(0.212611607142857,0.637120302391876)(0.212751116071428,0.636244454764652)(0.212890625,0.635366954759591)(0.213030133928571,0.634487803813029)(0.213169642857143,0.633607003364508)(0.213309151785714,0.632724554856789)(0.213448660714286,0.631840459735831)(0.213588169642857,0.630954719450805)(0.213727678571428,0.630067335454084)(0.2138671875,0.629178309201232)(0.214006696428571,0.628287642151021)(0.214146205357143,0.627395335765404)(0.214285714285714,0.626501391509538)(0.214425223214286,0.625605810851754)(0.214564732142857,0.624708595263578)(0.214704241071428,0.623809746219717)(0.21484375,0.622909265198049)(0.214983258928571,0.62200715367964)(0.215122767857143,0.621103413148717)(0.215262276785714,0.620198045092691)(0.215401785714286,0.619291051002128)(0.215541294642857,0.618382432370765)(0.215680803571428,0.617472190695505)(0.2158203125,0.616560327476398)(0.215959821428571,0.615646844216664)(0.216099330357143,0.614731742422661)(0.216238839285714,0.613815023603913)(0.216378348214286,0.612896689273077)(0.216517857142857,0.611976740945962)(0.216657366071428,0.611055180141521)(0.216796875,0.610132008381835)(0.216936383928571,0.609207227192133)(0.217075892857143,0.608280838100762)(0.217215401785714,0.607352842639214)(0.217354910714286,0.606423242342093)(0.217494419642857,0.605492038747136)(0.217633928571428,0.6045592333952)(0.2177734375,0.603624827830253)(0.217912946428571,0.602688823599385)(0.218052455357143,0.60175122225279)(0.218191964285714,0.60081202534378)(0.218331473214286,0.599871234428759)(0.218470982142857,0.598928851067247)(0.218610491071428,0.597984876821858)(0.21875,0.597039313258297)(0.218889508928571,0.596092161945373)(0.219029017857143,0.595143424454972)(0.219168526785714,0.594193102362082)(0.219308035714286,0.593241197244761)(0.219447544642857,0.592287710684157)(0.219587053571428,0.591332644264496)(0.2197265625,0.590375999573071)(0.219866071428571,0.589417778200257)(0.220005580357143,0.588457981739488)(0.220145089285714,0.587496611787273)(0.220284598214286,0.586533669943172)(0.220424107142857,0.585569157809816)(0.220563616071428,0.584603076992888)(0.220703125,0.583635429101119)(0.220842633928571,0.582666215746299)(0.220982142857143,0.581695438543254)(0.221121651785714,0.580723099109866)(0.221261160714286,0.579749199067045)(0.221400669642857,0.578773740038747)(0.221540178571428,0.577796723651964)(0.2216796875,0.576818151536709)(0.221819196428571,0.575838025326033)(0.221958705357143,0.574856346656003)(0.222098214285714,0.573873117165717)(0.222237723214286,0.57288833849728)(0.222377232142857,0.571902012295823)(0.222516741071428,0.570914140209485)(0.22265625,0.569924723889409)(0.222795758928571,0.568933764989752)(0.222935267857143,0.567941265167663)(0.223074776785714,0.566947226083303)(0.223214285714286,0.565951649399814)(0.223353794642857,0.564954536783343)(0.223493303571428,0.563955889903023)(0.2236328125,0.562955710430966)(0.223772321428571,0.56195400004228)(0.223911830357143,0.560950760415037)(0.224051339285714,0.559945993230301)(0.224190848214286,0.558939700172095)(0.224330357142857,0.55793188292742)(0.224469866071428,0.556922543186247)(0.224609375,0.555911682641497)(0.224748883928571,0.554899302989067)(0.224888392857143,0.553885405927794)(0.225027901785714,0.552869993159483)(0.225167410714286,0.551853066388879)(0.225306919642857,0.550834627323677)(0.225446428571428,0.549814677674522)(0.2255859375,0.548793219154984)(0.225725446428571,0.547770253481587)(0.225864955357143,0.546745782373771)(0.226004464285714,0.545719807553923)(0.226143973214286,0.54469233074734)(0.226283482142857,0.543663353682256)(0.226422991071428,0.542632878089821)(0.2265625,0.541600905704095)(0.226702008928571,0.540567438262061)(0.226841517857143,0.539532477503602)(0.226981026785714,0.538496025171519)(0.227120535714286,0.537458083011502)(0.227260044642857,0.536418652772152)(0.227399553571428,0.535377736204965)(0.2275390625,0.534335335064321)(0.227678571428571,0.533291451107502)(0.227818080357143,0.532246086094663)(0.227957589285714,0.531199241788853)(0.228097098214286,0.530150919955991)(0.228236607142857,0.529101122364877)(0.228376116071428,0.528049850787186)(0.228515625,0.526997106997452)(0.228655133928571,0.525942892773085)(0.228794642857143,0.524887209894346)(0.228934151785714,0.523830060144368)(0.229073660714286,0.522771445309123)(0.229213169642857,0.521711367177448)(0.229352678571428,0.520649827541024)(0.2294921875,0.51958682819437)(0.229631696428571,0.518522370934857)(0.229771205357143,0.517456457562682)(0.229910714285714,0.516389089880887)(0.230050223214286,0.515320269695334)(0.230189732142857,0.514249998814719)(0.230329241071428,0.513178279050564)(0.23046875,0.5121051122172)(0.230608258928571,0.511030500131787)(0.230747767857143,0.509954444614285)(0.230887276785714,0.508876947487477)(0.231026785714286,0.507798010576938)(0.231166294642857,0.506717635711056)(0.231305803571428,0.505635824721017)(0.2314453125,0.504552579440791)(0.231584821428571,0.503467901707155)(0.231724330357143,0.502381793359659)(0.231863839285714,0.501294256240653)(0.232003348214286,0.500205292195251)(0.232142857142857,0.499114903071356) 
};
\addplot [
color=blue,
solid,
forget plot
]
coordinates{
 (0.232142857142857,0.499114903071356)(0.232282366071428,0.498023090719645)(0.232421875,0.496929856993555)(0.232561383928571,0.495835203749301)(0.232700892857143,0.49473913284585)(0.232840401785714,0.49364164614494)(0.232979910714286,0.49254274551105)(0.233119419642857,0.491442432811424)(0.233258928571428,0.490340709916052)(0.2333984375,0.489237578697658)(0.233537946428571,0.488133041031724)(0.233677455357143,0.487027098796451)(0.233816964285714,0.48591975387279)(0.233956473214286,0.484811008144409)(0.234095982142857,0.483700863497711)(0.234235491071428,0.482589321821823)(0.234375,0.481476385008579)(0.234514508928571,0.480362054952544)(0.234654017857143,0.479246333550979)(0.234793526785714,0.478129222703868)(0.234933035714286,0.477010724313885)(0.235072544642857,0.475890840286416)(0.235212053571428,0.474769572529541)(0.2353515625,0.473646922954026)(0.235491071428571,0.472522893473339)(0.235630580357143,0.471397486003618)(0.235770089285714,0.4702707024637)(0.235909598214286,0.469142544775084)(0.236049107142857,0.468013014861955)(0.236188616071428,0.466882114651168)(0.236328125,0.465749846072235)(0.236467633928571,0.464616211057346)(0.236607142857143,0.463481211541334)(0.236746651785714,0.462344849461705)(0.236886160714286,0.461207126758599)(0.237025669642857,0.460068045374819)(0.237165178571428,0.458927607255809)(0.2373046875,0.457785814349644)(0.237444196428571,0.45664266860705)(0.237583705357143,0.455498171981374)(0.237723214285714,0.454352326428602)(0.237862723214286,0.453205133907334)(0.238002232142857,0.452056596378803)(0.238141741071428,0.450906715806856)(0.23828125,0.449755494157949)(0.238420758928571,0.448602933401156)(0.238560267857143,0.447449035508148)(0.238699776785714,0.446293802453211)(0.238839285714286,0.445137236213213)(0.238978794642857,0.443979338767632)(0.239118303571428,0.442820112098533)(0.2392578125,0.441659558190559)(0.239397321428571,0.44049767903095)(0.239536830357143,0.439334476609513)(0.239676339285714,0.43816995291864)(0.239815848214286,0.437004109953286)(0.239955357142857,0.43583694971098)(0.240094866071428,0.434668474191817)(0.240234375,0.433498685398441)(0.240373883928571,0.432327585336065)(0.240513392857143,0.43115517601244)(0.240652901785714,0.429981459437882)(0.240792410714286,0.428806437625234)(0.240931919642857,0.427630112589891)(0.241071428571428,0.426452486349785)(0.2412109375,0.42527356092537)(0.241350446428571,0.424093338339641)(0.241489955357143,0.422911820618105)(0.241629464285714,0.421729009788804)(0.241768973214286,0.420544907882282)(0.241908482142857,0.419359516931605)(0.242047991071428,0.418172838972352)(0.2421875,0.416984876042591)(0.242327008928571,0.41579563018291)(0.242466517857143,0.414605103436375)(0.242606026785714,0.413413297848563)(0.242745535714286,0.412220215467524)(0.242885044642857,0.411025858343803)(0.243024553571428,0.409830228530429)(0.2431640625,0.408633328082894)(0.243303571428571,0.407435159059179)(0.243443080357143,0.406235723519718)(0.243582589285714,0.405035023527428)(0.243722098214286,0.403833061147668)(0.243861607142857,0.402629838448267)(0.244001116071428,0.401425357499508)(0.244140625,0.400219620374111)(0.244280133928571,0.399012629147254)(0.244419642857143,0.397804385896543)(0.244559151785714,0.396594892702036)(0.244698660714286,0.395384151646206)(0.244838169642857,0.394172164813969)(0.244977678571428,0.392958934292665)(0.2451171875,0.391744462172042)(0.245256696428571,0.390528750544279)(0.245396205357143,0.389311801503956)(0.245535714285714,0.388093617148072)(0.245675223214286,0.386874199576018)(0.245814732142857,0.385653550889595)(0.245954241071428,0.384431673193002)(0.24609375,0.383208568592816)(0.246233258928571,0.38198423919802)(0.246372767857143,0.380758687119963)(0.246512276785714,0.379531914472391)(0.246651785714286,0.378303923371407)(0.246791294642857,0.377074715935501)(0.246930803571428,0.375844294285529)(0.2470703125,0.374612660544696)(0.247209821428571,0.373379816838586)(0.247349330357143,0.372145765295118)(0.247488839285714,0.370910508044581)(0.247628348214286,0.369674047219592)(0.247767857142857,0.368436384955125)(0.247907366071428,0.36719752338849)(0.248046875,0.365957464659322)(0.248186383928571,0.364716210909598)(0.248325892857143,0.363473764283608)(0.248465401785714,0.362230126927979)(0.248604910714286,0.360985300991639)(0.248744419642857,0.359739288625841)(0.248883928571428,0.358492091984149)(0.2490234375,0.357243713222417)(0.249162946428571,0.355994154498819)(0.249302455357143,0.354743417973806)(0.249441964285714,0.353491505810141)(0.249581473214286,0.352238420172856)(0.249720982142857,0.35098416322928)(0.249860491071428,0.349728737149023)(0.25,0.348472144103957)(0.250139508928571,0.34721438626824)(0.250279017857143,0.345955465818285)(0.250418526785714,0.344695384932779)(0.250558035714286,0.343434145792656)(0.250697544642857,0.342171750581112)(0.250837053571428,0.340908201483597)(0.2509765625,0.339643500687792)(0.251116071428571,0.338377650383636)(0.251255580357143,0.337110652763291)(0.251395089285714,0.335842510021166)(0.251534598214286,0.334573224353883)(0.251674107142857,0.333302797960302)(0.251813616071428,0.332031233041501)(0.251953125,0.330758531800762)(0.252092633928571,0.329484696443598)(0.252232142857143,0.32820972917771)(0.252371651785714,0.326933632213017)(0.252511160714286,0.325656407761625)(0.252650669642857,0.324378058037842)(0.252790178571428,0.323098585258169)(0.2529296875,0.32181799164128)(0.253069196428571,0.320536279408045)(0.253208705357143,0.319253450781498)(0.253348214285714,0.317969507986859)(0.253487723214286,0.316684453251502)(0.253627232142857,0.315398288804976)(0.253766741071428,0.314111016878992)(0.25390625,0.312822639707403)(0.254045758928571,0.311533159526228)(0.254185267857143,0.310242578573618)(0.254324776785714,0.308950899089884)(0.254464285714286,0.307658123317455)(0.254603794642857,0.30636425350091)(0.254743303571428,0.305069291886954)(0.2548828125,0.303773240724407)(0.255022321428571,0.302476102264225)(0.255161830357143,0.301177878759464)(0.255301339285714,0.299878572465307)(0.255440848214286,0.29857818563903)(0.255580357142857,0.297276720540023)(0.255719866071428,0.295974179429775)(0.255859375,0.294670564571856)(0.255998883928571,0.293365878231945)(0.256138392857143,0.292060122677787)(0.256277901785714,0.290753300179225)(0.256417410714286,0.289445413008163)(0.256556919642857,0.288136463438589)(0.256696428571428,0.286826453746559)(0.2568359375,0.285515386210179)(0.256975446428571,0.284203263109633)(0.257114955357143,0.282890086727139)(0.257254464285714,0.281575859346984)(0.257393973214286,0.280260583255483)(0.257533482142857,0.278944260741007)(0.257672991071428,0.277626894093959)(0.2578125,0.276308485606766)(0.257952008928571,0.274989037573898)(0.258091517857143,0.273668552291829)(0.258231026785714,0.272347032059073)(0.258370535714286,0.271024479176138)(0.258510044642857,0.269700895945555)(0.258649553571428,0.268376284671863)(0.2587890625,0.267050647661586)(0.258928571428571,0.265723987223263)(0.259068080357143,0.264396305667407)(0.259207589285714,0.263067605306535)(0.259347098214286,0.261737888455132)(0.259486607142857,0.260407157429672)(0.259626116071428,0.259075414548604)(0.259765625,0.257742662132333)(0.259905133928571,0.256408902503245)(0.260044642857143,0.255074137985673)(0.260184151785714,0.253738370905917)(0.260323660714286,0.252401603592217)(0.260463169642857,0.251063838374768)(0.260602678571428,0.24972507758571)(0.2607421875,0.248385323559107)(0.260881696428571,0.247044578630973)(0.261021205357143,0.245702845139233)(0.261160714285714,0.244360125423755)(0.261300223214286,0.243016421826307)(0.261439732142857,0.241671736690586)(0.261579241071428,0.240326072362199)(0.26171875,0.238979431188647)(0.261858258928571,0.237631815519347)(0.261997767857143,0.236283227705599)(0.262137276785714,0.234933670100608)(0.262276785714286,0.233583145059452)(0.262416294642857,0.232231654939105)(0.262555803571428,0.230879202098418)(0.2626953125,0.229525788898104)(0.262834821428571,0.22817141770076)(0.262974330357143,0.226816090870834)(0.263113839285714,0.225459810774648)(0.263253348214286,0.224102579780363)(0.263392857142857,0.222744400258005)(0.263532366071428,0.221385274579444)(0.263671875,0.22002520511838)(0.263811383928571,0.218664194250366)(0.263950892857143,0.217302244352772)(0.264090401785714,0.215939357804809)(0.264229910714286,0.214575536987499)(0.264369419642857,0.21321078428369)(0.264508928571428,0.211845102078048)(0.2646484375,0.210478492757032)(0.264787946428571,0.209110958708925)(0.264927455357143,0.207742502323791)(0.265066964285714,0.206373125993507)(0.265206473214286,0.205002832111723)(0.265345982142857,0.203631623073887)(0.265485491071428,0.20225950127723)(0.265625,0.200886469120746)(0.265764508928571,0.199512529005216)(0.265904017857143,0.198137683333174)(0.266043526785714,0.196761934508931)(0.266183035714286,0.19538528493854)(0.266322544642857,0.19400773702982)(0.266462053571428,0.192629293192338)(0.2666015625,0.191249955837392)(0.266741071428571,0.189869727378036)(0.266880580357143,0.188488610229042)(0.267020089285714,0.187106606806927)(0.267159598214286,0.185723719529918)(0.267299107142857,0.184339950817973)(0.267438616071428,0.182955303092767)(0.267578125,0.181569778777672)(0.267717633928571,0.180183380297785)(0.267857142857143,0.178796110079884)(0.267996651785714,0.177407970552464)(0.268136160714286,0.176018964145694)(0.268275669642857,0.174629093291441)(0.268415178571428,0.173238360423256)(0.2685546875,0.171846767976356)(0.268694196428571,0.170454318387646)(0.268833705357143,0.169061014095683)(0.268973214285714,0.167666857540705)(0.269112723214286,0.166271851164591)(0.269252232142857,0.164875997410888)(0.269391741071428,0.16347929872479)(0.26953125,0.162081757553124)(0.269670758928571,0.160683376344375)(0.269810267857143,0.159284157548643)(0.269949776785714,0.157884103617678)(0.270089285714286,0.156483217004837)(0.270228794642857,0.15508150016511)(0.270368303571428,0.153678955555104)(0.2705078125,0.152275585633023)(0.270647321428571,0.150871392858697)(0.270786830357143,0.149466379693536)(0.270926339285714,0.148060548600568)(0.271065848214286,0.146653902044393)(0.271205357142857,0.145246442491213)(0.271344866071428,0.143838172408811)(0.271484375,0.142429094266534)(0.271623883928571,0.141019210535322)(0.271763392857143,0.139608523687662)(0.271902901785714,0.138197036197623)(0.272042410714286,0.136784750540815)(0.272181919642857,0.135371669194415)(0.272321428571428,0.133957794637148)(0.2724609375,0.132543129349269)(0.272600446428571,0.13112767581259)(0.272739955357143,0.12971143651044)(0.272879464285714,0.128294413927695)(0.273018973214286,0.126876610550738)(0.273158482142857,0.125458028867483)(0.273297991071428,0.124038671367362)(0.2734375,0.1226185405413)(0.273577008928571,0.121197638881748)(0.273716517857143,0.119775968882638)(0.273856026785714,0.118353533039415)(0.273995535714286,0.116930333848997)(0.274135044642857,0.115506373809802)(0.274274553571428,0.114081655421728)(0.2744140625,0.112656181186134)(0.274553571428571,0.111229953605871)(0.274693080357143,0.109802975185237)(0.274832589285714,0.108375248430009)(0.274972098214286,0.106946775847401)(0.275111607142857,0.105517559946095)(0.275251116071428,0.104087603236219)(0.275390625,0.102656908229328)(0.275530133928571,0.101225477438432)(0.275669642857143,0.0997933133779568)(0.275809151785714,0.0983604185637716)(0.275948660714286,0.0969267955131505)(0.276088169642857,0.0954924467447985)(0.276227678571428,0.094057374778833)(0.2763671875,0.0926215821367652)(0.276506696428571,0.0911850713415274)(0.276646205357143,0.0897478449174312)(0.276785714285714,0.0883099053901992)(0.276925223214286,0.0868712552869237)(0.277064732142857,0.0854318971360938)(0.277204241071428,0.0839918334675765)(0.27734375,0.0825510668125984)(0.277483258928571,0.0811095997037722)(0.277622767857143,0.0796674346750561)(0.277762276785714,0.0782245742617846)(0.277901785714286,0.0767810210006273)(0.278041294642857,0.0753367774296162)(0.278180803571428,0.0738918460881268)(0.2783203125,0.0724462295168591)(0.278459821428571,0.0709999302578657)(0.278599330357143,0.069552950854509)(0.278738839285714,0.0681052938514937)(0.278878348214286,0.066656961794825)(0.279017857142857,0.0652079572318356)(0.279157366071428,0.0637582827111673)(0.279296875,0.0623079407827517)(0.279436383928571,0.060856933997838)(0.279575892857143,0.0594052649089509)(0.279715401785714,0.0579529360699225)(0.279854910714286,0.0564999500358506)(0.279994419642857,0.0550463093631259)(0.280133928571428,0.0535920166094133)(0.2802734375,0.0521370743336327)(0.280412946428571,0.0506814850959869)(0.280552455357143,0.0492252514579192)(0.280691964285714,0.0477683759821458)(0.280831473214286,0.0463108612326132)(0.280970982142857,0.0448527097745265)(0.281110491071428,0.0433939241743296)(0.28125,0.0419345069996869)(0.281389508928571,0.0404744608195104)(0.281529017857143,0.039013788203918)(0.281668526785714,0.0375524917242652)(0.281808035714286,0.0360905739531032)(0.281947544642857,0.0346280374642063)(0.282087053571428,0.0331648848325531)(0.2822265625,0.0317011186343072)(0.282366071428571,0.0302367414468453)(0.282505580357143,0.0287717558487144)(0.282645089285714,0.0273061644196644)(0.282784598214286,0.0258399697406055)(0.282924107142857,0.0243731743936361)(0.283063616071428,0.0229057809620237)(0.283203125,0.0214377920301858)(0.283342633928571,0.0199692101837174)(0.283482142857143,0.0185000380093492)(0.283621651785714,0.0170302780949791)(0.283761160714286,0.0155599330296304)(0.283900669642857,0.0140890054034793)(0.284040178571428,0.0126174978078359)(0.2841796875,0.0111454128351249)(0.284319196428571,0.00967275307891341)(0.284458705357143,0.00819952113386868)(0.284598214285714,0.00672571959579027)(0.284737723214286,0.00525135106156743)(0.284877232142857,0.00377641812920726)(0.285016741071428,0.00230092339781512)(0.28515625,0.000824869467575817)(0.285295758928571,-0.000651741060218758)(0.285435267857143,-0.00212890558321222)(0.285574776785714,-0.00360662149795105)(0.285714285714286,-0.00508488619992747)(0.285853794642857,-0.00656369708355153)(0.285993303571428,-0.00804305154217022)(0.2861328125,-0.0095229469680867)(0.286272321428571,-0.0110033807525324)(0.286411830357143,-0.0124843502857099)(0.286551339285714,-0.0139658529567601)(0.286690848214286,-0.0154478861538051)(0.286830357142857,-0.0169304472639205)(0.286969866071428,-0.0184135336731542)(0.287109375,-0.0198971427665461)(0.287248883928571,-0.0213812719280998)(0.287388392857143,-0.0228659185408255)(0.287527901785714,-0.0243510799867077)(0.287667410714286,-0.0258367536467474)(0.287806919642857,-0.0273229369009348)(0.287946428571428,-0.0288096271282684)(0.2880859375,-0.0302968217067738)(0.288225446428571,-0.0317845180134763)(0.288364955357143,-0.0332727134244437)(0.288504464285714,-0.0347614053147532)(0.288643973214286,-0.0362505910585349)(0.288783482142857,-0.0377402680289435)(0.288922991071428,-0.0392304335981775)(0.2890625,-0.0407210851374987)(0.289202008928571,-0.0422122200172043)(0.289341517857143,-0.0437038356066695)(0.289481026785714,-0.0451959292743151)(0.289620535714286,-0.0466884983876503)(0.289760044642857,-0.0481815403132448)(0.289899553571428,-0.0496750524167481)(0.2900390625,-0.0511690320629086)(0.290178571428571,-0.0526634766155462)(0.290318080357143,-0.0541583834375945)(0.290457589285714,-0.0556537498910686)(0.290597098214286,-0.0571495733371082)(0.290736607142857,-0.0586458511359489)(0.290876116071428,-0.0601425806469424)(0.291015625,-0.0616397592285753)(0.291155133928571,-0.0631373842384412)(0.291294642857143,-0.064635453033284)(0.291434151785714,-0.0661339629689645)(0.291573660714286,-0.0676329114005043)(0.291713169642857,-0.0691322956820571)(0.291852678571428,-0.0706321131669283)(0.2919921875,-0.0721323612075947)(0.292131696428571,-0.0736330371556758)(0.292271205357143,-0.0751341383619774)(0.292410714285714,-0.0766356621764587)(0.292550223214286,-0.0781376059482752)(0.292689732142857,-0.0796399670257509)(0.292829241071428,-0.0811427427563977)(0.29296875,-0.0826459304869347)(0.293108258928571,-0.0841495275632599)(0.293247767857143,-0.0856535313304939)(0.293387276785714,-0.0871579391329466)(0.293526785714286,-0.0886627483141607)(0.293666294642857,-0.0901679562168831)(0.293805803571428,-0.0916735601830851)(0.2939453125,-0.0931795575539813)(0.294084821428571,-0.0946859456700013)(0.294224330357143,-0.0961927218708336)(0.294363839285714,-0.0976998834953922)(0.294503348214286,-0.0992074278818601)(0.294642857142857,-0.100715352367661)(0.294782366071428,-0.102223654289478)(0.294921875,-0.103732330983276)(0.295061383928571,-0.105241379784269)(0.295200892857143,-0.106750798026968)(0.295340401785714,-0.108260583045144)(0.295479910714286,-0.109770732171875)(0.295619419642857,-0.111281242739514)(0.295758928571428,-0.112792112079712)(0.2958984375,-0.114303337523436)(0.296037946428571,-0.11581491640094)(0.296177455357143,-0.117326846041811)(0.296316964285714,-0.118839123774932)(0.296456473214286,-0.120351746928528)(0.296595982142857,-0.12186471283014)(0.296735491071428,-0.123378018806637)(0.296875,-0.124891662184246)(0.297014508928571,-0.126405640288515)(0.297154017857143,-0.127919950444359)(0.297293526785714,-0.12943458997603)(0.297433035714286,-0.130949556207156)(0.297572544642857,-0.132464846460717)(0.297712053571428,-0.133980458059059)(0.2978515625,-0.135496388323921)(0.297991071428571,-0.1370126345764)(0.298130580357143,-0.138529194136998)(0.298270089285714,-0.140046064325587)(0.298409598214286,-0.141563242461455)(0.298549107142857,-0.143080725863276)(0.298688616071428,-0.14459851184913)(0.298828125,-0.146116597736521)(0.298967633928571,-0.147634980842351)(0.299107142857143,-0.149153658482964)(0.299246651785714,-0.150672627974109)(0.299386160714286,-0.15219188663099)(0.299525669642857,-0.153711431768229)(0.299665178571428,-0.155231260699895)(0.2998046875,-0.156751370739515)(0.299944196428571,-0.158271759200053)(0.300083705357143,-0.159792423393948)(0.300223214285714,-0.161313360633085)(0.300362723214286,-0.162834568228836)(0.300502232142857,-0.164356043492031)(0.300641741071428,-0.165877783732981)(0.30078125,-0.167399786261495)(0.300920758928571,-0.168922048386853)(0.301060267857143,-0.170444567417847)(0.301199776785714,-0.171967340662753)(0.301339285714286,-0.173490365429367)(0.301478794642857,-0.175013639024985)(0.301618303571428,-0.176537158756418)(0.3017578125,-0.17806092193001)(0.301897321428571,-0.179584925851614)(0.302036830357143,-0.181109167826634)(0.302176339285714,-0.182633645159991)(0.302315848214286,-0.184158355156166)(0.302455357142857,-0.185683295119174)(0.302594866071428,-0.187208462352581)(0.302734375,-0.188733854159529)(0.302873883928571,-0.190259467842698)(0.303013392857143,-0.191785300704359)(0.303152901785714,-0.193311350046337)(0.303292410714286,-0.194837613170054)(0.303431919642857,-0.196364087376502)(0.303571428571428,-0.197890769966262)(0.3037109375,-0.199417658239523)(0.303850446428571,-0.200944749496057)(0.303989955357143,-0.202472041035257)(0.304129464285714,-0.20399953015611)(0.304268973214286,-0.205527214157234)(0.304408482142857,-0.207055090336857)(0.304547991071428,-0.208583155992831)(0.3046875,-0.210111408422654)(0.304827008928571,-0.211639844923441)(0.304966517857143,-0.213168462791969)(0.305106026785714,-0.214697259324641)(0.305245535714286,-0.216226231817534)(0.305385044642857,-0.217755377566365)(0.305524553571428,-0.219284693866517)(0.3056640625,-0.220814178013053)(0.305803571428571,-0.222343827300691)(0.305943080357143,-0.223873639023847)(0.306082589285714,-0.225403610476602)(0.306222098214286,-0.226933738952744)(0.306361607142857,-0.228464021745742)(0.306501116071428,-0.229994456148766)(0.306640625,-0.231525039454701)(0.306780133928571,-0.233055768956128)(0.306919642857143,-0.23458664194536)(0.307059151785714,-0.236117655714411)(0.307198660714286,-0.237648807555042)(0.307338169642857,-0.239180094758731)(0.307477678571428,-0.240711514616689)(0.3076171875,-0.242243064419889)(0.307756696428571,-0.243774741459026)(0.307896205357143,-0.245306543024571)(0.308035714285714,-0.24683846640673)(0.308175223214286,-0.248370508895495)(0.308314732142857,-0.249902667780609)(0.308454241071428,-0.251434940351588)(0.30859375,-0.252967323897742)(0.308733258928571,-0.254499815708148)(0.308872767857143,-0.256032413071688)(0.309012276785714,-0.257565113277021)(0.309151785714286,-0.259097913612626)(0.309291294642857,-0.26063081136677)(0.309430803571428,-0.262163803827534)(0.3095703125,-0.263696888282825)(0.309709821428571,-0.265230062020355)(0.309849330357143,-0.26676332232768)(0.309988839285714,-0.268296666492169)(0.310128348214286,-0.269830091801047)(0.310267857142857,-0.271363595541364)(0.310407366071428,-0.272897175000023)(0.310546875,-0.274430827463788)(0.310686383928571,-0.275964550219265)(0.310825892857143,-0.277498340552942)(0.310965401785714,-0.279032195751157)(0.311104910714286,-0.280566113100138)(0.311244419642857,-0.282100089885981)(0.311383928571428,-0.283634123394665)(0.3115234375,-0.285168210912072)(0.311662946428571,-0.286702349723962)(0.311802455357143,-0.288236537116011)(0.311941964285714,-0.289770770373785)(0.312081473214286,-0.291305046782778)(0.312220982142857,-0.292839363628385)(0.312360491071428,-0.294373718195924)(0.3125,-0.295908107770652)(0.312639508928571,-0.297442529637741)(0.312779017857143,-0.298976981082317)(0.312918526785714,-0.30051145938943)(0.313058035714286,-0.302045961844099)(0.313197544642857,-0.303580485731278)(0.313337053571428,-0.305115028335883)(0.3134765625,-0.306649586942805)(0.313616071428571,-0.308184158836885)(0.313755580357143,-0.30971874130296)(0.313895089285714,-0.311253331625823)(0.314034598214286,-0.312787927090276)(0.314174107142857,-0.314322524981091)(0.314313616071428,-0.31585712258304)(0.314453125,-0.317391717180908)(0.314592633928571,-0.318926306059465)(0.314732142857143,-0.320460886503516)(0.314871651785714,-0.321995455797857)(0.315011160714286,-0.32353001122733)(0.315150669642857,-0.325064550076788)(0.315290178571428,-0.326599069631115)(0.3154296875,-0.328133567175249)(0.315569196428571,-0.32966803999415)(0.315708705357143,-0.331202485372845)(0.315848214285714,-0.332736900596398)(0.315987723214286,-0.334271282949948)(0.316127232142857,-0.335805629718685)(0.316266741071428,-0.337339938187867)(0.31640625,-0.338874205642842)(0.316545758928571,-0.34040842936902)(0.316685267857143,-0.341942606651912)(0.316824776785714,-0.343476734777102)(0.316964285714286,-0.34501081103029)(0.317103794642857,-0.34654483269726)(0.317243303571428,-0.348078797063904)(0.3173828125,-0.34961270141624)(0.317522321428571,-0.351146543040383)(0.317661830357143,-0.35268031922259)(0.317801339285714,-0.354214027249224)(0.317940848214286,-0.355747664406802)(0.318080357142857,-0.357281227981964)(0.318219866071428,-0.358814715261494)(0.318359375,-0.360348123532338)(0.318498883928571,-0.361881450081575)(0.318638392857143,-0.363414692196465)(0.318777901785714,-0.364947847164413)(0.318917410714286,-0.36648091227301)(0.319056919642857,-0.36801388481001)(0.319196428571428,-0.369546762063347)(0.3193359375,-0.371079541321153)(0.319475446428571,-0.372612219871735)(0.319614955357143,-0.374144795003611)(0.319754464285714,-0.375677264005485)(0.319893973214286,-0.377209624166283)(0.320033482142857,-0.378741872775132)(0.320172991071428,-0.380274007121373)(0.3203125,-0.381806024494586)(0.320452008928571,-0.383337922184559)(0.320591517857143,-0.384869697481331)(0.320731026785714,-0.386401347675162)(0.320870535714286,-0.387932870056572)(0.321010044642857,-0.389464261916318)(0.321149553571428,-0.39099552054541)(0.3212890625,-0.392526643235131)(0.321428571428571,-0.394057627277011)(0.321568080357143,-0.395588469962867)(0.321707589285714,-0.397119168584773)(0.321847098214286,-0.398649720435102)(0.321986607142857,-0.400180122806501)(0.322126116071428,-0.401710372991904)(0.322265625,-0.403240468284558)(0.322405133928571,-0.40477040597799)(0.322544642857143,-0.406300183366057)(0.322684151785714,-0.407829797742905)(0.322823660714286,-0.409359246403018)(0.322963169642857,-0.410888526641187)(0.323102678571428,-0.412417635752534)(0.3232421875,-0.413946571032524)(0.323381696428571,-0.415475329776944)(0.323521205357143,-0.417003909281943)(0.323660714285714,-0.418532306844)(0.323800223214286,-0.420060519759967)(0.323939732142857,-0.42158854532704)(0.324079241071428,-0.423116380842783)(0.32421875,-0.42464402360514)(0.324358258928571,-0.426171470912413)(0.324497767857143,-0.427698720063304)(0.324637276785714,-0.429225768356881)(0.324776785714286,-0.430752613092622)(0.324916294642857,-0.432279251570386)(0.325055803571428,-0.433805681090436)(0.3251953125,-0.435331898953453)(0.325334821428571,-0.436857902460513)(0.325474330357143,-0.438383688913127)(0.325613839285714,-0.43990925561321)(0.325753348214286,-0.441434599863124)(0.325892857142857,-0.442959718965648)(0.326032366071428,-0.444484610224002)(0.326171875,-0.446009270941861)(0.326311383928571,-0.447533698423331)(0.326450892857143,-0.449057889972992)(0.326590401785714,-0.450581842895862)(0.326729910714286,-0.452105554497443)(0.326869419642857,-0.453629022083694)(0.327008928571428,-0.455152242961046)(0.3271484375,-0.456675214436429)(0.327287946428571,-0.458197933817234)(0.327427455357143,-0.459720398411365)(0.327566964285714,-0.461242605527203)(0.327706473214286,-0.462764552473648)(0.327845982142857,-0.464286236560091)(0.327985491071428,-0.465807655096437)(0.328125,-0.467328805393121)(0.328264508928571,-0.468849684761081)(0.328404017857143,-0.470370290511802)(0.328543526785714,-0.471890619957281)(0.328683035714286,-0.473410670410073)(0.328822544642857,-0.474930439183262)(0.328962053571428,-0.476449923590478)(0.3291015625,-0.477969120945923)(0.329241071428571,-0.479488028564334)(0.329380580357143,-0.481006643761033)(0.329520089285714,-0.482524963851894)(0.329659598214286,-0.48404298615338)(0.329799107142857,-0.485560707982524)(0.329938616071428,-0.48707812665694)(0.330078125,-0.488595239494849)(0.330217633928571,-0.490112043815046)(0.330357142857143,-0.491628536936947)(0.330496651785714,-0.493144716180555)(0.330636160714286,-0.494660578866501)(0.330775669642857,-0.49617612231602)(0.330915178571428,-0.497691343850966)(0.3310546875,-0.499206240793836)(0.331194196428571,-0.500720810467739)(0.331333705357143,-0.50223505019644)(0.331473214285714,-0.503748957304326)(0.331612723214286,-0.505262529116452)(0.331752232142857,-0.50677576295851)(0.331891741071428,-0.508288656156851)(0.33203125,-0.509801206038503)(0.332170758928571,-0.511313409931142)(0.332310267857143,-0.512825265163139)(0.332449776785714,-0.51433676906352)(0.332589285714286,-0.515847918962019)(0.332728794642857,-0.517358712189042)(0.332868303571428,-0.518869146075688)(0.3330078125,-0.520379217953774)(0.333147321428571,-0.521888925155797)(0.333286830357143,-0.523398265014986)(0.333426339285714,-0.524907234865267)(0.333565848214286,-0.526415832041302)(0.333705357142857,-0.527924053878466)(0.333844866071428,-0.529431897712865)(0.333984375,-0.530939360881354)(0.334123883928571,-0.532446440721508)(0.334263392857143,-0.533953134571672)(0.334402901785714,-0.535459439770918)(0.334542410714286,-0.536965353659098)(0.334681919642857,-0.538470873576807)(0.334821428571428,-0.539975996865409)(0.3349609375,-0.541480720867055)(0.335100446428571,-0.542985042924653)(0.335239955357143,-0.544488960381912)(0.335379464285714,-0.545992470583311)(0.335518973214286,-0.547495570874138)(0.335658482142857,-0.548998258600467)(0.335797991071428,-0.550500531109174)(0.3359375,-0.552002385747958)(0.336077008928571,-0.55350381986531)(0.336216517857143,-0.555004830810562)(0.336356026785714,-0.556505415933848)(0.336495535714286,-0.558005572586151)(0.336635044642857,-0.559505298119273)(0.336774553571428,-0.561004589885854)(0.3369140625,-0.562503445239396)(0.337053571428571,-0.564001861534228)(0.337193080357143,-0.565499836125554)(0.337332589285714,-0.566997366369419)(0.337472098214286,-0.568494449622749)(0.337611607142857,-0.56999108324333)(0.337751116071428,-0.571487264589821)(0.337890625,-0.572982991021774)(0.338030133928571,-0.57447825989961)(0.338169642857143,-0.575973068584659)(0.338309151785714,-0.577467414439125)(0.338448660714286,-0.578961294826136)(0.338588169642857,-0.580454707109709)(0.338727678571428,-0.581947648654772)(0.3388671875,-0.583440116827185)(0.339006696428571,-0.584932108993708)(0.339146205357143,-0.58642362252205)(0.339285714285714,-0.587914654780828)(0.339425223214286,-0.589405203139618)(0.339564732142857,-0.590895264968923)(0.339704241071428,-0.592384837640192)(0.33984375,-0.593873918525841)(0.339983258928571,-0.595362504999224)(0.340122767857143,-0.596850594434677)(0.340262276785714,-0.598338184207482)(0.340401785714286,-0.599825271693915)(0.340541294642857,-0.601311854271214)(0.340680803571428,-0.602797929317601)(0.3408203125,-0.604283494212298)(0.340959821428571,-0.605768546335503)(0.341099330357143,-0.607253083068429)(0.341238839285714,-0.608737101793275)(0.341378348214286,-0.610220599893266)(0.341517857142857,-0.611703574752627)(0.341657366071428,-0.613186023756599)(0.341796875,-0.614667944291464)(0.341936383928571,-0.616149333744512)(0.342075892857143,-0.617630189504084)(0.342215401785714,-0.619110508959544)(0.342354910714286,-0.620590289501317)(0.342494419642857,-0.622069528520861)(0.342633928571428,-0.623548223410691)(0.3427734375,-0.625026371564395)(0.342912946428571,-0.626503970376603)(0.343052455357143,-0.627981017243036)(0.343191964285714,-0.62945750956047)(0.343331473214286,-0.630933444726778)(0.343470982142857,-0.632408820140904)(0.343610491071428,-0.633883633202882)(0.34375,-0.635357881313853)(0.343889508928571,-0.63683156187604)(0.344029017857143,-0.63830467229279)(0.344168526785714,-0.639777209968539)(0.344308035714286,-0.641249172308858)(0.344447544642857,-0.642720556720422)(0.344587053571428,-0.644191360611033)(0.3447265625,-0.645661581389636)(0.344866071428571,-0.647131216466292)(0.345005580357143,-0.64860026325222)(0.345145089285714,-0.650068719159767)(0.345284598214286,-0.651536581602446)(0.345424107142857,-0.653003847994915)(0.345563616071428,-0.654470515752987)(0.345703125,-0.655936582293659)(0.345842633928571,-0.657402045035077)(0.345982142857143,-0.658866901396581)(0.346121651785714,-0.660331148798672)(0.346261160714286,-0.661794784663056)(0.346400669642857,-0.663257806412615)(0.346540178571428,-0.664720211471426)(0.3466796875,-0.66618199726478)(0.346819196428571,-0.667643161219155)(0.346958705357143,-0.669103700762258)(0.347098214285714,-0.670563613322992)(0.347237723214286,-0.672022896331499)(0.347377232142857,-0.673481547219134)(0.347516741071428,-0.67493956341848)(0.34765625,-0.676396942363369)(0.347795758928571,-0.677853681488858)(0.347935267857143,-0.679309778231265)(0.348074776785714,-0.680765230028139)(0.348214285714286,-0.682220034318306)(0.348353794642857,-0.683674188541835)(0.348493303571428,-0.685127690140062)(0.3486328125,-0.686580536555608)(0.348772321428571,-0.688032725232348)(0.348911830357143,-0.689484253615458)(0.349051339285714,-0.690935119151378)(0.349190848214286,-0.69238531928786)(0.349330357142857,-0.693834851473935)(0.349469866071428,-0.695283713159935)(0.349609375,-0.696731901797508)(0.349748883928571,-0.698179414839599)(0.349888392857143,-0.69962624974048)(0.350027901785714,-0.701072403955729)(0.350167410714286,-0.702517874942266)(0.350306919642857,-0.703962660158326)(0.350446428571428,-0.70540675706348)(0.3505859375,-0.706850163118651)(0.350725446428571,-0.70829287578609)(0.350864955357143,-0.709734892529414)(0.351004464285714,-0.711176210813577)(0.351143973214286,-0.712616828104911)(0.351283482142857,-0.714056741871099)(0.351422991071428,-0.715495949581191)(0.3515625,-0.716934448705628)(0.351702008928571,-0.718372236716209)(0.351841517857143,-0.719809311086137)(0.351981026785714,-0.721245669289986)(0.352120535714286,-0.722681308803741)(0.352260044642857,-0.724116227104772)(0.352399553571428,-0.725550421671853)(0.3525390625,-0.726983889985181)(0.352678571428571,-0.728416629526348)(0.352818080357143,-0.729848637778381)(0.352957589285714,-0.731279912225714)(0.353097098214286,-0.732710450354226)(0.353236607142857,-0.734140249651218)(0.353376116071428,-0.735569307605425)(0.353515625,-0.736997621707042)(0.353655133928571,-0.738425189447692)(0.353794642857143,-0.739852008320468)(0.353934151785714,-0.741278075819904)(0.354073660714286,-0.742703389442014)(0.354213169642857,-0.744127946684266)(0.354352678571428,-0.745551745045598)(0.3544921875,-0.746974782026441)(0.354631696428571,-0.74839705512869)(0.354771205357143,-0.749818561855742)(0.354910714285714,-0.75123929971247)(0.355050223214286,-0.75265926620526)(0.355189732142857,-0.754078458841989)(0.355329241071428,-0.755496875132034)(0.35546875,-0.756914512586303)(0.355608258928571,-0.758331368717198)(0.355747767857143,-0.75974744103866)(0.355887276785714,-0.761162727066139)(0.356026785714286,-0.762577224316632)(0.356166294642857,-0.763990930308658)(0.356305803571428,-0.765403842562276)(0.3564453125,-0.766815958599104)(0.356584821428571,-0.768227275942291)(0.356724330357143,-0.769637792116559)(0.356863839285714,-0.77104750464817)(0.357003348214286,-0.772456411064969)(0.357142857142857,-0.773864508896357)(0.357282366071428,-0.775271795673309)(0.357421875,-0.77667826892839)(0.357561383928571,-0.778083926195731)(0.357700892857143,-0.779488765011071)(0.357840401785714,-0.780892782911721)(0.357979910714286,-0.782295977436611)(0.358119419642857,-0.783698346126258)(0.358258928571428,-0.785099886522786)(0.3583984375,-0.786500596169946)(0.358537946428571,-0.787900472613086)(0.358677455357143,-0.789299513399195)(0.358816964285714,-0.790697716076871)(0.358956473214286,-0.79209507819636)(0.359095982142857,-0.79349159730953)(0.359235491071428,-0.794887270969892)(0.359375,-0.796282096732613)(0.359514508928571,-0.797676072154494)(0.359654017857143,-0.799069194794009)(0.359793526785714,-0.800461462211274)(0.359933035714286,-0.801852871968087)(0.360072544642857,-0.803243421627901)(0.360212053571428,-0.804633108755843)(0.3603515625,-0.806021930918735)(0.360491071428571,-0.807409885685061)(0.360630580357143,-0.808796970625012)(0.360770089285714,-0.810183183310456)(0.360909598214286,-0.811568521314975)(0.361049107142857,-0.812952982213838)(0.361188616071428,-0.814336563584026)(0.361328125,-0.81571926300424)(0.361467633928571,-0.817101078054884)(0.361607142857143,-0.818482006318097)(0.361746651785714,-0.81986204537773)(0.361886160714286,-0.821241192819379)(0.362025669642857,-0.822619446230365)(0.362165178571428,-0.823996803199747)(0.3623046875,-0.825373261318342)(0.362444196428571,-0.8267488181787)(0.362583705357143,-0.828123471375143)(0.362723214285714,-0.829497218503732)(0.362862723214286,-0.83087005716231)(0.363002232142857,-0.832241984950477)(0.363141741071428,-0.833612999469603)(0.36328125,-0.834983098322849)(0.363420758928571,-0.836352279115143)(0.363560267857143,-0.837720539453214)(0.363699776785714,-0.839087876945567)(0.363839285714286,-0.840454289202521)(0.363978794642857,-0.841819773836182)(0.364118303571428,-0.843184328460459)(0.3642578125,-0.844547950691089)(0.364397321428571,-0.845910638145602)(0.364536830357143,-0.847272388443367)(0.364676339285714,-0.84863319920556)(0.364815848214286,-0.849993068055199)(0.364955357142857,-0.851351992617128)(0.365094866071428,-0.852709970518025)(0.365234375,-0.854066999386424)(0.365373883928571,-0.85542307685269)(0.365513392857143,-0.856778200549055)(0.365652901785714,-0.858132368109592)(0.365792410714286,-0.859485577170251)(0.365931919642857,-0.860837825368836)(0.366071428571428,-0.862189110345019)(0.3662109375,-0.86353942974036)(0.366350446428571,-0.864888781198283)(0.366489955357143,-0.866237162364112)(0.366629464285714,-0.86758457088504)(0.366768973214286,-0.868931004410175)(0.366908482142857,-0.870276460590505)(0.367047991071428,-0.871620937078923)(0.3671875,-0.87296443153024)(0.367327008928571,-0.874306941601163)(0.367466517857143,-0.87564846495033)(0.367606026785714,-0.876988999238284)(0.367745535714286,-0.878328542127509)(0.367885044642857,-0.879667091282406)(0.368024553571428,-0.881004644369311)(0.3681640625,-0.88234119905651)(0.368303571428571,-0.883676753014215)(0.368443080357143,-0.885011303914604)(0.368582589285714,-0.88634484943179)(0.368722098214286,-0.887677387241858)(0.368861607142857,-0.889008915022845)(0.369001116071428,-0.89033943045475)(0.369140625,-0.891668931219558)(0.369280133928571,-0.892997415001209)(0.369419642857143,-0.894324879485642)(0.369559151785714,-0.895651322360762)(0.369698660714286,-0.896976741316478)(0.369838169642857,-0.898301134044681)(0.369977678571428,-0.899624498239258)(0.3701171875,-0.900946831596111)(0.370256696428571,-0.902268131813132)(0.370396205357143,-0.903588396590239)(0.370535714285714,-0.904907623629351)(0.370675223214286,-0.90622581063442)(0.370814732142857,-0.907542955311412)(0.370954241071428,-0.908859055368322)(0.37109375,-0.91017410851519)(0.371233258928571,-0.911488112464076)(0.371372767857143,-0.912801064929099)(0.371512276785714,-0.91411296362641)(0.371651785714286,-0.915423806274224)(0.371791294642857,-0.916733590592802)(0.371930803571428,-0.918042314304463)(0.3720703125,-0.919349975133603)(0.372209821428571,-0.920656570806671)(0.372349330357143,-0.921962099052204)(0.372488839285714,-0.923266557600801)(0.372628348214286,-0.92456994418516)(0.372767857142857,-0.925872256540052)(0.372907366071428,-0.927173492402339)(0.373046875,-0.928473649510991)(0.373186383928571,-0.929772725607059)(0.373325892857143,-0.931070718433718)(0.373465401785714,-0.932367625736232)(0.373604910714286,-0.933663445261995)(0.373744419642857,-0.934958174760506)(0.373883928571428,-0.936251811983386)(0.3740234375,-0.937544354684392)(0.374162946428571,-0.938835800619398)(0.374302455357143,-0.940126147546426)(0.374441964285714,-0.941415393225624)(0.374581473214286,-0.942703535419295)(0.374720982142857,-0.943990571891882)(0.374860491071428,-0.945276500409978)(0.375,-0.946561318742344)(0.375139508928571,-0.947845024659886)(0.375279017857143,-0.949127615935693)(0.375418526785714,-0.950409090345004)(0.375558035714286,-0.95168944566525)(0.375697544642857,-0.952968679676028)(0.375837053571428,-0.954246790159117)(0.3759765625,-0.955523774898494)(0.376116071428571,-0.956799631680312)(0.376255580357143,-0.958074358292935)(0.376395089285714,-0.95934795252691)(0.376534598214286,-0.960620412175006)(0.376674107142857,-0.961891735032185)(0.376813616071428,-0.963161918895624)(0.376953125,-0.964430961564728)(0.377092633928571,-0.965698860841106)(0.377232142857143,-0.966965614528611)(0.377371651785714,-0.968231220433308)(0.377511160714286,-0.969495676363511)(0.377650669642857,-0.970758980129761)(0.377790178571428,-0.972021129544843)(0.3779296875,-0.973282122423798)(0.378069196428571,-0.974541956583903)(0.378208705357143,-0.975800629844707)(0.378348214285714,-0.977058140028002)(0.378487723214286,-0.978314484957858)(0.378627232142857,-0.979569662460605)(0.378766741071428,-0.98082367036484)(0.37890625,-0.982076506501451)(0.379045758928571,-0.983328168703592)(0.379185267857143,-0.984578654806716)(0.379324776785714,-0.985827962648548)(0.379464285714286,-0.987076090069124)(0.379603794642857,-0.988323034910765)(0.379743303571428,-0.989568795018093)(0.3798828125,-0.990813368238049)(0.380022321428571,-0.992056752419868)(0.380161830357143,-0.993298945415112)(0.380301339285714,-0.994539945077651)(0.380440848214286,-0.995779749263688)(0.380580357142857,-0.997018355831744)(0.380719866071428,-0.998255762642671)(0.380859375,-0.999491967559666)(0.380998883928571,-1.00072696844825)(0.381138392857143,-1.00196076317631)(0.381277901785714,-1.00319334961404)(0.381417410714286,-1.00442472563404)(0.381556919642857,-1.00565488911123)(0.381696428571428,-1.00688383792288)(0.3818359375,-1.00811156994867)(0.381975446428571,-1.0093380830706)(0.382114955357143,-1.01056337517306)(0.382254464285714,-1.01178744414283)(0.382393973214286,-1.01301028786906)(0.382533482142857,-1.01423190424328)(0.382672991071428,-1.01545229115941)(0.3828125,-1.01667144651377)(0.382952008928571,-1.01788936820506)(0.383091517857143,-1.01910605413441)(0.383231026785714,-1.02032150220533)(0.383370535714286,-1.02153571032375)(0.383510044642857,-1.02274867639802)(0.383649553571428,-1.02396039833888)(0.3837890625,-1.02517087405952)(0.383928571428571,-1.02638010147554)(0.384068080357143,-1.02758807850499)(0.384207589285714,-1.02879480306832)(0.384347098214286,-1.03000027308845)(0.384486607142857,-1.03120448649072)(0.384626116071428,-1.03240744120293)(0.384765625,-1.03360913515531)(0.384905133928571,-1.03480956628057)(0.385044642857143,-1.03600873251387)(0.385184151785714,-1.03720663179281)(0.385323660714286,-1.03840326205748)(0.385463169642857,-1.03959862125043)(0.385602678571428,-1.0407927073167)(0.3857421875,-1.04198551820377)(0.385881696428571,-1.04317705186165)(0.386021205357143,-1.0443673062428)(0.386160714285714,-1.04555627930217)(0.386300223214286,-1.04674396899724)(0.386439732142857,-1.04793037328796)(0.386579241071428,-1.04911549013676)(0.38671875,-1.05029931750862)(0.386858258928571,-1.05148185337101)(0.386997767857143,-1.05266309569392)(0.387137276785714,-1.05384304244983)(0.387276785714286,-1.05502169161379)(0.387416294642857,-1.05619904116333)(0.387555803571428,-1.05737508907852)(0.3876953125,-1.058549833342)(0.387834821428571,-1.0597232719389)(0.387974330357143,-1.06089540285691)(0.388113839285714,-1.06206622408626)(0.388253348214286,-1.06323573361975)(0.388392857142857,-1.06440392945271)(0.388532366071428,-1.06557080958303)(0.388671875,-1.06673637201117)(0.388811383928571,-1.06790061474014)(0.388950892857143,-1.06906353577553)(0.389090401785714,-1.0702251331255)(0.389229910714286,-1.0713854048008)(0.389369419642857,-1.07254434881473)(0.389508928571428,-1.07370196318319)(0.3896484375,-1.07485824592468)(0.389787946428571,-1.07601319506026)(0.389927455357143,-1.07716680861363)(0.390066964285714,-1.07831908461105)(0.390206473214286,-1.0794700210814)(0.390345982142857,-1.08061961605617)(0.390485491071428,-1.08176786756945)(0.390625,-1.08291477365795)(0.390764508928571,-1.084060332361)(0.390904017857143,-1.08520454172056)(0.391043526785714,-1.0863473997812)(0.391183035714286,-1.08748890459013)(0.391322544642857,-1.08862905419719)(0.391462053571428,-1.08976784665485)(0.3916015625,-1.09090528001825)(0.391741071428571,-1.09204135234515)(0.391880580357143,-1.09317606169596)(0.392020089285714,-1.09430940613374)(0.392159598214286,-1.09544138372423)(0.392299107142857,-1.09657199253581)(0.392438616071428,-1.09770123063952)(0.392578125,-1.09882909610909)(0.392717633928571,-1.09995558702089)(0.392857142857143,-1.101080701454)(0.392996651785714,-1.10220443749014)(0.393136160714286,-1.10332679321376)(0.393275669642857,-1.10444776671196)(0.393415178571428,-1.10556735607454)(0.3935546875,-1.10668555939399)(0.393694196428571,-1.10780237476551)(0.393833705357143,-1.10891780028699)(0.393973214285714,-1.11003183405904)(0.394112723214286,-1.11114447418495)(0.394252232142857,-1.11225571877076)(0.394391741071428,-1.11336556592519)(0.39453125,-1.1144740137597)(0.394670758928571,-1.11558106038848)(0.394810267857143,-1.11668670392842)(0.394949776785714,-1.11779094249917)(0.395089285714286,-1.1188937742231)(0.395228794642857,-1.11999519722531)(0.395368303571428,-1.12109520963367)(0.3955078125,-1.12219380957876)(0.395647321428571,-1.12329099519394)(0.395786830357143,-1.1243867646153)(0.395926339285714,-1.12548111598169)(0.396065848214286,-1.12657404743474)(0.396205357142857,-1.12766555711882)(0.396344866071428,-1.12875564318108)(0.396484375,-1.12984430377143)(0.396623883928571,-1.13093153704257)(0.396763392857143,-1.13201734114996)(0.396902901785714,-1.13310171425185)(0.397042410714286,-1.13418465450929)(0.397181919642857,-1.13526616008609)(0.397321428571428,-1.13634622914887)(0.3974609375,-1.13742485986705)(0.397600446428571,-1.13850205041282)(0.397739955357143,-1.13957779896123)(0.397879464285714,-1.14065210369007)(0.398018973214286,-1.14172496277998)(0.398158482142857,-1.14279637441442)(0.398297991071428,-1.14386633677963)(0.3984375,-1.1449348480647)(0.398577008928571,-1.14600190646155)(0.398716517857143,-1.1470675101649)(0.398856026785714,-1.14813165737232)(0.398995535714286,-1.14919434628422)(0.399135044642857,-1.15025557510382)(0.399274553571428,-1.15131534203721)(0.3994140625,-1.15237364529333)(0.399553571428571,-1.15343048308393)(0.399693080357143,-1.15448585362366)(0.399832589285714,-1.15553975512998)(0.399972098214286,-1.15659218582325)(0.400111607142857,-1.15764314392667)(0.400251116071428,-1.1586926276663)(0.400390625,-1.15974063527109)(0.400530133928571,-1.16078716497285)(0.400669642857143,-1.16183221500627)(0.400809151785714,-1.16287578360892)(0.400948660714286,-1.16391786902125)(0.401088169642857,-1.1649584694866)(0.401227678571428,-1.1659975832512)(0.4013671875,-1.16703520856417)(0.401506696428571,-1.16807134367753)(0.401646205357143,-1.16910598684621)(0.401785714285714,-1.17013913632802)(0.401925223214286,-1.17117079038369)(0.402064732142857,-1.17220094727688)(0.402204241071428,-1.17322960527412)(0.40234375,-1.1742567626449)(0.402483258928571,-1.17528241766161)(0.402622767857143,-1.17630656859956)(0.402762276785714,-1.17732921373701)(0.402901785714286,-1.17835035135512)(0.403041294642857,-1.179369979738)(0.403180803571428,-1.1803880971727)(0.4033203125,-1.18140470194921)(0.403459821428571,-1.18241979236046)(0.403599330357143,-1.18343336670232)(0.403738839285714,-1.18444542327364)(0.403878348214286,-1.18545596037618)(0.404017857142857,-1.1864649763147)(0.404157366071428,-1.18747246939689)(0.404296875,-1.18847843793341)(0.404436383928571,-1.18948288023791)(0.404575892857143,-1.19048579462698)(0.404715401785714,-1.19148717942019)(0.404854910714286,-1.19248703294011)(0.404994419642857,-1.19348535351227)(0.405133928571428,-1.19448213946518)(0.4052734375,-1.19547738913036)(0.405412946428571,-1.19647110084228)(0.405552455357143,-1.19746327293844)(0.405691964285714,-1.19845390375933)(0.405831473214286,-1.19944299164843)(0.405970982142857,-1.20043053495222)(0.406110491071428,-1.2014165320202)(0.40625,-1.20240098120487)(0.406389508928571,-1.20338388086174)(0.406529017857143,-1.20436522934935)(0.406668526785714,-1.20534502502924)(0.406808035714286,-1.20632326626599)(0.406947544642857,-1.2072999514272)(0.407087053571428,-1.20827507888348)(0.4072265625,-1.20924864700851)(0.407366071428571,-1.21022065417896)(0.407505580357143,-1.21119109877458)(0.407645089285714,-1.21215997917813)(0.407784598214286,-1.21312729377542)(0.407924107142857,-1.21409304095533)(0.408063616071428,-1.21505721910977)(0.408203125,-1.2160198266337)(0.408342633928571,-1.21698086192514)(0.408482142857143,-1.21794032338519)(0.408621651785714,-1.21889820941799)(0.408761160714286,-1.21985451843076)(0.408900669642857,-1.22080924883378)(0.409040178571428,-1.2217623990404)(0.4091796875,-1.22271396746706)(0.409319196428571,-1.22366395253326)(0.409458705357143,-1.22461235266161)(0.409598214285714,-1.22555916627776)(0.409737723214286,-1.22650439181049)(0.409877232142857,-1.22744802769166)(0.410016741071428,-1.22839007235621)(0.41015625,-1.22933052424218)(0.410295758928571,-1.23026938179073)(0.410435267857143,-1.2312066434461)(0.410574776785714,-1.23214230765564)(0.410714285714286,-1.23307637286983)(0.410853794642857,-1.23400883754223)(0.410993303571428,-1.23493970012954)(0.4111328125,-1.23586895909156)(0.411272321428571,-1.23679661289123)(0.411411830357143,-1.2377226599946)(0.411551339285714,-1.23864709887086)(0.411690848214286,-1.23956992799231)(0.411830357142857,-1.24049114583439)(0.411969866071428,-1.24141075087569)(0.412109375,-1.24232874159794)(0.412248883928571,-1.24324511648597)(0.412388392857143,-1.24415987402782)(0.412527901785714,-1.24507301271462)(0.412667410714286,-1.24598453104069)(0.412806919642857,-1.24689442750349)(0.412946428571428,-1.24780270060361)(0.4130859375,-1.24870934884485)(0.413225446428571,-1.24961437073414)(0.413364955357143,-1.25051776478157)(0.413504464285714,-1.25141952950043)(0.413643973214286,-1.25231966340715)(0.413783482142857,-1.25321816502135)(0.413922991071428,-1.25411503286582)(0.4140625,-1.25501026546653)(0.414202008928571,-1.25590386135264)(0.414341517857143,-1.25679581905649)(0.414481026785714,-1.2576861371136)(0.414620535714286,-1.2585748140627)(0.414760044642857,-1.2594618484457)(0.414899553571428,-1.26034723880771)(0.4150390625,-1.26123098369704)(0.415178571428571,-1.26211308166521)(0.415318080357143,-1.26299353126693)(0.415457589285714,-1.26387233106012)(0.415597098214286,-1.26474947960594)(0.415736607142857,-1.26562497546873)(0.415876116071428,-1.26649881721604)(0.416015625,-1.26737100341869)(0.416155133928571,-1.26824153265066)(0.416294642857143,-1.26911040348921)(0.416434151785714,-1.26997761451478)(0.416573660714286,-1.27084316431108)(0.416713169642857,-1.27170705146502)(0.416852678571428,-1.27256927456677)(0.4169921875,-1.27342983220973)(0.417131696428571,-1.27428872299054)(0.417271205357143,-1.2751459455091)(0.417410714285714,-1.27600149836851)(0.417550223214286,-1.27685538017519)(0.417689732142857,-1.27770758953875)(0.417829241071428,-1.27855812507208)(0.41796875,-1.27940698539134)(0.418108258928571,-1.28025416911594)(0.418247767857143,-1.28109967486854)(0.418387276785714,-1.28194350127507)(0.418526785714286,-1.28278564696476)(0.418666294642857,-1.28362611057006)(0.418805803571428,-1.28446489072674)(0.4189453125,-1.28530198607382)(0.419084821428571,-1.2861373952536)(0.419224330357143,-1.28697111691168)(0.419363839285714,-1.28780314969692)(0.419503348214286,-1.28863349226149)(0.419642857142857,-1.28946214326082)(0.419782366071428,-1.29028910135367)(0.419921875,-1.29111436520207)(0.420061383928571,-1.29193793347135)(0.420200892857143,-1.29275980483015)(0.420340401785714,-1.2935799779504)(0.420479910714286,-1.29439845150734)(0.420619419642857,-1.29521522417952)(0.420758928571428,-1.2960302946488)(0.4208984375,-1.29684366160036)(0.421037946428571,-1.29765532372267)(0.421177455357143,-1.29846527970755)(0.421316964285714,-1.29927352825013)(0.421456473214286,-1.30008006804885)(0.421595982142857,-1.30088489780548)(0.421735491071428,-1.30168801622514)(0.421875,-1.30248942201625)(0.422014508928571,-1.30328911389058)(0.422154017857143,-1.30408709056324)(0.422293526785714,-1.30488335075267)(0.422433035714286,-1.30567789318065)(0.422572544642857,-1.30647071657231)(0.422712053571428,-1.3072618196561)(0.4228515625,-1.30805120116387)(0.422991071428571,-1.30883885983077)(0.423130580357143,-1.30962479439534)(0.423270089285714,-1.31040900359943)(0.423409598214286,-1.31119148618831)(0.423549107142857,-1.31197224091056)(0.423688616071428,-1.31275126651813)(0.423828125,-1.31352856176637)(0.423967633928571,-1.31430412541395)(0.424107142857143,-1.31507795622295)(0.424246651785714,-1.3158500529588)(0.424386160714286,-1.31662041439031)(0.424525669642857,-1.31738903928967)(0.424665178571428,-1.31815592643245)(0.4248046875,-1.31892107459759)(0.424944196428571,-1.31968448256744)(0.425083705357143,-1.32044614912771)(0.425223214285714,-1.32120607306751)(0.425362723214286,-1.32196425317936)(0.425502232142857,-1.32272068825913)(0.425641741071428,-1.32347537710613)(0.42578125,-1.32422831852305)(0.425920758928571,-1.32497951131598)(0.426060267857143,-1.32572895429442)(0.426199776785714,-1.32647664627126)(0.426339285714286,-1.32722258606282)(0.426478794642857,-1.32796677248882)(0.426618303571428,-1.32870920437239)(0.4267578125,-1.32944988054007)(0.426897321428571,-1.33018879982182)(0.427036830357143,-1.33092596105105)(0.427176339285714,-1.33166136306453)(0.427315848214286,-1.33239500470252)(0.427455357142857,-1.33312688480866)(0.427594866071428,-1.33385700223002)(0.427734375,-1.33458535581715)(0.427873883928571,-1.33531194442396)(0.428013392857143,-1.33603676690785)(0.428152901785714,-1.33675982212965)(0.428292410714286,-1.33748110895359)(0.428431919642857,-1.3382006262474)(0.428571428571428,-1.33891837288221)(0.4287109375,-1.33963434773261)(0.428850446428571,-1.34034854967665)(0.428989955357143,-1.34106097759581)(0.429129464285714,-1.34177163037504)(0.429268973214286,-1.34248050690273)(0.429408482142857,-1.34318760607074)(0.429547991071428,-1.34389292677438)(0.4296875,-1.34459646791243)(0.429827008928571,-1.34529822838712)(0.429966517857143,-1.34599820710417)(0.430106026785714,-1.34669640297274)(0.430245535714286,-1.34739281490548)(0.430385044642857,-1.3480874418185)(0.430524553571428,-1.34878028263138)(0.4306640625,-1.3494713362672)(0.430803571428571,-1.3501606016525)(0.430943080357143,-1.35084807771729)(0.431082589285714,-1.35153376339509)(0.431222098214286,-1.35221765762289)(0.431361607142857,-1.35289975934116)(0.431501116071428,-1.35358006749387)(0.431640625,-1.35425858102848)(0.431780133928571,-1.35493529889593)(0.431919642857143,-1.35561022005067)(0.432059151785714,-1.35628334345064)(0.432198660714286,-1.35695466805728)(0.432338169642857,-1.35762419283554)(0.432477678571428,-1.35829191675385)(0.4326171875,-1.35895783878417)(0.432756696428571,-1.35962195790194)(0.432896205357143,-1.36028427308614)(0.433035714285714,-1.36094478331925)(0.433175223214286,-1.36160348758724)(0.433314732142857,-1.36226038487963)(0.433454241071428,-1.36291547418944)(0.43359375,-1.3635687545132)(0.433733258928571,-1.36422022485097)(0.433872767857143,-1.36486988420634)(0.434012276785714,-1.36551773158642)(0.434151785714286,-1.36616376600183)(0.434291294642857,-1.36680798646675)(0.434430803571428,-1.36745039199886)(0.4345703125,-1.3680909816194)(0.434709821428571,-1.36872975435311)(0.434849330357143,-1.3693667092283)(0.434988839285714,-1.3700018452768)(0.435128348214286,-1.37063516153398)(0.435267857142857,-1.37126665703877)(0.435407366071428,-1.37189633083361)(0.435546875,-1.37252418196452)(0.435686383928571,-1.37315020948104)(0.435825892857143,-1.37377441243629)(0.435965401785714,-1.3743967898869)(0.436104910714286,-1.3750173408931)(0.436244419642857,-1.37563606451862)(0.436383928571428,-1.37625295983081)(0.4365234375,-1.37686802590052)(0.436662946428571,-1.3774812618022)(0.436802455357143,-1.37809266661384)(0.436941964285714,-1.378702239417)(0.437081473214286,-1.37930997929682)(0.437220982142857,-1.37991588534198)(0.437360491071428,-1.38051995664475)(0.4375,-1.38112219230096)(0.437639508928571,-1.38172259141003)(0.437779017857143,-1.38232115307492)(0.437918526785714,-1.38291787640221)(0.438058035714286,-1.38351276050202)(0.438197544642857,-1.38410580448807)(0.438337053571428,-1.38469700747766)(0.4384765625,-1.38528636859167)(0.438616071428571,-1.38587388695455)(0.438755580357143,-1.38645956169437)(0.438895089285714,-1.38704339194276)(0.439034598214286,-1.38762537683496)(0.439174107142857,-1.38820551550977)(0.439313616071428,-1.38878380710962)(0.439453125,-1.38936025078051)(0.439592633928571,-1.38993484567205)(0.439732142857143,-1.39050759093744)(0.439871651785714,-1.39107848573349)(0.440011160714286,-1.3916475292206)(0.440150669642857,-1.39221472056278)(0.440290178571428,-1.39278005892765)(0.4404296875,-1.39334354348641)(0.440569196428571,-1.3939051734139)(0.440708705357143,-1.39446494788856)(0.440848214285714,-1.39502286609244)(0.440987723214286,-1.39557892721119)(0.441127232142857,-1.39613313043408)(0.441266741071428,-1.39668547495402)(0.44140625,-1.39723595996752)(0.441545758928571,-1.39778458467469)(0.441685267857143,-1.39833134827928)(0.441824776785714,-1.39887624998868)(0.441964285714286,-1.39941928901387)(0.442103794642857,-1.39996046456947)(0.442243303571428,-1.40049977587373)(0.4423828125,-1.40103722214853)(0.442522321428571,-1.40157280261936)(0.442661830357143,-1.40210651651538)(0.442801339285714,-1.40263836306933)(0.442940848214286,-1.40316834151764)(0.443080357142857,-1.40369645110034)(0.443219866071428,-1.4042226910611)(0.443359375,-1.40474706064724)(0.443498883928571,-1.40526955910971)(0.443638392857143,-1.40579018570312)(0.443777901785714,-1.40630893968571)(0.443917410714286,-1.40682582031935)(0.444056919642857,-1.40734082686959)(0.444196428571428,-1.40785395860559)(0.4443359375,-1.4083652148002)(0.444475446428571,-1.40887459472988)(0.444614955357143,-1.40938209767476)(0.444754464285714,-1.40988772291863)(0.444893973214286,-1.41039146974893)(0.445033482142857,-1.41089333745675)(0.445172991071428,-1.41139332533684)(0.4453125,-1.4118914326876)(0.445452008928571,-1.41238765881112)(0.445591517857143,-1.41288200301311)(0.445731026785714,-1.41337446460298)(0.445870535714286,-1.41386504289377)(0.446010044642857,-1.41435373720222)(0.446149553571428,-1.41484054684871)(0.4462890625,-1.4153254711573)(0.446428571428571,-1.41580850945572)(0.446568080357143,-1.41628966107536)(0.446707589285714,-1.4167689253513)(0.446847098214286,-1.41724630162227)(0.446986607142857,-1.4177217892307)(0.447126116071428,-1.41819538752268)(0.447265625,-1.41866709584798)(0.447405133928571,-1.41913691356005)(0.447544642857143,-1.41960484001601)(0.447684151785714,-1.42007087457668)(0.447823660714286,-1.42053501660655)(0.447963169642857,-1.4209972654738)(0.448102678571428,-1.42145762055027)(0.4482421875,-1.42191608121153)(0.448381696428571,-1.42237264683679)(0.448521205357143,-1.42282731680899)(0.448660714285714,-1.42328009051474)(0.448800223214286,-1.42373096734432)(0.448939732142857,-1.42417994669174)(0.449079241071428,-1.42462702795469)(0.44921875,-1.42507221053453)(0.449358258928571,-1.42551549383636)(0.449497767857143,-1.42595687726893)(0.449637276785714,-1.42639636024472)(0.449776785714286,-1.42683394217991)(0.449916294642857,-1.42726962249434)(0.450055803571428,-1.4277034006116)(0.4501953125,-1.42813527595896)(0.450334821428571,-1.42856524796739)(0.450474330357143,-1.42899331607157)(0.450613839285714,-1.42941947970989)(0.450753348214286,-1.42984373832443)(0.450892857142857,-1.430266091361)(0.451032366071428,-1.43068653826911)(0.451171875,-1.43110507850196)(0.451311383928571,-1.4315217115165)(0.451450892857143,-1.43193643677336)(0.451590401785714,-1.43234925373689)(0.451729910714286,-1.43276016187517)(0.451869419642857,-1.43316916065997)(0.452008928571428,-1.4335762495668)(0.4521484375,-1.43398142807487)(0.452287946428571,-1.43438469566711)(0.452427455357143,-1.43478605183018)(0.452566964285714,-1.43518549605446)(0.452706473214286,-1.43558302783403)(0.452845982142857,-1.43597864666671)(0.452985491071428,-1.43637235205405)(0.453125,-1.43676414350131)(0.453264508928571,-1.43715402051747)(0.453404017857143,-1.43754198261525)(0.453543526785714,-1.4379280293111)(0.453683035714286,-1.43831216012518)(0.453822544642857,-1.4386943745814)(0.453962053571428,-1.43907467220738)(0.4541015625,-1.43945305253448)(0.454241071428571,-1.43982951509778)(0.454380580357143,-1.44020405943613)(0.454520089285714,-1.44057668509207)(0.454659598214286,-1.44094739161189)(0.454799107142857,-1.44131617854561)(0.454938616071428,-1.44168304544701)(0.455078125,-1.44204799187358)(0.455217633928571,-1.44241101738656)(0.455357142857143,-1.44277212155092)(0.455496651785714,-1.44313130393538)(0.455636160714286,-1.44348856411239)(0.455775669642857,-1.44384390165815)(0.455915178571428,-1.4441973161526)(0.4560546875,-1.44454880717942)(0.456194196428571,-1.44489837432603)(0.456333705357143,-1.4452460171836)(0.456473214285714,-1.44559173534705)(0.456612723214286,-1.44593552841504)(0.456752232142857,-1.44627739598998)(0.456891741071428,-1.44661733767801)(0.45703125,-1.44695535308905)(0.457170758928571,-1.44729144183674)(0.457310267857143,-1.44762560353849)(0.457449776785714,-1.44795783781545)(0.457589285714286,-1.44828814429252)(0.457728794642857,-1.44861652259836)(0.457868303571428,-1.44894297236538)(0.4580078125,-1.44926749322973)(0.458147321428571,-1.44959008483134)(0.458286830357143,-1.44991074681388)(0.458426339285714,-1.45022947882476)(0.458565848214286,-1.45054628051519)(0.458705357142857,-1.45086115154009)(0.458844866071428,-1.45117409155816)(0.458984375,-1.45148510023186)(0.459123883928571,-1.4517941772274)(0.459263392857143,-1.45210132221475)(0.459402901785714,-1.45240653486766)(0.459542410714286,-1.45270981486362)(0.459681919642857,-1.45301116188388)(0.459821428571428,-1.45331057561346)(0.4599609375,-1.45360805574114)(0.460100446428571,-1.45390360195947)(0.460239955357143,-1.45419721396476)(0.460379464285714,-1.45448889145707)(0.460518973214286,-1.45477863414026)(0.460658482142857,-1.45506644172192)(0.460797991071428,-1.45535231391341)(0.4609375,-1.45563625042989)(0.461077008928571,-1.45591825099025)(0.461216517857143,-1.45619831531717)(0.461356026785714,-1.45647644313709)(0.461495535714286,-1.45675263418021)(0.461635044642857,-1.45702688818052)(0.461774553571428,-1.45729920487576)(0.4619140625,-1.45756958400745)(0.462053571428571,-1.4578380253209)(0.462193080357143,-1.45810452856514)(0.462332589285714,-1.45836909349303)(0.462472098214286,-1.45863171986117)(0.462611607142857,-1.45889240742994)(0.462751116071428,-1.45915115596349)(0.462890625,-1.45940796522975)(0.463030133928571,-1.45966283500042)(0.463169642857143,-1.45991576505097)(0.463309151785714,-1.46016675516066)(0.463448660714286,-1.46041580511252)(0.463588169642857,-1.46066291469334)(0.463727678571428,-1.46090808369371)(0.4638671875,-1.46115131190799)(0.464006696428571,-1.4613925991343)(0.464146205357143,-1.46163194517457)(0.464285714285714,-1.46186934983447)(0.464425223214286,-1.46210481292348)(0.464564732142857,-1.46233833425485)(0.464704241071428,-1.46256991364559)(0.46484375,-1.46279955091652)(0.464983258928571,-1.46302724589222)(0.465122767857143,-1.46325299840106)(0.465262276785714,-1.46347680827519)(0.465401785714286,-1.46369867535052)(0.465541294642857,-1.46391859946678)(0.465680803571428,-1.46413658046744)(0.4658203125,-1.46435261819979)(0.465959821428571,-1.46456671251487)(0.466099330357143,-1.46477886326752)(0.466238839285714,-1.46498907031637)(0.466378348214286,-1.46519733352381)(0.466517857142857,-1.46540365275602)(0.466657366071428,-1.46560802788299)(0.466796875,-1.46581045877845)(0.466936383928571,-1.46601094531995)(0.467075892857143,-1.46620948738881)(0.467215401785714,-1.46640608487013)(0.467354910714286,-1.4666007376528)(0.467494419642857,-1.4667934456295)(0.467633928571428,-1.46698420869669)(0.4677734375,-1.46717302675461)(0.467912946428571,-1.4673598997073)(0.468052455357143,-1.46754482746257)(0.468191964285714,-1.46772780993202)(0.468331473214286,-1.46790884703104)(0.468470982142857,-1.46808793867881)(0.468610491071428,-1.46826508479829)(0.46875,-1.46844028531622)(0.468889508928571,-1.46861354016315)(0.469029017857143,-1.46878484927338)(0.469168526785714,-1.46895421258504)(0.469308035714286,-1.46912163004001)(0.469447544642857,-1.46928710158397)(0.469587053571428,-1.4694506271664)(0.4697265625,-1.46961220674056)(0.469866071428571,-1.46977184026348)(0.470005580357143,-1.46992952769601)(0.470145089285714,-1.47008526900275)(0.470284598214286,-1.47023906415213)(0.470424107142857,-1.47039091311632)(0.470563616071428,-1.47054081587132)(0.470703125,-1.4706887723969)(0.470842633928571,-1.47083478267661)(0.470982142857143,-1.4709788466978)(0.471121651785714,-1.4711209644516)(0.471261160714286,-1.47126113593295)(0.471400669642857,-1.47139936114054)(0.471540178571428,-1.47153564007688)(0.4716796875,-1.47166997274826)(0.471819196428571,-1.47180235916474)(0.471958705357143,-1.4719327993402)(0.472098214285714,-1.47206129329227)(0.472237723214286,-1.4721878410424)(0.472377232142857,-1.47231244261582)(0.472516741071428,-1.47243509804154)(0.47265625,-1.47255580735236)(0.472795758928571,-1.47267457058486)(0.472935267857143,-1.47279138777944)(0.473074776785714,-1.47290625898024)(0.473214285714286,-1.47301918423523)(0.473353794642857,-1.47313016359613)(0.473493303571428,-1.47323919711849)(0.4736328125,-1.4733462848616)(0.473772321428571,-1.47345142688858)(0.473911830357143,-1.47355462326631)(0.474051339285714,-1.47365587406546)(0.474190848214286,-1.47375517936049)(0.474330357142857,-1.47385253922966)(0.474469866071428,-1.47394795375499)(0.474609375,-1.47404142302231)(0.474748883928571,-1.47413294712121)(0.474888392857143,-1.4742225261451)(0.475027901785714,-1.47431016019115)(0.475167410714286,-1.47439584936032)(0.475306919642857,-1.47447959375736)(0.475446428571428,-1.47456139349081)(0.4755859375,-1.47464124867297)(0.475725446428571,-1.47471915941996)(0.475864955357143,-1.47479512585166)(0.476004464285714,-1.47486914809174)(0.476143973214286,-1.47494122626766)(0.476283482142857,-1.47501136051065)(0.476422991071428,-1.47507955095574)(0.4765625,-1.47514579774174)(0.476702008928571,-1.47521010101122)(0.476841517857143,-1.47527246091057)(0.476981026785714,-1.47533287758993)(0.477120535714286,-1.47539135120324)(0.477260044642857,-1.47544788190822)(0.477399553571428,-1.47550246986636)(0.4775390625,-1.47555511524295)(0.477678571428571,-1.47560581820705)(0.477818080357143,-1.47565457893149)(0.477957589285714,-1.47570139759289)(0.478097098214286,-1.47574627437167)(0.478236607142857,-1.47578920945199)(0.478376116071428,-1.47583020302182)(0.478515625,-1.47586925527289)(0.478655133928571,-1.47590636640072)(0.478794642857143,-1.47594153660461)(0.478934151785714,-1.47597476608762)(0.479073660714286,-1.47600605505661)(0.479213169642857,-1.4760354037222)(0.479352678571428,-1.4760628122988)(0.4794921875,-1.47608828100457)(0.479631696428571,-1.47611181006148)(0.479771205357143,-1.47613339969526)(0.479910714285714,-1.47615305013539)(0.480050223214286,-1.47617076161518)(0.480189732142857,-1.47618653437166)(0.480329241071428,-1.47620036864565)(0.48046875,-1.47621226468177)(0.480608258928571,-1.47622222272837)(0.480747767857143,-1.4762302430376)(0.480887276785714,-1.47623632586536)(0.481026785714286,-1.47624047147136)(0.481166294642857,-1.47624268011903)(0.481305803571428,-1.4762429520756)(0.4814453125,-1.47624128761208)(0.481584821428571,-1.4762376870032)(0.481724330357143,-1.47623215052752)(0.481863839285714,-1.47622467846732)(0.482003348214286,-1.47621527110868)(0.482142857142857,-1.47620392874141)(0.482282366071428,-1.47619065165913)(0.482421875,-1.4761754401592)(0.482561383928571,-1.47615829454273)(0.482700892857143,-1.47613921511464)(0.482840401785714,-1.47611820218356)(0.482979910714286,-1.47609525606193)(0.483119419642857,-1.47607037706591)(0.483258928571428,-1.47604356551547)(0.4833984375,-1.4760148217343)(0.483537946428571,-1.47598414604987)(0.483677455357143,-1.4759515387934)(0.483816964285714,-1.47591700029988)(0.483956473214286,-1.47588053090806)(0.484095982142857,-1.47584213096043)(0.484235491071428,-1.47580180080325)(0.484375,-1.47575954078655)(0.484514508928571,-1.47571535126409)(0.484654017857143,-1.4756692325934)(0.484793526785714,-1.47562118513576)(0.484933035714286,-1.47557120925621)(0.485072544642857,-1.47551930532354)(0.485212053571428,-1.47546547371029)(0.4853515625,-1.47540971479276)(0.485491071428571,-1.47535202895098)(0.485630580357143,-1.47529241656875)(0.485770089285714,-1.47523087803363)(0.485909598214286,-1.47516741373689)(0.486049107142857,-1.47510202407359)(0.486188616071428,-1.4750347094425)(0.486328125,-1.47496547024618)(0.486467633928571,-1.47489430689089)(0.486607142857143,-1.47482121978667)(0.486746651785714,-1.47474620934728)(0.486886160714286,-1.47466927599024)(0.487025669642857,-1.4745904201368)(0.487165178571428,-1.47450964221196)(0.4873046875,-1.47442694264446)(0.487444196428571,-1.47434232186678)(0.487583705357143,-1.47425578031513)(0.487723214285714,-1.47416731842948)(0.487862723214286,-1.47407693665351)(0.488002232142857,-1.47398463543466)(0.488141741071428,-1.47389041522409)(0.48828125,-1.4737942764767)(0.488420758928571,-1.47369621965113)(0.488560267857143,-1.47359624520974)(0.488699776785714,-1.47349435361865)(0.488839285714286,-1.47339054534766)(0.488978794642857,-1.47328482087036)(0.489118303571428,-1.47317718066403)(0.4892578125,-1.47306762520969)(0.489397321428571,-1.47295615499208)(0.489536830357143,-1.47284277049969)(0.489676339285714,-1.47272747222471)(0.489815848214286,-1.47261026066307)(0.489955357142857,-1.47249113631441)(0.490094866071428,-1.47237009968211)(0.490234375,-1.47224715127325)(0.490373883928571,-1.47212229159866)(0.490513392857143,-1.47199552117287)(0.490652901785714,-1.47186684051412)(0.490792410714286,-1.4717362501444)(0.490931919642857,-1.47160375058938)(0.491071428571428,-1.47146934237847)(0.4912109375,-1.47133302604478)(0.491350446428571,-1.47119480212516)(0.491489955357143,-1.47105467116014)(0.491629464285714,-1.47091263369398)(0.491768973214286,-1.47076869027464)(0.491908482142857,-1.47062284145381)(0.492047991071428,-1.47047508778686)(0.4921875,-1.47032542983288)(0.492327008928571,-1.47017386815468)(0.492466517857143,-1.47002040331876)(0.492606026785714,-1.46986503589533)(0.492745535714286,-1.46970776645829)(0.492885044642857,-1.46954859558525)(0.493024553571428,-1.46938752385754)(0.4931640625,-1.46922455186016)(0.493303571428571,-1.46905968018182)(0.493443080357143,-1.46889290941494)(0.493582589285714,-1.4687242401556)(0.493722098214286,-1.46855367300363)(0.493861607142857,-1.4683812085625)(0.494001116071428,-1.4682068474394)(0.494140625,-1.46803059024521)(0.494280133928571,-1.4678524375945)(0.494419642857143,-1.46767239010552)(0.494559151785714,-1.46749044840022)(0.494698660714286,-1.46730661310423)(0.494838169642857,-1.46712088484686)(0.494977678571428,-1.46693326426112)(0.4951171875,-1.46674375198369)(0.495256696428571,-1.46655234865494)(0.495396205357143,-1.4663590549189)(0.495535714285714,-1.46616387142331)(0.495675223214286,-1.46596679881957)(0.495814732142857,-1.46576783776276)(0.495954241071428,-1.46556698891163)(0.49609375,-1.46536425292861)(0.496233258928571,-1.46515963047981)(0.496372767857143,-1.464953122235)(0.496512276785714,-1.46474472886763)(0.496651785714286,-1.4645344510548)(0.496791294642857,-1.46432228947729)(0.496930803571428,-1.46410824481956)(0.4970703125,-1.46389231776972)(0.497209821428571,-1.46367450901954)(0.497349330357143,-1.46345481926447)(0.497488839285714,-1.46323324920359)(0.497628348214286,-1.46300979953967)(0.497767857142857,-1.46278447097912)(0.497907366071428,-1.46255726423202)(0.498046875,-1.4623281800121)(0.498186383928571,-1.46209721903674)(0.498325892857143,-1.46186438202698)(0.498465401785714,-1.4616296697075)(0.498604910714286,-1.46139308280664)(0.498744419642857,-1.46115462205639)(0.498883928571428,-1.46091428819238)(0.4990234375,-1.46067208195388)(0.499162946428571,-1.46042800408382)(0.499302455357143,-1.46018205532877)(0.499441964285714,-1.45993423643893)(0.499581473214286,-1.45968454816814)(0.499720982142857,-1.45943299127389)(0.499860491071428,-1.45917956651731)(0.5,-1.45892427466314)(0.500139508928571,-1.45866711647978)(0.500279017857143,-1.45840809273925)(0.500418526785714,-1.45814720421721)(0.500558035714286,-1.45788445169294)(0.500697544642857,-1.45761983594934)(0.500837053571428,-1.45735335777296)(0.5009765625,-1.45708501795396)(0.501116071428571,-1.45681481728613)(0.501255580357143,-1.45654275656686)(0.501395089285714,-1.45626883659719)(0.501534598214286,-1.45599305818177)(0.501674107142857,-1.45571542212885)(0.501813616071428,-1.45543592925032)(0.501953125,-1.45515458036167)(0.502092633928571,-1.45487137628199)(0.502232142857143,-1.45458631783402)(0.502371651785714,-1.45429940584406)(0.502511160714286,-1.45401064114206)(0.502650669642857,-1.45372002456155)(0.502790178571428,-1.45342755693967)(0.5029296875,-1.45313323911717)(0.503069196428571,-1.4528370719384)(0.503208705357143,-1.45253905625129)(0.503348214285714,-1.45223919290739)(0.503487723214286,-1.45193748276185)(0.503627232142857,-1.45163392667338)(0.503766741071428,-1.45132852550433)(0.50390625,-1.4510212801206)(0.504045758928571,-1.4507121913917)(0.504185267857143,-1.45040126019073)(0.504324776785714,-1.45008848739436)(0.504464285714286,-1.44977387388287)(0.504603794642857,-1.44945742054008)(0.504743303571428,-1.44913912825345)(0.5048828125,-1.44881899791396)(0.505022321428571,-1.44849703041621)(0.505161830357143,-1.44817322665835)(0.505301339285714,-1.44784758754211)(0.505440848214286,-1.44752011397281)(0.505580357142857,-1.44719080685932)(0.505719866071428,-1.44685966711408)(0.505859375,-1.4465266956531)(0.505998883928571,-1.44619189339597)(0.506138392857143,-1.44585526126581)(0.506277901785714,-1.44551680018933)(0.506417410714286,-1.4451765110968)(0.506556919642857,-1.44483439492202)(0.506696428571428,-1.44449045260237)(0.5068359375,-1.44414468507879)(0.506975446428571,-1.44379709329575)(0.507114955357143,-1.44344767820128)(0.507254464285714,-1.44309644074697)(0.507393973214286,-1.44274338188794)(0.507533482142857,-1.44238850258286)(0.507672991071428,-1.44203180379395)(0.5078125,-1.44167328648696)(0.507952008928571,-1.44131295163119)(0.508091517857143,-1.44095080019947)(0.508231026785714,-1.44058683316818)(0.508370535714286,-1.4402210515172)(0.508510044642857,-1.43985345622998)(0.508649553571428,-1.43948404829348)(0.5087890625,-1.4391128286982)(0.508928571428571,-1.43873979843814)(0.509068080357143,-1.43836495851085)(0.509207589285714,-1.4379883099174)(0.509347098214286,-1.43760985366236)(0.509486607142857,-1.43722959075384)(0.509626116071428,-1.43684752220346)(0.509765625,-1.43646364902636)(0.509905133928571,-1.43607797224117)(0.510044642857143,-1.43569049287005)(0.510184151785714,-1.43530121193866)(0.510323660714286,-1.43491013047618)(0.510463169642857,-1.43451724951528)(0.510602678571428,-1.43412257009213)(0.5107421875,-1.43372609324641)(0.510881696428571,-1.43332782002131)(0.511021205357143,-1.43292775146347)(0.511160714285714,-1.43252588862309)(0.511300223214286,-1.4321222325538)(0.511439732142857,-1.43171678431277)(0.511579241071428,-1.43130954496062)(0.51171875,-1.43090051556147)(0.511858258928571,-1.43048969718294)(0.511997767857143,-1.43007709089611)(0.512137276785714,-1.42966269777555)(0.512276785714286,-1.4292465188993)(0.512416294642857,-1.42882855534888)(0.512555803571428,-1.42840880820929)(0.5126953125,-1.42798727856899)(0.512834821428571,-1.42756396751993)(0.512974330357143,-1.42713887615749)(0.513113839285714,-1.42671200558054)(0.513253348214286,-1.42628335689143)(0.513392857142857,-1.42585293119592)(0.513532366071428,-1.42542072960329)(0.513671875,-1.42498675322622)(0.513811383928571,-1.42455100318089)(0.513950892857143,-1.42411348058689)(0.514090401785714,-1.4236741865673)(0.514229910714286,-1.42323312224861)(0.514369419642857,-1.42279028876079)(0.514508928571428,-1.42234568723724)(0.5146484375,-1.42189931881479)(0.514787946428571,-1.42145118463372)(0.514927455357143,-1.42100128583774)(0.515066964285714,-1.42054962357401)(0.515206473214286,-1.4200961989931)(0.515345982142857,-1.41964101324903)(0.515485491071428,-1.41918406749923)(0.515625,-1.41872536290457)(0.515764508928571,-1.41826490062933)(0.515904017857143,-1.41780268184123)(0.516043526785714,-1.41733870771139)(0.516183035714286,-1.41687297941435)(0.516322544642857,-1.41640549812807)(0.516462053571428,-1.41593626503393)(0.5166015625,-1.41546528131669)(0.516741071428571,-1.41499254816455)(0.516880580357143,-1.41451806676909)(0.517020089285714,-1.41404183832531)(0.517159598214286,-1.41356386403161)(0.517299107142857,-1.41308414508977)(0.517438616071428,-1.41260268270497)(0.517578125,-1.41211947808581)(0.517717633928571,-1.41163453244426)(0.517857142857143,-1.41114784699566)(0.517996651785714,-1.41065942295877)(0.518136160714286,-1.41016926155572)(0.518275669642857,-1.40967736401201)(0.518415178571428,-1.40918373155654)(0.5185546875,-1.40868836542158)(0.518694196428571,-1.40819126684276)(0.518833705357143,-1.4076924370591)(0.518973214285714,-1.40719187731297)(0.519112723214286,-1.40668958885014)(0.519252232142857,-1.4061855729197)(0.519391741071428,-1.40567983077415)(0.51953125,-1.4051723636693)(0.519670758928571,-1.40466317286436)(0.519810267857143,-1.40415225962187)(0.519949776785714,-1.40363962520774)(0.520089285714286,-1.4031252708912)(0.520228794642857,-1.40260919794487)(0.520368303571428,-1.40209140764467)(0.5205078125,-1.4015719012699)(0.520647321428571,-1.40105068010317)(0.520786830357143,-1.40052774543046)(0.520926339285714,-1.40000309854105)(0.521065848214286,-1.39947674072758)(0.521205357142857,-1.398948673286)(0.521344866071428,-1.39841889751561)(0.521484375,-1.397887414719)(0.521623883928571,-1.39735422620213)(0.521763392857143,-1.39681933327423)(0.521902901785714,-1.39628273724788)(0.522042410714286,-1.39574443943897)(0.522181919642857,-1.39520444116669)(0.522321428571428,-1.39466274375355)(0.5224609375,-1.39411934852536)(0.522600446428571,-1.39357425681125)(0.522739955357143,-1.39302746994363)(0.522879464285714,-1.39247898925822)(0.523018973214286,-1.39192881609404)(0.523158482142857,-1.3913769517934)(0.523297991071428,-1.3908233977019)(0.5234375,-1.39026815516843)(0.523577008928571,-1.38971122554517)(0.523716517857143,-1.38915261018758)(0.523856026785714,-1.38859231045439)(0.523995535714286,-1.38803032770763)(0.524135044642857,-1.38746666331259)(0.524274553571428,-1.38690131863785)(0.5244140625,-1.38633429505523)(0.524553571428571,-1.38576559393985)(0.524693080357143,-1.38519521667007)(0.524832589285714,-1.38462316462753)(0.524972098214286,-1.38404943919712)(0.525111607142857,-1.38347404176699)(0.525251116071428,-1.38289697372854)(0.525390625,-1.38231823647643)(0.525530133928571,-1.38173783140856)(0.525669642857143,-1.38115575992609)(0.525809151785714,-1.3805720234334)(0.525948660714286,-1.37998662333812)(0.526088169642857,-1.37939956105114)(0.526227678571428,-1.37881083798655)(0.5263671875,-1.37822045556169)(0.526506696428571,-1.37762841519715)(0.526646205357143,-1.37703471831669)(0.526785714285714,-1.37643936634736)(0.526925223214286,-1.37584236071939)(0.527064732142857,-1.37524370286623)(0.527204241071428,-1.37464339422456)(0.52734375,-1.37404143623428)(0.527483258928571,-1.37343783033847)(0.527622767857143,-1.37283257798344)(0.527762276785714,-1.37222568061871)(0.527901785714286,-1.37161713969698)(0.528041294642857,-1.37100695667416)(0.528180803571428,-1.37039513300935)(0.5283203125,-1.36978167016487)(0.528459821428571,-1.36916656960619)(0.528599330357143,-1.368549832802)(0.528738839285714,-1.36793146122415)(0.528878348214286,-1.36731145634768)(0.529017857142857,-1.36668981965083)(0.529157366071428,-1.36606655261498)(0.529296875,-1.36544165672471)(0.529436383928571,-1.36481513346776)(0.529575892857143,-1.36418698433503)(0.529715401785714,-1.36355721082061)(0.529854910714286,-1.36292581442172)(0.529994419642857,-1.36229279663875)(0.530133928571428,-1.36165815897527)(0.5302734375,-1.36102190293795)(0.530412946428571,-1.36038403003667)(0.530552455357143,-1.35974454178441)(0.530691964285714,-1.35910343969732)(0.530831473214286,-1.35846072529468)(0.530970982142857,-1.35781640009891)(0.531110491071428,-1.35717046563557)(0.53125,-1.35652292343335)(0.531389508928571,-1.35587377502405)(0.531529017857143,-1.35522302194263)(0.531668526785714,-1.35457066572715)(0.531808035714286,-1.3539167079188)(0.531947544642857,-1.35326115006188)(0.532087053571428,-1.35260399370381)(0.5322265625,-1.35194524039511)(0.532366071428571,-1.35128489168943)(0.532505580357143,-1.3506229491435)(0.532645089285714,-1.34995941431718)(0.532784598214286,-1.3492942887734)(0.532924107142857,-1.3486275740782)(0.533063616071428,-1.34795927180072)(0.533203125,-1.34728938351317)(0.533342633928571,-1.34661791079086)(0.533482142857143,-1.3459448552122)(0.533621651785714,-1.34527021835864)(0.533761160714286,-1.34459400181474)(0.533900669642857,-1.34391620716813)(0.534040178571428,-1.3432368360095)(0.5341796875,-1.34255588993262)(0.534319196428571,-1.34187337053432)(0.534458705357143,-1.34118927941448)(0.534598214285714,-1.34050361817607)(0.534737723214286,-1.33981638842508)(0.534877232142857,-1.33912759177058)(0.535016741071428,-1.33843722982468)(0.53515625,-1.33774530420252)(0.535295758928571,-1.33705181652233)(0.535435267857143,-1.33635676840532)(0.535574776785714,-1.33566016147578)(0.535714285714286,-1.33496199736102)(0.535853794642857,-1.33426227769138)(0.535993303571428,-1.33356100410023)(0.5361328125,-1.33285817822396)(0.536272321428571,-1.33215380170198)(0.536411830357143,-1.33144787617673)(0.536551339285714,-1.33074040329366)(0.536690848214286,-1.33003138470122)(0.536830357142857,-1.32932082205088)(0.536969866071428,-1.32860871699711)(0.537109375,-1.32789507119738)(0.537248883928571,-1.32717988631218)(0.537388392857143,-1.32646316400496)(0.537527901785714,-1.32574490594219)(0.537667410714286,-1.32502511379331)(0.537806919642857,-1.32430378923077)(0.537946428571428,-1.32358093392998)(0.5380859375,-1.32285654956933)(0.538225446428571,-1.3221306378302)(0.538364955357143,-1.32140320039694)(0.538504464285714,-1.32067423895686)(0.538643973214286,-1.31994375520023)(0.538783482142857,-1.31921175082031)(0.538922991071428,-1.3184782275133)(0.5390625,-1.31774318697836)(0.539202008928571,-1.3170066309176)(0.539341517857143,-1.31626856103608)(0.539481026785714,-1.31552897904182)(0.539620535714286,-1.31478788664575)(0.539760044642857,-1.31404528556178)(0.539899553571428,-1.31330117750674)(0.5400390625,-1.31255556420038)(0.540178571428571,-1.31180844736539)(0.540318080357143,-1.31105982872739)(0.540457589285714,-1.31030971001492)(0.540597098214286,-1.30955809295943)(0.540736607142857,-1.30880497929531)(0.540876116071428,-1.30805037075984)(0.541015625,-1.30729426909322)(0.541155133928571,-1.30653667603856)(0.541294642857143,-1.30577759334184)(0.541434151785714,-1.305017022752)(0.541573660714286,-1.30425496602081)(0.541713169642857,-1.30349142490299)(0.541852678571428,-1.30272640115611)(0.5419921875,-1.30195989654063)(0.542131696428571,-1.30119191281992)(0.542271205357143,-1.30042245176019)(0.542410714285714,-1.29965151513056)(0.542550223214286,-1.29887910470299)(0.542689732142857,-1.29810522225234)(0.542829241071428,-1.29732986955632)(0.54296875,-1.29655304839549)(0.543108258928571,-1.29577476055328)(0.543247767857143,-1.29499500781598)(0.543387276785714,-1.29421379197273)(0.543526785714286,-1.2934311148155)(0.543666294642857,-1.29264697813912)(0.543805803571428,-1.29186138374126)(0.5439453125,-1.29107433342243)(0.544084821428571,-1.29028582898596)(0.544224330357143,-1.28949587223801)(0.544363839285714,-1.28870446498759)(0.544503348214286,-1.28791160904652)(0.544642857142857,-1.28711730622942)(0.544782366071428,-1.28632155835376)(0.544921875,-1.28552436723979)(0.545061383928571,-1.2847257347106)(0.545200892857143,-1.28392566259207)(0.545340401785714,-1.28312415271287)(0.545479910714286,-1.28232120690448)(0.545619419642857,-1.28151682700118)(0.545758928571428,-1.28071101484005)(0.5458984375,-1.27990377226092)(0.546037946428571,-1.27909510110644)(0.546177455357143,-1.27828500322203)(0.546316964285714,-1.27747348045588)(0.546456473214286,-1.27666053465896)(0.546595982142857,-1.275846167685)(0.546735491071428,-1.27503038139052)(0.546875,-1.27421317763477)(0.547014508928571,-1.27339455827978)(0.547154017857143,-1.27257452519034)(0.547293526785714,-1.27175308023397)(0.547433035714286,-1.27093022528095)(0.547572544642857,-1.2701059622043)(0.547712053571428,-1.2692802928798)(0.5478515625,-1.26845321918594)(0.547991071428571,-1.26762474300396)(0.548130580357143,-1.26679486621781)(0.548270089285714,-1.26596359071419)(0.548409598214286,-1.26513091838252)(0.548549107142857,-1.26429685111492)(0.548688616071428,-1.26346139080623)(0.548828125,-1.26262453935403)(0.548967633928571,-1.26178629865856)(0.549107142857143,-1.26094667062281)(0.549246651785714,-1.26010565715243)(0.549386160714286,-1.25926326015579)(0.549525669642857,-1.25841948154395)(0.549665178571428,-1.25757432323067)(0.5498046875,-1.25672778713236)(0.549944196428571,-1.25587987516814)(0.550083705357143,-1.25503058925981)(0.550223214285714,-1.25417993133183)(0.550362723214286,-1.25332790331133)(0.550502232142857,-1.25247450712812)(0.550641741071428,-1.25161974471467)(0.55078125,-1.25076361800609)(0.550920758928571,-1.24990612894017)(0.551060267857143,-1.24904727945734)(0.551199776785714,-1.24818707150068)(0.551339285714286,-1.2473255070159)(0.551478794642857,-1.24646258795138)(0.551618303571428,-1.24559831625811)(0.5517578125,-1.24473269388972)(0.551897321428571,-1.24386572280248)(0.552036830357143,-1.24299740495527)(0.552176339285714,-1.24212774230959)(0.552315848214286,-1.24125673682958)(0.552455357142857,-1.24038439048196)(0.552594866071428,-1.2395107052361)(0.552734375,-1.23863568306393)(0.552873883928571,-1.23775932594002)(0.553013392857143,-1.23688163584151)(0.553152901785714,-1.23600261474817)(0.553292410714286,-1.23512226464231)(0.553431919642857,-1.23424058750888)(0.553571428571428,-1.23335758533537)(0.5537109375,-1.23247326011188)(0.553850446428571,-1.23158761383107)(0.553989955357143,-1.23070064848816)(0.554129464285714,-1.22981236608096)(0.554268973214286,-1.22892276860983)(0.554408482142857,-1.22803185807769)(0.554547991071428,-1.22713963649003)(0.5546875,-1.22624610585487)(0.554827008928571,-1.2253512681828)(0.554966517857143,-1.22445512548692)(0.555106026785714,-1.22355767978291)(0.555245535714286,-1.22265893308896)(0.555385044642857,-1.2217588874258)(0.555524553571428,-1.22085754481669)(0.5556640625,-1.21995490728741)(0.555803571428571,-1.21905097686626)(0.555943080357143,-1.21814575558407)(0.556082589285714,-1.21723924547416)(0.556222098214286,-1.21633144857237)(0.556361607142857,-1.21542236691705)(0.556501116071428,-1.21451200254905)(0.556640625,-1.21360035751169)(0.556780133928571,-1.21268743385082)(0.556919642857143,-1.21177323361475)(0.557059151785714,-1.2108577588543)(0.557198660714286,-1.20994101162275)(0.557338169642857,-1.20902299397586)(0.557477678571428,-1.20810370797188)(0.5576171875,-1.2071831556715)(0.557756696428571,-1.2062613391379)(0.557896205357143,-1.2053382604367)(0.558035714285714,-1.20441392163599)(0.558175223214286,-1.20348832480632)(0.558314732142857,-1.20256147202066)(0.558454241071428,-1.20163336535445)(0.55859375,-1.20070400688556)(0.558733258928571,-1.19977339869431)(0.558872767857143,-1.19884154286342)(0.559012276785714,-1.19790844147808)(0.559151785714286,-1.19697409662587)(0.559291294642857,-1.19603851039681)(0.559430803571428,-1.19510168488332)(0.5595703125,-1.19416362218025)(0.559709821428571,-1.19322432438486)(0.559849330357143,-1.19228379359678)(0.559988839285714,-1.19134203191808)(0.560128348214286,-1.1903990414532)(0.560267857142857,-1.18945482430898)(0.560407366071428,-1.18850938259466)(0.560546875,-1.18756271842183)(0.560686383928571,-1.1866148339045)(0.560825892857143,-1.18566573115903)(0.560965401785714,-1.18471541230415)(0.561104910714286,-1.18376387946096)(0.561244419642857,-1.18281113475294)(0.561383928571428,-1.18185718030591)(0.5615234375,-1.18090201824804)(0.561662946428571,-1.17994565070988)(0.561802455357143,-1.17898807982428)(0.561941964285714,-1.17802930772647)(0.562081473214286,-1.17706933655401)(0.562220982142857,-1.17610816844678)(0.562360491071428,-1.17514580554701)(0.5625,-1.17418224999923)(0.562639508928571,-1.17321750395032)(0.562779017857143,-1.17225156954945)(0.562918526785714,-1.17128444894813)(0.563058035714286,-1.17031614430015)(0.563197544642857,-1.16934665776162)(0.563337053571428,-1.16837599149095)(0.5634765625,-1.16740414764886)(0.563616071428571,-1.16643112839833)(0.563755580357143,-1.16545693590464)(0.563895089285714,-1.16448157233537)(0.564034598214286,-1.16350503986037)(0.564174107142857,-1.16252734065175)(0.564313616071428,-1.16154847688391)(0.564453125,-1.16056845073351)(0.564592633928571,-1.15958726437947)(0.564732142857143,-1.15860492000297)(0.564871651785714,-1.15762141978745)(0.565011160714286,-1.15663676591857)(0.565150669642857,-1.15565096058429)(0.565290178571428,-1.15466400597476)(0.5654296875,-1.15367590428239)(0.565569196428571,-1.15268665770183)(0.565708705357143,-1.15169626842993)(0.565848214285714,-1.15070473866579)(0.565987723214286,-1.14971207061072)(0.566127232142857,-1.14871826646824)(0.566266741071428,-1.1477233284441)(0.56640625,-1.14672725874624)(0.566545758928571,-1.14573005958481)(0.566685267857143,-1.14473173317214)(0.566824776785714,-1.14373228172279)(0.566964285714286,-1.14273170745347)(0.567103794642857,-1.1417300125831)(0.567243303571428,-1.14072719933279)(0.5673828125,-1.13972326992579)(0.567522321428571,-1.13871822658756)(0.567661830357143,-1.1377120715457)(0.567801339285714,-1.13670480702999)(0.567940848214286,-1.13569643527237)(0.568080357142857,-1.13468695850693)(0.568219866071428,-1.1336763789699)(0.568359375,-1.13266469889968)(0.568498883928571,-1.13165192053679)(0.568638392857143,-1.1306380461239)(0.568777901785714,-1.12962307790581)(0.568917410714286,-1.12860701812944)(0.569056919642857,-1.12758986904386)(0.569196428571428,-1.12657163290024)(0.5693359375,-1.12555231195185)(0.569475446428571,-1.12453190845411)(0.569614955357143,-1.12351042466452)(0.569754464285714,-1.1224878628427)(0.569893973214286,-1.12146422525033)(0.570033482142857,-1.12043951415124)(0.570172991071428,-1.11941373181131)(0.5703125,-1.11838688049852)(0.570452008928571,-1.11735896248292)(0.570591517857143,-1.11632998003665)(0.570731026785714,-1.1152999354339)(0.570870535714286,-1.11426883095096)(0.571010044642857,-1.11323666886615)(0.571149553571428,-1.11220345145987)(0.5712890625,-1.11116918101456)(0.571428571428571,-1.11013385981473)(0.571568080357143,-1.1090974901469)(0.571707589285714,-1.10806007429968)(0.571847098214286,-1.10702161456366)(0.571986607142857,-1.1059821132315)(0.572126116071428,-1.10494157259788)(0.572265625,-1.10389999495949)(0.572405133928571,-1.10285738261506)(0.572544642857143,-1.10181373786532)(0.572684151785714,-1.10076906301301)(0.572823660714286,-1.09972336036286)(0.572963169642857,-1.09867663222164)(0.573102678571428,-1.09762888089808)(0.5732421875,-1.09658010870291)(0.573381696428571,-1.09553031794887)(0.573521205357143,-1.09447951095063)(0.573660714285714,-1.0934276900249)(0.573800223214286,-1.09237485749033)(0.573939732142857,-1.09132101566752)(0.574079241071428,-1.09026616687908)(0.57421875,-1.08921031344956)(0.574358258928571,-1.08815345770544)(0.574497767857143,-1.08709560197519)(0.574637276785714,-1.08603674858921)(0.574776785714286,-1.08497689987984)(0.574916294642857,-1.08391605818134)(0.575055803571428,-1.08285422582995)(0.5751953125,-1.08179140516379)(0.575334821428571,-1.08072759852293)(0.575474330357143,-1.07966280824935)(0.575613839285714,-1.07859703668695)(0.575753348214286,-1.07753028618152)(0.575892857142857,-1.07646255908079)(0.576032366071428,-1.07539385773436)(0.576171875,-1.07432418449374)(0.576311383928571,-1.07325354171234)(0.576450892857143,-1.07218193174544)(0.576590401785714,-1.0711093569502)(0.576729910714286,-1.07003581968568)(0.576869419642857,-1.06896132231279)(0.577008928571428,-1.06788586719432)(0.5771484375,-1.06680945669493)(0.577287946428571,-1.06573209318113)(0.577427455357143,-1.06465377902128)(0.577566964285714,-1.06357451658561)(0.577706473214286,-1.06249430824616)(0.577845982142857,-1.06141315637686)(0.577985491071428,-1.06033106335344)(0.578125,-1.05924803155346)(0.578264508928571,-1.05816406335634)(0.578404017857143,-1.05707916114328)(0.578543526785714,-1.05599332729734)(0.578683035714286,-1.05490656420336)(0.578822544642857,-1.053818874248)(0.578962053571428,-1.05273025981973)(0.5791015625,-1.05164072330881)(0.579241071428571,-1.05055026710731)(0.579380580357143,-1.04945889360907)(0.579520089285714,-1.04836660520972)(0.579659598214286,-1.04727340430669)(0.579799107142857,-1.04617929329915)(0.579938616071428,-1.04508427458808)(0.580078125,-1.0439883505762)(0.580217633928571,-1.042891523668)(0.580357142857143,-1.04179379626973)(0.580496651785714,-1.0406951707894)(0.580636160714286,-1.03959564963674)(0.580775669642857,-1.03849523522325)(0.580915178571428,-1.03739392996217)(0.5810546875,-1.03629173626845)(0.581194196428571,-1.0351886565588)(0.581333705357143,-1.03408469325162)(0.581473214285714,-1.03297984876705)(0.581612723214286,-1.03187412552695)(0.581752232142857,-1.03076752595488)(0.581891741071428,-1.02966005247612)(0.58203125,-1.02855170751761)(0.582170758928571,-1.02744249350804)(0.582310267857143,-1.02633241287776)(0.582449776785714,-1.02522146805881)(0.582589285714286,-1.02410966148492)(0.582728794642857,-1.02299699559149)(0.582868303571428,-1.0218834728156)(0.5830078125,-1.02076909559598)(0.583147321428571,-1.01965386637305)(0.583286830357143,-1.01853778758887)(0.583426339285714,-1.01742086168716)(0.583565848214286,-1.01630309111327)(0.583705357142857,-1.01518447831423)(0.583844866071428,-1.01406502573868)(0.583984375,-1.0129447358369)(0.584123883928571,-1.01182361106081)(0.584263392857143,-1.01070165386395)(0.584402901785714,-1.00957886670148)(0.584542410714286,-1.00845525203016)(0.584681919642857,-1.00733081230839)(0.584821428571428,-1.00620554999615)(0.5849609375,-1.00507946755504)(0.585100446428571,-1.00395256744824)(0.585239955357143,-1.00282485214053)(0.585379464285714,-1.00169632409828)(0.585518973214286,-1.00056698578943)(0.585658482142857,-0.999436839683502)(0.585797991071428,-0.998305888251606)(0.5859375,-0.997174133966394)(0.586077008928571,-0.996041579302099)(0.586216517857143,-0.994908226734499)(0.586356026785714,-0.993774078740937)(0.586495535714286,-0.992639137800292)(0.586635044642857,-0.991503406392995)(0.586774553571428,-0.990366887001017)(0.5869140625,-0.989229582107851)(0.587053571428571,-0.988091494198534)(0.587193080357143,-0.986952625759612)(0.587332589285714,-0.985812979279164)(0.587472098214286,-0.98467255724677)(0.587611607142857,-0.983531362153529)(0.587751116071428,-0.982389396492046)(0.587890625,-0.981246662756414)(0.588030133928571,-0.980103163442235)(0.588169642857143,-0.978958901046588)(0.588309151785714,-0.977813878068051)(0.588448660714286,-0.976668097006666)(0.588588169642857,-0.975521560363964)(0.588727678571428,-0.974374270642944)(0.5888671875,-0.973226230348064)(0.589006696428571,-0.972077441985252)(0.589146205357143,-0.970927908061881)(0.589285714285714,-0.969777631086789)(0.589425223214286,-0.968626613570245)(0.589564732142857,-0.967474858023969)(0.589704241071428,-0.966322366961121)(0.58984375,-0.965169142896279)(0.589983258928571,-0.964015188345462)(0.590122767857143,-0.962860505826096)(0.590262276785714,-0.961705097857041)(0.590401785714286,-0.960548966958551)(0.590541294642857,-0.959392115652301)(0.590680803571428,-0.958234546461366)(0.5908203125,-0.957076261910207)(0.590959821428571,-0.955917264524693)(0.591099330357143,-0.954757556832067)(0.591238839285714,-0.953597141360966)(0.591378348214286,-0.952436020641392)(0.591517857142857,-0.95127419720473)(0.591657366071428,-0.950111673583733)(0.591796875,-0.948948452312505)(0.591936383928571,-0.947784535926524)(0.592075892857143,-0.946619926962604)(0.592215401785714,-0.945454627958923)(0.592354910714286,-0.944288641454988)(0.592494419642857,-0.943121969991654)(0.592633928571428,-0.941954616111108)(0.5927734375,-0.940786582356858)(0.592912946428571,-0.939617871273745)(0.593052455357143,-0.938448485407917)(0.593191964285714,-0.937278427306849)(0.593331473214286,-0.93610769951931)(0.593470982142857,-0.934936304595382)(0.593610491071428,-0.933764245086448)(0.59375,-0.93259152354517)(0.593889508928571,-0.931418142525517)(0.594029017857143,-0.930244104582725)(0.594168526785714,-0.929069412273321)(0.594308035714286,-0.927894068155095)(0.594447544642857,-0.926718074787114)(0.594587053571428,-0.92554143472971)(0.5947265625,-0.924364150544461)(0.594866071428571,-0.923186224794215)(0.595005580357143,-0.922007660043053)(0.595145089285714,-0.920828458856314)(0.595284598214286,-0.919648623800563)(0.595424107142857,-0.918468157443607)(0.595563616071428,-0.917287062354484)(0.595703125,-0.916105341103443)(0.595842633928571,-0.914922996261967)(0.595982142857143,-0.913740030402739)(0.596121651785714,-0.912556446099662)(0.596261160714286,-0.911372245927833)(0.596400669642857,-0.910187432463555)(0.596540178571428,-0.909002008284325)(0.5966796875,-0.907815975968819)(0.596819196428571,-0.906629338096911)(0.596958705357143,-0.905442097249637)(0.597098214285714,-0.904254256009224)(0.597237723214286,-0.903065816959051)(0.597377232142857,-0.901876782683672)(0.597516741071428,-0.900687155768799)(0.59765625,-0.899496938801288)(0.597795758928571,-0.898306134369156)(0.597935267857143,-0.89711474506155)(0.598074776785714,-0.89592277346877)(0.598214285714286,-0.894730222182235)(0.598353794642857,-0.893537093794501)(0.598493303571428,-0.892343390899249)(0.5986328125,-0.891149116091268)(0.598772321428571,-0.889954271966473)(0.598911830357143,-0.888758861121873)(0.599051339285714,-0.887562886155595)(0.599190848214286,-0.886366349666849)(0.599330357142857,-0.885169254255949)(0.599469866071428,-0.883971602524296)(0.599609375,-0.882773397074363)(0.599748883928571,-0.881574640509717)(0.599888392857143,-0.880375335434982)(0.600027901785714,-0.879175484455863)(0.600167410714286,-0.877975090179113)(0.600306919642857,-0.876774155212556)(0.600446428571428,-0.875572682165066)(0.6005859375,-0.874370673646553)(0.600725446428571,-0.873168132267985)(0.600864955357143,-0.871965060641352)(0.601004464285714,-0.870761461379691)(0.601143973214286,-0.86955733709705)(0.601283482142857,-0.868352690408512)(0.601422991071428,-0.867147523930175)(0.6015625,-0.865941840279138)(0.601702008928571,-0.864735642073521)(0.601841517857143,-0.863528931932433)(0.601981026785714,-0.86232171247599)(0.602120535714286,-0.861113986325288)(0.602260044642857,-0.859905756102419)(0.602399553571428,-0.858697024430456)(0.6025390625,-0.857487793933437)(0.602678571428571,-0.856278067236386)(0.602818080357143,-0.855067846965276)(0.602957589285714,-0.853857135747059)(0.603097098214286,-0.852645936209624)(0.603236607142857,-0.851434250981824)(0.603376116071428,-0.850222082693456)(0.603515625,-0.849009433975248)(0.603655133928571,-0.847796307458875)(0.603794642857143,-0.846582705776929)(0.603934151785714,-0.84536863156294)(0.604073660714286,-0.844154087451346)(0.604213169642857,-0.842939076077508)(0.604352678571428,-0.841723600077696)(0.6044921875,-0.840507662089075)(0.604631696428571,-0.839291264749724)(0.604771205357143,-0.838074410698599)(0.604910714285714,-0.836857102575562)(0.605050223214286,-0.835639343021343)(0.605189732142857,-0.834421134677561)(0.605329241071428,-0.833202480186711)(0.60546875,-0.831983382192142)(0.605608258928571,-0.830763843338083)(0.605747767857143,-0.829543866269605)(0.605887276785714,-0.828323453632648)(0.606026785714286,-0.827102608073985)(0.606166294642857,-0.825881332241241)(0.606305803571428,-0.824659628782879)(0.6064453125,-0.823437500348185)(0.606584821428571,-0.822214949587285)(0.606724330357143,-0.820991979151112)(0.606863839285714,-0.819768591691432)(0.607003348214286,-0.818544789860806)(0.607142857142857,-0.817320576312616)(0.607282366071428,-0.816095953701041)(0.607421875,-0.814870924681047)(0.607561383928571,-0.813645491908408)(0.607700892857143,-0.812419658039665)(0.607840401785714,-0.811193425732156)(0.607979910714286,-0.809966797643982)(0.608119419642857,-0.808739776434021)(0.608258928571428,-0.80751236476192)(0.6083984375,-0.806284565288073)(0.608537946428571,-0.805056380673645)(0.608677455357143,-0.803827813580532)(0.608816964285714,-0.802598866671394)(0.608956473214286,-0.801369542609613)(0.609095982142857,-0.800139844059317)(0.609235491071428,-0.798909773685361)(0.609375,-0.797679334153315)(0.609514508928571,-0.79644852812948)(0.609654017857143,-0.795217358280858)(0.609793526785714,-0.793985827275173)(0.609933035714286,-0.792753937780834)(0.610072544642857,-0.791521692466965)(0.610212053571428,-0.790289094003376)(0.6103515625,-0.789056145060559)(0.610491071428571,-0.787822848309698)(0.610630580357143,-0.786589206422642)(0.610770089285714,-0.785355222071926)(0.610909598214286,-0.784120897930737)(0.611049107142857,-0.782886236672933)(0.611188616071428,-0.781651240973031)(0.611328125,-0.780415913506183)(0.611467633928571,-0.779180256948207)(0.611607142857143,-0.777944273975543)(0.611746651785714,-0.776707967265283)(0.611886160714286,-0.775471339495133)(0.612025669642857,-0.774234393343435)(0.612165178571428,-0.772997131489153)(0.6123046875,-0.77175955661185)(0.612444196428571,-0.770521671391717)(0.612583705357143,-0.769283478509532)(0.612723214285714,-0.768044980646689)(0.612862723214286,-0.766806180485157)(0.613002232142857,-0.765567080707508)(0.613141741071428,-0.764327683996895)(0.61328125,-0.763087993037039)(0.613420758928571,-0.761848010512247)(0.613560267857143,-0.760607739107381)(0.613699776785714,-0.759367181507879)(0.613839285714286,-0.758126340399718)(0.613978794642857,-0.756885218469444)(0.614118303571428,-0.755643818404146)(0.6142578125,-0.754402142891442)(0.614397321428571,-0.753160194619503)(0.614536830357143,-0.751917976277017)(0.614676339285714,-0.750675490553209)(0.614815848214286,-0.749432740137812)(0.614955357142857,-0.748189727721086)(0.615094866071428,-0.746946455993799)(0.615234375,-0.745702927647213)(0.615373883928571,-0.744459145373106)(0.615513392857143,-0.743215111863732)(0.615652901785714,-0.741970829811852)(0.615792410714286,-0.740726301910695)(0.615931919642857,-0.739481530853979)(0.616071428571428,-0.738236519335895)(0.6162109375,-0.736991270051092)(0.616350446428571,-0.735745785694697)(0.616489955357143,-0.734500068962276)(0.616629464285714,-0.733254122549868)(0.616768973214286,-0.732007949153939)(0.616908482142857,-0.73076155147141)(0.617047991071428,-0.729514932199639)(0.6171875,-0.728268094036404)(0.617327008928571,-0.727021039679923)(0.617466517857143,-0.725773771828819)(0.617606026785714,-0.724526293182149)(0.617745535714286,-0.723278606439362)(0.617885044642857,-0.722030714300325)(0.618024553571428,-0.720782619465303)(0.6181640625,-0.719534324634946)(0.618303571428571,-0.718285832510308)(0.618443080357143,-0.71703714579281)(0.618582589285714,-0.71578826718427)(0.618722098214286,-0.714539199386863)(0.618861607142857,-0.713289945103141)(0.619001116071428,-0.712040507036024)(0.619140625,-0.710790887888773)(0.619280133928571,-0.709541090365021)(0.619419642857143,-0.708291117168731)(0.619559151785714,-0.707040971004223)(0.619698660714286,-0.705790654576137)(0.619838169642857,-0.70454017058946)(0.619977678571428,-0.703289521749502)(0.6201171875,-0.702038710761882)(0.620256696428571,-0.700787740332552)(0.620396205357143,-0.699536613167757)(0.620535714285714,-0.698285331974064)(0.620675223214286,-0.697033899458323)(0.620814732142857,-0.695782318327691)(0.620954241071428,-0.694530591289614)(0.62109375,-0.693278721051811)(0.621233258928571,-0.692026710322293)(0.621372767857143,-0.690774561809332)(0.621512276785714,-0.689522278221482)(0.621651785714286,-0.688269862267544)(0.621791294642857,-0.687017316656589)(0.621930803571428,-0.685764644097941)(0.6220703125,-0.684511847301157)(0.622209821428571,-0.683258928976053)(0.622349330357143,-0.682005891832668)(0.622488839285714,-0.680752738581283)(0.622628348214286,-0.679499471932395)(0.622767857142857,-0.678246094596728)(0.622907366071428,-0.676992609285225)(0.623046875,-0.675739018709026)(0.623186383928571,-0.674485325579492)(0.623325892857143,-0.673231532608169)(0.623465401785714,-0.67197764250681)(0.623604910714286,-0.670723657987345)(0.623744419642857,-0.669469581761896)(0.623883928571428,-0.668215416542766)(0.6240234375,-0.666961165042418)(0.624162946428571,-0.6657068299735)(0.624302455357143,-0.664452414048807)(0.624441964285714,-0.663197919981304)(0.624581473214286,-0.661943350484097)(0.624720982142857,-0.660688708270446)(0.624860491071428,-0.659433996053757)(0.625,-0.658179216547557)(0.625139508928571,-0.656924372465521)(0.625279017857143,-0.655669466521434)(0.625418526785714,-0.654414501429216)(0.625558035714286,-0.653159479902889)(0.625697544642857,-0.651904404656595)(0.625837053571428,-0.650649278404579)(0.6259765625,-0.649394103861176)(0.626116071428571,-0.648138883740831)(0.626255580357143,-0.646883620758058)(0.626395089285714,-0.645628317627475)(0.626534598214286,-0.64437297706376)(0.626674107142857,-0.643117601781675)(0.626813616071428,-0.641862194496051)(0.626953125,-0.640606757921769)(0.627092633928571,-0.639351294773784)(0.627232142857143,-0.638095807767084)(0.627371651785714,-0.636840299616722)(0.627511160714286,-0.635584773037774)(0.627650669642857,-0.634329230745366)(0.627790178571428,-0.633073675454654)(0.6279296875,-0.631818109880806)(0.628069196428571,-0.630562536739028)(0.628208705357143,-0.629306958744523)(0.628348214285714,-0.62805137861252)(0.628487723214286,-0.626795799058236)(0.628627232142857,-0.6255402227969)(0.628766741071428,-0.624284652543734)(0.62890625,-0.623029091013935)(0.629045758928571,-0.6217735409227)(0.629185267857143,-0.620518004985189)(0.629324776785714,-0.619262485916549)(0.629464285714286,-0.618006986431878)(0.629603794642857,-0.616751509246248)(0.629743303571428,-0.615496057074688)(0.6298828125,-0.614240632632167)(0.630022321428571,-0.612985238633615)(0.630161830357143,-0.611729877793886)(0.630301339285714,-0.610474552827788)(0.630440848214286,-0.609219266450041)(0.630580357142857,-0.607964021375303)(0.630719866071428,-0.60670882031815)(0.630859375,-0.605453665993063)(0.630998883928571,-0.604198561114445)(0.631138392857143,-0.60294350839659)(0.631277901785714,-0.601688510553703)(0.631417410714286,-0.600433570299869)(0.631556919642857,-0.599178690349072)(0.631696428571428,-0.597923873415179)(0.6318359375,-0.59666912221192)(0.631975446428571,-0.595414439452917)(0.632114955357143,-0.59415982785164)(0.632254464285714,-0.592905290121437)(0.632393973214286,-0.591650828975497)(0.632533482142857,-0.590396447126871)(0.632672991071428,-0.589142147288456)(0.6328125,-0.587887932172978)(0.632952008928571,-0.586633804493013)(0.633091517857143,-0.585379766960952)(0.633231026785714,-0.584125822289025)(0.633370535714286,-0.582871973189268)(0.633510044642857,-0.581618222373539)(0.633649553571428,-0.580364572553509)(0.6337890625,-0.579111026440637)(0.633928571428571,-0.577857586746199)(0.634068080357143,-0.576604256181244)(0.634207589285714,-0.575351037456629)(0.634347098214286,-0.574097933282975)(0.634486607142857,-0.572844946370692)(0.634626116071428,-0.571592079429962)(0.634765625,-0.570339335170722)(0.634905133928571,-0.569086716302686)(0.635044642857143,-0.567834225535308)(0.635184151785714,-0.566581865577808)(0.635323660714286,-0.565329639139137)(0.635463169642857,-0.564077548927995)(0.635602678571428,-0.562825597652821)(0.6357421875,-0.561573788021769)(0.635881696428571,-0.560322122742733)(0.636021205357143,-0.559070604523312)(0.636160714285714,-0.557819236070834)(0.636300223214286,-0.556568020092317)(0.636439732142857,-0.555316959294499)(0.636579241071428,-0.554066056383812)(0.63671875,-0.55281531406637)(0.636858258928571,-0.551564735047991)(0.636997767857143,-0.55031432203416)(0.637137276785714,-0.549064077730051)(0.637276785714286,-0.547814004840499)(0.637416294642857,-0.546564106070012)(0.637555803571428,-0.545314384122764)(0.6376953125,-0.54406484170257)(0.637834821428571,-0.542815481512912)(0.637974330357143,-0.541566306256902)(0.638113839285714,-0.540317318637308)(0.638253348214286,-0.539068521356515)(0.638392857142857,-0.537819917116551)(0.638532366071428,-0.53657150861907)(0.638671875,-0.535323298565331)(0.638811383928571,-0.534075289656222)(0.638950892857143,-0.532827484592225)(0.639090401785714,-0.531579886073441)(0.639229910714286,-0.530332496799553)(0.639369419642857,-0.529085319469848)(0.639508928571428,-0.527838356783202)(0.6396484375,-0.52659161143806)(0.639787946428571,-0.525345086132461)(0.639927455357143,-0.524098783563998)(0.640066964285714,-0.522852706429848)(0.640206473214286,-0.521606857426734)(0.640345982142857,-0.520361239250944)(0.640485491071428,-0.519115854598319)(0.640625,-0.517870706164234)(0.640764508928571,-0.516625796643619)(0.640904017857143,-0.515381128730923)(0.641043526785714,-0.514136705120142)(0.641183035714286,-0.512892528504778)(0.641322544642857,-0.511648601577867)(0.641462053571428,-0.510404927031958)(0.6416015625,-0.509161507559097)(0.641741071428571,-0.507918345850849)(0.641880580357143,-0.506675444598261)(0.642020089285714,-0.505432806491891)(0.642159598214286,-0.504190434221766)(0.642299107142857,-0.50294833047741)(0.642438616071428,-0.501706497947824)(0.642578125,-0.500464939321469)(0.642717633928571,-0.499223657286287)(0.642857142857143,-0.497982654529668)(0.642996651785714,-0.496741933738474)(0.643136160714286,-0.495501497599002)(0.643275669642857,-0.494261348797006)(0.643415178571428,-0.493021490017683)(0.6435546875,-0.491781923945652)(0.643694196428571,-0.490542653264977)(0.643833705357143,-0.489303680659135)(0.643973214285714,-0.488065008811035)(0.644112723214286,-0.486826640402986)(0.644252232142857,-0.485588578116719)(0.644391741071428,-0.484350824633369)(0.64453125,-0.483113382633458)(0.644670758928571,-0.481876254796916)(0.644810267857143,-0.480639443803049)(0.644949776785714,-0.479402952330559)(0.645089285714286,-0.478166783057512)(0.645228794642857,-0.476930938661358)(0.645368303571428,-0.475695421818917)(0.6455078125,-0.474460235206355)(0.645647321428571,-0.473225381499216)(0.645786830357143,-0.471990863372376)(0.645926339285714,-0.470756683500078)(0.646065848214286,-0.469522844555887)(0.646205357142857,-0.468289349212719)(0.646344866071428,-0.467056200142819)(0.646484375,-0.465823400017749)(0.646623883928571,-0.464590951508404)(0.646763392857143,-0.463358857284983)(0.646902901785714,-0.462127120017008)(0.647042410714286,-0.460895742373292)(0.647181919642857,-0.459664727021959)(0.647321428571428,-0.458434076630429)(0.6474609375,-0.457203793865399)(0.647600446428571,-0.455973881392868)(0.647739955357143,-0.454744341878095)(0.647879464285714,-0.453515177985631)(0.648018973214286,-0.452286392379282)(0.648158482142857,-0.451057987722127)(0.648297991071428,-0.449829966676503)(0.6484375,-0.448602331903992)(0.648577008928571,-0.447375086065437)(0.648716517857143,-0.446148231820908)(0.648856026785714,-0.444921771829731)(0.648995535714286,-0.443695708750448)(0.649135044642857,-0.442470045240838)(0.649274553571428,-0.441244783957906)(0.6494140625,-0.440019927557861)(0.649553571428571,-0.438795478696137)(0.649693080357143,-0.437571440027364)(0.649832589285714,-0.436347814205386)(0.649972098214286,-0.435124603883228)(0.650111607142857,-0.433901811713119)(0.650251116071428,-0.432679440346474)(0.650390625,-0.431457492433877)(0.650530133928571,-0.430235970625104)(0.650669642857143,-0.429014877569085)(0.650809151785714,-0.427794215913932)(0.650948660714286,-0.426573988306902)(0.651088169642857,-0.425354197394418)(0.651227678571428,-0.424134845822054)(0.6513671875,-0.422915936234516)(0.651506696428571,-0.421697471275667)(0.651646205357143,-0.420479453588488)(0.651785714285714,-0.419261885815105)(0.651925223214286,-0.418044770596753)(0.652064732142857,-0.416828110573798)(0.652204241071428,-0.415611908385721)(0.65234375,-0.414396166671098)(0.652483258928571,-0.413180888067626)(0.652622767857143,-0.411966075212084)(0.652762276785714,-0.410751730740362)(0.652901785714286,-0.409537857287421)(0.653041294642857,-0.408324457487317)(0.653180803571428,-0.407111533973186)(0.6533203125,-0.405899089377224)(0.653459821428571,-0.404687126330709)(0.653599330357143,-0.403475647463969)(0.653738839285714,-0.402264655406403)(0.653878348214286,-0.401054152786449)(0.654017857142857,-0.399844142231602)(0.654157366071428,-0.398634626368401)(0.654296875,-0.397425607822409)(0.654436383928571,-0.396217089218239)(0.654575892857143,-0.395009073179512)(0.654715401785714,-0.39380156232889)(0.654854910714286,-0.392594559288033)(0.654994419642857,-0.391388066677627)(0.655133928571428,-0.390182087117364)(0.6552734375,-0.388976623225923)(0.655412946428571,-0.387771677620999)(0.655552455357143,-0.386567252919259)(0.655691964285714,-0.385363351736376)(0.655831473214286,-0.384159976686985)(0.655970982142857,-0.38295713038471)(0.656110491071428,-0.381754815442144)(0.65625,-0.380553034470839)(0.656389508928571,-0.379351790081317)(0.656529017857143,-0.378151084883046)(0.656668526785714,-0.376950921484456)(0.656808035714286,-0.375751302492908)(0.656947544642857,-0.374552230514716)(0.657087053571428,-0.373353708155129)(0.6572265625,-0.372155738018315)(0.657366071428571,-0.370958322707381)(0.657505580357143,-0.36976146482434)(0.657645089285714,-0.368565166970136)(0.657784598214286,-0.367369431744608)(0.657924107142857,-0.366174261746508)(0.658063616071428,-0.364979659573493)(0.658203125,-0.363785627822099)(0.658342633928571,-0.36259216908777)(0.658482142857143,-0.361399285964818)(0.658621651785714,-0.360206981046451)(0.658761160714286,-0.359015256924735)(0.658900669642857,-0.357824116190617)(0.659040178571428,-0.356633561433912)(0.6591796875,-0.355443595243278)(0.659319196428571,-0.354254220206246)(0.659458705357143,-0.353065438909181)(0.659598214285714,-0.351877253937306)(0.659737723214286,-0.35068966787467)(0.659877232142857,-0.349502683304168)(0.660016741071428,-0.348316302807521)(0.66015625,-0.347130528965267)(0.660295758928571,-0.345945364356776)(0.660435267857143,-0.344760811560218)(0.660574776785714,-0.343576873152587)(0.660714285714286,-0.342393551709667)(0.660853794642857,-0.341210849806051)(0.660993303571428,-0.340028770015128)(0.6611328125,-0.338847314909063)(0.661272321428571,-0.337666487058822)(0.661411830357143,-0.336486289034134)(0.661551339285714,-0.335306723403517)(0.661690848214286,-0.334127792734245)(0.661830357142857,-0.332949499592365)(0.661969866071428,-0.331771846542686)(0.662109375,-0.330594836148756)(0.662248883928571,-0.329418470972892)(0.662388392857143,-0.328242753576136)(0.662527901785714,-0.327067686518287)(0.662667410714286,-0.325893272357861)(0.662806919642857,-0.324719513652118)(0.662946428571428,-0.323546412957038)(0.6630859375,-0.322373972827313)(0.663225446428571,-0.321202195816362)(0.663364955357143,-0.320031084476298)(0.663504464285714,-0.318860641357955)(0.663643973214286,-0.317690869010851)(0.663783482142857,-0.31652176998321)(0.663922991071428,-0.315353346821945)(0.6640625,-0.314185602072641)(0.664202008928571,-0.313018538279581)(0.664341517857143,-0.311852157985705)(0.664481026785714,-0.310686463732638)(0.664620535714286,-0.309521458060656)(0.664760044642857,-0.308357143508705)(0.664899553571428,-0.307193522614386)(0.6650390625,-0.30603059791394)(0.665178571428571,-0.304868371942265)(0.665318080357143,-0.303706847232888)(0.665457589285714,-0.302546026317983)(0.665597098214286,-0.301385911728341)(0.665736607142857,-0.300226505993389)(0.665876116071428,-0.299067811641174)(0.666015625,-0.297909831198347)(0.666155133928571,-0.296752567190186)(0.666294642857143,-0.295596022140558)(0.666434151785714,-0.294440198571947)(0.666573660714286,-0.293285099005417)(0.666713169642857,-0.292130725960635)(0.666852678571428,-0.290977081955851)(0.6669921875,-0.28982416950789)(0.667131696428571,-0.288671991132164)(0.667271205357143,-0.287520549342643)(0.667410714285714,-0.286369846651877)(0.667550223214286,-0.285219885570965)(0.667689732142857,-0.284070668609572)(0.667829241071428,-0.282922198275915)(0.66796875,-0.281774477076747)(0.668108258928571,-0.280627507517377)(0.668247767857143,-0.279481292101638)(0.668387276785714,-0.278335833331909)(0.668526785714286,-0.277191133709081)(0.668666294642857,-0.276047195732579)(0.668805803571428,-0.274904021900346)(0.6689453125,-0.273761614708826)(0.669084821428571,-0.272619976652987)(0.669224330357143,-0.271479110226285)(0.669363839285714,-0.270339017920687)(0.669503348214286,-0.26919970222664)(0.669642857142857,-0.268061165633091)(0.669782366071428,-0.266923410627471)(0.669921875,-0.265786439695677)(0.670061383928571,-0.264650255322095)(0.670200892857143,-0.263514859989568)(0.670340401785714,-0.262380256179414)(0.670479910714286,-0.2612464463714)(0.670619419642857,-0.260113433043755)(0.670758928571428,-0.258981218673159)(0.6708984375,-0.257849805734729)(0.671037946428571,-0.256719196702031)(0.671177455357143,-0.255589394047058)(0.671316964285714,-0.254460400240243)(0.671456473214286,-0.253332217750433)(0.671595982142857,-0.252204849044906)(0.671735491071428,-0.251078296589357)(0.671875,-0.249952562847881)(0.672014508928571,-0.248827650282993)(0.672154017857143,-0.247703561355597)(0.672293526785714,-0.246580298525007)(0.672433035714286,-0.245457864248916)(0.672572544642857,-0.244336260983414)(0.672712053571428,-0.243215491182974)(0.6728515625,-0.242095557300437)(0.672991071428571,-0.240976461787031)(0.673130580357143,-0.239858207092336)(0.673270089285714,-0.238740795664314)(0.673409598214286,-0.237624229949267)(0.673549107142857,-0.236508512391865)(0.673688616071428,-0.235393645435125)(0.673828125,-0.234279631520399)(0.673967633928571,-0.233166473087393)(0.674107142857143,-0.232054172574134)(0.674246651785714,-0.230942732416992)(0.674386160714286,-0.22983215505065)(0.674525669642857,-0.228722442908122)(0.674665178571428,-0.227613598420738)(0.6748046875,-0.226505624018127)(0.674944196428571,-0.225398522128241)(0.675083705357143,-0.22429229517732)(0.675223214285714,-0.223186945589912)(0.675362723214286,-0.222082475788847)(0.675502232142857,-0.220978888195252)(0.675641741071428,-0.219876185228536)(0.67578125,-0.218774369306376)(0.675920758928571,-0.217673442844737)(0.676060267857143,-0.216573408257841)(0.676199776785714,-0.215474267958183)(0.676339285714286,-0.214376024356509)(0.676478794642857,-0.213278679861827)(0.676618303571428,-0.212182236881395)(0.6767578125,-0.211086697820709)(0.676897321428571,-0.209992065083515)(0.677036830357143,-0.208898341071785)(0.677176339285714,-0.207805528185735)(0.677315848214286,-0.206713628823793)(0.677455357142857,-0.20562264538262)(0.677594866071428,-0.204532580257096)(0.677734375,-0.203443435840301)(0.677873883928571,-0.202355214523537)(0.678013392857143,-0.201267918696297)(0.678152901785714,-0.200181550746287)(0.678292410714286,-0.19909611305939)(0.678431919642857,-0.198011608019692)(0.678571428571428,-0.196928038009462)(0.6787109375,-0.195845405409141)(0.678850446428571,-0.194763712597358)(0.678989955357143,-0.193682961950897)(0.679129464285714,-0.192603155844726)(0.679268973214286,-0.191524296651958)(0.679408482142857,-0.190446386743875)(0.679547991071428,-0.189369428489911)(0.6796875,-0.188293424257637)(0.679827008928571,-0.187218376412781)(0.679966517857143,-0.186144287319196)(0.680106026785714,-0.185071159338884)(0.680245535714286,-0.18399899483196)(0.680385044642857,-0.182927796156677)(0.680524553571428,-0.181857565669407)(0.6806640625,-0.180788305724627)(0.680803571428571,-0.179720018674938)(0.680943080357143,-0.178652706871036)(0.681082589285714,-0.177586372661731)(0.681222098214286,-0.176521018393917)(0.681361607142857,-0.175456646412591)(0.681501116071428,-0.174393259060839)(0.681640625,-0.173330858679819)(0.681780133928571,-0.172269447608782)(0.681919642857143,-0.171209028185042)(0.682059151785714,-0.170149602743993)(0.682198660714286,-0.169091173619084)(0.682338169642857,-0.168033743141833)(0.682477678571428,-0.166977313641816)(0.6826171875,-0.16592188744665)(0.682756696428571,-0.164867466882011)(0.682896205357143,-0.163814054271608)(0.683035714285714,-0.162761651937199)(0.683175223214286,-0.161710262198562)(0.683314732142857,-0.160659887373516)(0.683454241071428,-0.159610529777904)(0.68359375,-0.158562191725579)(0.683733258928571,-0.157514875528423)(0.683872767857143,-0.156468583496317)(0.684012276785714,-0.155423317937159)(0.684151785714286,-0.154379081156839)(0.684291294642857,-0.153335875459254)(0.684430803571428,-0.152293703146293)(0.6845703125,-0.151252566517825)(0.684709821428571,-0.150212467871716)(0.684849330357143,-0.149173409503799)(0.684988839285714,-0.148135393707895)(0.685128348214286,-0.147098422775783)(0.685267857142857,-0.146062498997218)(0.685407366071428,-0.145027624659918)(0.685546875,-0.143993802049547)(0.685686383928571,-0.142961033449736)(0.685825892857143,-0.141929321142051)(0.685965401785714,-0.140898667406017)(0.686104910714286,-0.139869074519084)(0.686244419642857,-0.138840544756647)(0.686383928571428,-0.137813080392033)(0.6865234375,-0.136786683696484)(0.686662946428571,-0.13576135693918)(0.686802455357143,-0.134737102387201)(0.686941964285714,-0.133713922305558)(0.687081473214286,-0.132691818957155)(0.687220982142857,-0.131670794602811)(0.687360491071428,-0.130650851501244)(0.6875,-0.12963199190906)(0.687639508928571,-0.128614218080766)(0.687779017857143,-0.127597532268746)(0.687918526785714,-0.126581936723278)(0.688058035714286,-0.125567433692504)(0.688197544642857,-0.124554025422451)(0.688337053571428,-0.123541714157016)(0.6884765625,-0.122530502137948)(0.688616071428571,-0.121520391604872)(0.688755580357143,-0.120511384795257)(0.688895089285714,-0.119503483944435)(0.689034598214286,-0.118496691285572)(0.689174107142857,-0.117491009049689)(0.689313616071428,-0.116486439465645)(0.689453125,-0.115482984760123)(0.689592633928571,-0.11448064715765)(0.689732142857143,-0.113479428880566)(0.689871651785714,-0.112479332149043)(0.690011160714286,-0.111480359181061)(0.690150669642857,-0.110482512192421)(0.690290178571428,-0.10948579339673)(0.6904296875,-0.108490205005395)(0.690569196428571,-0.107495749227629)(0.690708705357143,-0.106502428270434)(0.690848214285714,-0.105510244338613)(0.690987723214286,-0.104519199634744)(0.691127232142857,-0.103529296359196)(0.691266741071428,-0.102540536710118)(0.69140625,-0.101552922883424)(0.691545758928571,-0.100566457072809)(0.691685267857143,-0.0995811414697217)(0.691824776785714,-0.0985969782633855)(0.691964285714286,-0.0976139696407677)(0.692103794642857,-0.0966321177865976)(0.692243303571428,-0.0956514248833538)(0.6923828125,-0.0946718931112491)(0.692522321428571,-0.0936935246482498)(0.692661830357143,-0.0927163216700447)(0.692801339285714,-0.0917402863500676)(0.692940848214286,-0.0907654208594664)(0.693080357142857,-0.0897917273671222)(0.693219866071428,-0.0888192080396347)(0.693359375,-0.0878478650413093)(0.693498883928571,-0.086877700534174)(0.693638392857143,-0.0859087166779511)(0.693777901785714,-0.0849409156300771)(0.693917410714286,-0.0839742995456744)(0.694056919642857,-0.083008870577568)(0.694196428571428,-0.0820446308762729)(0.6943359375,-0.0810815825899798)(0.694475446428571,-0.0801197278645731)(0.694614955357143,-0.079159068843602)(0.694754464285714,-0.0781996076683006)(0.694893973214286,-0.0772413464775594)(0.695033482142857,-0.0762842874079429)(0.695172991071428,-0.0753284325936755)(0.6953125,-0.074373784166629)(0.695452008928571,-0.0734203442563393)(0.695591517857143,-0.0724681149899778)(0.695731026785714,-0.0715170984923722)(0.695870535714286,-0.0705672968859775)(0.696010044642857,-0.0696187122908932)(0.696149553571428,-0.0686713468248503)(0.6962890625,-0.0677252026031979)(0.696428571428571,-0.0667802817389204)(0.696568080357143,-0.0658365863426089)(0.696707589285714,-0.0648941185224816)(0.696847098214286,-0.0639528803843558)(0.696986607142857,-0.0630128740316638)(0.697126116071428,-0.0620741015654411)(0.697265625,-0.0611365650843126)(0.697405133928571,-0.0602002666845095)(0.697544642857143,-0.0592652084598413)(0.697684151785714,-0.058331392501716)(0.697823660714286,-0.0573988208991119)(0.697963169642857,-0.0564674957385939)(0.698102678571428,-0.0555374191043014)(0.6982421875,-0.054608593077935)(0.698381696428571,-0.0536810197387727)(0.698521205357143,-0.052754701163643)(0.698660714285714,-0.051829639426944)(0.698800223214286,-0.0509058366006157)(0.698939732142857,-0.0499832947541569)(0.699079241071428,-0.049062015954612)(0.69921875,-0.0481420022665586)(0.699358258928571,-0.0472232557521238)(0.699497767857143,-0.0463057784709564)(0.699637276785714,-0.0453895724802469)(0.699776785714286,-0.0444746398346997)(0.699916294642857,-0.0435609825865497)(0.700055803571428,-0.0426486027855497)(0.7001953125,-0.0417375024789572)(0.700334821428571,-0.0408276837115513)(0.700474330357143,-0.0399191485256052)(0.700613839285714,-0.0390118989609054)(0.700753348214286,-0.0381059370547249)(0.700892857142857,-0.0372012648418391)(0.701032366071428,-0.036297884354513)(0.701171875,-0.0353957976224895)(0.701311383928571,-0.0344950066730045)(0.701450892857143,-0.0335955135307608)(0.701590401785714,-0.0326973202179466)(0.701729910714286,-0.0318004287542087)(0.701869419642857,-0.0309048411566691)(0.702008928571428,-0.0300105594399118)(0.7021484375,-0.0291175856159708)(0.702287946428571,-0.0282259216943459)(0.702427455357143,-0.027335569681976)(0.702566964285714,-0.0264465315832584)(0.702706473214286,-0.0255588094000213)(0.702845982142857,-0.0246724051315407)(0.702985491071428,-0.0237873207745273)(0.703125,-0.0229035583231145)(0.703264508928571,-0.0220211197688743)(0.703404017857143,-0.0211400071007908)(0.703543526785714,-0.0202602223052784)(0.703683035714286,-0.0193817673661564)(0.703822544642857,-0.0185046442646638)(0.703962053571428,-0.0176288549794479)(0.7041015625,-0.0167544014865513)(0.704241071428571,-0.015881285759428)(0.704380580357143,-0.0150095097689175)(0.704520089285714,-0.0141390754832625)(0.704659598214286,-0.0132699848680835)(0.704799107142857,-0.0124022398863942)(0.704938616071428,-0.0115358424985893)(0.705078125,-0.0106707946624321)(0.705217633928571,-0.00980709833307103)(0.705357142857143,-0.00894475546301243)(0.705496651785714,-0.00808376800213984)(0.705636160714286,-0.0072241378976875)(0.705775669642857,-0.00636586709425646)(0.705915178571428,-0.00550895753380198)(0.7060546875,-0.00465341115562168)(0.706194196428571,-0.00379922989637116)(0.706333705357143,-0.00294641569003817)(0.706473214285714,-0.00209497046796059)(0.706612723214286,-0.00124489615880119)(0.706752232142857,-0.000396194688562557)(0.706891741071428,0.000451132019424594)(0.70703125,0.0012970820445134)(0.707170758928571,0.0021416534687253)(0.707310267857143,0.00298484437677615)(0.707449776785714,0.00382665285605743)(0.707589285714286,0.00466707699666224)(0.707728794642857,0.00550611489136976)(0.707868303571428,0.00634376463565733)(0.7080078125,0.00718002432771203)(0.708147321428571,0.00801489206841555)(0.708286830357143,0.0088483659613694)(0.708426339285714,0.00968044411287738)(0.708565848214286,0.0105111246319702)(0.708705357142857,0.0113404056303911)(0.708844866071428,0.0121682852226068)(0.708984375,0.0129947615258197)(0.709123883928571,0.0138198326599528)(0.709263392857143,0.0146434967476745)(0.709402901785714,0.015465751914381)(0.709542410714286,0.0162865962882216)(0.709681919642857,0.0171060280000828)(0.709821428571428,0.0179240451836011)(0.7099609375,0.0187406459751739)(0.710100446428571,0.0195558285139446)(0.710239955357143,0.0203695909418272)(0.710379464285714,0.0211819314034897)(0.710518973214286,0.0219928480463777)(0.710658482142857,0.0228023390207002)(0.710797991071428,0.0236104024794409)(0.7109375,0.0244170365783697)(0.711077008928571,0.0252222394760282)(0.711216517857143,0.0260260093337534)(0.711356026785714,0.0268283443156616)(0.711495535714286,0.0276292425886717)(0.711635044642857,0.0284287023224913)(0.711774553571428,0.0292267216896277)(0.7119140625,0.0300232988653992)(0.712053571428571,0.030818432027921)(0.712193080357143,0.0316121193581286)(0.712332589285714,0.0324043590397616)(0.712472098214286,0.033195149259387)(0.712611607142857,0.0339844882063857)(0.712751116071428,0.0347723740729626)(0.712890625,0.0355588050541588)(0.713030133928571,0.0363437793478364)(0.713169642857143,0.0371272951547033)(0.713309151785714,0.0379093506782953)(0.713448660714286,0.0386899441250008)(0.713588169642857,0.0394690737040462)(0.713727678571428,0.0402467376275069)(0.7138671875,0.0410229341103182)(0.714006696428571,0.0417976613702621)(0.714146205357143,0.0425709176279893)(0.714285714285714,0.0433427011070039)(0.714425223214286,0.0441130100336865)(0.714564732142857,0.0448818426372798)(0.714704241071428,0.0456491971499)(0.71484375,0.0464150718065476)(0.714983258928571,0.0471794648450933)(0.715122767857143,0.0479423745063011)(0.715262276785714,0.0487037990338127)(0.715401785714286,0.0494637366741697)(0.715541294642857,0.0502221856768004)(0.715680803571428,0.0509791442940303)(0.7158203125,0.051734610781093)(0.715959821428571,0.0524885833961165)(0.716099330357143,0.0532410604001459)(0.716238839285714,0.0539920400571274)(0.716378348214286,0.0547415206339313)(0.716517857142857,0.0554895004003384)(0.716657366071428,0.05623597762905)(0.716796875,0.0569809505956994)(0.716936383928571,0.0577244175788378)(0.717075892857143,0.0584663768599573)(0.717215401785714,0.0592068267234744)(0.717354910714286,0.0599457654567532)(0.717494419642857,0.0606831913500918)(0.717633928571428,0.0614191026967323)(0.7177734375,0.0621534977928722)(0.717912946428571,0.0628863749376499)(0.718052455357143,0.0636177324331683)(0.718191964285714,0.0643475685844779)(0.718331473214286,0.0650758816996001)(0.718470982142857,0.065802670089513)(0.718610491071428,0.0665279320681628)(0.71875,0.0672516659524729)(0.718889508928571,0.067973870062332)(0.719029017857143,0.068694542720615)(0.719168526785714,0.0694136822531687)(0.719308035714286,0.0701312869888324)(0.719447544642857,0.0708473552594262)(0.719587053571428,0.0715618853997602)(0.7197265625,0.0722748757476449)(0.719866071428571,0.0729863246438791)(0.720005580357143,0.0736962304322702)(0.720145089285714,0.0744045914596202)(0.720284598214286,0.0751114060757464)(0.720424107142857,0.0758166726334693)(0.720563616071428,0.0765203894886222)(0.720703125,0.0772225550000614)(0.720842633928571,0.0779231675296533)(0.720982142857143,0.0786222254422964)(0.721121651785714,0.0793197271059052)(0.721261160714286,0.0800156708914325)(0.721400669642857,0.0807100551728562)(0.721540178571428,0.08140287832719)(0.7216796875,0.0820941387344917)(0.721819196428571,0.0827838347778526)(0.721958705357143,0.0834719648434171)(0.722098214285714,0.0841585273203688)(0.722237723214286,0.0848435206009507)(0.722377232142857,0.0855269430804538)(0.722516741071428,0.0862087931572261)(0.72265625,0.0868890692326822)(0.722795758928571,0.087567769711292)(0.722935267857143,0.0882448930006001)(0.723074776785714,0.0889204375112123)(0.723214285714286,0.0895944016568158)(0.723353794642857,0.0902667838541671)(0.723493303571428,0.0909375825231017)(0.7236328125,0.0916067960865433)(0.723772321428571,0.0922744229704923)(0.723911830357143,0.0929404616040457)(0.724051339285714,0.0936049104193831)(0.724190848214286,0.0942677678517868)(0.724330357142857,0.0949290323396302)(0.724469866071428,0.0955887023243868)(0.724609375,0.0962467762506398)(0.724748883928571,0.0969032525660704)(0.724888392857143,0.0975581297214776)(0.725027901785714,0.0982114061707643)(0.725167410714286,0.0988630803709574)(0.725306919642857,0.0995131507821957)(0.725446428571428,0.10016161586774)(0.7255859375,0.10080847409398)(0.725725446428571,0.101453723930427)(0.725864955357143,0.10209736384973)(0.726004464285714,0.102739392327661)(0.726143973214286,0.10337980784314)(0.726283482142857,0.104018608878218)(0.726422991071428,0.104655793918089)(0.7265625,0.105291361451097)(0.726702008928571,0.105925309968728)(0.726841517857143,0.106557637965625)(0.726981026785714,0.107188343939578)(0.727120535714286,0.107817426391543)(0.727260044642857,0.108444883825629)(0.727399553571428,0.109070714749105)(0.7275390625,0.109694917672417)(0.727678571428571,0.110317491109167)(0.727818080357143,0.110938433576139)(0.727957589285714,0.111557743593281)(0.728097098214286,0.112175419683729)(0.728236607142857,0.11279146037379)(0.728376116071428,0.113405864192958)(0.728515625,0.114018629673914)(0.728655133928571,0.114629755352523)(0.728794642857143,0.115239239767849)(0.728934151785714,0.115847081462141)(0.729073660714286,0.116453278980855)(0.729213169642857,0.117057830872639)(0.729352678571428,0.117660735689345)(0.7294921875,0.118261991986038)(0.729631696428571,0.118861598320981)(0.729771205357143,0.119459553255657)(0.729910714285714,0.120055855354756)(0.730050223214286,0.120650503186192)(0.730189732142857,0.121243495321093)(0.730329241071428,0.12183483033381)(0.73046875,0.122424506801923)(0.730608258928571,0.123012523306236)(0.730747767857143,0.123598878430786)(0.730887276785714,0.124183570762842)(0.731026785714286,0.124766598892913)(0.731166294642857,0.125347961414741)(0.731305803571428,0.125927656925311)(0.7314453125,0.12650568402486)(0.731584821428571,0.127082041316861)(0.731724330357143,0.127656727408046)(0.731863839285714,0.128229740908393)(0.732003348214286,0.128801080431142)(0.732142857142857,0.129370744592785)(0.732282366071428,0.129938732013075)(0.732421875,0.130505041315034)(0.732561383928571,0.131069671124944)(0.732700892857143,0.13163262007236)(0.732840401785714,0.132193886790104)(0.732979910714286,0.132753469914279)(0.733119419642857,0.133311368084258)(0.733258928571428,0.133867579942693)(0.7333984375,0.134422104135526)(0.733537946428571,0.134974939311975)(0.733677455357143,0.135526084124551)(0.733816964285714,0.136075537229051)(0.733956473214286,0.136623297284568)(0.734095982142857,0.137169362953487)(0.734235491071428,0.137713732901491)(0.734375,0.138256405797568)(0.734514508928571,0.138797380314001)(0.734654017857143,0.139336655126387)(0.734793526785714,0.139874228913624)(0.734933035714286,0.140410100357927)(0.735072544642857,0.140944268144818)(0.735212053571428,0.141476730963136)(0.7353515625,0.142007487505045)(0.735491071428571,0.142536536466021)(0.735630580357143,0.143063876544869)(0.735770089285714,0.143589506443717)(0.735909598214286,0.144113424868025)(0.736049107142857,0.14463563052658)(0.736188616071428,0.145156122131504)(0.736328125,0.145674898398256)(0.736467633928571,0.146191958045631)(0.736607142857143,0.146707299795771)(0.736746651785714,0.147220922374152)(0.736886160714286,0.147732824509606)(0.737025669642857,0.148243004934306)(0.737165178571428,0.148751462383778)(0.7373046875,0.149258195596905)(0.737444196428571,0.149763203315919)(0.737583705357143,0.150266484286418)(0.737723214285714,0.150768037257354)(0.737862723214286,0.151267860981047)(0.738002232142857,0.15176595421318)(0.738141741071428,0.152262315712802)(0.73828125,0.15275694424234)(0.738420758928571,0.153249838567584)(0.738560267857143,0.153740997457708)(0.738699776785714,0.154230419685257)(0.738839285714286,0.154718104026161)(0.738978794642857,0.155204049259729)(0.739118303571428,0.155688254168654)(0.7392578125,0.156170717539022)(0.739397321428571,0.156651438160302)(0.739536830357143,0.157130414825359)(0.739676339285714,0.157607646330448)(0.739815848214286,0.158083131475227)(0.739955357142857,0.158556869062748)(0.740094866071428,0.159028857899462)(0.740234375,0.159499096795232)(0.740373883928571,0.159967584563317)(0.740513392857143,0.160434320020393)(0.740652901785714,0.16089930198654)(0.740792410714286,0.161362529285255)(0.740931919642857,0.161824000743447)(0.741071428571428,0.162283715191442)(0.7412109375,0.162741671462992)(0.741350446428571,0.163197868395261)(0.741489955357143,0.163652304828845)(0.741629464285714,0.164104979607763)(0.741768973214286,0.164555891579465)(0.741908482142857,0.165005039594828)(0.742047991071428,0.165452422508165)(0.7421875,0.165898039177224)(0.742327008928571,0.16634188846319)(0.742466517857143,0.16678396923069)(0.742606026785714,0.167224280347789)(0.742745535714286,0.167662820686)(0.742885044642857,0.16809958912028)(0.743024553571428,0.168534584529034)(0.7431640625,0.168967805794123)(0.743303571428571,0.169399251800853)(0.743443080357143,0.169828921437994)(0.743582589285714,0.170256813597763)(0.743722098214286,0.170682927175848)(0.743861607142857,0.17110726107139)(0.744001116071428,0.171529814186994)(0.744140625,0.171950585428738)(0.744280133928571,0.17236957370616)(0.744419642857143,0.172786777932274)(0.744559151785714,0.173202197023561)(0.744698660714286,0.173615829899982)(0.744838169642857,0.174027675484971)(0.744977678571428,0.174437732705438)(0.7451171875,0.174846000491781)(0.745256696428571,0.175252477777874)(0.745396205357143,0.17565716350108)(0.745535714285714,0.176060056602244)(0.745675223214286,0.176461156025708)(0.745814732142857,0.176860460719297)(0.745954241071428,0.177257969634332)(0.74609375,0.17765368172563)(0.746233258928571,0.178047595951504)(0.746372767857143,0.178439711273768)(0.746512276785714,0.178830026657734)(0.746651785714286,0.179218541072219)(0.746791294642857,0.179605253489546)(0.746930803571428,0.179990162885542)(0.7470703125,0.180373268239549)(0.747209821428571,0.180754568534412)(0.747349330357143,0.181134062756498)(0.747488839285714,0.181511749895681)(0.747628348214286,0.181887628945359)(0.747767857142857,0.182261698902444)(0.747907366071428,0.182633958767371)(0.748046875,0.183004407544099)(0.748186383928571,0.183373044240109)(0.748325892857143,0.183739867866412)(0.748465401785714,0.184104877437544)(0.748604910714286,0.184468071971577)(0.748744419642857,0.184829450490111)(0.748883928571428,0.185189012018281)(0.7490234375,0.18554675558476)(0.749162946428571,0.185902680221758)(0.749302455357143,0.186256784965027)(0.749441964285714,0.186609068853858)(0.749581473214286,0.186959530931089)(0.749720982142857,0.187308170243101)(0.749860491071428,0.187654985839824)(0.75,0.187999976774739)(0.750139508928571,0.188343142104874)(0.750279017857143,0.188684480890815)(0.750418526785714,0.189023992196699)(0.750558035714286,0.189361675090223)(0.750697544642857,0.18969752864264)(0.750837053571428,0.190031551928764)(0.7509765625,0.190363744026973)(0.751116071428571,0.190694104019206)(0.751255580357143,0.191022630990972)(0.751395089285714,0.191349324031343)(0.751534598214286,0.191674182232963)(0.751674107142857,0.191997204692046)(0.751813616071428,0.192318390508379)(0.751953125,0.192637738785325)(0.752092633928571,0.192955248629821)(0.752232142857143,0.193270919152385)(0.752371651785714,0.19358474946711)(0.752511160714286,0.193896738691678)(0.752650669642857,0.194206885947347)(0.752790178571428,0.194515190358964)(0.7529296875,0.194821651054961)(0.753069196428571,0.195126267167359)(0.753208705357143,0.195429037831771)(0.753348214285714,0.195729962187396)(0.753487723214286,0.196029039377035)(0.753627232142857,0.196326268547075)(0.753766741071428,0.196621648847506)(0.75390625,0.196915179431916)(0.754045758928571,0.197206859457488)(0.754185267857143,0.197496688085013)(0.754324776785714,0.197784664478882)(0.754464285714286,0.198070787807091)(0.754603794642857,0.198355057241244)(0.754743303571428,0.19863747195655)(0.7548828125,0.198918031131833)(0.755022321428571,0.199196733949524)(0.755161830357143,0.199473579595668)(0.755301339285714,0.199748567259926)(0.755440848214286,0.200021696135574)(0.755580357142857,0.200292965419505)(0.755719866071428,0.200562374312233)(0.755859375,0.200829922017892)(0.755998883928571,0.201095607744237)(0.756138392857143,0.20135943070265)(0.756277901785714,0.201621390108136)(0.756417410714286,0.201881485179328)(0.756556919642857,0.202139715138486)(0.756696428571428,0.202396079211503)(0.7568359375,0.2026505766279)(0.756975446428571,0.202903206620834)(0.757114955357143,0.203153968427095)(0.757254464285714,0.203402861287109)(0.757393973214286,0.20364988444494)(0.757533482142857,0.203895037148291)(0.757672991071428,0.204138318648505)(0.7578125,0.204379728200567)(0.757952008928571,0.204619265063104)(0.758091517857143,0.204856928498392)(0.758231026785714,0.205092717772347)(0.758370535714286,0.205326632154539)(0.758510044642857,0.205558670918182)(0.758649553571428,0.205788833340142)(0.7587890625,0.20601711870094)(0.758928571428571,0.206243526284745)(0.759068080357143,0.206468055379385)(0.759207589285714,0.206690705276341)(0.759347098214286,0.206911475270753)(0.759486607142857,0.207130364661421)(0.759626116071428,0.207347372750802)(0.759765625,0.207562498845017)(0.759905133928571,0.207775742253849)(0.760044642857143,0.207987102290746)(0.760184151785714,0.20819657827282)(0.760323660714286,0.208404169520851)(0.760463169642857,0.208609875359288)(0.760602678571428,0.208813695116247)(0.7607421875,0.209015628123517)(0.760881696428571,0.209215673716559)(0.761021205357143,0.209413831234507)(0.761160714285714,0.209610100020168)(0.761300223214286,0.209804479420029)(0.761439732142857,0.20999696878425)(0.761579241071428,0.210187567466673)(0.76171875,0.210376274824817)(0.761858258928571,0.210563090219883)(0.761997767857143,0.210748013016756)(0.762137276785714,0.210931042584001)(0.762276785714286,0.211112178293872)(0.762416294642857,0.211291419522306)(0.762555803571428,0.211468765648926)(0.7626953125,0.211644216057048)(0.762834821428571,0.211817770133674)(0.762974330357143,0.211989427269497)(0.763113839285714,0.212159186858903)(0.763253348214286,0.212327048299972)(0.763392857142857,0.212493010994475)(0.763532366071428,0.212657074347881)(0.763671875,0.212819237769357)(0.763811383928571,0.212979500671764)(0.763950892857143,0.213137862471666)(0.764090401785714,0.213294322589322)(0.764229910714286,0.213448880448698)(0.764369419642857,0.213601535477459)(0.764508928571428,0.213752287106973)(0.7646484375,0.213901134772314)(0.764787946428571,0.21404807791226)(0.764927455357143,0.214193115969299)(0.765066964285714,0.214336248389623)(0.765206473214286,0.214477474623135)(0.765345982142857,0.214616794123448)(0.765485491071428,0.214754206347885)(0.765625,0.214889710757482)(0.765764508928571,0.215023306816986)(0.765904017857143,0.215154993994863)(0.766043526785714,0.215284771763289)(0.766183035714286,0.215412639598159)(0.766322544642857,0.215538596979085)(0.766462053571428,0.215662643389395)(0.7666015625,0.215784778316141)(0.766741071428571,0.215905001250089)(0.766880580357143,0.216023311685734)(0.767020089285714,0.216139709121285)(0.767159598214286,0.21625419305868)(0.767299107142857,0.216366763003579)(0.767438616071428,0.216477418465367)(0.767578125,0.216586158957157)(0.767717633928571,0.216692983995787)(0.767857142857143,0.216797893101823)(0.767996651785714,0.21690088579956)(0.768136160714286,0.217001961617026)(0.768275669642857,0.217101120085975)(0.768415178571428,0.217198360741896)(0.7685546875,0.217293683124008)(0.768694196428571,0.217387086775267)(0.768833705357143,0.21747857124236)(0.768973214285714,0.217568136075711)(0.769112723214286,0.21765578082948)(0.769252232142857,0.217741505061563)(0.769391741071428,0.217825308333595)(0.76953125,0.21790719021095)(0.769670758928571,0.21798715026274)(0.769810267857143,0.218065188061819)(0.769949776785714,0.21814130318478)(0.770089285714286,0.218215495211961)(0.770228794642857,0.218287763727441)(0.770368303571428,0.218358108319043)(0.7705078125,0.218426528578335)(0.770647321428571,0.21849302410063)(0.770786830357143,0.218557594484986)(0.770926339285714,0.218620239334209)(0.771065848214286,0.218680958254853)(0.771205357142857,0.21873975085722)(0.771344866071428,0.218796616755359)(0.771484375,0.218851555567073)(0.771623883928571,0.218904566913913)(0.771763392857143,0.218955650421182)(0.771902901785714,0.219004805717933)(0.772042410714286,0.219052032436976)(0.772181919642857,0.219097330214872)(0.772321428571428,0.219140698691935)(0.7724609375,0.219182137512237)(0.772600446428571,0.219221646323604)(0.772739955357143,0.219259224777618)(0.772879464285714,0.219294872529619)(0.773018973214286,0.219328589238703)(0.773158482142857,0.219360374567727)(0.773297991071428,0.219390228183304)(0.7734375,0.219418149755809)(0.773577008928571,0.219444138959377)(0.773716517857143,0.219468195471902)(0.773856026785714,0.219490318975041)(0.773995535714286,0.219510509154214)(0.774135044642857,0.219528765698602)(0.774274553571428,0.219545088301151)(0.7744140625,0.219559476658569)(0.774553571428571,0.219571930471331)(0.774693080357143,0.219582449443675)(0.774832589285714,0.219591033283606)(0.774972098214286,0.219597681702894)(0.775111607142857,0.219602394417079)(0.775251116071428,0.219605171145464)(0.775390625,0.219606011611123)(0.775530133928571,0.219604915540898)(0.775669642857143,0.219601882665399)(0.775809151785714,0.219596912719006)(0.775948660714286,0.219590005439871)(0.776088169642857,0.219581160569913)(0.776227678571428,0.219570377854825)(0.7763671875,0.21955765704407)(0.776506696428571,0.219542997890884)(0.776646205357143,0.219526400152275)(0.776785714285714,0.219507863589024)(0.776925223214286,0.219487387965686)(0.777064732142857,0.219464973050589)(0.777204241071428,0.219440618615837)(0.77734375,0.219414324437307)(0.777483258928571,0.219386090294652)(0.777622767857143,0.219355915971302)(0.777762276785714,0.21932380125446)(0.777901785714286,0.219289745935109)(0.778041294642857,0.219253749808007)(0.778180803571428,0.21921581267169)(0.7783203125,0.21917593432847)(0.778459821428571,0.219134114584441)(0.778599330357143,0.219090353249471)(0.778738839285714,0.219044650137211)(0.778878348214286,0.218997005065087)(0.779017857142857,0.218947417854308)(0.779157366071428,0.218895888329861)(0.779296875,0.218842416320515)(0.779436383928571,0.218787001658818)(0.779575892857143,0.2187296441811)(0.779715401785714,0.218670343727471)(0.779854910714286,0.218609100141824)(0.779994419642857,0.218545913271833)(0.780133928571428,0.218480782968954)(0.7802734375,0.218413709088426)(0.780412946428571,0.218344691489272)(0.780552455357143,0.218273730034294)(0.780691964285714,0.218200824590082)(0.780831473214286,0.218125975027006)(0.780970982142857,0.218049181219222)(0.781110491071428,0.217970443044669)(0.78125,0.21788976038507)(0.781389508928571,0.217807133125934)(0.781529017857143,0.217722561156552)(0.781668526785714,0.217636044370004)(0.781808035714286,0.217547582663151)(0.781947544642857,0.217457175936641)(0.782087053571428,0.217364824094909)(0.7822265625,0.217270527046173)(0.782366071428571,0.21717428470244)(0.782505580357143,0.2170760969795)(0.782645089285714,0.216975963796931)(0.782784598214286,0.216873885078096)(0.782924107142857,0.216769860750148)(0.783063616071428,0.216663890744022)(0.783203125,0.216555974994443)(0.783342633928571,0.216446113439922)(0.783482142857143,0.216334306022757)(0.783621651785714,0.216220552689034)(0.783761160714286,0.216104853388624)(0.783900669642857,0.215987208075188)(0.784040178571428,0.215867616706173)(0.7841796875,0.215746079242814)(0.784319196428571,0.215622595650135)(0.784458705357143,0.215497165896944)(0.784598214285714,0.215369789955841)(0.784737723214286,0.21524046780321)(0.784877232142857,0.215109199419226)(0.785016741071428,0.214975984787851)(0.78515625,0.214840823896834)(0.785295758928571,0.214703716737712)(0.785435267857143,0.214564663305812)(0.785574776785714,0.214423663600246)(0.785714285714286,0.214280717623915)(0.785853794642857,0.214135825383511)(0.785993303571428,0.213988986889509)(0.7861328125,0.213840202156175)(0.786272321428571,0.213689471201563)(0.786411830357143,0.213536794047513)(0.786551339285714,0.213382170719655)(0.786690848214286,0.213225601247405)(0.786830357142857,0.213067085663968)(0.786969866071428,0.212906624006336)(0.787109375,0.212744216315288)(0.787248883928571,0.212579862635393)(0.787388392857143,0.212413563015003)(0.787527901785714,0.212245317506263)(0.787667410714286,0.212075126165101)(0.787806919642857,0.211902989051232)(0.787946428571428,0.211728906228162)(0.7880859375,0.211552877763179)(0.788225446428571,0.211374903727362)(0.788364955357143,0.211194984195573)(0.788504464285714,0.211013119246463)(0.788643973214286,0.210829308962467)(0.788783482142857,0.210643553429809)(0.788922991071428,0.210455852738498)(0.7890625,0.210266206982327)(0.789202008928571,0.210074616258877)(0.789341517857143,0.209881080669512)(0.789481026785714,0.209685600319385)(0.789620535714286,0.209488175317429)(0.789760044642857,0.209288805776367)(0.789899553571428,0.209087491812703)(0.7900390625,0.208884233546726)(0.790178571428571,0.208679031102511) 
};
\addplot [
color=blue,
solid,
forget plot
]
coordinates{
 (0.790178571428571,0.208679031102511)(0.790318080357143,0.208471884607914)(0.790457589285714,0.208262794194578)(0.790597098214286,0.208051759997925)(0.790736607142857,0.207838782157166)(0.790876116071428,0.207623860815289)(0.791015625,0.207406996119069)(0.791155133928571,0.207188188219061)(0.791294642857143,0.206967437269601)(0.791434151785714,0.206744743428811)(0.791573660714286,0.206520106858588)(0.791713169642857,0.206293527724617)(0.791852678571428,0.20606500619636)(0.7919921875,0.205834542447058)(0.792131696428571,0.205602136653736)(0.792271205357143,0.205367788997195)(0.792410714285714,0.205131499662019)(0.792550223214286,0.204893268836567)(0.792689732142857,0.20465309671298)(0.792829241071428,0.204410983487177)(0.79296875,0.204166929358852)(0.793108258928571,0.203920934531481)(0.793247767857143,0.203672999212313)(0.793387276785714,0.203423123612376)(0.793526785714286,0.203171307946474)(0.793666294642857,0.202917552433185)(0.793805803571428,0.202661857294867)(0.7939453125,0.202404222757648)(0.794084821428571,0.202144649051434)(0.794224330357143,0.201883136409903)(0.794363839285714,0.201619685070508)(0.794503348214286,0.201354295274474)(0.794642857142857,0.201086967266801)(0.794782366071428,0.200817701296261)(0.794921875,0.200546497615393)(0.795061383928571,0.200273356480514)(0.795200892857143,0.199998278151708)(0.795340401785714,0.199721262892831)(0.795479910714286,0.199442310971506)(0.795619419642857,0.199161422659129)(0.795758928571428,0.198878598230863)(0.7958984375,0.198593837965637)(0.796037946428571,0.198307142146154)(0.796177455357143,0.198018511058875)(0.796316964285714,0.197727944994037)(0.796456473214286,0.197435444245635)(0.796595982142857,0.197141009111435)(0.796735491071428,0.196844639892966)(0.796875,0.19654633689552)(0.797014508928571,0.196246100428154)(0.797154017857143,0.195943930803688)(0.797293526785714,0.195639828338705)(0.797433035714286,0.195333793353548)(0.797572544642857,0.195025826172322)(0.797712053571428,0.194715927122894)(0.7978515625,0.194404096536889)(0.797991071428571,0.194090334749693)(0.798130580357143,0.193774642100449)(0.798270089285714,0.193457018932059)(0.798409598214286,0.193137465591182)(0.798549107142857,0.192815982428233)(0.798688616071428,0.192492569797386)(0.798828125,0.192167228056566)(0.798967633928571,0.191839957567455)(0.799107142857143,0.191510758695488)(0.799246651785714,0.191179631809854)(0.799386160714286,0.190846577283494)(0.799525669642857,0.1905115954931)(0.799665178571428,0.190174686819118)(0.7998046875,0.189835851645738)(0.799944196428571,0.189495090360908)(0.800083705357143,0.189152403356317)(0.800223214285714,0.188807791027407)(0.800362723214286,0.188461253773363)(0.800502232142857,0.188112791997121)(0.800641741071428,0.18776240610536)(0.80078125,0.187410096508504)(0.800920758928571,0.187055863620722)(0.801060267857143,0.186699707859924)(0.801199776785714,0.186341629647766)(0.801339285714286,0.185981629409642)(0.801478794642857,0.185619707574689)(0.801618303571428,0.185255864575785)(0.8017578125,0.184890100849543)(0.801897321428571,0.184522416836319)(0.802036830357143,0.184152812980202)(0.802176339285714,0.183781289729023)(0.802315848214286,0.183407847534341)(0.802455357142857,0.183032486851456)(0.802594866071428,0.182655208139403)(0.802734375,0.182276011860943)(0.802873883928571,0.181894898482576)(0.803013392857143,0.181511868474528)(0.803152901785714,0.18112692231076)(0.803292410714286,0.180740060468958)(0.803431919642857,0.18035128343054)(0.803571428571428,0.17996059168065)(0.8037109375,0.179567985708157)(0.803850446428571,0.17917346600566)(0.803989955357143,0.178777033069477)(0.804129464285714,0.178378687399654)(0.804268973214286,0.177978429499956)(0.804408482142857,0.177576259877874)(0.804547991071428,0.177172179044618)(0.8046875,0.176766187515116)(0.804827008928571,0.176358285808018)(0.804966517857143,0.175948474445686)(0.805106026785714,0.175536753954207)(0.805245535714286,0.175123124863376)(0.805385044642857,0.174707587706707)(0.805524553571428,0.174290143021428)(0.8056640625,0.173870791348475)(0.805803571428571,0.173449533232502)(0.805943080357143,0.173026369221868)(0.806082589285714,0.172601299868644)(0.806222098214286,0.172174325728607)(0.806361607142857,0.171745447361245)(0.806501116071428,0.171314665329751)(0.806640625,0.17088198020102)(0.806780133928571,0.170447392545655)(0.806919642857143,0.170010902937959)(0.807059151785714,0.169572511955938)(0.807198660714286,0.169132220181297)(0.807338169642857,0.168690028199442)(0.807477678571428,0.168245936599481)(0.8076171875,0.16779994597421)(0.807756696428571,0.16735205692013)(0.807896205357143,0.166902270037431)(0.808035714285714,0.166450585930001)(0.808175223214286,0.165997005205414)(0.808314732142857,0.165541528474944)(0.808454241071428,0.165084156353551)(0.80859375,0.16462488945988)(0.808733258928571,0.164163728416272)(0.808872767857143,0.163700673848747)(0.809012276785714,0.163235726387015)(0.809151785714286,0.162768886664467)(0.809291294642857,0.162300155318178)(0.809430803571428,0.161829532988909)(0.8095703125,0.161357020321093)(0.809709821428571,0.160882617962849)(0.809849330357143,0.160406326565969)(0.809988839285714,0.159928146785926)(0.810128348214286,0.159448079281864)(0.810267857142857,0.158966124716605)(0.810407366071428,0.158482283756643)(0.810546875,0.15799655707214)(0.810686383928571,0.157508945336932)(0.810825892857143,0.157019449228521)(0.810965401785714,0.156528069428081)(0.811104910714286,0.156034806620444)(0.811244419642857,0.155539661494115)(0.811383928571428,0.155042634741261)(0.8115234375,0.154543727057706)(0.811662946428571,0.154042939142942)(0.811802455357143,0.153540271700113)(0.811941964285714,0.153035725436029)(0.812081473214286,0.152529301061148)(0.812220982142857,0.152020999289591)(0.812360491071428,0.15151082083913)(0.8125,0.150998766431188)(0.812639508928571,0.150484836790842)(0.812779017857143,0.149969032646816)(0.812918526785714,0.149451354731484)(0.813058035714286,0.148931803780865)(0.813197544642857,0.148410380534626)(0.813337053571428,0.147887085736078)(0.8134765625,0.147361920132172)(0.813616071428571,0.146834884473503)(0.813755580357143,0.146305979514301)(0.813895089285714,0.14577520601244)(0.814034598214286,0.145242564729426)(0.814174107142857,0.144708056430404)(0.814313616071428,0.144171681884151)(0.814453125,0.143633441863074)(0.814592633928571,0.143093337143217)(0.814732142857143,0.142551368504244)(0.814871651785714,0.142007536729456)(0.815011160714286,0.141461842605772)(0.815150669642857,0.140914286923741)(0.815290178571428,0.140364870477538)(0.8154296875,0.139813594064949)(0.815569196428571,0.139260458487391)(0.815708705357143,0.138705464549891)(0.815848214285714,0.1381486130611)(0.815987723214286,0.137589904833277)(0.816127232142857,0.137029340682301)(0.816266741071428,0.136466921427663)(0.81640625,0.135902647892457)(0.816545758928571,0.135336520903397)(0.816685267857143,0.134768541290793)(0.816824776785714,0.134198709888571)(0.816964285714286,0.133627027534252)(0.817103794642857,0.133053495068966)(0.817243303571428,0.132478113337444)(0.8173828125,0.131900883188009)(0.817522321428571,0.131321805472591)(0.817661830357143,0.130740881046708)(0.817801339285714,0.130158110769478)(0.817940848214286,0.129573495503605)(0.818080357142857,0.12898703611539)(0.818219866071428,0.128398733474723)(0.818359375,0.127808588455077)(0.818498883928571,0.127216601933515)(0.818638392857143,0.126622774790679)(0.818777901785714,0.126027107910801)(0.818917410714286,0.125429602181685)(0.819056919642857,0.12483025849472)(0.819196428571428,0.124229077744872)(0.8193359375,0.123626060830679)(0.819475446428571,0.123021208654255)(0.819614955357143,0.122414522121285)(0.819754464285714,0.121806002141025)(0.819893973214286,0.121195649626297)(0.820033482142857,0.120583465493493)(0.820172991071428,0.119969450662571)(0.8203125,0.119353606057045)(0.820452008928571,0.118735932603999)(0.820591517857143,0.118116431234067)(0.820731026785714,0.11749510288145)(0.820870535714286,0.116871948483897)(0.821010044642857,0.116246968982716)(0.821149553571428,0.115620165322769)(0.8212890625,0.11499153845246)(0.821428571428571,0.11436108932375)(0.821568080357143,0.113728818892139)(0.821707589285714,0.113094728116681)(0.821847098214286,0.112458817959963)(0.821986607142857,0.111821089388119)(0.822126116071428,0.111181543370824)(0.822265625,0.110540180881283)(0.822405133928571,0.109897002896243)(0.822544642857143,0.10925201039598)(0.822684151785714,0.108605204364306)(0.822823660714286,0.107956585788556)(0.822963169642857,0.107306155659599)(0.823102678571428,0.10665391497183)(0.8232421875,0.105999864723162)(0.823381696428571,0.105344005915036)(0.823521205357143,0.104686339552406)(0.823660714285714,0.104026866643754)(0.823800223214286,0.103365588201066)(0.823939732142857,0.102702505239851)(0.824079241071428,0.102037618779131)(0.82421875,0.10137092984143)(0.824358258928571,0.100702439452787)(0.824497767857143,0.100032148642743)(0.824637276785714,0.0993600584443469)(0.824776785714286,0.0986861698941441)(0.824916294642857,0.098010484032186)(0.825055803571428,0.0973330019020223)(0.8251953125,0.0966537245506919)(0.825334821428571,0.0959726530287353)(0.825474330357143,0.0952897883901782)(0.825613839285714,0.094605131692543)(0.825753348214286,0.0939186839968325)(0.825892857142857,0.0932304463675415)(0.826032366071428,0.0925404198726489)(0.826171875,0.0918486055836084)(0.826311383928571,0.091155004575361)(0.826450892857143,0.0904596179263178)(0.826590401785714,0.0897624467183731)(0.826729910714286,0.0890634920368856)(0.826869419642857,0.0883627549706927)(0.827008928571428,0.0876602366121004)(0.8271484375,0.086955938056875)(0.827287946428571,0.0862498604042552)(0.827427455357143,0.0855420047569354)(0.827566964285714,0.0848323722210775)(0.827706473214286,0.0841209639062938)(0.827845982142857,0.0834077809256595)(0.827985491071428,0.0826928243957041)(0.828125,0.0819760954364018)(0.828264508928571,0.0812575951711845)(0.828404017857143,0.080537324726924)(0.828543526785714,0.0798152852339451)(0.828683035714286,0.0790914778260072)(0.828822544642857,0.0783659036403179)(0.828962053571428,0.0776385638175231)(0.8291015625,0.0769094595016983)(0.829241071428571,0.0761785918403611)(0.829380580357143,0.075445961984454)(0.829520089285714,0.0747115710883564)(0.829659598214286,0.0739754203098668)(0.829799107142857,0.0732375108102165)(0.829938616071428,0.0724978437540594)(0.830078125,0.0717564203094629)(0.830217633928571,0.0710132416479214)(0.830357142857143,0.0702683089443376)(0.830496651785714,0.0695216233770357)(0.830636160714286,0.0687731861277432)(0.830775669642857,0.0680229983816039)(0.830915178571428,0.0672710613271688)(0.8310546875,0.0665173761563859)(0.831194196428571,0.0657619440646144)(0.831333705357143,0.0650047662506055)(0.831473214285714,0.0642458439165159)(0.831612723214286,0.0634851782678888)(0.831752232142857,0.062722770513668)(0.831891741071428,0.0619586218661882)(0.83203125,0.0611927335411643)(0.832170758928571,0.0604251067577063)(0.832310267857143,0.0596557427382993)(0.832449776785714,0.058884642708818)(0.832589285714286,0.0581118078985066)(0.832728794642857,0.0573372395399935)(0.832868303571428,0.0565609388692813)(0.8330078125,0.0557829071257359)(0.833147321428571,0.0550031455521017)(0.833286830357143,0.0542216553944818)(0.833426339285714,0.0534384379023511)(0.833565848214286,0.0526534943285382)(0.833705357142857,0.0518668259292383)(0.833844866071428,0.0510784339640038)(0.833984375,0.0502883196957339)(0.834123883928571,0.0494964843906889)(0.834263392857143,0.0487029293184701)(0.834402901785714,0.0479076557520349)(0.834542410714286,0.0471106649676756)(0.834681919642857,0.0463119582450349)(0.834821428571428,0.0455115368670952)(0.8349609375,0.0447094021201678)(0.835100446428571,0.0439055552939085)(0.835239955357143,0.0430999976812964)(0.835379464285714,0.0422927305786487)(0.835518973214286,0.0414837552856008)(0.835658482142857,0.0406730731051201)(0.835797991071428,0.0398606853434971)(0.8359375,0.0390465933103326)(0.836077008928571,0.0382307983185546)(0.836216517857143,0.0374133016843965)(0.836356026785714,0.0365941047274123)(0.836495535714286,0.035773208770456)(0.836635044642857,0.0349506151396961)(0.836774553571428,0.0341263251646058)(0.8369140625,0.033300340177951)(0.837053571428571,0.0324726615158066)(0.837193080357143,0.0316432905175349)(0.837332589285714,0.0308122285258007)(0.837472098214286,0.0299794768865504)(0.837611607142857,0.0291450369490273)(0.837751116071428,0.0283089100657602)(0.837890625,0.0274710975925534)(0.838030133928571,0.026631600888501)(0.838169642857143,0.0257904213159666)(0.838309151785714,0.0249475602405977)(0.838448660714286,0.024103019031305)(0.838588169642857,0.0232567990602771)(0.838727678571428,0.0224089017029707)(0.8388671875,0.0215593283380981)(0.839006696428571,0.0207080803476437)(0.839146205357143,0.019855159116842)(0.839285714285714,0.0190005660341931)(0.839425223214286,0.0181443024914413)(0.839564732142857,0.0172863698835903)(0.839704241071428,0.0164267696088927)(0.83984375,0.0155655030688377)(0.839983258928571,0.014702571668168)(0.840122767857143,0.0138379768148572)(0.840262276785714,0.0129717199201256)(0.840401785714286,0.0121038023984181)(0.840541294642857,0.0112342256674206)(0.840680803571428,0.0103629911480482)(0.8408203125,0.00949010026443353)(0.840959821428571,0.00861555444394346)(0.841099330357143,0.0077393551171564)(0.841238839285714,0.00686150371787819)(0.841378348214286,0.00598200168311969)(0.841517857142857,0.00510085045311359)(0.841657366071428,0.00421805147130205)(0.841796875,0.00333360618432543)(0.841936383928571,0.00244751604203841)(0.842075892857143,0.00155978249748789)(0.842215401785714,0.000670407006928553)(0.842354910714286,-0.000220608970199709)(0.842494419642857,-0.00111326397125511)(0.842633928571428,-0.00200755653040574)(0.8427734375,-0.00290348517864181)(0.842912946428571,-0.00380104844375873)(0.843052455357143,-0.00470024485037979)(0.843191964285714,-0.0056010729199405)(0.843331473214286,-0.00650353117071101)(0.843470982142857,-0.0074076181177799)(0.843610491071428,-0.00831333227306552)(0.84375,-0.0092206721453284)(0.843889508928571,-0.0101296362401545)(0.844029017857143,-0.0110402230599783)(0.844168526785714,-0.0119524311040663)(0.844308035714286,-0.0128662588685401)(0.844447544642857,-0.01378170484636)(0.844587053571428,-0.0146987675273366)(0.8447265625,-0.0156174453981431)(0.844866071428571,-0.0165377369422989)(0.845005580357143,-0.0174596406401917)(0.845145089285714,-0.0183831549690624)(0.845284598214286,-0.0193082784030277)(0.845424107142857,-0.0202350094130632)(0.845563616071428,-0.0211633464670158)(0.845703125,-0.0220932880296159)(0.845842633928571,-0.02302483256246)(0.845982142857143,-0.0239579785240353)(0.846121651785714,-0.0248927243697016)(0.846261160714286,-0.0258290685517161)(0.846400669642857,-0.0267670095192158)(0.846540178571428,-0.0277065457182301)(0.8466796875,-0.0286476755916923)(0.846819196428571,-0.0295903975794241)(0.846958705357143,-0.0305347101181577)(0.847098214285714,-0.0314806116415199)(0.847237723214286,-0.0324281005800561)(0.847377232142857,-0.0333771753612123)(0.847516741071428,-0.0343278344093483)(0.84765625,-0.0352800761457498)(0.847795758928571,-0.0362338989886113)(0.847935267857143,-0.03718930135306)(0.848074776785714,-0.0381462816511386)(0.848214285714286,-0.0391048382918304)(0.848353794642857,-0.0400649696810407)(0.848493303571428,-0.0410266742216098)(0.8486328125,-0.041989950313326)(0.848772321428571,-0.0429547963529078)(0.848911830357143,-0.0439212107340284)(0.849051339285714,-0.0448891918472982)(0.849190848214286,-0.0458587380802897)(0.849330357142857,-0.0468298478175194)(0.849469866071428,-0.0478025194404609)(0.849609375,-0.0487767513275574)(0.849748883928571,-0.0497525418542047)(0.849888392857143,-0.0507298893927748)(0.850027901785714,-0.0517087923125992)(0.850167410714286,-0.0526892489799935)(0.850306919642857,-0.0536712577582398)(0.850446428571428,-0.0546548170075988)(0.8505859375,-0.0556399250853238)(0.850725446428571,-0.0566265803456421)(0.850864955357143,-0.0576147811397802)(0.851004464285714,-0.0586045258159462)(0.851143973214286,-0.059595812719355)(0.851283482142857,-0.0605886401922093)(0.851422991071428,-0.0615830065737144)(0.8515625,-0.0625789102000895)(0.851702008928571,-0.0635763494045511)(0.851841517857143,-0.0645753225173371)(0.851981026785714,-0.0655758278656897)(0.852120535714286,-0.0665778637738804)(0.852260044642857,-0.0675814285631919)(0.852399553571428,-0.068586520551931)(0.8525390625,-0.0695931380554428)(0.852678571428571,-0.0706012793860908)(0.852818080357143,-0.0716109428532843)(0.852957589285714,-0.0726221267634588)(0.853097098214286,-0.0736348294201025)(0.853236607142857,-0.0746490491237376)(0.853376116071428,-0.0756647841719337)(0.853515625,-0.0766820328593213)(0.853655133928571,-0.0777007934775731)(0.853794642857143,-0.07872106431543)(0.853934151785714,-0.0797428436586828)(0.854073660714286,-0.0807661297901981)(0.854213169642857,-0.0817909209898998)(0.854352678571428,-0.0828172155347824)(0.8544921875,-0.0838450116989247)(0.854631696428571,-0.0848743077534712)(0.854771205357143,-0.0859051019666578)(0.854910714285714,-0.0869373926037941)(0.855050223214286,-0.0879711779272889)(0.855189732142857,-0.0890064561966316)(0.855329241071428,-0.0900432256684062)(0.85546875,-0.0910814845963042)(0.855608258928571,-0.0921212312311067)(0.855747767857143,-0.0931624638207103)(0.855887276785714,-0.0942051806101082)(0.856026785714286,-0.0952493798414173)(0.856166294642857,-0.0962950597538587)(0.856305803571428,-0.0973422185837716)(0.8564453125,-0.0983908545646274)(0.856584821428571,-0.0994409659270105)(0.856724330357143,-0.100492550898645)(0.856863839285714,-0.101545607704375)(0.857003348214286,-0.102600134566193)(0.857142857142857,-0.103656129703217)(0.857282366071428,-0.10471359133172)(0.857421875,-0.105772517665111)(0.857561383928571,-0.106832906913949)(0.857700892857143,-0.107894757285956)(0.857840401785714,-0.108958066985996)(0.857979910714286,-0.110022834216107)(0.858119419642857,-0.111089057175475)(0.858258928571428,-0.112156734060469)(0.8583984375,-0.113225863064615)(0.858537946428571,-0.114296442378616)(0.858677455357143,-0.115368470190359)(0.858816964285714,-0.116441944684902)(0.858956473214286,-0.117516864044499)(0.859095982142857,-0.118593226448578)(0.859235491071428,-0.119671030073772)(0.859375,-0.120750273093899)(0.859514508928571,-0.121830953679976)(0.859654017857143,-0.122913070000232)(0.859793526785714,-0.123996620220089)(0.859933035714286,-0.125081602502191)(0.860072544642857,-0.126168015006379)(0.860212053571428,-0.12725585588973)(0.8603515625,-0.128345123306522)(0.860491071428571,-0.129435815408266)(0.860630580357143,-0.130527930343703)(0.860770089285714,-0.131621466258796)(0.860909598214286,-0.132716421296754)(0.861049107142857,-0.133812793598008)(0.861188616071428,-0.134910581300251)(0.861328125,-0.136009782538402)(0.861467633928571,-0.137110395444636)(0.861607142857143,-0.138212418148388)(0.861746651785714,-0.139315848776336)(0.861886160714286,-0.140420685452433)(0.862025669642857,-0.141526926297877)(0.862165178571428,-0.142634569431151)(0.8623046875,-0.143743612967997)(0.862444196428571,-0.144854055021433)(0.862583705357143,-0.145965893701763)(0.862723214285714,-0.147079127116561)(0.862862723214286,-0.148193753370699)(0.863002232142857,-0.149309770566325)(0.863141741071428,-0.150427176802893)(0.86328125,-0.151545970177144)(0.863420758928571,-0.152666148783118)(0.863560267857143,-0.153787710712171)(0.863699776785714,-0.154910654052953)(0.863839285714286,-0.156034976891438)(0.863978794642857,-0.1571606773109)(0.864118303571428,-0.158287753391949)(0.8642578125,-0.159416203212503)(0.864397321428571,-0.160546024847812)(0.864536830357143,-0.161677216370461)(0.864676339285714,-0.162809775850357)(0.864815848214286,-0.163943701354762)(0.864955357142857,-0.165078990948258)(0.865094866071428,-0.166215642692793)(0.865234375,-0.167353654647648)(0.865373883928571,-0.168493024869461)(0.865513392857143,-0.169633751412235)(0.865652901785714,-0.170775832327319)(0.865792410714286,-0.171919265663442)(0.865931919642857,-0.173064049466683)(0.866071428571428,-0.17421018178051)(0.8662109375,-0.175357660645754)(0.866350446428571,-0.176506484100626)(0.866489955357143,-0.177656650180731)(0.866629464285714,-0.178808156919047)(0.866768973214286,-0.179961002345957)(0.866908482142857,-0.181115184489221)(0.867047991071428,-0.182270701374017)(0.8671875,-0.183427551022909)(0.867327008928571,-0.184585731455871)(0.867466517857143,-0.185745240690297)(0.867606026785714,-0.18690607674098)(0.867745535714286,-0.188068237620145)(0.867885044642857,-0.189231721337424)(0.868024553571428,-0.190396525899889)(0.8681640625,-0.191562649312028)(0.868303571428571,-0.192730089575768)(0.868443080357143,-0.193898844690478)(0.868582589285714,-0.195068912652957)(0.868722098214286,-0.196240291457463)(0.868861607142857,-0.197412979095686)(0.869001116071428,-0.198586973556786)(0.869140625,-0.199762272827365)(0.869280133928571,-0.200938874891489)(0.869419642857143,-0.202116777730698)(0.869559151785714,-0.203295979323987)(0.869698660714286,-0.204476477647836)(0.869838169642857,-0.205658270676187)(0.869977678571428,-0.206841356380477)(0.8701171875,-0.208025732729616)(0.870256696428571,-0.209211397690002)(0.870396205357143,-0.210398349225537)(0.870535714285714,-0.211586585297604)(0.870675223214286,-0.2127761038651)(0.870814732142857,-0.21396690288441)(0.870954241071428,-0.215158980309446)(0.87109375,-0.216352334091613)(0.871233258928571,-0.217546962179841)(0.871372767857143,-0.218742862520584)(0.871512276785714,-0.21994003305781)(0.871651785714286,-0.221138471733027)(0.871791294642857,-0.222338176485258)(0.871930803571428,-0.223539145251081)(0.8720703125,-0.2247413759646)(0.872209821428571,-0.225944866557464)(0.872349330357143,-0.227149614958881)(0.872488839285714,-0.228355619095596)(0.872628348214286,-0.229562876891925)(0.872767857142857,-0.230771386269728)(0.872907366071428,-0.231981145148446)(0.873046875,-0.233192151445074)(0.873186383928571,-0.234404403074182)(0.873325892857143,-0.235617897947927)(0.873465401785714,-0.236832633976031)(0.873604910714286,-0.238048609065815)(0.873744419642857,-0.239265821122172)(0.873883928571428,-0.240484268047607)(0.8740234375,-0.241703947742204)(0.874162946428571,-0.242924858103654)(0.874302455357143,-0.244146997027259)(0.874441964285714,-0.24537036240592)(0.874581473214286,-0.246594952130161)(0.874720982142857,-0.247820764088111)(0.874860491071428,-0.249047796165534)(0.875,-0.250276046245808)(0.875139508928571,-0.251505512209941)(0.875279017857143,-0.252736191936586)(0.875418526785714,-0.253968083302018)(0.875558035714286,-0.255201184180171)(0.875697544642857,-0.256435492442606)(0.875837053571428,-0.257671005958553)(0.8759765625,-0.258907722594881)(0.876116071428571,-0.26014564021612)(0.876255580357143,-0.261384756684475)(0.876395089285714,-0.2626250698598)(0.876534598214286,-0.263866577599638)(0.876674107142857,-0.265109277759188)(0.876813616071428,-0.266353168191346)(0.876953125,-0.267598246746678)(0.877092633928571,-0.268844511273441)(0.877232142857143,-0.270091959617592)(0.877371651785714,-0.271340589622771)(0.877511160714286,-0.272590399130333)(0.877650669642857,-0.273841385979319)(0.877790178571428,-0.275093548006498)(0.8779296875,-0.276346883046337)(0.878069196428571,-0.277601388931023)(0.878208705357143,-0.278857063490473)(0.878348214285714,-0.280113904552318)(0.878487723214286,-0.28137190994193)(0.878627232142857,-0.2826310774824)(0.878766741071428,-0.283891404994575)(0.87890625,-0.285152890297029)(0.879045758928571,-0.286415531206088)(0.879185267857143,-0.287679325535834)(0.879324776785714,-0.288944271098095)(0.879464285714286,-0.290210365702471)(0.879603794642857,-0.291477607156308)(0.879743303571428,-0.292745993264738)(0.8798828125,-0.294015521830653)(0.880022321428571,-0.295286190654723)(0.880161830357143,-0.296557997535406)(0.880301339285714,-0.297830940268936)(0.880440848214286,-0.299105016649346)(0.880580357142857,-0.30038022446845)(0.880719866071428,-0.301656561515875)(0.880859375,-0.302934025579036)(0.880998883928571,-0.30421261444316)(0.881138392857143,-0.305492325891291)(0.881277901785714,-0.306773157704277)(0.881417410714286,-0.308055107660798)(0.881556919642857,-0.309338173537343)(0.881696428571428,-0.310622353108246)(0.8818359375,-0.311907644145658)(0.881975446428571,-0.313194044419571)(0.882114955357143,-0.314481551697829)(0.882254464285714,-0.315770163746104)(0.882393973214286,-0.317059878327935)(0.882533482142857,-0.318350693204699)(0.882672991071428,-0.319642606135648)(0.8828125,-0.320935614877882)(0.882952008928571,-0.322229717186373)(0.883091517857143,-0.323524910813972)(0.883231026785714,-0.324821193511394)(0.883370535714286,-0.326118563027249)(0.883510044642857,-0.327417017108013)(0.883649553571428,-0.328716553498071)(0.8837890625,-0.330017169939686)(0.883928571428571,-0.331318864173024)(0.884068080357143,-0.332621633936159)(0.884207589285714,-0.333925476965062)(0.884347098214286,-0.335230390993627)(0.884486607142857,-0.336536373753647)(0.884626116071428,-0.337843422974854)(0.884765625,-0.33915153638489)(0.884905133928571,-0.340460711709327)(0.885044642857143,-0.341770946671683)(0.885184151785714,-0.343082238993395)(0.885323660714286,-0.344394586393862)(0.885463169642857,-0.345707986590409)(0.885602678571428,-0.347022437298332)(0.8857421875,-0.348337936230865)(0.885881696428571,-0.349654481099209)(0.886021205357143,-0.350972069612537)(0.886160714285714,-0.352290699477975)(0.886300223214286,-0.35361036840064)(0.886439732142857,-0.354931074083608)(0.886579241071428,-0.356252814227956)(0.88671875,-0.357575586532733)(0.886858258928571,-0.358899388694981)(0.886997767857143,-0.360224218409749)(0.887137276785714,-0.361550073370071)(0.887276785714286,-0.362876951267002)(0.887416294642857,-0.364204849789585)(0.887555803571428,-0.365533766624901)(0.8876953125,-0.366863699458028)(0.887834821428571,-0.368194645972074)(0.887974330357143,-0.369526603848182)(0.888113839285714,-0.370859570765513)(0.888253348214286,-0.37219354440128)(0.888392857142857,-0.373528522430716)(0.888532366071428,-0.374864502527123)(0.888671875,-0.376201482361832)(0.888811383928571,-0.377539459604237)(0.888950892857143,-0.378878431921795)(0.889090401785714,-0.380218396980017)(0.889229910714286,-0.381559352442493)(0.889369419642857,-0.382901295970871)(0.889508928571428,-0.384244225224894)(0.8896484375,-0.385588137862368)(0.889787946428571,-0.386933031539194)(0.889927455357143,-0.388278903909369)(0.890066964285714,-0.389625752624973)(0.890206473214286,-0.3909735753362)(0.890345982142857,-0.392322369691331)(0.890485491071428,-0.393672133336776)(0.890625,-0.395022863917041)(0.890764508928571,-0.396374559074754)(0.890904017857143,-0.397727216450677)(0.891043526785714,-0.399080833683684)(0.891183035714286,-0.400435408410795)(0.891322544642857,-0.401790938267151)(0.891462053571428,-0.403147420886051)(0.8916015625,-0.404504853898925)(0.891741071428571,-0.405863234935357)(0.891880580357143,-0.407222561623096)(0.892020089285714,-0.408582831588035)(0.892159598214286,-0.409944042454247)(0.892299107142857,-0.411306191843956)(0.892438616071428,-0.412669277377579)(0.892578125,-0.414033296673694)(0.892717633928571,-0.415398247349066)(0.892857142857143,-0.416764127018659)(0.892996651785714,-0.41813093329561)(0.893136160714286,-0.419498663791273)(0.893275669642857,-0.420867316115181)(0.893415178571428,-0.422236887875096)(0.8935546875,-0.423607376676971)(0.893694196428571,-0.424978780124981)(0.893833705357143,-0.426351095821528)(0.893973214285714,-0.427724321367227)(0.894112723214286,-0.429098454360935)(0.894252232142857,-0.430473492399725)(0.894391741071428,-0.431849433078931)(0.89453125,-0.43322627399211)(0.894670758928571,-0.434604012731074)(0.894810267857143,-0.435982646885896)(0.894949776785714,-0.437362174044891)(0.895089285714286,-0.438742591794654)(0.895228794642857,-0.440123897720026)(0.895368303571428,-0.441506089404139)(0.8955078125,-0.442889164428387)(0.895647321428571,-0.444273120372447)(0.895786830357143,-0.445657954814292)(0.895926339285714,-0.44704366533017)(0.896065848214286,-0.448430249494641)(0.896205357142857,-0.449817704880544)(0.896344866071428,-0.451206029059044)(0.896484375,-0.452595219599598)(0.896623883928571,-0.453985274069982)(0.896763392857143,-0.455376190036298)(0.896902901785714,-0.456767965062959)(0.897042410714286,-0.458160596712721)(0.897181919642857,-0.459554082546653)(0.897321428571428,-0.460948420124184)(0.8974609375,-0.462343607003066)(0.897600446428571,-0.463739640739404)(0.897739955357143,-0.465136518887665)(0.897879464285714,-0.466534239000656)(0.898018973214286,-0.467932798629563)(0.898158482142857,-0.469332195323918)(0.898297991071428,-0.470732426631645)(0.8984375,-0.472133490099025)(0.898577008928571,-0.473535383270727)(0.898716517857143,-0.474938103689813)(0.898856026785714,-0.476341648897719)(0.898995535714286,-0.477746016434293)(0.899135044642857,-0.479151203837765)(0.899274553571428,-0.480557208644788)(0.8994140625,-0.481964028390407)(0.899553571428571,-0.483371660608086)(0.899693080357143,-0.484780102829718)(0.899832589285714,-0.486189352585604)(0.899972098214286,-0.487599407404488)(0.900111607142857,-0.489010264813531)(0.900251116071428,-0.490421922338351)(0.900390625,-0.491834377502991)(0.900530133928571,-0.493247627829947)(0.900669642857143,-0.494661670840178)(0.900809151785714,-0.496076504053084)(0.900948660714286,-0.497492124986543)(0.901088169642857,-0.498908531156881)(0.901227678571428,-0.500325720078917)(0.9013671875,-0.50174368926593)(0.901506696428571,-0.503162436229684)(0.901646205357143,-0.504581958480438)(0.901785714285714,-0.506002253526928)(0.901925223214286,-0.507423318876404)(0.902064732142857,-0.508845152034595)(0.902204241071428,-0.510267750505758)(0.90234375,-0.511691111792641)(0.902483258928571,-0.513115233396515)(0.902622767857143,-0.514540112817179)(0.902762276785714,-0.515965747552943)(0.902901785714286,-0.517392135100663)(0.903041294642857,-0.51881927295571)(0.903180803571428,-0.520247158612017)(0.9033203125,-0.521675789562042)(0.903459821428571,-0.523105163296798)(0.903599330357143,-0.524535277305865)(0.903738839285714,-0.525966129077362)(0.903878348214286,-0.527397716097992)(0.904017857142857,-0.528830035853008)(0.904157366071428,-0.530263085826255)(0.904296875,-0.531696863500142)(0.904436383928571,-0.533131366355666)(0.904575892857143,-0.53456659187242)(0.904715401785714,-0.536002537528579)(0.904854910714286,-0.53743920080093)(0.904994419642857,-0.538876579164846)(0.905133928571428,-0.540314670094328)(0.9052734375,-0.541753471061972)(0.905412946428571,-0.543192979538999)(0.905552455357143,-0.54463319299526)(0.905691964285714,-0.546074108899222)(0.905831473214286,-0.547515724717999)(0.905970982142857,-0.548958037917324)(0.906110491071428,-0.550401045961594)(0.90625,-0.551844746313835)(0.906389508928571,-0.553289136435733)(0.906529017857143,-0.554734213787637)(0.906668526785714,-0.556179975828548)(0.906808035714286,-0.557626420016148)(0.906947544642857,-0.559073543806772)(0.907087053571428,-0.560521344655455)(0.9072265625,-0.561969820015896)(0.907366071428571,-0.563418967340484)(0.907505580357143,-0.564868784080314)(0.907645089285714,-0.566319267685159)(0.907784598214286,-0.567770415603513)(0.907924107142857,-0.569222225282557)(0.908063616071428,-0.570674694168205)(0.908203125,-0.57212781970507)(0.908342633928571,-0.573581599336492)(0.908482142857143,-0.575036030504549)(0.908621651785714,-0.576491110650035)(0.908761160714286,-0.577946837212496)(0.908900669642857,-0.579403207630201)(0.909040178571428,-0.58086021934019)(0.9091796875,-0.582317869778233)(0.909319196428571,-0.583776156378864)(0.909458705357143,-0.585235076575388)(0.909598214285714,-0.586694627799863)(0.909737723214286,-0.588154807483133)(0.909877232142857,-0.589615613054801)(0.910016741071428,-0.591077041943272)(0.91015625,-0.592539091575722)(0.910295758928571,-0.594001759378121)(0.910435267857143,-0.595465042775249)(0.910574776785714,-0.59692893919067)(0.910714285714286,-0.598393446046772)(0.910853794642857,-0.599858560764736)(0.910993303571428,-0.601324280764581)(0.9111328125,-0.602790603465129)(0.911272321428571,-0.604257526284037)(0.911411830357143,-0.6057250466378)(0.911551339285714,-0.607193161941738)(0.911690848214286,-0.608661869610027)(0.911830357142857,-0.610131167055672)(0.911969866071428,-0.611601051690552)(0.912109375,-0.613071520925384)(0.912248883928571,-0.614542572169752)(0.912388392857143,-0.616014202832118)(0.912527901785714,-0.617486410319802)(0.912667410714286,-0.618959192039016)(0.912806919642857,-0.620432545394835)(0.912946428571428,-0.621906467791243)(0.9130859375,-0.6233809566311)(0.913225446428571,-0.624856009316167)(0.913364955357143,-0.626331623247119)(0.913504464285714,-0.62780779582352)(0.913643973214286,-0.629284524443867)(0.913783482142857,-0.630761806505553)(0.913922991071428,-0.632239639404917)(0.9140625,-0.633718020537206)(0.914202008928571,-0.635196947296606)(0.914341517857143,-0.636676417076252)(0.914481026785714,-0.638156427268205)(0.914620535714286,-0.639636975263491)(0.914760044642857,-0.64111805845207)(0.914899553571428,-0.642599674222884)(0.9150390625,-0.644081819963816)(0.915178571428571,-0.645564493061727)(0.915318080357143,-0.647047690902458)(0.915457589285714,-0.648531410870815)(0.915597098214286,-0.650015650350606)(0.915736607142857,-0.651500406724604)(0.915876116071428,-0.652985677374602)(0.916015625,-0.654471459681371)(0.916155133928571,-0.655957751024692)(0.916294642857143,-0.657444548783367)(0.916434151785714,-0.658931850335193)(0.916573660714286,-0.660419653057008)(0.916713169642857,-0.661907954324651)(0.916852678571428,-0.663396751513014)(0.9169921875,-0.664886041996006)(0.917131696428571,-0.666375823146579)(0.917271205357143,-0.667866092336742)(0.917410714285714,-0.669356846937539)(0.917550223214286,-0.670848084319084)(0.917689732142857,-0.672339801850532)(0.917829241071428,-0.673831996900127)(0.91796875,-0.675324666835164)(0.918108258928571,-0.676817809022019)(0.918247767857143,-0.678311420826158)(0.918387276785714,-0.679805499612119)(0.918526785714286,-0.681300042743548)(0.918666294642857,-0.682795047583165)(0.918805803571428,-0.684290511492816)(0.9189453125,-0.685786431833435)(0.919084821428571,-0.687282805965069)(0.919224330357143,-0.688779631246897)(0.919363839285714,-0.690276905037201)(0.919503348214286,-0.691774624693409)(0.919642857142857,-0.693272787572059)(0.919782366071428,-0.69477139102885)(0.919921875,-0.696270432418605)(0.920061383928571,-0.697769909095298)(0.920200892857143,-0.699269818412069)(0.920340401785714,-0.700770157721198)(0.920479910714286,-0.702270924374146)(0.920619419642857,-0.70377211572152)(0.920758928571428,-0.705273729113127)(0.9208984375,-0.706775761897929)(0.921037946428571,-0.708278211424081)(0.921177455357143,-0.709781075038935)(0.921316964285714,-0.711284350089021)(0.921456473214286,-0.712788033920087)(0.921595982142857,-0.714292123877064)(0.921735491071428,-0.715796617304114)(0.921875,-0.717301511544596)(0.922014508928571,-0.718806803941095)(0.922154017857143,-0.720312491835431)(0.922293526785714,-0.721818572568636)(0.922433035714286,-0.723325043480997)(0.922572544642857,-0.724831901912022)(0.922712053571428,-0.726339145200484)(0.9228515625,-0.727846770684391)(0.922991071428571,-0.729354775701011)(0.923130580357143,-0.730863157586884)(0.923270089285714,-0.7323719136778)(0.923409598214286,-0.73388104130884)(0.923549107142857,-0.735390537814338)(0.923688616071428,-0.736900400527936)(0.923828125,-0.738410626782542)(0.923967633928571,-0.739921213910363)(0.924107142857143,-0.741432159242913)(0.924246651785714,-0.742943460110993)(0.924386160714286,-0.744455113844731)(0.924525669642857,-0.745967117773546)(0.924665178571428,-0.747479469226197)(0.9248046875,-0.74899216553075)(0.924944196428571,-0.750505204014604)(0.925083705357143,-0.752018582004503)(0.925223214285714,-0.753532296826513)(0.925362723214286,-0.755046345806063)(0.925502232142857,-0.756560726267911)(0.925641741071428,-0.75807543553619)(0.92578125,-0.759590470934378)(0.925920758928571,-0.761105829785321)(0.926060267857143,-0.762621509411247)(0.926199776785714,-0.764137507133741)(0.926339285714286,-0.765653820273791)(0.926478794642857,-0.767170446151747)(0.926618303571428,-0.768687382087372)(0.9267578125,-0.770204625399811)(0.926897321428571,-0.771722173407612)(0.927036830357143,-0.773240023428742)(0.927176339285714,-0.774758172780563)(0.927315848214286,-0.776276618779875)(0.927455357142857,-0.777795358742875)(0.927594866071428,-0.779314389985215)(0.927734375,-0.780833709821959)(0.927873883928571,-0.782353315567615)(0.928013392857143,-0.783873204536144)(0.928152901785714,-0.785393374040941)(0.928292410714286,-0.786913821394872)(0.928431919642857,-0.788434543910242)(0.928571428571428,-0.789955538898842)(0.9287109375,-0.791476803671914)(0.928850446428571,-0.792998335540182)(0.928989955357143,-0.794520131813857)(0.929129464285714,-0.796042189802623)(0.929268973214286,-0.79756450681567)(0.929408482142857,-0.799087080161664)(0.929547991071428,-0.800609907148793)(0.9296875,-0.802132985084735)(0.929827008928571,-0.803656311276683)(0.929966517857143,-0.805179883031359)(0.930106026785714,-0.806703697654989)(0.930245535714286,-0.808227752453344)(0.930385044642857,-0.809752044731708)(0.930524553571428,-0.811276571794925)(0.9306640625,-0.812801330947362)(0.930803571428571,-0.814326319492943)(0.930943080357143,-0.81585153473515)(0.931082589285714,-0.817376973977014)(0.931222098214286,-0.818902634521143)(0.931361607142857,-0.820428513669696)(0.931501116071428,-0.821954608724428)(0.931640625,-0.823480916986655)(0.931780133928571,-0.825007435757284)(0.931919642857143,-0.826534162336822)(0.932059151785714,-0.828061094025356)(0.932198660714286,-0.82958822812259)(0.932338169642857,-0.831115561927815)(0.932477678571428,-0.832643092739955)(0.9326171875,-0.834170817857532)(0.932756696428571,-0.835698734578695)(0.932896205357143,-0.83722684020123)(0.933035714285714,-0.83875513202254)(0.933175223214286,-0.840283607339683)(0.933314732142857,-0.841812263449338)(0.933454241071428,-0.843341097647857)(0.93359375,-0.844870107231223)(0.933733258928571,-0.846399289495087)(0.933872767857143,-0.847928641734772)(0.934012276785714,-0.849458161245253)(0.934151785714286,-0.8509878453212)(0.934291294642857,-0.852517691256939)(0.934430803571428,-0.854047696346506)(0.9345703125,-0.855577857883607)(0.934709821428571,-0.857108173161648)(0.934849330357143,-0.858638639473748)(0.934988839285714,-0.860169254112715)(0.935128348214286,-0.861700014371086)(0.935267857142857,-0.863230917541094)(0.935407366071428,-0.864761960914715)(0.935546875,-0.866293141783636)(0.935686383928571,-0.86782445743928)(0.935825892857143,-0.869355905172817)(0.935965401785714,-0.870887482275145)(0.936104910714286,-0.872419186036929)(0.936244419642857,-0.873951013748564)(0.936383928571428,-0.875482962700229)(0.9365234375,-0.877015030181846)(0.936662946428571,-0.878547213483114)(0.936802455357143,-0.880079509893513)(0.936941964285714,-0.881611916702292)(0.937081473214286,-0.883144431198498)(0.937220982142857,-0.88467705067095)(0.937360491071428,-0.886209772408285)(0.9375,-0.887742593698921)(0.937639508928571,-0.88927551183109)(0.937779017857143,-0.890808524092844)(0.937918526785714,-0.892341627772034)(0.938058035714286,-0.893874820156355)(0.938197544642857,-0.895408098533305)(0.938337053571428,-0.896941460190238)(0.9384765625,-0.898474902414329)(0.938616071428571,-0.900008422492598)(0.938755580357143,-0.901542017711927)(0.938895089285714,-0.903075685359034)(0.939034598214286,-0.904609422720515)(0.939174107142857,-0.906143227082806)(0.939313616071428,-0.90767709573224)(0.939453125,-0.909211025955002)(0.939592633928571,-0.910745015037166)(0.939732142857143,-0.912279060264698)(0.939871651785714,-0.913813158923442)(0.940011160714286,-0.915347308299154)(0.940150669642857,-0.916881505677469)(0.940290178571428,-0.918415748343954)(0.9404296875,-0.919950033584068)(0.940569196428571,-0.921484358683192)(0.940708705357143,-0.92301872092664)(0.940848214285714,-0.924553117599637)(0.940987723214286,-0.926087545987359)(0.941127232142857,-0.9276220033749)(0.941266741071428,-0.92915648704732)(0.94140625,-0.930690994289608)(0.941545758928571,-0.932225522386713)(0.941685267857143,-0.933760068623555)(0.941824776785714,-0.935294630285)(0.941964285714286,-0.936829204655904)(0.942103794642857,-0.938363789021076)(0.942243303571428,-0.939898380665329)(0.9423828125,-0.941432976873442)(0.942522321428571,-0.942967574930191)(0.942661830357143,-0.944502172120359)(0.942801339285714,-0.946036765728714)(0.942940848214286,-0.947571353040049)(0.943080357142857,-0.949105931339148)(0.943219866071428,-0.950640497910836)(0.943359375,-0.952175050039941)(0.943498883928571,-0.953709585011324)(0.943638392857143,-0.955244100109891)(0.943777901785714,-0.956778592620571)(0.943917410714286,-0.958313059828352)(0.944056919642857,-0.959847499018252)(0.944196428571428,-0.961381907475368)(0.9443359375,-0.962916282484834)(0.944475446428571,-0.964450621331858)(0.944614955357143,-0.965984921301726)(0.944754464285714,-0.967519179679786)(0.944893973214286,-0.969053393751482)(0.945033482142857,-0.970587560802325)(0.945172991071428,-0.97212167811794)(0.9453125,-0.97365574298403)(0.945452008928571,-0.975189752686404)(0.945591517857143,-0.976723704510991)(0.945731026785714,-0.978257595743815)(0.945870535714286,-0.979791423671036)(0.946010044642857,-0.981325185578915)(0.946149553571428,-0.982858878753867)(0.9462890625,-0.984392500482419)(0.946428571428571,-0.985926048051244)(0.946568080357143,-0.98745951874717)(0.946707589285714,-0.988992909857155)(0.946847098214286,-0.990526218668336)(0.946986607142857,-0.992059442467983)(0.947126116071428,-0.993592578543557)(0.947265625,-0.99512562418267)(0.947405133928571,-0.996658576673118)(0.947544642857143,-0.998191433302884)(0.947684151785714,-0.999724191360125)(0.947823660714286,-1.00125684813321)(0.947963169642857,-1.00278940091067)(0.948102678571428,-1.00432184698128)(0.9482421875,-1.00585418363399)(0.948381696428571,-1.00738640815799)(0.948521205357143,-1.00891851784266)(0.948660714285714,-1.01045050997762)(0.948800223214286,-1.01198238185273)(0.948939732142857,-1.01351413075805)(0.949079241071428,-1.01504575398391)(0.94921875,-1.01657724882088)(0.949358258928571,-1.01810861255976)(0.949497767857143,-1.01963984249164)(0.949637276785714,-1.02117093590783)(0.949776785714286,-1.02270189009994)(0.949916294642857,-1.02423270235982)(0.950055803571428,-1.02576336997963)(0.9501953125,-1.02729389025179)(0.950334821428571,-1.028824260469)(0.950474330357143,-1.03035447792427)(0.950613839285714,-1.03188453991091)(0.950753348214286,-1.03341444372251)(0.950892857142857,-1.03494418665299)(0.951032366071428,-1.03647376599656)(0.951171875,-1.03800317904778)(0.951311383928571,-1.03953242310151)(0.951450892857143,-1.04106149545294)(0.951590401785714,-1.0425903933976)(0.951729910714286,-1.04411911423136)(0.951869419642857,-1.04564765525043)(0.952008928571428,-1.04717601375139)(0.9521484375,-1.04870418703114)(0.952287946428571,-1.05023217238696)(0.952427455357143,-1.0517599671165)(0.952566964285714,-1.05328756851777)(0.952706473214286,-1.05481497388918)(0.952845982142857,-1.05634218052948)(0.952985491071428,-1.05786918573785)(0.953125,-1.05939598681383)(0.953264508928571,-1.06092258105737)(0.953404017857143,-1.06244896576882)(0.953543526785714,-1.06397513824895)(0.953683035714286,-1.06550109579893)(0.953822544642857,-1.06702683572035)(0.953962053571428,-1.06855235531522)(0.9541015625,-1.07007765188599)(0.954241071428571,-1.07160272273554)(0.954380580357143,-1.07312756516718)(0.954520089285714,-1.07465217648467)(0.954659598214286,-1.07617655399223)(0.954799107142857,-1.07770069499452)(0.954938616071428,-1.07922459679668)(0.955078125,-1.08074825670428)(0.955217633928571,-1.0822716720234)(0.955357142857143,-1.08379484006057)(0.955496651785714,-1.08531775812282)(0.955636160714286,-1.08684042351766)(0.955775669642857,-1.08836283355307)(0.955915178571428,-1.08988498553756)(0.9560546875,-1.09140687678012)(0.956194196428571,-1.09292850459026)(0.956333705357143,-1.09444986627801)(0.956473214285714,-1.09597095915388)(0.956612723214286,-1.09749178052896)(0.956752232142857,-1.09901232771481)(0.956891741071428,-1.10053259802357)(0.95703125,-1.10205258876789)(0.957170758928571,-1.10357229726097)(0.957310267857143,-1.10509172081658)(0.957449776785714,-1.10661085674902)(0.957589285714286,-1.10812970237316)(0.957728794642857,-1.10964825500443)(0.957868303571428,-1.11116651195884)(0.9580078125,-1.11268447055296)(0.958147321428571,-1.11420212810394)(0.958286830357143,-1.11571948192955)(0.958426339285714,-1.11723652934812)(0.958565848214286,-1.11875326767857)(0.958705357142857,-1.12026969424044)(0.958844866071428,-1.12178580635389)(0.958984375,-1.12330160133964)(0.959123883928571,-1.12481707651908)(0.959263392857143,-1.12633222921421)(0.959402901785714,-1.12784705674763)(0.959542410714286,-1.12936155644261)(0.959681919642857,-1.13087572562303)(0.959821428571428,-1.13238956161344)(0.9599609375,-1.13390306173901)(0.960100446428571,-1.13541622332558)(0.960239955357143,-1.13692904369965)(0.960379464285714,-1.13844152018837)(0.960518973214286,-1.13995365011959)(0.960658482142857,-1.14146543082179)(0.960797991071428,-1.14297685962416)(0.9609375,-1.14448793385658)(0.961077008928571,-1.14599865084959)(0.961216517857143,-1.14750900793446)(0.961356026785714,-1.14901900244313)(0.961495535714286,-1.15052863170828)(0.961635044642857,-1.15203789306325)(0.961774553571428,-1.15354678384216)(0.9619140625,-1.1550553013798)(0.962053571428571,-1.1565634430117)(0.962193080357143,-1.15807120607414)(0.962332589285714,-1.15957858790411)(0.962472098214286,-1.16108558583937)(0.962611607142857,-1.1625921972184)(0.962751116071428,-1.16409841938045)(0.962890625,-1.16560424966551)(0.963030133928571,-1.16710968541436)(0.963169642857143,-1.16861472396852)(0.963309151785714,-1.17011936267031)(0.963448660714286,-1.1716235988628)(0.963588169642857,-1.17312742988986)(0.963727678571428,-1.17463085309615)(0.9638671875,-1.17613386582712)(0.964006696428571,-1.17763646542901)(0.964146205357143,-1.17913864924888)(0.964285714285714,-1.18064041463459)(0.964425223214286,-1.18214175893482)(0.964564732142857,-1.18364267949905)(0.964704241071428,-1.18514317367762)(0.96484375,-1.18664323882167)(0.964983258928571,-1.18814287228318)(0.965122767857143,-1.18964207141499)(0.965262276785714,-1.19114083357076)(0.965401785714286,-1.19263915610501)(0.965541294642857,-1.19413703637312)(0.965680803571428,-1.19563447173133)(0.9658203125,-1.19713145953674)(0.965959821428571,-1.19862799714732)(0.966099330357143,-1.20012408192193)(0.966238839285714,-1.20161971122029)(0.966378348214286,-1.20311488240304)(0.966517857142857,-1.20460959283167)(0.966657366071428,-1.20610383986859)(0.966796875,-1.20759762087712)(0.966936383928571,-1.20909093322145)(0.967075892857143,-1.21058377426674)(0.967215401785714,-1.21207614137902)(0.967354910714286,-1.21356803192525)(0.967494419642857,-1.21505944327334)(0.967633928571428,-1.21655037279211)(0.9677734375,-1.21804081785133)(0.967912946428571,-1.21953077582169)(0.968052455357143,-1.22102024407488)(0.968191964285714,-1.22250921998348)(0.968331473214286,-1.22399770092107)(0.968470982142857,-1.22548568426218)(0.968610491071428,-1.22697316738231)(0.96875,-1.22846014765793)(0.968889508928571,-1.22994662246649)(0.969029017857143,-1.23143258918644)(0.969168526785714,-1.23291804519719)(0.969308035714286,-1.23440298787916)(0.969447544642857,-1.23588741461377)(0.969587053571428,-1.23737132278345)(0.9697265625,-1.23885470977162)(0.969866071428571,-1.24033757296272)(0.970005580357143,-1.24181990974223)(0.970145089285714,-1.24330171749663)(0.970284598214286,-1.24478299361345)(0.970424107142857,-1.24626373548122)(0.970563616071428,-1.24774394048956)(0.970703125,-1.24922360602909)(0.970842633928571,-1.25070272949151)(0.970982142857143,-1.25218130826956)(0.971121651785714,-1.25365933975703)(0.971261160714286,-1.25513682134882)(0.971400669642857,-1.25661375044083)(0.971540178571428,-1.25809012443011)(0.9716796875,-1.25956594071473)(0.971819196428571,-1.26104119669387)(0.971958705357143,-1.26251588976782)(0.972098214285714,-1.26399001733793)(0.972237723214286,-1.26546357680668)(0.972377232142857,-1.26693656557762)(0.972516741071428,-1.26840898105546)(0.97265625,-1.26988082064598)(0.972795758928571,-1.2713520817561)(0.972935267857143,-1.27282276179388)(0.973074776785714,-1.27429285816848)(0.973214285714286,-1.27576236829023)(0.973353794642857,-1.27723128957056)(0.973493303571428,-1.2786996194221)(0.9736328125,-1.28016735525858)(0.973772321428571,-1.2816344944949)(0.973911830357143,-1.28310103454715)(0.974051339285714,-1.28456697283254)(0.974190848214286,-1.2860323067695)(0.974330357142857,-1.2874970337776)(0.974469866071428,-1.2889611512776)(0.974609375,-1.29042465669145)(0.974748883928571,-1.29188754744229)(0.974888392857143,-1.29334982095446)(0.975027901785714,-1.29481147465349)(0.975167410714286,-1.29627250596614)(0.975306919642857,-1.29773291232035)(0.975446428571428,-1.29919269114531)(0.9755859375,-1.30065183987139)(0.975725446428571,-1.30211035593022)(0.975864955357143,-1.30356823675466)(0.976004464285714,-1.30502547977878)(0.976143973214286,-1.30648208243793)(0.976283482142857,-1.30793804216866)(0.976422991071428,-1.30939335640882)(0.9765625,-1.31084802259748)(0.976702008928571,-1.31230203817498)(0.976841517857143,-1.31375540058294)(0.976981026785714,-1.31520810726424)(0.977120535714286,-1.31666015566303)(0.977260044642857,-1.31811154322476)(0.977399553571428,-1.31956226739615)(0.9775390625,-1.32101232562521)(0.977678571428571,-1.32246171536125)(0.977818080357143,-1.3239104340549)(0.977957589285714,-1.32535847915806)(0.978097098214286,-1.32680584812398)(0.978236607142857,-1.3282525384072)(0.978376116071428,-1.32969854746358)(0.978515625,-1.33114387275031)(0.978655133928571,-1.33258851172592)(0.978794642857143,-1.33403246185028)(0.978934151785714,-1.33547572058457)(0.979073660714286,-1.33691828539135)(0.979213169642857,-1.33836015373452)(0.979352678571428,-1.33980132307932)(0.9794921875,-1.34124179089236)(0.979631696428571,-1.34268155464163)(0.979771205357143,-1.34412061179648)(0.979910714285714,-1.34555895982761)(0.980050223214286,-1.34699659620716)(0.980189732142857,-1.34843351840858)(0.980329241071428,-1.34986972390678)(0.98046875,-1.35130521017802)(0.980608258928571,-1.35273997469997)(0.980747767857143,-1.35417401495171)(0.980887276785714,-1.35560732841373)(0.981026785714286,-1.35703991256793)(0.981166294642857,-1.35847176489763)(0.981305803571428,-1.35990288288759)(0.9814453125,-1.36133326402397)(0.981584821428571,-1.36276290579438)(0.981724330357143,-1.36419180568787)(0.981863839285714,-1.36561996119493)(0.982003348214286,-1.36704736980752)(0.982142857142857,-1.368474029019)(0.982282366071428,-1.36989993632426)(0.982421875,-1.37132508921959)(0.982561383928571,-1.37274948520278)(0.982700892857143,-1.3741731217731)(0.982840401785714,-1.37559599643127)(0.982979910714286,-1.37701810667952)(0.983119419642857,-1.37843945002155)(0.983258928571428,-1.37986002396257)(0.9833984375,-1.38127982600927)(0.983537946428571,-1.38269885366984)(0.983677455357143,-1.384117104454)(0.983816964285714,-1.38553457587295)(0.983956473214286,-1.38695126543946)(0.984095982142857,-1.38836717066775)(0.984235491071428,-1.38978228907362)(0.984375,-1.39119661817438)(0.984514508928571,-1.39261015548889)(0.984654017857143,-1.39402289853753)(0.984793526785714,-1.39543484484224)(0.984933035714286,-1.39684599192652)(0.985072544642857,-1.3982563373154)(0.985212053571428,-1.3996658785355)(0.9853515625,-1.40107461311497)(0.985491071428571,-1.40248253858356)(0.985630580357143,-1.40388965247259)(0.985770089285714,-1.40529595231493)(0.985909598214286,-1.40670143564509)(0.986049107142857,-1.40810609999912)(0.986188616071428,-1.40950994291468)(0.986328125,-1.41091296193103)(0.986467633928571,-1.41231515458904)(0.986607142857143,-1.41371651843118)(0.986746651785714,-1.41511705100152)(0.986886160714286,-1.41651674984579)(0.987025669642857,-1.4179156125113)(0.987165178571428,-1.419313636547)(0.9873046875,-1.42071081950348)(0.987444196428571,-1.42210715893295)(0.987583705357143,-1.4235026523893)(0.987723214285714,-1.42489729742802)(0.987862723214286,-1.42629109160627)(0.988002232142857,-1.42768403248288)(0.988141741071428,-1.42907611761832)(0.98828125,-1.43046734457474)(0.988420758928571,-1.43185771091594)(0.988560267857143,-1.43324721420743)(0.988699776785714,-1.43463585201636)(0.988839285714286,-1.43602362191159)(0.988978794642857,-1.43741052146366)(0.989118303571428,-1.43879654824482)(0.9892578125,-1.44018169982899)(0.989397321428571,-1.44156597379181)(0.989536830357143,-1.44294936771063)(0.989676339285714,-1.44433187916451)(0.989815848214286,-1.44571350573422)(0.989955357142857,-1.44709424500227)(0.990094866071428,-1.44847409455288)(0.990234375,-1.449853051972)(0.990373883928571,-1.45123111484733)(0.990513392857143,-1.45260828076832)(0.990652901785714,-1.45398454732612)(0.990792410714286,-1.45535991211369)(0.990931919642857,-1.4567343727257)(0.991071428571428,-1.4581079267586)(0.9912109375,-1.4594805718106)(0.991350446428571,-1.46085230548168)(0.991489955357143,-1.4622231253736)(0.991629464285714,-1.46359302908987)(0.991768973214286,-1.46496201423583)(0.991908482142857,-1.46633007841857)(0.992047991071428,-1.46769721924699)(0.9921875,-1.46906343433177)(0.992327008928571,-1.47042872128542)(0.992466517857143,-1.47179307772224)(0.992606026785714,-1.47315650125832)(0.992745535714286,-1.47451898951162)(0.992885044642857,-1.47588054010185)(0.993024553571428,-1.47724115065062)(0.9931640625,-1.47860081878131)(0.993303571428571,-1.47995954211915)(0.993443080357143,-1.48131731829123)(0.993582589285714,-1.48267414492646)(0.993722098214286,-1.48403001965562)(0.993861607142857,-1.48538494011131)(0.994001116071428,-1.48673890392802)(0.994140625,-1.48809190874208)(0.994280133928571,-1.4894439521917)(0.994419642857143,-1.49079503191696)(0.994559151785714,-1.49214514555979)(0.994698660714286,-1.49349429076405)(0.994838169642857,-1.49484246517544)(0.994977678571428,-1.49618966644157)(0.9951171875,-1.49753589221195)(0.995256696428571,-1.49888114013795)(0.995396205357143,-1.5002254078729)(0.995535714285714,-1.501568693072)(0.995675223214286,-1.50291099339237)(0.995814732142857,-1.50425230649305)(0.995954241071428,-1.50559263003501)(0.99609375,-1.50693196168112)(0.996233258928571,-1.5082702990962)(0.996372767857143,-1.50960763994701)(0.996512276785714,-1.51094398190222)(0.996651785714286,-1.5122793226325)(0.996791294642857,-1.5136136598104)(0.996930803571428,-1.51494699111048)(0.9970703125,-1.51627931420923)(0.997209821428571,-1.5176106267851)(0.997349330357143,-1.51894092651851)(0.997488839285714,-1.52027021109187)(0.997628348214286,-1.52159847818954)(0.997767857142857,-1.52292572549788)(0.997907366071428,-1.52425195070521)(0.998046875,-1.52557715150188)(0.998186383928571,-1.52690132558018)(0.998325892857143,-1.52822447063445)(0.998465401785714,-1.529546584361)(0.998604910714286,-1.53086766445816)(0.998744419642857,-1.53218770862626)(0.998883928571428,-1.53350671456767)(0.9990234375,-1.53482467998676)(0.999162946428571,-1.53614160258993)(0.999302455357143,-1.53745748008561)(0.999441964285714,-1.53877231018426)(0.999581473214286,-1.5400860905984)(0.999720982142857,-1.54139881904255)(0.999860491071428,-1.54271049323333)(1,-1.54402111088937)(1.00013950892857,-1.54533066973137)(1.00027901785714,-1.54663916748211)(1.00041852678571,-1.54794660186639)(1.00055803571429,-1.54925297061114)(1.00069754464286,-1.5505582714453)(1.00083705357143,-1.55186250209994)(1.0009765625,-1.55316566030819)(1.00111607142857,-1.55446774380526)(1.00125558035714,-1.55576875032848)(1.00139508928571,-1.55706867761723)(1.00153459821429,-1.55836752341305)(1.00167410714286,-1.55966528545952)(1.00181361607143,-1.56096196150239)(1.001953125,-1.56225754928949)(1.00209263392857,-1.56355204657076)(1.00223214285714,-1.56484545109829)(1.00237165178571,-1.56613776062627)(1.00251116071429,-1.56742897291105)(1.00265066964286,-1.5687190857111)(1.00279017857143,-1.57000809678702)(1.0029296875,-1.57129600390156)(1.00306919642857,-1.57258280481963)(1.00320870535714,-1.57386849730829)(1.00334821428571,-1.57515307913675)(1.00348772321429,-1.57643654807638)(1.00362723214286,-1.57771890190073)(1.00376674107143,-1.5790001383855)(1.00390625,-1.58028025530858)(1.00404575892857,-1.58155925045003)(1.00418526785714,-1.58283712159211)(1.00432477678571,-1.58411386651924)(1.00446428571429,-1.58538948301807)(1.00460379464286,-1.58666396887739)(1.00474330357143,-1.58793732188826)(1.0048828125,-1.58920953984389)(1.00502232142857,-1.59048062053971)(1.00516183035714,-1.59175056177339)(1.00530133928571,-1.59301936134478)(1.00544084821429,-1.594287017056)(1.00558035714286,-1.59555352671133)(1.00571986607143,-1.59681888811735)(1.005859375,-1.59808309908283)(1.00599888392857,-1.59934615741879)(1.00613839285714,-1.6006080609385)(1.00627790178571,-1.60186880745747)(1.00641741071429,-1.60312839479346)(1.00655691964286,-1.6043868207665)(1.00669642857143,-1.60564408319887)(1.0068359375,-1.60690017991511)(1.00697544642857,-1.60815510874203)(1.00711495535714,-1.60940886750874)(1.00725446428571,-1.61066145404657)(1.00739397321429,-1.6119128661892)(1.00753348214286,-1.61316310177254)(1.00767299107143,-1.61441215863483)(1.0078125,-1.61566003461656)(1.00795200892857,-1.61690672756057)(1.00809151785714,-1.61815223531196)(1.00823102678571,-1.61939655571816)(1.00837053571429,-1.62063968662891)(1.00851004464286,-1.62188162589625)(1.00864955357143,-1.62312237137455)(1.0087890625,-1.62436192092051)(1.00892857142857,-1.62560027239313)(1.00906808035714,-1.62683742365379)(1.00920758928571,-1.62807337256615)(1.00934709821429,-1.62930811699627)(1.00948660714286,-1.63054165481248)(1.00962611607143,-1.63177398388554)(1.009765625,-1.6330051020885)(1.00990513392857,-1.63423500729679)(1.01004464285714,-1.63546369738821)(1.01018415178571,-1.63669117024291)(1.01032366071429,-1.63791742374342)(1.01046316964286,-1.63914245577462)(1.01060267857143,-1.64036626422381)(1.0107421875,-1.64158884698063)(1.01088169642857,-1.64281020193712)(1.01102120535714,-1.64403032698772)(1.01116071428571,-1.64524922002925)(1.01130022321429,-1.64646687896094)(1.01143973214286,-1.6476833016844)(1.01157924107143,-1.64889848610368)(1.01171875,-1.65011243012521)(1.01185825892857,-1.65132513165784)(1.01199776785714,-1.65253658861285)(1.01213727678571,-1.65374679890394)(1.01227678571429,-1.65495576044723)(1.01241629464286,-1.65616347116127)(1.01255580357143,-1.65736992896704)(1.0126953125,-1.65857513178798)(1.01283482142857,-1.65977907754995)(1.01297433035714,-1.66098176418126)(1.01311383928571,-1.66218318961268)(1.01325334821429,-1.66338335177743)(1.01339285714286,-1.66458224861119)(1.01353236607143,-1.6657798780521)(1.013671875,-1.66697623804076)(1.01381138392857,-1.66817132652026)(1.01395089285714,-1.66936514143615)(1.01409040178571,-1.67055768073645)(1.01422991071429,-1.6717489423717)(1.01436941964286,-1.67293892429487)(1.01450892857143,-1.67412762446148)(1.0146484375,-1.6753150408295)(1.01478794642857,-1.67650117135942)(1.01492745535714,-1.67768601401423)(1.01506696428571,-1.67886956675942)(1.01520647321429,-1.680051827563)(1.01534598214286,-1.68123279439548)(1.01548549107143,-1.68241246522991)(1.015625,-1.68359083804184)(1.01576450892857,-1.68476791080936)(1.01590401785714,-1.68594368151309)(1.01604352678571,-1.68711814813618)(1.01618303571429,-1.68829130866431)(1.01632254464286,-1.68946316108572)(1.01646205357143,-1.69063370339119)(1.0166015625,-1.69180293357404)(1.01674107142857,-1.69297084963015)(1.01688058035714,-1.69413744955796)(1.01702008928571,-1.69530273135847)(1.01715959821429,-1.69646669303525)(1.01729910714286,-1.69762933259442)(1.01743861607143,-1.6987906480447)(1.017578125,-1.69995063739738)(1.01771763392857,-1.70110929866631)(1.01785714285714,-1.70226662986795)(1.01799665178571,-1.70342262902134)(1.01813616071429,-1.7045772941481)(1.01827566964286,-1.70573062327248)(1.01841517857143,-1.70688261442129)(1.0185546875,-1.70803326562397)(1.01869419642857,-1.70918257491255)(1.01883370535714,-1.7103305403217)(1.01897321428571,-1.71147715988867)(1.01911272321429,-1.71262243165336)(1.01925223214286,-1.71376635365827)(1.01939174107143,-1.71490892394854)(1.01953125,-1.71605014057195)(1.01967075892857,-1.71719000157888)(1.01981026785714,-1.7183285050224)(1.01994977678571,-1.71946564895816)(1.02008928571429,-1.72060143144451)(1.02022879464286,-1.72173585054243)(1.02036830357143,-1.72286890431555)(1.0205078125,-1.72400059083015)(1.02064732142857,-1.72513090815518)(1.02078683035714,-1.72625985436228)(1.02092633928571,-1.7273874275257)(1.02106584821429,-1.72851362572243)(1.02120535714286,-1.72963844703207)(1.02134486607143,-1.73076188953696)(1.021484375,-1.73188395132208)(1.02162388392857,-1.73300463047512)(1.02176339285714,-1.73412392508645)(1.02190290178571,-1.73524183324914)(1.02204241071429,-1.73635835305896)(1.02218191964286,-1.73747348261437)(1.02232142857143,-1.73858722001656)(1.0224609375,-1.73969956336941)(1.02260044642857,-1.7408105107795)(1.02273995535714,-1.74192006035617)(1.02287946428571,-1.74302821021144)(1.02301897321429,-1.74413495846008)(1.02315848214286,-1.74524030321956)(1.02329799107143,-1.74634424261012)(1.0234375,-1.74744677475471)(1.02357700892857,-1.74854789777901)(1.02371651785714,-1.74964760981146)(1.02385602678571,-1.75074590898325)(1.02399553571429,-1.75184279342831)(1.02413504464286,-1.75293826128332)(1.02427455357143,-1.75403231068772)(1.0244140625,-1.75512493978372)(1.02455357142857,-1.75621614671629)(1.02469308035714,-1.75730592963316)(1.02483258928571,-1.75839428668483)(1.02497209821429,-1.75948121602459)(1.02511160714286,-1.76056671580849)(1.02525111607143,-1.7616507841954)(1.025390625,-1.76273341934692)(1.02553013392857,-1.76381461942747)(1.02566964285714,-1.76489438260428)(1.02580915178571,-1.76597270704734)(1.02594866071429,-1.76704959092948)(1.02608816964286,-1.76812503242628)(1.02622767857143,-1.76919902971618)(1.0263671875,-1.77027158098041)(1.02650669642857,-1.77134268440301)(1.02664620535714,-1.77241233817084)(1.02678571428571,-1.77348054047358)(1.02692522321429,-1.77454728950375)(1.02706473214286,-1.77561258345668)(1.02720424107143,-1.77667642053054)(1.02734375,-1.77773879892633)(1.02748325892857,-1.7787997168479)(1.02762276785714,-1.77985917250193)(1.02776227678571,-1.78091716409795)(1.02790178571429,-1.78197368984836)(1.02804129464286,-1.78302874796837)(1.02818080357143,-1.78408233667608)(1.0283203125,-1.78513445419245)(1.02845982142857,-1.78618509874127)(1.02859933035714,-1.78723426854925)(1.02873883928571,-1.78828196184592)(1.02887834821429,-1.78932817686371)(1.02901785714286,-1.79037291183792)(1.02915736607143,-1.79141616500674)(1.029296875,-1.79245793461124)(1.02943638392857,-1.79349821889535)(1.02957589285714,-1.79453701610594)(1.02971540178571,-1.79557432449273)(1.02985491071429,-1.79661014230837)(1.02999441964286,-1.79764446780839)(1.03013392857143,-1.79867729925123)(1.0302734375,-1.79970863489825)(1.03041294642857,-1.8007384730137)(1.03055245535714,-1.80176681186477)(1.03069196428571,-1.80279364972153)(1.03083147321429,-1.80381898485702)(1.03097098214286,-1.80484281554717)(1.03111049107143,-1.80586514007084)(1.03125,-1.80688595670984)(1.03138950892857,-1.80790526374889)(1.03152901785714,-1.80892305947568)(1.03166852678571,-1.80993934218082)(1.03180803571429,-1.81095411015785)(1.03194754464286,-1.8119673617033)(1.03208705357143,-1.81297909511661)(1.0322265625,-1.81398930870021)(1.03236607142857,-1.81499800075945)(1.03250558035714,-1.81600516960267)(1.03264508928571,-1.81701081354117)(1.03278459821429,-1.81801493088922)(1.03292410714286,-1.81901751996404)(1.03306361607143,-1.82001857908584)(1.033203125,-1.82101810657783)(1.03334263392857,-1.82201610076615)(1.03348214285714,-1.82301255997997)(1.03362165178571,-1.82400748255142)(1.03376116071429,-1.82500086681564)(1.03390066964286,-1.82599271111075)(1.03404017857143,-1.82698301377788)(1.0341796875,-1.82797177316114)(1.03431919642857,-1.82895898760765)(1.03445870535714,-1.82994465546757)(1.03459821428571,-1.83092877509402)(1.03473772321429,-1.83191134484317)(1.03487723214286,-1.83289236307418)(1.03501674107143,-1.83387182814925)(1.03515625,-1.8348497384336)(1.03529575892857,-1.83582609229546)(1.03543526785714,-1.83680088810611)(1.03557477678571,-1.83777412423985)(1.03571428571429,-1.83874579907402)(1.03585379464286,-1.83971591098899)(1.03599330357143,-1.84068445836819)(1.0361328125,-1.84165143959807)(1.03627232142857,-1.84261685306816)(1.03641183035714,-1.84358069717101)(1.03655133928571,-1.84454297030224)(1.03669084821429,-1.84550367086054)(1.03683035714286,-1.84646279724762)(1.03696986607143,-1.8474203478683)(1.037109375,-1.84837632113045)(1.03724888392857,-1.84933071544499)(1.03738839285714,-1.85028352922594)(1.03752790178571,-1.85123476089038)(1.03766741071429,-1.85218440885849)(1.03780691964286,-1.8531324715535)(1.03794642857143,-1.85407894740175)(1.0380859375,-1.85502383483266)(1.03822544642857,-1.85596713227875)(1.03836495535714,-1.85690883817562)(1.03850446428571,-1.85784895096198)(1.03864397321429,-1.85878746907963)(1.03878348214286,-1.85972439097348)(1.03892299107143,-1.86065971509155)(1.0390625,-1.86159343988497)(1.03920200892857,-1.86252556380796)(1.03934151785714,-1.8634560853179)(1.03948102678571,-1.86438500287524)(1.03962053571429,-1.86531231494359)(1.03976004464286,-1.86623801998966)(1.03989955357143,-1.8671621164833)(1.0400390625,-1.86808460289749)(1.04017857142857,-1.86900547770834)(1.04031808035714,-1.8699247393951)(1.04045758928571,-1.87084238644015)(1.04059709821429,-1.87175841732903)(1.04073660714286,-1.87267283055041)(1.04087611607143,-1.87358562459612)(1.041015625,-1.87449679796113)(1.04115513392857,-1.87540634914357)(1.04129464285714,-1.87631427664473)(1.04143415178571,-1.87722057896905)(1.04157366071429,-1.87812525462414)(1.04171316964286,-1.87902830212079)(1.04185267857143,-1.87992971997293)(1.0419921875,-1.88082950669769)(1.04213169642857,-1.88172766081535)(1.04227120535714,-1.88262418084938)(1.04241071428571,-1.88351906532643)(1.04255022321429,-1.88441231277634)(1.04268973214286,-1.88530392173211)(1.04282924107143,-1.88619389072997)(1.04296875,-1.88708221830931)(1.04310825892857,-1.88796890301272)(1.04324776785714,-1.888853943386)(1.04338727678571,-1.88973733797813)(1.04352678571429,-1.89061908534131)(1.04366629464286,-1.89149918403094)(1.04380580357143,-1.89237763260564)(1.0439453125,-1.89325442962721)(1.04408482142857,-1.8941295736607)(1.04422433035714,-1.89500306327437)(1.04436383928571,-1.89587489703967)(1.04450334821429,-1.89674507353132)(1.04464285714286,-1.89761359132723)(1.04478236607143,-1.89848044900854)(1.044921875,-1.89934564515964)(1.04506138392857,-1.90020917836814)(1.04520089285714,-1.9010710472249)(1.04534040178571,-1.901931250324)(1.04547991071429,-1.90278978626278)(1.04561941964286,-1.9036466536418)(1.04575892857143,-1.9045018510649)(1.0458984375,-1.90535537713916)(1.04603794642857,-1.90620723047488)(1.04617745535714,-1.90705740968567)(1.04631696428571,-1.90790591338835)(1.04645647321429,-1.90875274020305)(1.04659598214286,-1.9095978887531)(1.04673549107143,-1.91044135766516)(1.046875,-1.91128314556912)(1.04701450892857,-1.91212325109815)(1.04715401785714,-1.91296167288871)(1.04729352678571,-1.91379840958051)(1.04743303571429,-1.91463345981656)(1.04757254464286,-1.91546682224314)(1.04771205357143,-1.91629849550984)(1.0478515625,-1.9171284782695)(1.04799107142857,-1.91795676917827)(1.04813058035714,-1.9187833668956)(1.04827008928571,-1.91960827008422)(1.04840959821429,-1.92043147741017)(1.04854910714286,-1.92125298754278)(1.04868861607143,-1.9220727991547)(1.048828125,-1.92289091092187)(1.04896763392857,-1.92370732152353)(1.04910714285714,-1.92452202964226)(1.04924665178571,-1.92533503396393)(1.04938616071429,-1.92614633317774)(1.04952566964286,-1.9269559259762)(1.04966517857143,-1.92776381105513)(1.0498046875,-1.92856998711371)(1.04994419642857,-1.92937445285441)(1.05008370535714,-1.93017720698304)(1.05022321428571,-1.93097824820874)(1.05036272321429,-1.931777575244)(1.05050223214286,-1.93257518680463)(1.05064174107143,-1.93337108160978)(1.05078125,-1.93416525838194)(1.05092075892857,-1.93495771584695)(1.05106026785714,-1.93574845273399)(1.05119977678571,-1.9365374677756)(1.05133928571429,-1.93732475970765)(1.05147879464286,-1.93811032726939)(1.05161830357143,-1.93889416920341)(1.0517578125,-1.93967628425566)(1.05189732142857,-1.94045667117545)(1.05203683035714,-1.94123532871546)(1.05217633928571,-1.94201225563173)(1.05231584821429,-1.94278745068367)(1.05245535714286,-1.94356091263406)(1.05259486607143,-1.94433264024906)(1.052734375,-1.94510263229819)(1.05287388392857,-1.94587088755436)(1.05301339285714,-1.94663740479386)(1.05315290178571,-1.94740218279636)(1.05329241071429,-1.94816522034491)(1.05343191964286,-1.94892651622596)(1.05357142857143,-1.94968606922934)(1.0537109375,-1.95044387814828)(1.05385044642857,-1.95119994177938)(1.05398995535714,-1.95195425892267)(1.05412946428571,-1.95270682838157)(1.05426897321429,-1.95345764896289)(1.05440848214286,-1.95420671947684)(1.05454799107143,-1.95495403873707)(1.0546875,-1.95569960556059)(1.05482700892857,-1.95644341876786)(1.05496651785714,-1.95718547718273)(1.05510602678571,-1.95792577963248)(1.05524553571429,-1.9586643249478)(1.05538504464286,-1.9594011119628)(1.05552455357143,-1.96013613951501)(1.0556640625,-1.9608694064454)(1.05580357142857,-1.96160091159834)(1.05594308035714,-1.96233065382165)(1.05608258928571,-1.96305863196658)(1.05622209821429,-1.9637848448878)(1.05636160714286,-1.96450929144342)(1.05650111607143,-1.96523197049499)(1.056640625,-1.96595288090752)(1.05678013392857,-1.96667202154942)(1.05691964285714,-1.96738939129259)(1.05705915178571,-1.96810498901233)(1.05719866071429,-1.96881881358744)(1.05733816964286,-1.96953086390012)(1.05747767857143,-1.97024113883607)(1.0576171875,-1.97094963728441)(1.05775669642857,-1.97165635813773)(1.05789620535714,-1.97236130029208)(1.05803571428571,-1.97306446264698)(1.05817522321429,-1.9737658441054)(1.05831473214286,-1.97446544357378)(1.05845424107143,-1.97516325996204)(1.05859375,-1.97585929218355)(1.05873325892857,-1.97655353915517)(1.05887276785714,-1.97724599979723)(1.05901227678571,-1.97793667303352)(1.05915178571429,-1.97862555779134)(1.05929129464286,-1.97931265300145)(1.05943080357143,-1.9799979575981)(1.0595703125,-1.98068147051901)(1.05970982142857,-1.98136319070541)(1.05984933035714,-1.982043117102)(1.05998883928571,-1.98272124865699)(1.06012834821429,-1.98339758432206)(1.06026785714286,-1.98407212305241)(1.06040736607143,-1.98474486380673)(1.060546875,-1.98541580554719)(1.06068638392857,-1.98608494723948)(1.06082589285714,-1.98675228785279)(1.06096540178571,-1.98741782635982)(1.06110491071429,-1.98808156173678)(1.06124441964286,-1.98874349296336)(1.06138392857143,-1.9894036190228)(1.0615234375,-1.99006193890183)(1.06166294642857,-1.9907184515907)(1.06180245535714,-1.99137315608318)(1.06194196428571,-1.99202605137656)(1.06208147321429,-1.99267713647166)(1.06222098214286,-1.99332641037279)(1.06236049107143,-1.99397387208783)(1.0625,-1.99461952062815)(1.06263950892857,-1.99526335500867)(1.06277901785714,-1.99590537424784)(1.06291852678571,-1.99654557736762)(1.06305803571429,-1.99718396339354)(1.06319754464286,-1.99782053135464)(1.06333705357143,-1.99845528028351)(1.0634765625,-1.99908820921628)(1.06361607142857,-1.99971931719261)(1.06375558035714,-2.00034860325572)(1.06389508928571,-2.00097606645237)(1.06403459821429,-2.00160170583286)(1.06417410714286,-2.00222552045106)(1.06431361607143,-2.00284750936437)(1.064453125,-2.00346767163375)(1.06459263392857,-2.00408600632373)(1.06473214285714,-2.00470251250237)(1.06487165178571,-2.0053171892413)(1.06501116071429,-2.00593003561573)(1.06515066964286,-2.0065410507044)(1.06529017857143,-2.00715023358964)(1.0654296875,-2.00775758335733)(1.06556919642857,-2.00836309909693)(1.06570870535714,-2.00896677990145)(1.06584821428571,-2.0095686248675)(1.06598772321429,-2.01016863309524)(1.06612723214286,-2.01076680368842)(1.06626674107143,-2.01136313575434)(1.06640625,-2.01195762840392)(1.06654575892857,-2.01255028075162)(1.06668526785714,-2.0131410919155)(1.06682477678571,-2.01373006101721)(1.06696428571429,-2.01431718718198)(1.06710379464286,-2.01490246953862)(1.06724330357143,-2.01548590721953)(1.0673828125,-2.0160674993607)(1.06752232142857,-2.01664724510172)(1.06766183035714,-2.01722514358578)(1.06780133928571,-2.01780119395963)(1.06794084821429,-2.01837539537366)(1.06808035714286,-2.01894774698183)(1.06821986607143,-2.01951824794171)(1.068359375,-2.02008689741448)(1.06849888392857,-2.02065369456491)(1.06863839285714,-2.02121863856137)(1.06877790178571,-2.02178172857586)(1.06891741071429,-2.02234296378398)(1.06905691964286,-2.02290234336491)(1.06919642857143,-2.0234598665015)(1.0693359375,-2.02401553238015)(1.06947544642857,-2.02456934019092)(1.06961495535714,-2.02512128912748)(1.06975446428571,-2.02567137838708)(1.06989397321429,-2.02621960717065)(1.07003348214286,-2.02676597468268)(1.07017299107143,-2.02731048013134)(1.0703125,-2.02785312272838)(1.07045200892857,-2.02839390168919)(1.07059151785714,-2.0289328162328)(1.07073102678571,-2.02946986558184)(1.07087053571429,-2.0300050489626)(1.07101004464286,-2.03053836560499)(1.07114955357143,-2.03106981474255)(1.0712890625,-2.03159939561246)(1.07142857142857,-2.03212710745552)(1.07156808035714,-2.03265294951619)(1.07170758928571,-2.03317692104257)(1.07184709821429,-2.03369902128637)(1.07198660714286,-2.03421924950298)(1.07212611607143,-2.03473760495141)(1.072265625,-2.03525408689431)(1.07240513392857,-2.03576869459801)(1.07254464285714,-2.03628142733244)(1.07268415178571,-2.03679228437123)(1.07282366071429,-2.03730126499161)(1.07296316964286,-2.03780836847451)(1.07310267857143,-2.03831359410447)(1.0732421875,-2.03881694116972)(1.07338169642857,-2.03931840896212)(1.07352120535714,-2.03981799677721)(1.07366071428571,-2.04031570391417)(1.07380022321429,-2.04081152967585)(1.07393973214286,-2.04130547336876)(1.07407924107143,-2.04179753430307)(1.07421875,-2.04228771179263)(1.07435825892857,-2.04277600515493)(1.07449776785714,-2.04326241371114)(1.07463727678571,-2.0437469367861)(1.07477678571429,-2.04422957370833)(1.07491629464286,-2.04471032380999)(1.07505580357143,-2.04518918642694)(1.0751953125,-2.04566616089871)(1.07533482142857,-2.04614124656849)(1.07547433035714,-2.04661444278316)(1.07561383928571,-2.04708574889328)(1.07575334821429,-2.04755516425307)(1.07589285714286,-2.04802268822045)(1.07603236607143,-2.04848832015701)(1.076171875,-2.04895205942803)(1.07631138392857,-2.04941390540245)(1.07645089285714,-2.04987385745294)(1.07659040178571,-2.05033191495582)(1.07672991071429,-2.0507880772911)(1.07686941964286,-2.05124234384249)(1.07700892857143,-2.05169471399739)(1.0771484375,-2.05214518714688)(1.07728794642857,-2.05259376268575)(1.07742745535714,-2.05304044001245)(1.07756696428571,-2.05348521852917)(1.07770647321429,-2.05392809764176)(1.07784598214286,-2.05436907675978)(1.07798549107143,-2.05480815529648)(1.078125,-2.05524533266883)(1.07826450892857,-2.05568060829747)(1.07840401785714,-2.05611398160677)(1.07854352678571,-2.05654545202478)(1.07868303571429,-2.05697501898327)(1.07882254464286,-2.05740268191771)(1.07896205357143,-2.05782844026726)(1.0791015625,-2.05825229347481)(1.07924107142857,-2.05867424098695)(1.07938058035714,-2.05909428225397)(1.07952008928571,-2.05951241672988)(1.07965959821429,-2.0599286438724)(1.07979910714286,-2.06034296314297)(1.07993861607143,-2.06075537400671)(1.080078125,-2.0611658759325)(1.08021763392857,-2.0615744683929)(1.08035714285714,-2.06198115086421)(1.08049665178571,-2.06238592282643)(1.08063616071429,-2.06278878376329)(1.08077566964286,-2.06318973316223)(1.08091517857143,-2.06358877051442)(1.0810546875,-2.06398589531475)(1.08119419642857,-2.06438110706181)(1.08133370535714,-2.06477440525796)(1.08147321428571,-2.06516578940923)(1.08161272321429,-2.06555525902542)(1.08175223214286,-2.06594281362003)(1.08189174107143,-2.0663284527103)(1.08203125,-2.06671217581719)(1.08217075892857,-2.0670939824654)(1.08231026785714,-2.06747387218334)(1.08244977678571,-2.06785184450317)(1.08258928571429,-2.06822789896077)(1.08272879464286,-2.06860203509576)(1.08286830357143,-2.0689742524515)(1.0830078125,-2.06934455057506)(1.08314732142857,-2.06971292901728)(1.08328683035714,-2.0700793873327)(1.08342633928571,-2.07044392507962)(1.08356584821429,-2.07080654182007)(1.08370535714286,-2.07116723711983)(1.08384486607143,-2.0715260105484)(1.083984375,-2.07188286167904)(1.08412388392857,-2.07223779008874)(1.08426339285714,-2.07259079535823)(1.08440290178571,-2.07294187707199)(1.08454241071429,-2.07329103481825)(1.08468191964286,-2.07363826818897)(1.08482142857143,-2.07398357677985)(1.0849609375,-2.07432696019036)(1.08510044642857,-2.07466841802371)(1.08523995535714,-2.07500794988683)(1.08537946428571,-2.07534555539045)(1.08551897321429,-2.075681234149)(1.08565848214286,-2.07601498578068)(1.08579799107143,-2.07634680990744)(1.0859375,-2.076676706155)(1.08607700892857,-2.07700467415279)(1.08621651785714,-2.07733071353403)(1.08635602678571,-2.07765482393568)(1.08649553571429,-2.07797700499846)(1.08663504464286,-2.07829725636683)(1.08677455357143,-2.07861557768902)(1.0869140625,-2.07893196861701)(1.08705357142857,-2.07924642880654)(1.08719308035714,-2.0795589579171)(1.08733258928571,-2.07986955561196)(1.08747209821429,-2.08017822155813)(1.08761160714286,-2.08048495542638)(1.08775111607143,-2.08078975689125)(1.087890625,-2.08109262563103)(1.08803013392857,-2.08139356132777)(1.08816964285714,-2.08169256366731)(1.08830915178571,-2.08198963233921)(1.08844866071429,-2.08228476703683)(1.08858816964286,-2.08257796745728)(1.08872767857143,-2.08286923330142)(1.0888671875,-2.08315856427391)(1.08900669642857,-2.08344596008313)(1.08914620535714,-2.08373142044128)(1.08928571428571,-2.08401494506428)(1.08942522321429,-2.08429653367184)(1.08956473214286,-2.08457618598743)(1.08970424107143,-2.0848539017383)(1.08984375,-2.08512968065546)(1.08998325892857,-2.08540352247369)(1.09012276785714,-2.08567542693154)(1.09026227678571,-2.08594539377134)(1.09040178571429,-2.08621342273917)(1.09054129464286,-2.08647951358491)(1.09068080357143,-2.08674366606218)(1.0908203125,-2.0870058799284)(1.09095982142857,-2.08726615494475)(1.09109933035714,-2.08752449087619)(1.09123883928571,-2.08778088749145)(1.09137834821429,-2.08803534456302)(1.09151785714286,-2.0882878618672)(1.09165736607143,-2.08853843918404)(1.091796875,-2.08878707629736)(1.09193638392857,-2.08903377299477)(1.09207589285714,-2.08927852906766)(1.09221540178571,-2.08952134431118)(1.09235491071429,-2.08976221852428)(1.09249441964286,-2.09000115150967)(1.09263392857143,-2.09023814307384)(1.0927734375,-2.09047319302707)(1.09291294642857,-2.0907063011834)(1.09305245535714,-2.09093746736067)(1.09319196428571,-2.09116669138049)(1.09333147321429,-2.09139397306824)(1.09347098214286,-2.0916193122531)(1.09361049107143,-2.09184270876801)(1.09375,-2.09206416244972)(1.09388950892857,-2.09228367313872)(1.09402901785714,-2.09250124067933)(1.09416852678571,-2.0927168649196)(1.09430803571429,-2.09293054571141)(1.09444754464286,-2.09314228291039)(1.09458705357143,-2.09335207637597)(1.0947265625,-2.09355992597135)(1.09486607142857,-2.09376583156353)(1.09500558035714,-2.09396979302328)(1.09514508928571,-2.09417181022515)(1.09528459821429,-2.09437188304749)(1.09542410714286,-2.09457001137243)(1.09556361607143,-2.09476619508587)(1.095703125,-2.09496043407751)(1.09584263392857,-2.09515272824083)(1.09598214285714,-2.09534307747309)(1.09612165178571,-2.09553148167535)(1.09626116071429,-2.09571794075245)(1.09640066964286,-2.095902454613)(1.09654017857143,-2.09608502316942)(1.0966796875,-2.09626564633789)(1.09681919642857,-2.0964443240384)(1.09695870535714,-2.09662105619471)(1.09709821428571,-2.09679584273439)(1.09723772321429,-2.09696868358876)(1.09737723214286,-2.09713957869296)(1.09751674107143,-2.09730852798589)(1.09765625,-2.09747553141027)(1.09779575892857,-2.09764058891257)(1.09793526785714,-2.09780370044308)(1.09807477678571,-2.09796486595586)(1.09821428571429,-2.09812408540875)(1.09835379464286,-2.0982813587634)(1.09849330357143,-2.09843668598523)(1.0986328125,-2.09859006704345)(1.09877232142857,-2.09874150191107)(1.09891183035714,-2.09889099056487)(1.09905133928571,-2.09903853298543)(1.09919084821429,-2.09918412915712)(1.09933035714286,-2.09932777906808)(1.09946986607143,-2.09946948271026)(1.099609375,-2.09960924007939)(1.09974888392857,-2.09974705117498)(1.09988839285714,-2.09988291600034)(1.10002790178571,-2.10001683456256)(1.10016741071429,-2.10014880687252)(1.10030691964286,-2.10027883294489)(1.10044642857143,-2.10040691279813)(1.1005859375,-2.10053304645447)(1.10072544642857,-2.10065723393995)(1.10086495535714,-2.1007794752844)(1.10100446428571,-2.10089977052141)(1.10114397321429,-2.10101811968839)(1.10128348214286,-2.10113452282651)(1.10142299107143,-2.10124897998075)(1.1015625,-2.10136149119986)(1.10170200892857,-2.10147205653639)(1.10184151785714,-2.10158067604667)(1.10198102678571,-2.10168734979082)(1.10212053571429,-2.10179207783275)(1.10226004464286,-2.10189486024015)(1.10239955357143,-2.1019956970845)(1.1025390625,-2.10209458844106)(1.10267857142857,-2.10219153438889)(1.10281808035714,-2.10228653501083)(1.10295758928571,-2.10237959039351)(1.10309709821429,-2.10247070062732)(1.10323660714286,-2.10255986580648)(1.10337611607143,-2.10264708602897)(1.103515625,-2.10273236139654)(1.10365513392857,-2.10281569201477)(1.10379464285714,-2.10289707799297)(1.10393415178571,-2.10297651944429)(1.10407366071429,-2.10305401648562)(1.10421316964286,-2.10312956923765)(1.10435267857143,-2.10320317782487)(1.1044921875,-2.10327484237554)(1.10463169642857,-2.10334456302169)(1.10477120535714,-2.10341233989916)(1.10491071428571,-2.10347817314755)(1.10505022321429,-2.10354206291026)(1.10518973214286,-2.10360400933446)(1.10532924107143,-2.10366401257111)(1.10546875,-2.10372207277496)(1.10560825892857,-2.10377819010452)(1.10574776785714,-2.10383236472209)(1.10588727678571,-2.10388459679376)(1.10602678571429,-2.1039348864894)(1.10616629464286,-2.10398323398265)(1.10630580357143,-2.10402963945093)(1.1064453125,-2.10407410307545)(1.10658482142857,-2.10411662504119)(1.10672433035714,-2.10415720553692)(1.10686383928571,-2.10419584475518)(1.10700334821429,-2.10423254289228)(1.10714285714286,-2.10426730014833)(1.10728236607143,-2.10430011672721)(1.107421875,-2.10433099283656)(1.10756138392857,-2.10435992868781)(1.10770089285714,-2.10438692449617)(1.10784040178571,-2.10441198048062)(1.10797991071429,-2.10443509686392)(1.10811941964286,-2.1044562738726)(1.10825892857143,-2.10447551173696)(1.1083984375,-2.10449281069109)(1.10853794642857,-2.10450817097284)(1.10867745535714,-2.10452159282383)(1.10881696428571,-2.10453307648947)(1.10895647321429,-2.10454262221893)(1.10909598214286,-2.10455023026516)(1.10923549107143,-2.10455590088486)(1.109375,-2.10455963433853)(1.10951450892857,-2.10456143089041)(1.10965401785714,-2.10456129080855)(1.10979352678571,-2.10455921436472)(1.10993303571429,-2.1045552018345)(1.11007254464286,-2.10454925349722)(1.11021205357143,-2.10454136963598)(1.1103515625,-2.10453155053764)(1.11049107142857,-2.10451979649284)(1.11063058035714,-2.10450610779597)(1.11077008928571,-2.10449048474521)(1.11090959821429,-2.10447292764247)(1.11104910714286,-2.10445343679346)(1.11118861607143,-2.10443201250763)(1.111328125,-2.1044086550982)(1.11146763392857,-2.10438336488216)(1.11160714285714,-2.10435614218024)(1.11174665178571,-2.10432698731696)(1.11188616071429,-2.10429590062057)(1.11202566964286,-2.10426288242311)(1.11216517857143,-2.10422793306036)(1.1123046875,-2.10419105287187)(1.11244419642857,-2.10415224220093)(1.11258370535714,-2.1041115013946)(1.11272321428571,-2.1040688308037)(1.11286272321429,-2.1040242307828)(1.11300223214286,-2.10397770169023)(1.11314174107143,-2.10392924388807)(1.11328125,-2.10387885774214)(1.11342075892857,-2.10382654362205)(1.11356026785714,-2.10377230190113)(1.11369977678571,-2.10371613295648)(1.11383928571429,-2.10365803716892)(1.11397879464286,-2.10359801492307)(1.11411830357143,-2.10353606660725)(1.1142578125,-2.10347219261357)(1.11439732142857,-2.10340639333785)(1.11453683035714,-2.10333866917969)(1.11467633928571,-2.10326902054242)(1.11481584821429,-2.10319744783312)(1.11495535714286,-2.1031239514626)(1.11509486607143,-2.10304853184544)(1.115234375,-2.10297118939995)(1.11537388392857,-2.10289192454817)(1.11551339285714,-2.10281073771591)(1.11565290178571,-2.10272762933269)(1.11579241071429,-2.10264259983179)(1.11593191964286,-2.10255564965022)(1.11607142857143,-2.10246677922873)(1.1162109375,-2.10237598901182)(1.11635044642857,-2.10228327944771)(1.11648995535714,-2.10218865098836)(1.11662946428571,-2.10209210408946)(1.11676897321429,-2.10199363921044)(1.11690848214286,-2.10189325681448)(1.11704799107143,-2.10179095736845)(1.1171875,-2.10168674134298)(1.11732700892857,-2.10158060921244)(1.11746651785714,-2.10147256145491)(1.11760602678571,-2.10136259855219)(1.11774553571429,-2.10125072098983)(1.11788504464286,-2.1011369292571)(1.11802455357143,-2.10102122384698)(1.1181640625,-2.10090360525621)(1.11830357142857,-2.10078407398521)(1.11844308035714,-2.10066263053814)(1.11858258928571,-2.10053927542291)(1.11872209821429,-2.1004140091511)(1.11886160714286,-2.10028683223805)(1.11900111607143,-2.10015774520279)(1.119140625,-2.10002674856809)(1.11928013392857,-2.09989384286044)(1.11941964285714,-2.09975902861001)(1.11955915178571,-2.09962230635072)(1.11969866071429,-2.09948367662019)(1.11983816964286,-2.09934313995976)(1.11997767857143,-2.09920069691447)(1.1201171875,-2.09905634803307)(1.12025669642857,-2.09891009386803)(1.12039620535714,-2.09876193497553)(1.12053571428571,-2.09861187191544)(1.12067522321429,-2.09845990525135)(1.12081473214286,-2.09830603555055)(1.12095424107143,-2.09815026338404)(1.12109375,-2.09799258932652)(1.12123325892857,-2.09783301395638)(1.12137276785714,-2.09767153785572)(1.12151227678571,-2.09750816161035)(1.12165178571429,-2.09734288580975)(1.12179129464286,-2.09717571104714)(1.12193080357143,-2.09700663791939)(1.1220703125,-2.09683566702709)(1.12220982142857,-2.09666279897452)(1.12234933035714,-2.09648803436965)(1.12248883928571,-2.09631137382414)(1.12262834821429,-2.09613281795335)(1.12276785714286,-2.09595236737632)(1.12290736607143,-2.09577002271577)(1.123046875,-2.09558578459813)(1.12318638392857,-2.09539965365348)(1.12332589285714,-2.09521163051562)(1.12346540178571,-2.09502171582202)(1.12360491071429,-2.09482991021381)(1.12374441964286,-2.09463621433584)(1.12388392857143,-2.0944406288366)(1.1240234375,-2.09424315436829)(1.12416294642857,-2.09404379158676)(1.12430245535714,-2.09384254115155)(1.12444196428571,-2.09363940372588)(1.12458147321429,-2.09343437997663)(1.12472098214286,-2.09322747057434)(1.12486049107143,-2.09301867619326)(1.125,-2.09280799751126)(1.12513950892857,-2.09259543520992)(1.12527901785714,-2.09238098997445)(1.12541852678571,-2.09216466249376)(1.12555803571429,-2.09194645346039)(1.12569754464286,-2.09172636357056)(1.12583705357143,-2.09150439352415)(1.1259765625,-2.0912805440247)(1.12611607142857,-2.09105481577941)(1.12625558035714,-2.09082720949911)(1.12639508928571,-2.09059772589833)(1.12653459821429,-2.09036636569521)(1.12667410714286,-2.09013312961159)(1.12681361607143,-2.08989801837291)(1.126953125,-2.08966103270829)(1.12709263392857,-2.08942217335049)(1.12723214285714,-2.08918144103592)(1.12737165178571,-2.08893883650464)(1.12751116071429,-2.08869436050034)(1.12765066964286,-2.08844801377037)(1.12779017857143,-2.08819979706569)(1.1279296875,-2.08794971114094)(1.12806919642857,-2.08769775675437)(1.12820870535714,-2.08744393466787)(1.12834821428571,-2.08718824564698)(1.12848772321429,-2.08693069046086)(1.12862723214286,-2.0866712698823)(1.12876674107143,-2.08640998468774)(1.12890625,-2.08614683565722)(1.12904575892857,-2.08588182357443)(1.12918526785714,-2.08561494922667)(1.12932477678571,-2.08534621340489)(1.12946428571429,-2.08507561690363)(1.12960379464286,-2.08480316052108)(1.12974330357143,-2.08452884505902)(1.1298828125,-2.08425267132289)(1.13002232142857,-2.0839746401217)(1.13016183035714,-2.08369475226811)(1.13030133928571,-2.08341300857837)(1.13044084821429,-2.08312940987237)(1.13058035714286,-2.08284395697358)(1.13071986607143,-2.08255665070909)(1.130859375,-2.08226749190961)(1.13099888392857,-2.08197648140944)(1.13113839285714,-2.08168362004648)(1.13127790178571,-2.08138890866226)(1.13141741071429,-2.08109234810188)(1.13155691964286,-2.08079393921405)(1.13169642857143,-2.08049368285107)(1.1318359375,-2.08019157986886)(1.13197544642857,-2.07988763112691)(1.13211495535714,-2.07958183748831)(1.13225446428571,-2.07927419981975)(1.13239397321429,-2.07896471899148)(1.13253348214286,-2.07865339587737)(1.13267299107143,-2.07834023135487)(1.1328125,-2.07802522630499)(1.13295200892857,-2.07770838161234)(1.13309151785714,-2.07738969816513)(1.13323102678571,-2.0770691768551)(1.13337053571429,-2.07674681857761)(1.13351004464286,-2.07642262423158)(1.13364955357143,-2.0760965947195)(1.1337890625,-2.07576873094744)(1.13392857142857,-2.07543903382503)(1.13406808035714,-2.07510750426547)(1.13420758928571,-2.07477414318554)(1.13434709821429,-2.07443895150557)(1.13448660714286,-2.07410193014945)(1.13462611607143,-2.07376308004465)(1.134765625,-2.07342240212219)(1.13490513392857,-2.07307989731663)(1.13504464285714,-2.07273556656611)(1.13518415178571,-2.07238941081231)(1.13532366071429,-2.07204143100047)(1.13546316964286,-2.07169162807938)(1.13560267857143,-2.07134000300137)(1.1357421875,-2.07098655672233)(1.13588169642857,-2.07063129020167)(1.13602120535714,-2.07027420440236)(1.13616071428571,-2.06991530029093)(1.13630022321429,-2.0695545788374)(1.13643973214286,-2.06919204101538)(1.13657924107143,-2.06882768780198)(1.13671875,-2.06846152017785)(1.13685825892857,-2.06809353912718)(1.13699776785714,-2.06772374563768)(1.13713727678571,-2.0673521407006)(1.13727678571429,-2.06697872531069)(1.13741629464286,-2.06660350046627)(1.13755580357143,-2.06622646716913)(1.1376953125,-2.0658476264246)(1.13783482142857,-2.06546697924155)(1.13797433035714,-2.06508452663234)(1.13811383928571,-2.06470026961284)(1.13825334821429,-2.06431420920244)(1.13839285714286,-2.06392634642406)(1.13853236607143,-2.0635366823041)(1.138671875,-2.06314521787247)(1.13881138392857,-2.06275195416259)(1.13895089285714,-2.06235689221139)(1.13909040178571,-2.06196003305929)(1.13922991071429,-2.06156137775021)(1.13936941964286,-2.06116092733155)(1.13950892857143,-2.06075868285424)(1.1396484375,-2.06035464537267)(1.13978794642857,-2.05994881594472)(1.13992745535714,-2.05954119563179)(1.14006696428571,-2.05913178549873)(1.14020647321429,-2.05872058661388)(1.14034598214286,-2.05830760004909)(1.14048549107143,-2.05789282687964)(1.140625,-2.05747626818434)(1.14076450892857,-2.05705792504543)(1.14090401785714,-2.05663779854866)(1.14104352678571,-2.05621588978322)(1.14118303571429,-2.05579219984179)(1.14132254464286,-2.05536672982051)(1.14146205357143,-2.05493948081897)(1.1416015625,-2.05451045394026)(1.14174107142857,-2.05407965029088)(1.14188058035714,-2.05364707098084)(1.14202008928571,-2.05321271712356)(1.14215959821429,-2.05277658983595)(1.14229910714286,-2.05233869023834)(1.14243861607143,-2.05189901945454)(1.142578125,-2.05145757861178)(1.14271763392857,-2.05101436884075)(1.14285714285714,-2.05056939127559)(1.14299665178571,-2.05012264705386)(1.14313616071429,-2.04967413731657)(1.14327566964286,-2.04922386320818)(1.14341517857143,-2.04877182587654)(1.1435546875,-2.04831802647299)(1.14369419642857,-2.04786246615225)(1.14383370535714,-2.04740514607251)(1.14397321428571,-2.04694606739534)(1.14411272321429,-2.04648523128576)(1.14425223214286,-2.04602263891222)(1.14439174107143,-2.04555829144655)(1.14453125,-2.04509219006405)(1.14467075892857,-2.04462433594338)(1.14481026785714,-2.04415473026664)(1.14494977678571,-2.04368337421934)(1.14508928571429,-2.04321026899039)(1.14522879464286,-2.04273541577209)(1.14536830357143,-2.04225881576018)(1.1455078125,-2.04178047015376)(1.14564732142857,-2.04130038015535)(1.14578683035714,-2.04081854697086)(1.14592633928571,-2.04033497180959)(1.14606584821429,-2.03984965588423)(1.14620535714286,-2.03936260041086)(1.14634486607143,-2.03887380660896)(1.146484375,-2.03838327570136)(1.14662388392857,-2.0378910089143)(1.14676339285714,-2.03739700747738)(1.14690290178571,-2.0369012726236)(1.14704241071429,-2.0364038055893)(1.14718191964286,-2.03590460761422)(1.14732142857143,-2.03540367994146)(1.1474609375,-2.03490102381749)(1.14760044642857,-2.03439664049212)(1.14773995535714,-2.03389053121856)(1.14787946428571,-2.03338269725335)(1.14801897321429,-2.0328731398564)(1.14815848214286,-2.03236186029096)(1.14829799107143,-2.03184885982366)(1.1484375,-2.03133413972445)(1.14857700892857,-2.03081770126665)(1.14871651785714,-2.0302995457269)(1.14885602678571,-2.0297796743852)(1.14899553571429,-2.02925808852488)(1.14913504464286,-2.02873478943263)(1.14927455357143,-2.02820977839843)(1.1494140625,-2.02768305671564)(1.14955357142857,-2.02715462568091)(1.14969308035714,-2.02662448659425)(1.14983258928571,-2.02609264075897)(1.14997209821429,-2.0255590894817)(1.15011160714286,-2.02502383407242)(1.15025111607143,-2.02448687584439)(1.150390625,-2.02394821611421)(1.15053013392857,-2.02340785620177)(1.15066964285714,-2.0228657974303)(1.15080915178571,-2.0223220411263)(1.15094866071429,-2.0217765886196)(1.15108816964286,-2.02122944124332)(1.15122767857143,-2.02068060033388)(1.1513671875,-2.020130067231)(1.15150669642857,-2.01957784327769)(1.15164620535714,-2.01902392982024)(1.15178571428571,-2.01846832820827)(1.15192522321429,-2.01791103979462)(1.15206473214286,-2.01735206593547)(1.15220424107143,-2.01679140799025)(1.15234375,-2.01622906732169)(1.15248325892857,-2.01566504529576)(1.15262276785714,-2.01509934328174)(1.15276227678571,-2.01453196265216)(1.15290178571429,-2.01396290478282)(1.15304129464286,-2.01339217105278)(1.15318080357143,-2.01281976284437)(1.1533203125,-2.01224568154318)(1.15345982142857,-2.01166992853805)(1.15359933035714,-2.01109250522107)(1.15373883928571,-2.01051341298759)(1.15387834821429,-2.00993265323621)(1.15401785714286,-2.00935022736876)(1.15415736607143,-2.00876613679033)(1.154296875,-2.00818038290924)(1.15443638392857,-2.00759296713705)(1.15457589285714,-2.00700389088856)(1.15471540178571,-2.00641315558179)(1.15485491071429,-2.005820762638)(1.15499441964286,-2.00522671348166)(1.15513392857143,-2.00463100954049)(1.1552734375,-2.0040336522454)(1.15541294642857,-2.00343464303055)(1.15555245535714,-2.00283398333329)(1.15569196428571,-2.00223167459419)(1.15583147321429,-2.00162771825703)(1.15597098214286,-2.0010221157688)(1.15611049107143,-2.00041486857969)(1.15625,-1.99980597814309)(1.15638950892857,-1.99919544591559)(1.15652901785714,-1.99858327335698)(1.15666852678571,-1.99796946193023)(1.15680803571429,-1.99735401310152)(1.15694754464286,-1.99673692834019)(1.15708705357143,-1.99611820911879)(1.1572265625,-1.99549785691303)(1.15736607142857,-1.99487587320182)(1.15750558035714,-1.99425225946722)(1.15764508928571,-1.99362701719448)(1.15778459821429,-1.99300014787202)(1.15792410714286,-1.99237165299142)(1.15806361607143,-1.99174153404743)(1.158203125,-1.99110979253794)(1.15834263392857,-1.99047642996403)(1.15848214285714,-1.98984144782992)(1.15862165178571,-1.98920484764297)(1.15876116071429,-1.98856663091371)(1.15890066964286,-1.98792679915581)(1.15904017857143,-1.98728535388607)(1.1591796875,-1.98664229662444)(1.15931919642857,-1.98599762889402)(1.15945870535714,-1.98535135222103)(1.15959821428571,-1.98470346813481)(1.15973772321429,-1.98405397816786)(1.15987723214286,-1.98340288385578)(1.16001674107143,-1.9827501867373)(1.16015625,-1.98209588835427)(1.16029575892857,-1.98143999025166)(1.16043526785714,-1.98078249397755)(1.16057477678571,-1.98012340108312)(1.16071428571429,-1.97946271312268)(1.16085379464286,-1.97880043165362)(1.16099330357143,-1.97813655823646)(1.1611328125,-1.97747109443479)(1.16127232142857,-1.97680404181531)(1.16141183035714,-1.97613540194781)(1.16155133928571,-1.97546517640517)(1.16169084821429,-1.97479336676335)(1.16183035714286,-1.9741199746014)(1.16196986607143,-1.97344500150145)(1.162109375,-1.97276844904871)(1.16224888392857,-1.97209031883146)(1.16238839285714,-1.97141061244103)(1.16252790178571,-1.97072933147186)(1.16266741071429,-1.97004647752142)(1.16280691964286,-1.96936205219026)(1.16294642857143,-1.96867605708197)(1.1630859375,-1.96798849380322)(1.16322544642857,-1.96729936396371)(1.16336495535714,-1.9666086691762)(1.16350446428571,-1.9659164110565)(1.16364397321429,-1.96522259122345)(1.16378348214286,-1.96452721129894)(1.16392299107143,-1.96383027290789)(1.1640625,-1.96313177767825)(1.16420200892857,-1.96243172724102)(1.16434151785714,-1.96173012323019)(1.16448102678571,-1.96102696728282)(1.16462053571429,-1.96032226103895)(1.16476004464286,-1.95961600614167)(1.16489955357143,-1.95890820423704)(1.1650390625,-1.95819885697419)(1.16517857142857,-1.95748796600521)(1.16531808035714,-1.95677553298522)(1.16545758928571,-1.95606155957232)(1.16559709821429,-1.95534604742763)(1.16573660714286,-1.95462899821525)(1.16587611607143,-1.95391041360227)(1.166015625,-1.9531902952588)(1.16615513392857,-1.95246864485788)(1.16629464285714,-1.95174546407558)(1.16643415178571,-1.95102075459093)(1.16657366071429,-1.95029451808593)(1.16671316964286,-1.94956675624556)(1.16685267857143,-1.94883747075776)(1.1669921875,-1.94810666331345)(1.16713169642857,-1.94737433560651)(1.16727120535714,-1.94664048933375)(1.16741071428571,-1.94590512619498)(1.16755022321429,-1.94516824789293)(1.16768973214286,-1.9444298561333)(1.16782924107143,-1.94368995262471)(1.16796875,-1.94294853907875)(1.16810825892857,-1.94220561720993)(1.16824776785714,-1.9414611887357)(1.16838727678571,-1.94071525537645)(1.16852678571429,-1.93996781885549)(1.16866629464286,-1.93921888089906)(1.16880580357143,-1.93846844323632)(1.1689453125,-1.93771650759935)(1.16908482142857,-1.93696307572315)(1.16922433035714,-1.93620814934561)(1.16936383928571,-1.93545173020757)(1.16950334821429,-1.93469382005274)(1.16964285714286,-1.93393442062774)(1.16978236607143,-1.93317353368209)(1.169921875,-1.93241116096822)(1.17006138392857,-1.93164730424142)(1.17020089285714,-1.93088196525991)(1.17034040178571,-1.93011514578475)(1.17047991071429,-1.92934684757991)(1.17061941964286,-1.92857707241224)(1.17075892857143,-1.92780582205145)(1.1708984375,-1.92703309827012)(1.17103794642857,-1.92625890284372)(1.17117745535714,-1.92548323755055)(1.17131696428571,-1.92470610417181)(1.17145647321429,-1.92392750449152)(1.17159598214286,-1.92314744029658)(1.17173549107143,-1.92236591337672)(1.171875,-1.92158292552454)(1.17201450892857,-1.92079847853547)(1.17215401785714,-1.92001257420778)(1.17229352678571,-1.91922521434257)(1.17243303571429,-1.91843640074379)(1.17257254464286,-1.91764613521821)(1.17271205357143,-1.91685441957541)(1.1728515625,-1.91606125562783)(1.17299107142857,-1.9152666451907)(1.17313058035714,-1.91447059008206)(1.17327008928571,-1.91367309212279)(1.17340959821429,-1.91287415313655)(1.17354910714286,-1.91207377494982)(1.17368861607143,-1.91127195939188)(1.173828125,-1.9104687082948)(1.17396763392857,-1.90966402349346)(1.17410714285714,-1.90885790682551)(1.17424665178571,-1.9080503601314)(1.17438616071429,-1.90724138525436)(1.17452566964286,-1.9064309840404)(1.17466517857143,-1.90561915833829)(1.1748046875,-1.90480590999961)(1.17494419642857,-1.90399124087868)(1.17508370535714,-1.90317515283258)(1.17522321428571,-1.90235764772118)(1.17536272321429,-1.90153872740708)(1.17550223214286,-1.90071839375565)(1.17564174107143,-1.89989664863501)(1.17578125,-1.89907349391601)(1.17592075892857,-1.89824893147227)(1.17606026785714,-1.89742296318014)(1.17619977678571,-1.89659559091869)(1.17633928571429,-1.89576681656974)(1.17647879464286,-1.89493664201784)(1.17661830357143,-1.89410506915025)(1.1767578125,-1.89327209985697)(1.17689732142857,-1.89243773603071)(1.17703683035714,-1.89160197956689)(1.17717633928571,-1.89076483236363)(1.17731584821429,-1.88992629632179)(1.17745535714286,-1.88908637334491)(1.17759486607143,-1.88824506533922)(1.177734375,-1.88740237421367)(1.17787388392857,-1.88655830187989)(1.17801339285714,-1.88571285025219)(1.17815290178571,-1.88486602124758)(1.17829241071429,-1.88401781678574)(1.17843191964286,-1.88316823878904)(1.17857142857143,-1.8823172891825)(1.1787109375,-1.88146496989383)(1.17885044642857,-1.8806112828534)(1.17898995535714,-1.87975622999424)(1.17912946428571,-1.87889981325204)(1.17926897321429,-1.87804203456515)(1.17940848214286,-1.87718289587455)(1.17954799107143,-1.87632239912389)(1.1796875,-1.87546054625946)(1.17982700892857,-1.87459733923019)(1.17996651785714,-1.87373277998762)(1.18010602678571,-1.87286687048595)(1.18024553571429,-1.871999612682)(1.18038504464286,-1.87113100853521)(1.18052455357143,-1.87026106000766)(1.1806640625,-1.86938976906401)(1.18080357142857,-1.86851713767157)(1.18094308035714,-1.86764316780022)(1.18108258928571,-1.8667678614225)(1.18122209821429,-1.86589122051349)(1.18136160714286,-1.86501324705091)(1.18150111607143,-1.86413394301505)(1.181640625,-1.86325331038881)(1.18178013392857,-1.86237135115765)(1.18191964285714,-1.86148806730964)(1.18205915178571,-1.86060346083542)(1.18219866071429,-1.85971753372818)(1.18233816964286,-1.85883028798371)(1.18247767857143,-1.85794172560036)(1.1826171875,-1.85705184857902)(1.18275669642857,-1.85616065892318)(1.18289620535714,-1.85526815863885)(1.18303571428571,-1.85437434973461)(1.18317522321429,-1.85347923422156)(1.18331473214286,-1.85258281411339)(1.18345424107143,-1.85168509142627)(1.18359375,-1.85078606817897)(1.18373325892857,-1.84988574639273)(1.18387276785714,-1.84898412809136)(1.18401227678571,-1.84808121530118)(1.18415178571429,-1.84717701005102)(1.18429129464286,-1.84627151437225)(1.18443080357143,-1.84536473029871)(1.1845703125,-1.84445665986678)(1.18470982142857,-1.84354730511536)(1.18484933035714,-1.84263666808579)(1.18498883928571,-1.84172475082197)(1.18512834821429,-1.84081155537024)(1.18526785714286,-1.83989708377948)(1.18540736607143,-1.83898133810099)(1.185546875,-1.83806432038861)(1.18568638392857,-1.83714603269863)(1.18582589285714,-1.8362264770898)(1.18596540178571,-1.83530565562335)(1.18610491071429,-1.83438357036298)(1.18624441964286,-1.83346022337484)(1.18638392857143,-1.83253561672753)(1.1865234375,-1.83160975249213)(1.18666294642857,-1.83068263274213)(1.18680245535714,-1.82975425955349)(1.18694196428571,-1.8288246350046)(1.18708147321429,-1.82789376117629)(1.18722098214286,-1.82696164015181)(1.18736049107143,-1.82602827401684)(1.1875,-1.82509366485951)(1.18763950892857,-1.82415781477033)(1.18777901785714,-1.82322072584225)(1.18791852678571,-1.82228240017062)(1.18805803571429,-1.82134283985321)(1.18819754464286,-1.82040204699017)(1.18833705357143,-1.81946002368408)(1.1884765625,-1.81851677203988)(1.18861607142857,-1.81757229416493)(1.18875558035714,-1.81662659216897)(1.18889508928571,-1.8156796681641)(1.18903459821429,-1.81473152426483)(1.18917410714286,-1.81378216258803)(1.18931361607143,-1.81283158525294)(1.189453125,-1.81187979438115)(1.18959263392857,-1.81092679209665)(1.18973214285714,-1.80997258052576)(1.18987165178571,-1.80901716179715)(1.19001116071429,-1.80806053804186)(1.19015066964286,-1.80710271139326)(1.19029017857143,-1.80614368398706)(1.1904296875,-1.80518345796132)(1.19056919642857,-1.80422203545643)(1.19070870535714,-1.80325941861509)(1.19084821428571,-1.80229560958236)(1.19098772321429,-1.80133061050557)(1.19112723214286,-1.80036442353441)(1.19126674107143,-1.79939705082087)(1.19140625,-1.79842849451924)(1.19154575892857,-1.79745875678613)(1.19168526785714,-1.79648783978042)(1.19182477678571,-1.79551574566331)(1.19196428571429,-1.79454247659828)(1.19210379464286,-1.79356803475111)(1.19224330357143,-1.79259242228985)(1.1923828125,-1.79161564138482)(1.19252232142857,-1.79063769420865)(1.19266183035714,-1.78965858293619)(1.19280133928571,-1.7886783097446)(1.19294084821429,-1.78769687681327)(1.19308035714286,-1.78671428632386)(1.19321986607143,-1.78573054046029)(1.193359375,-1.78474564140872)(1.19349888392857,-1.78375959135756)(1.19363839285714,-1.78277239249744)(1.19377790178571,-1.78178404702126)(1.19391741071429,-1.78079455712412)(1.19405691964286,-1.77980392500337)(1.19419642857143,-1.77881215285857)(1.1943359375,-1.7778192428915)(1.19447544642857,-1.77682519730617)(1.19461495535714,-1.77583001830877)(1.19475446428571,-1.77483370810772)(1.19489397321429,-1.77383626891364)(1.19503348214286,-1.77283770293933)(1.19517299107143,-1.77183801239981)(1.1953125,-1.77083719951227)(1.19545200892857,-1.76983526649609)(1.19559151785714,-1.76883221557281)(1.19573102678571,-1.7678280489662)(1.19587053571429,-1.76682276890213)(1.19601004464286,-1.76581637760869)(1.19614955357143,-1.76480887731611)(1.1962890625,-1.76380027025679)(1.19642857142857,-1.76279055866528)(1.19656808035714,-1.76177974477827)(1.19670758928571,-1.76076783083461)(1.19684709821429,-1.75975481907528)(1.19698660714286,-1.7587407117434)(1.19712611607143,-1.75772551108422)(1.197265625,-1.75670921934513)(1.19740513392857,-1.75569183877563)(1.19754464285714,-1.75467337162735)(1.19768415178571,-1.75365382015401)(1.19782366071429,-1.75263318661148)(1.19796316964286,-1.7516114732577)(1.19810267857143,-1.75058868235273)(1.1982421875,-1.74956481615874)(1.19838169642857,-1.74853987693995)(1.19852120535714,-1.74751386696271)(1.19866071428571,-1.74648678849544)(1.19880022321429,-1.74545864380864)(1.19893973214286,-1.74442943517487)(1.19907924107143,-1.74339916486878)(1.19921875,-1.74236783516709)(1.19935825892857,-1.74133544834857)(1.19949776785714,-1.74030200669404)(1.19963727678571,-1.73926751248639)(1.19977678571429,-1.73823196801055)(1.19991629464286,-1.73719537555351)(1.20005580357143,-1.73615773740425)(1.2001953125,-1.73511905585386)(1.20033482142857,-1.73407933319539)(1.20047433035714,-1.73303857172397)(1.20061383928571,-1.73199677373672)(1.20075334821429,-1.73095394153279)(1.20089285714286,-1.72991007741333)(1.20103236607143,-1.72886518368151)(1.201171875,-1.72781926264252)(1.20131138392857,-1.72677231660351)(1.20145089285714,-1.72572434787366)(1.20159040178571,-1.72467535876411)(1.20172991071429,-1.72362535158802)(1.20186941964286,-1.72257432866051)(1.20200892857143,-1.72152229229867)(1.2021484375,-1.72046924482159)(1.20228794642857,-1.7194151885503)(1.20242745535714,-1.71836012580781)(1.20256696428571,-1.71730405891908)(1.20270647321429,-1.71624699021103)(1.20284598214286,-1.71518892201252)(1.20298549107143,-1.71412985665437)(1.203125,-1.71306979646934)(1.20326450892857,-1.71200874379212)(1.20340401785714,-1.71094670095931)(1.20354352678571,-1.70988367030948)(1.20368303571429,-1.7088196541831)(1.20382254464286,-1.70775465492255)(1.20396205357143,-1.70668867487213)(1.2041015625,-1.70562171637807)(1.20424107142857,-1.70455378178847)(1.20438058035714,-1.70348487345335)(1.20452008928571,-1.70241499372462)(1.20465959821429,-1.70134414495609)(1.20479910714286,-1.70027232950343)(1.20493861607143,-1.69919954972422)(1.205078125,-1.69812580797791)(1.20521763392857,-1.69705110662581)(1.20535714285714,-1.69597544803111)(1.20549665178571,-1.69489883455886)(1.20563616071429,-1.69382126857596)(1.20577566964286,-1.69274275245118)(1.20591517857143,-1.69166328855512)(1.2060546875,-1.69058287926024)(1.20619419642857,-1.68950152694084)(1.20633370535714,-1.68841923397304)(1.20647321428571,-1.68733600273481)(1.20661272321429,-1.68625183560593)(1.20675223214286,-1.68516673496802)(1.20689174107143,-1.68408070320449)(1.20703125,-1.68299374270059)(1.20717075892857,-1.68190585584336)(1.20731026785714,-1.68081704502165)(1.20744977678571,-1.6797273126261)(1.20758928571429,-1.67863666104916)(1.20772879464286,-1.67754509268506)(1.20786830357143,-1.6764526099298)(1.2080078125,-1.67535921518119)(1.20814732142857,-1.67426491083879)(1.20828683035714,-1.67316969930393)(1.20842633928571,-1.67207358297973)(1.20856584821429,-1.67097656427104)(1.20870535714286,-1.66987864558449)(1.20884486607143,-1.66877982932846)(1.208984375,-1.66768011791305)(1.20912388392857,-1.66657951375014)(1.20926339285714,-1.66547801925332)(1.20940290178571,-1.66437563683793)(1.20954241071429,-1.66327236892103)(1.20968191964286,-1.66216821792139)(1.20982142857143,-1.66106318625953)(1.2099609375,-1.65995727635766)(1.21010044642857,-1.6588504906397)(1.21023995535714,-1.65774283153129)(1.21037946428571,-1.65663430145976)(1.21051897321429,-1.65552490285413)(1.21065848214286,-1.65441463814512)(1.21079799107143,-1.65330350976513)(1.2109375,-1.65219152014824)(1.21107700892857,-1.65107867173021)(1.21121651785714,-1.64996496694848)(1.21135602678571,-1.64885040824213)(1.21149553571429,-1.64773499805193)(1.21163504464286,-1.6466187388203)(1.21177455357143,-1.6455016329913)(1.2119140625,-1.64438368301064)(1.21205357142857,-1.6432648913257)(1.21219308035714,-1.64214526038546)(1.21233258928571,-1.64102479264057)(1.21247209821429,-1.63990349054326)(1.21261160714286,-1.63878135654744)(1.21275111607143,-1.6376583931086)(1.212890625,-1.63653460268385)(1.21303013392857,-1.63540998773194)(1.21316964285714,-1.63428455071318)(1.21330915178571,-1.63315829408951)(1.21344866071429,-1.63203122032445)(1.21358816964286,-1.63090333188313)(1.21372767857143,-1.62977463123224)(1.2138671875,-1.62864512084006)(1.21400669642857,-1.62751480317647)(1.21414620535714,-1.62638368071288)(1.21428571428571,-1.62525175592229)(1.21442522321429,-1.62411903127926)(1.21456473214286,-1.62298550925991)(1.21470424107143,-1.62185119234189)(1.21484375,-1.62071608300444)(1.21498325892857,-1.6195801837283)(1.21512276785714,-1.61844349699576)(1.21526227678571,-1.61730602529066)(1.21540178571429,-1.61616777109834)(1.21554129464286,-1.61502873690568)(1.21568080357143,-1.61388892520108)(1.2158203125,-1.61274833847445)(1.21595982142857,-1.6116069792172)(1.21609933035714,-1.61046484992226)(1.21623883928571,-1.60932195308404)(1.21637834821429,-1.60817829119844)(1.21651785714286,-1.60703386676289)(1.21665736607143,-1.60588868227626)(1.216796875,-1.60474274023892)(1.21693638392857,-1.6035960431527)(1.21707589285714,-1.60244859352092)(1.21721540178571,-1.60130039384835)(1.21735491071429,-1.60015144664122)(1.21749441964286,-1.59900175440723)(1.21763392857143,-1.5978513196555)(1.2177734375,-1.59670014489663)(1.21791294642857,-1.59554823264264)(1.21805245535714,-1.59439558540698)(1.21819196428571,-1.59324220570455)(1.21833147321429,-1.59208809605165)(1.21847098214286,-1.59093325896604)(1.21861049107143,-1.58977769696686)(1.21875,-1.58862141257467)(1.21888950892857,-1.58746440831144)(1.21902901785714,-1.58630668670054)(1.21916852678571,-1.58514825026673)(1.21930803571429,-1.58398910153618)(1.21944754464286,-1.58282924303643)(1.21958705357143,-1.58166867729639)(1.2197265625,-1.58050740684638)(1.21986607142857,-1.57934543421807)(1.22000558035714,-1.57818276194448)(1.22014508928571,-1.57701939256004)(1.22028459821429,-1.57585532860049)(1.22042410714286,-1.57469057260294)(1.22056361607143,-1.57352512710585)(1.220703125,-1.57235899464902)(1.22084263392857,-1.57119217777359)(1.22098214285714,-1.57002467902202)(1.22112165178571,-1.5688565009381)(1.22126116071429,-1.56768764606696)(1.22140066964286,-1.56651811695503)(1.22154017857143,-1.56534791615005)(1.2216796875,-1.56417704620108)(1.22181919642857,-1.56300550965847)(1.22195870535714,-1.56183330907389)(1.22209821428571,-1.56066044700026)(1.22223772321429,-1.55948692599183)(1.22237723214286,-1.55831274860412)(1.22251674107143,-1.55713791739391)(1.22265625,-1.55596243491927)(1.22279575892857,-1.55478630373954)(1.22293526785714,-1.55360952641531)(1.22307477678571,-1.55243210550844)(1.22321428571429,-1.55125404358202)(1.22335379464286,-1.55007534320042)(1.22349330357143,-1.54889600692923)(1.2236328125,-1.54771603733528)(1.22377232142857,-1.54653543698664)(1.22391183035714,-1.54535420845261)(1.22405133928571,-1.5441723543037)(1.22419084821429,-1.54298987711165)(1.22433035714286,-1.5418067794494)(1.22446986607143,-1.5406230638911)(1.224609375,-1.53943873301212)(1.22474888392857,-1.53825378938901)(1.22488839285714,-1.53706823559952)(1.22502790178571,-1.53588207422258)(1.22516741071429,-1.5346953078383)(1.22530691964286,-1.53350793902799)(1.22544642857143,-1.53231997037411)(1.2255859375,-1.53113140446029)(1.22572544642857,-1.52994224387135)(1.22586495535714,-1.52875249119321)(1.22600446428571,-1.52756214901301)(1.22614397321429,-1.52637121991897)(1.22628348214286,-1.52517970650052)(1.22642299107143,-1.52398761134818)(1.2265625,-1.52279493705361)(1.22670200892857,-1.52160168620961)(1.22684151785714,-1.5204078614101)(1.22698102678571,-1.51921346525011)(1.22712053571429,-1.51801850032577)(1.22726004464286,-1.51682296923435)(1.22739955357143,-1.51562687457419)(1.2275390625,-1.51443021894475)(1.22767857142857,-1.51323300494658)(1.22781808035714,-1.51203523518129)(1.22795758928571,-1.51083691225159)(1.22809709821429,-1.50963803876128)(1.22823660714286,-1.50843861731522)(1.22837611607143,-1.50723865051933)(1.228515625,-1.50603814098059)(1.22865513392857,-1.50483709130705)(1.22879464285714,-1.5036355041078)(1.22893415178571,-1.50243338199298)(1.22907366071429,-1.50123072757378)(1.22921316964286,-1.50002754346241)(1.22935267857143,-1.49882383227212)(1.2294921875,-1.49761959661718)(1.22963169642857,-1.4964148391129)(1.22977120535714,-1.49520956237557)(1.22991071428571,-1.49400376902252)(1.23005022321429,-1.49279746167208)(1.23018973214286,-1.49159064294357)(1.23032924107143,-1.49038331545732)(1.23046875,-1.48917548183464)(1.23060825892857,-1.48796714469782)(1.23074776785714,-1.48675830667015)(1.23088727678571,-1.48554897037587)(1.23102678571429,-1.48433913844021)(1.23116629464286,-1.48312881348936)(1.23130580357143,-1.48191799815045)(1.2314453125,-1.4807066950516)(1.23158482142857,-1.47949490682185)(1.23172433035714,-1.4782826360912)(1.23186383928571,-1.47706988549058)(1.23200334821429,-1.47585665765185)(1.23214285714286,-1.47464295520782)(1.23228236607143,-1.47342878079219)(1.232421875,-1.47221413703962)(1.23256138392857,-1.47099902658566)(1.23270089285714,-1.46978345206676)(1.23284040178571,-1.46856741612028)(1.23297991071429,-1.46735092138449)(1.23311941964286,-1.46613397049855)(1.23325892857143,-1.46491656610248)(1.2333984375,-1.46369871083723)(1.23353794642857,-1.46248040734459)(1.23367745535714,-1.46126165826724)(1.23381696428571,-1.46004246624872)(1.23395647321429,-1.45882283393342)(1.23409598214286,-1.45760276396663)(1.23423549107143,-1.45638225899443)(1.234375,-1.45516132166381)(1.23451450892857,-1.45393995462255)(1.23465401785714,-1.4527181605193)(1.23479352678571,-1.45149594200352)(1.23493303571429,-1.4502733017255)(1.23507254464286,-1.44905024233637)(1.23521205357143,-1.44782676648805)(1.2353515625,-1.44660287683328)(1.23549107142857,-1.44537857602562)(1.23563058035714,-1.44415386671941)(1.23577008928571,-1.44292875156979)(1.23590959821429,-1.44170323323269)(1.23604910714286,-1.44047731436483)(1.23618861607143,-1.43925099762371)(1.236328125,-1.43802428566759)(1.23646763392857,-1.43679718115552)(1.23660714285714,-1.43556968674731)(1.23674665178571,-1.43434180510351)(1.23688616071429,-1.43311353888543)(1.23702566964286,-1.43188489075515)(1.23716517857143,-1.43065586337548)(1.2373046875,-1.42942645940995)(1.23744419642857,-1.42819668152286)(1.23758370535714,-1.4269665323792)(1.23772321428571,-1.4257360146447)(1.23786272321429,-1.42450513098582)(1.23800223214286,-1.42327388406971)(1.23814174107143,-1.42204227656424)(1.23828125,-1.42081031113797)(1.23842075892857,-1.41957799046018)(1.23856026785714,-1.41834531720081)(1.23869977678571,-1.41711229403051)(1.23883928571429,-1.4158789236206)(1.23897879464286,-1.41464520864309)(1.23911830357143,-1.41341115177063)(1.2392578125,-1.41217675567657)(1.23939732142857,-1.41094202303491)(1.23953683035714,-1.40970695652028)(1.23967633928571,-1.408471558808)(1.23981584821429,-1.407235832574)(1.23995535714286,-1.40599978049488)(1.24009486607143,-1.40476340524784)(1.240234375,-1.40352670951074)(1.24037388392857,-1.40228969596205)(1.24051339285714,-1.40105236728085)(1.24065290178571,-1.39981472614686)(1.24079241071429,-1.39857677524037)(1.24093191964286,-1.39733851724232)(1.24107142857143,-1.3960999548342)(1.2412109375,-1.39486109069812)(1.24135044642857,-1.39362192751677)(1.24148995535714,-1.39238246797343)(1.24162946428571,-1.39114271475195)(1.24176897321429,-1.38990267053674)(1.24190848214286,-1.3886623380128)(1.24204799107143,-1.38742171986567)(1.2421875,-1.38618081878146)(1.24232700892857,-1.38493963744684)(1.24246651785714,-1.38369817854899)(1.24260602678571,-1.38245644477566)(1.24274553571429,-1.38121443881512)(1.24288504464286,-1.37997216335619)(1.24302455357143,-1.37872962108818)(1.2431640625,-1.37748681470096)(1.24330357142857,-1.37624374688489)(1.24344308035714,-1.37500042033084)(1.24358258928571,-1.37375683773018)(1.24372209821429,-1.37251300177479)(1.24386160714286,-1.37126891515705)(1.24400111607143,-1.37002458056979)(1.244140625,-1.36878000070637)(1.24428013392857,-1.36753517826061)(1.24441964285714,-1.36629011592677)(1.24455915178571,-1.36504481639963)(1.24469866071429,-1.36379928237439)(1.24483816964286,-1.36255351654674)(1.24497767857143,-1.36130752161279)(1.2451171875,-1.36006130026911)(1.24525669642857,-1.35881485521272)(1.24539620535714,-1.35756818914106)(1.24553571428571,-1.35632130475201)(1.24567522321429,-1.35507420474386)(1.24581473214286,-1.35382689181535)(1.24595424107143,-1.35257936866559)(1.24609375,-1.35133163799415)(1.24623325892857,-1.35008370250096)(1.24637276785714,-1.34883556488637)(1.24651227678571,-1.34758722785113)(1.24665178571429,-1.34633869409636)(1.24679129464286,-1.34508996632357)(1.24693080357143,-1.34384104723464)(1.2470703125,-1.34259193953186)(1.24720982142857,-1.34134264591783)(1.24734933035714,-1.34009316909555)(1.24748883928571,-1.33884351176837)(1.24762834821429,-1.33759367663998)(1.24776785714286,-1.33634366641443)(1.24790736607143,-1.33509348379611)(1.248046875,-1.33384313148972)(1.24818638392857,-1.33259261220034)(1.24832589285714,-1.33134192863332)(1.24846540178571,-1.33009108349437)(1.24860491071429,-1.32884007948948)(1.24874441964286,-1.327588919325)(1.24888392857143,-1.32633760570753)(1.2490234375,-1.32508614134399)(1.24916294642857,-1.32383452894161)(1.24930245535714,-1.32258277120789)(1.24944196428571,-1.32133087085061)(1.24958147321429,-1.32007883057782)(1.24972098214286,-1.31882665309789)(1.24986049107143,-1.31757434111939)(1.25,-1.3163218973512)(1.25013950892857,-1.31506932450244)(1.25027901785714,-1.31381662528249)(1.25041852678571,-1.31256380240095)(1.25055803571429,-1.31131085856769)(1.25069754464286,-1.3100577964928)(1.25083705357143,-1.30880461888661)(1.2509765625,-1.30755132845966)(1.25111607142857,-1.30629792792274)(1.25125558035714,-1.30504441998681)(1.25139508928571,-1.30379080736307)(1.25153459821429,-1.30253709276291)(1.25167410714286,-1.30128327889794)(1.25181361607143,-1.30002936847994)(1.251953125,-1.29877536422088)(1.25209263392857,-1.29752126883292)(1.25223214285714,-1.29626708502841)(1.25237165178571,-1.29501281551984)(1.25251116071429,-1.29375846301988)(1.25265066964286,-1.29250403024139)(1.25279017857143,-1.29124951989734)(1.2529296875,-1.28999493470088)(1.25306919642857,-1.28874027736531)(1.25320870535714,-1.28748555060404)(1.25334821428571,-1.28623075713066)(1.25348772321429,-1.28497589965883)(1.25362723214286,-1.2837209809024)(1.25376674107143,-1.28246600357528)(1.25390625,-1.28121097039155)(1.25404575892857,-1.27995588406536)(1.25418526785714,-1.27870074731097)(1.25432477678571,-1.27744556284274)(1.25446428571429,-1.27619033337513)(1.25460379464286,-1.27493506162269)(1.25474330357143,-1.27367975030003)(1.2548828125,-1.27242440212186)(1.25502232142857,-1.27116901980296)(1.25516183035714,-1.26991360605816)(1.25530133928571,-1.26865816360236)(1.25544084821429,-1.26740269515053)(1.25558035714286,-1.26614720341767)(1.25571986607143,-1.26489169111883)(1.255859375,-1.26363616096911)(1.25599888392857,-1.26238061568364)(1.25613839285714,-1.26112505797757)(1.25627790178571,-1.25986949056608)(1.25641741071429,-1.25861391616438)(1.25655691964286,-1.25735833748767)(1.25669642857143,-1.25610275725119)(1.2568359375,-1.25484717817015)(1.25697544642857,-1.25359160295979)(1.25711495535714,-1.25233603433531)(1.25725446428571,-1.25108047501192)(1.25739397321429,-1.2498249277048)(1.25753348214286,-1.24856939512912)(1.25767299107143,-1.24731388000001)(1.2578125,-1.24605838503256)(1.25795200892857,-1.24480291294184)(1.25809151785714,-1.24354746644286)(1.25823102678571,-1.24229204825059)(1.25837053571429,-1.24103666107992)(1.25851004464286,-1.23978130764573)(1.25864955357143,-1.23852599066279)(1.2587890625,-1.23727071284581)(1.25892857142857,-1.23601547690943)(1.25906808035714,-1.23476028556821)(1.25920758928571,-1.23350514153661)(1.25934709821429,-1.23225004752901)(1.25948660714286,-1.2309950062597)(1.25962611607143,-1.22974002044283)(1.259765625,-1.22848509279249)(1.25990513392857,-1.22723022602264)(1.26004464285714,-1.2259754228471)(1.26018415178571,-1.22472068597959)(1.26032366071429,-1.22346601813369)(1.26046316964286,-1.22221142202286)(1.26060267857143,-1.2209569003604)(1.2607421875,-1.21970245585948)(1.26088169642857,-1.21844809123312)(1.26102120535714,-1.21719380919417)(1.26116071428571,-1.21593961245534)(1.26130022321429,-1.21468550372915)(1.26143973214286,-1.21343148572796)(1.26157924107143,-1.21217756116397)(1.26171875,-1.21092373274917)(1.26185825892857,-1.20967000319539)(1.26199776785714,-1.20841637521424)(1.26213727678571,-1.20716285151715)(1.26227678571429,-1.20590943481534)(1.26241629464286,-1.20465612781984)(1.26255580357143,-1.20340293324143)(1.2626953125,-1.2021498537907)(1.26283482142857,-1.20089689217802)(1.26297433035714,-1.1996440511135)(1.26311383928571,-1.19839133330704)(1.26325334821429,-1.1971387414683)(1.26339285714286,-1.1958862783067)(1.26353236607143,-1.19463394653137)(1.263671875,-1.19338174885123)(1.26381138392857,-1.19212968797493)(1.26395089285714,-1.19087776661083)(1.26409040178571,-1.18962598746705)(1.26422991071429,-1.1883743532514)(1.26436941964286,-1.18712286667143)(1.26450892857143,-1.18587153043441)(1.2646484375,-1.18462034724729)(1.26478794642857,-1.18336931981676)(1.26492745535714,-1.18211845084916)(1.26506696428571,-1.18086774305055)(1.26520647321429,-1.17961719912669)(1.26534598214286,-1.17836682178298)(1.26548549107143,-1.17711661372454)(1.265625,-1.17586657765613)(1.26576450892857,-1.17461671628218)(1.26590401785714,-1.17336703230679)(1.26604352678571,-1.17211752843372)(1.26618303571429,-1.17086820736635)(1.26632254464286,-1.16961907180774)(1.26646205357143,-1.16837012446056)(1.2666015625,-1.16712136802714)(1.26674107142857,-1.16587280520941)(1.26688058035714,-1.16462443870894)(1.26702008928571,-1.16337627122693)(1.26715959821429,-1.16212830546416)(1.26729910714286,-1.16088054412104)(1.26743861607143,-1.15963298989759)(1.267578125,-1.1583856454934)(1.26771763392857,-1.15713851360767)(1.26785714285714,-1.15589159693919)(1.26799665178571,-1.15464489818632)(1.26813616071429,-1.15339842004699)(1.26827566964286,-1.15215216521872)(1.26841517857143,-1.15090613639859)(1.2685546875,-1.14966033628322)(1.26869419642857,-1.14841476756882)(1.26883370535714,-1.14716943295112)(1.26897321428571,-1.14592433512541)(1.26911272321429,-1.1446794767865)(1.26925223214286,-1.14343486062876)(1.26939174107143,-1.14219048934607)(1.26953125,-1.14094636563185)(1.26967075892857,-1.13970249217902)(1.26981026785714,-1.13845887168002)(1.26994977678571,-1.13721550682681)(1.27008928571429,-1.13597240031082)(1.27022879464286,-1.13472955482302)(1.27036830357143,-1.13348697305383)(1.2705078125,-1.13224465769319)(1.27064732142857,-1.13100261143051)(1.27078683035714,-1.12976083695467)(1.27092633928571,-1.12851933695403)(1.27106584821429,-1.12727811411639)(1.27120535714286,-1.12603717112907)(1.27134486607143,-1.12479651067877)(1.271484375,-1.12355613545171)(1.27162388392857,-1.1223160481335)(1.27176339285714,-1.12107625140922)(1.27190290178571,-1.11983674796337)(1.27204241071429,-1.11859754047989)(1.27218191964286,-1.11735863164215)(1.27232142857143,-1.11612002413291)(1.2724609375,-1.11488172063438)(1.27260044642857,-1.11364372382815)(1.27273995535714,-1.11240603639522)(1.27287946428571,-1.111168661016)(1.27301897321429,-1.10993160037027)(1.27315848214286,-1.10869485713723)(1.27329799107143,-1.10745843399543)(1.2734375,-1.10622233362281)(1.27357700892857,-1.1049865586967)(1.27371651785714,-1.10375111189376)(1.27385602678571,-1.10251599589003)(1.27399553571429,-1.10128121336092)(1.27413504464286,-1.10004676698117)(1.27427455357143,-1.09881265942487)(1.2744140625,-1.09757889336546)(1.27455357142857,-1.09634547147571)(1.27469308035714,-1.09511239642771)(1.27483258928571,-1.09387967089289)(1.27497209821429,-1.092647297542)(1.27511160714286,-1.09141527904509)(1.27525111607143,-1.09018361807153)(1.275390625,-1.08895231728999)(1.27553013392857,-1.08772137936844)(1.27566964285714,-1.08649080697416)(1.27580915178571,-1.08526060277368)(1.27594866071429,-1.08403076943286)(1.27608816964286,-1.08280130961679)(1.27622767857143,-1.08157222598988)(1.2763671875,-1.08034352121578)(1.27650669642857,-1.0791151979574)(1.27664620535714,-1.07788725887692)(1.27678571428571,-1.07665970663578)(1.27692522321429,-1.07543254389463)(1.27706473214286,-1.07420577331342)(1.27720424107143,-1.07297939755127)(1.27734375,-1.07175341926659)(1.27748325892857,-1.07052784111699)(1.27762276785714,-1.06930266575929)(1.27776227678571,-1.06807789584955)(1.27790178571429,-1.06685353404302)(1.27804129464286,-1.06562958299418)(1.27818080357143,-1.06440604535668)(1.2783203125,-1.0631829237834)(1.27845982142857,-1.06196022092638)(1.27859933035714,-1.06073793943686)(1.27873883928571,-1.05951608196526)(1.27887834821429,-1.05829465116116)(1.27901785714286,-1.05707364967334)(1.27915736607143,-1.05585308014971)(1.279296875,-1.05463294523737)(1.27943638392857,-1.05341324758254)(1.27957589285714,-1.05219398983062)(1.27971540178571,-1.05097517462615)(1.27985491071429,-1.04975680461278)(1.27999441964286,-1.04853888243333)(1.28013392857143,-1.04732141072972)(1.2802734375,-1.04610439214302)(1.28041294642857,-1.0448878293134)(1.28055245535714,-1.04367172488014)(1.28069196428571,-1.04245608148164)(1.28083147321429,-1.0412409017554)(1.28097098214286,-1.04002618833801)(1.28111049107143,-1.03881194386516)(1.28125,-1.03759817097162)(1.28138950892857,-1.03638487229126)(1.28152901785714,-1.035172050457)(1.28166852678571,-1.03395970810085)(1.28180803571429,-1.03274784785388)(1.28194754464286,-1.03153647234623)(1.28208705357143,-1.03032558420709)(1.2822265625,-1.0291151860647)(1.28236607142857,-1.02790528054636)(1.28250558035714,-1.02669587027838)(1.28264508928571,-1.02548695788614)(1.28278459821429,-1.02427854599403)(1.28292410714286,-1.02307063722549)(1.28306361607143,-1.02186323420293)(1.283203125,-1.02065633954784)(1.28334263392857,-1.01944995588068)(1.28348214285714,-1.01824408582093)(1.28362165178571,-1.01703873198705)(1.28376116071429,-1.01583389699653)(1.28390066964286,-1.01462958346582)(1.28404017857143,-1.01342579401037)(1.2841796875,-1.01222253124461)(1.28431919642857,-1.01101979778193)(1.28445870535714,-1.00981759623472)(1.28459821428571,-1.0086159292143)(1.28473772321429,-1.00741479933097)(1.28487723214286,-1.00621420919398)(1.28501674107143,-1.00501416141153)(1.28515625,-1.00381465859076)(1.28529575892857,-1.00261570333777)(1.28543526785714,-1.00141729825756)(1.28557477678571,-1.00021944595408)(1.28571428571429,-0.999022149030203)(1.28585379464286,-0.997825410087725)(1.28599330357143,-0.996629231727341)(1.2861328125,-0.995433616548671)(1.28627232142857,-0.994238567150231)(1.28641183035714,-0.993044086129434)(1.28655133928571,-0.991850176082593)(1.28669084821429,-0.9906568396049)(1.28683035714286,-0.989464079290447)(1.28696986607143,-0.988271897732185)(1.287109375,-0.987080297521954)(1.28724888392857,-0.985889281250458)(1.28738839285714,-0.984698851507259)(1.28752790178571,-0.983509010880786)(1.28766741071429,-0.98231976195831)(1.28780691964286,-0.981131107325968)(1.28794642857143,-0.979943049568719)(1.2880859375,-0.978755591270379)(1.28822544642857,-0.97756873501359)(1.28836495535714,-0.976382483379815)(1.28850446428571,-0.975196838949355)(1.28864397321429,-0.974011804301313)(1.28878348214286,-0.972827382013622)(1.28892299107143,-0.971643574663007)(1.2890625,-0.970460384825008)(1.28920200892857,-0.96927781507396)(1.28934151785714,-0.968095867982987)(1.28948102678571,-0.966914546124008)(1.28962053571429,-0.965733852067714)(1.28976004464286,-0.964553788383592)(1.28989955357143,-0.96337435763988)(1.2900390625,-0.962195562403604)(1.29017857142857,-0.961017405240546)(1.29031808035714,-0.959839888715238)(1.29045758928571,-0.958663015390978)(1.29059709821429,-0.9574867878298)(1.29073660714286,-0.956311208592496)(1.29087611607143,-0.955136280238579)(1.291015625,-0.953962005326309)(1.29115513392857,-0.952788386412671)(1.29129464285714,-0.951615426053366)(1.29143415178571,-0.950443126802823)(1.29157366071429,-0.949271491214173)(1.29171316964286,-0.948100521839272)(1.29185267857143,-0.94693022122866)(1.2919921875,-0.945760591931591)(1.29213169642857,-0.944591636496009)(1.29227120535714,-0.943423357468536)(1.29241071428571,-0.942255757394492)(1.29255022321429,-0.941088838817862)(1.29268973214286,-0.939922604281321)(1.29282924107143,-0.938757056326193)(1.29296875,-0.937592197492483)(1.29310825892857,-0.93642803031885)(1.29324776785714,-0.935264557342597)(1.29338727678571,-0.93410178109969)(1.29352678571429,-0.932939704124726)(1.29366629464286,-0.931778328950955)(1.29380580357143,-0.930617658110246)(1.2939453125,-0.92945769413311)(1.29408482142857,-0.928298439548679)(1.29422433035714,-0.927139896884697)(1.29436383928571,-0.925982068667532)(1.29450334821429,-0.924824957422151)(1.29464285714286,-0.923668565672141)(1.29478236607143,-0.92251289593967)(1.294921875,-0.921357950745517)(1.29506138392857,-0.920203732609044)(1.29520089285714,-0.919050244048195)(1.29534040178571,-0.9178974875795)(1.29547991071429,-0.916745465718056)(1.29561941964286,-0.915594180977544)(1.29575892857143,-0.914443635870192)(1.2958984375,-0.913293832906805)(1.29603794642857,-0.912144774596739)(1.29617745535714,-0.910996463447892)(1.29631696428571,-0.90984890196672)(1.29645647321429,-0.908702092658208)(1.29659598214286,-0.907556038025892)(1.29673549107143,-0.906410740571821)(1.296875,-0.905266202796587)(1.29701450892857,-0.904122427199298)(1.29715401785714,-0.902979416277571)(1.29729352678571,-0.901837172527546)(1.29743303571429,-0.900695698443858)(1.29757254464286,-0.899554996519658)(1.29771205357143,-0.89841506924658)(1.2978515625,-0.897275919114763)(1.29799107142857,-0.896137548612828)(1.29813058035714,-0.894999960227872)(1.29827008928571,-0.893863156445483)(1.29840959821429,-0.892727139749708)(1.29854910714286,-0.891591912623076)(1.29868861607143,-0.890457477546566)(1.298828125,-0.889323836999626)(1.29896763392857,-0.888190993460155)(1.29910714285714,-0.887058949404494)(1.29924665178571,-0.885927707307439)(1.29938616071429,-0.884797269642213)(1.29952566964286,-0.883667638880486)(1.29966517857143,-0.882538817492345)(1.2998046875,-0.881410807946313)(1.29994419642857,-0.88028361270933)(1.30008370535714,-0.879157234246743)(1.30022321428571,-0.87803167502232)(1.30036272321429,-0.876906937498225)(1.30050223214286,-0.875783024135034)(1.30064174107143,-0.874659937391706)(1.30078125,-0.873537679725604)(1.30092075892857,-0.87241625359247)(1.30106026785714,-0.871295661446426)(1.30119977678571,-0.870175905739978)(1.30133928571429,-0.869056988923993)(1.30147879464286,-0.86793891344772)(1.30161830357143,-0.866821681758754)(1.3017578125,-0.865705296303064)(1.30189732142857,-0.864589759524963)(1.30203683035714,-0.86347507386711)(1.30217633928571,-0.862361241770516)(1.30231584821429,-0.861248265674518)(1.30245535714286,-0.860136148016804)(1.30259486607143,-0.859024891233374)(1.302734375,-0.857914497758565)(1.30287388392857,-0.856804970025032)(1.30301339285714,-0.855696310463736)(1.30315290178571,-0.85458852150396)(1.30329241071429,-0.853481605573281)(1.30343191964286,-0.85237556509759)(1.30357142857143,-0.851270402501061)(1.3037109375,-0.850166120206169)(1.30385044642857,-0.849062720633675)(1.30398995535714,-0.847960206202613)(1.30412946428571,-0.846858579330306)(1.30426897321429,-0.845757842432338)(1.30440848214286,-0.844657997922575)(1.30454799107143,-0.843559048213129)(1.3046875,-0.842460995714386)(1.30482700892857,-0.84136384283498)(1.30496651785714,-0.840267591981789)(1.30510602678571,-0.839172245559945)(1.30524553571429,-0.838077805972807)(1.30538504464286,-0.836984275621985)(1.30552455357143,-0.835891656907304)(1.3056640625,-0.834799952226826)(1.30580357142857,-0.833709163976832)(1.30594308035714,-0.832619294551812)(1.30608258928571,-0.831530346344477)(1.30622209821429,-0.830442321745735)(1.30636160714286,-0.82935522314471)(1.30650111607143,-0.828269052928708)(1.306640625,-0.827183813483243)(1.30678013392857,-0.82609950719201)(1.30691964285714,-0.825016136436885)(1.30705915178571,-0.82393370359793)(1.30719866071429,-0.822852211053373)(1.30733816964286,-0.821771661179625)(1.30747767857143,-0.820692056351246)(1.3076171875,-0.819613398940972)(1.30775669642857,-0.818535691319687)(1.30789620535714,-0.817458935856424)(1.30803571428571,-0.816383134918372)(1.30817522321429,-0.815308290870849)(1.30831473214286,-0.814234406077328)(1.30845424107143,-0.813161482899395)(1.30859375,-0.812089523696783)(1.30873325892857,-0.81101853082734)(1.30887276785714,-0.80994850664703)(1.30901227678571,-0.808879453509939)(1.30915178571429,-0.807811373768255)(1.30929129464286,-0.806744269772283)(1.30943080357143,-0.805678143870414)(1.3095703125,-0.80461299840915)(1.30970982142857,-0.803548835733079)(1.30984933035714,-0.80248565818487)(1.30998883928571,-0.801423468105286)(1.31012834821429,-0.800362267833154)(1.31026785714286,-0.799302059705393)(1.31040736607143,-0.798242846056972)(1.310546875,-0.797184629220938)(1.31068638392857,-0.796127411528396)(1.31082589285714,-0.795071195308497)(1.31096540178571,-0.794015982888456)(1.31110491071429,-0.79296177659352)(1.31124441964286,-0.791908578746995)(1.31138392857143,-0.790856391670206)(1.3115234375,-0.789805217682525)(1.31166294642857,-0.788755059101349)(1.31180245535714,-0.787705918242091)(1.31194196428571,-0.786657797418194)(1.31208147321429,-0.785610698941104)(1.31222098214286,-0.784564625120293)(1.31236049107143,-0.78351957826322)(1.3125,-0.782475560675361)(1.31263950892857,-0.781432574660184)(1.31277901785714,-0.780390622519143)(1.31291852678571,-0.77934970655169)(1.31305803571429,-0.778309829055247)(1.31319754464286,-0.777270992325232)(1.31333705357143,-0.776233198655021)(1.3134765625,-0.775196450335971)(1.31361607142857,-0.774160749657403)(1.31375558035714,-0.773126098906592)(1.31389508928571,-0.772092500368777)(1.31403459821429,-0.771059956327142)(1.31417410714286,-0.770028469062831)(1.31431361607143,-0.768998040854913)(1.314453125,-0.767968673980414)(1.31459263392857,-0.766940370714286)(1.31473214285714,-0.765913133329407)(1.31487165178571,-0.764886964096589)(1.31501116071429,-0.763861865284554)(1.31515066964286,-0.762837839159958)(1.31529017857143,-0.761814887987348)(1.3154296875,-0.760793014029196)(1.31556919642857,-0.759772219545873)(1.31570870535714,-0.758752506795641)(1.31584821428571,-0.757733878034668)(1.31598772321429,-0.756716335517001)(1.31612723214286,-0.755699881494587)(1.31626674107143,-0.754684518217237)(1.31640625,-0.753670247932656)(1.31654575892857,-0.752657072886414)(1.31668526785714,-0.751644995321945)(1.31682477678571,-0.750634017480557)(1.31696428571429,-0.749624141601405)(1.31710379464286,-0.748615369921515)(1.31724330357143,-0.747607704675747)(1.3173828125,-0.746601148096822)(1.31752232142857,-0.745595702415298)(1.31766183035714,-0.744591369859565)(1.31780133928571,-0.743588152655858)(1.31794084821429,-0.742586053028227)(1.31808035714286,-0.741585073198563)(1.31821986607143,-0.740585215386562)(1.318359375,-0.739586481809748)(1.31849888392857,-0.738588874683455)(1.31863839285714,-0.737592396220817)(1.31877790178571,-0.73659704863278)(1.31891741071429,-0.735602834128081)(1.31905691964286,-0.734609754913262)(1.31919642857143,-0.733617813192644)(1.3193359375,-0.732627011168344)(1.31947544642857,-0.731637351040257)(1.31961495535714,-0.730648835006051)(1.31975446428571,-0.729661465261177)(1.31989397321429,-0.728675243998842)(1.32003348214286,-0.727690173410033)(1.32017299107143,-0.726706255683483)(1.3203125,-0.725723493005692)(1.32045200892857,-0.724741887560909)(1.32059151785714,-0.723761441531125)(1.32073102678571,-0.722782157096083)(1.32087053571429,-0.721804036433256)(1.32101004464286,-0.720827081717865)(1.32114955357143,-0.719851295122846)(1.3212890625,-0.718876678818874)(1.32142857142857,-0.717903234974343)(1.32156808035714,-0.716930965755359)(1.32170758928571,-0.715959873325752)(1.32184709821429,-0.71498995984705)(1.32198660714286,-0.714021227478501)(1.32212611607143,-0.713053678377036)(1.322265625,-0.712087314697301)(1.32240513392857,-0.711122138591627)(1.32254464285714,-0.710158152210028)(1.32268415178571,-0.709195357700213)(1.32282366071429,-0.70823375720756)(1.32296316964286,-0.707273352875138)(1.32310267857143,-0.706314146843672)(1.3232421875,-0.705356141251565)(1.32338169642857,-0.704399338234884)(1.32352120535714,-0.703443739927347)(1.32366071428571,-0.702489348460336)(1.32380022321429,-0.701536165962876)(1.32393973214286,-0.700584194561651)(1.32407924107143,-0.699633436380973)(1.32421875,-0.698683893542805)(1.32435825892857,-0.697735568166741)(1.32449776785714,-0.696788462369999)(1.32463727678571,-0.695842578267435)(1.32477678571429,-0.694897917971515)(1.32491629464286,-0.693954483592335)(1.32505580357143,-0.693012277237593)(1.3251953125,-0.692071301012607)(1.32533482142857,-0.691131557020299)(1.32547433035714,-0.690193047361186)(1.32561383928571,-0.689255774133392)(1.32575334821429,-0.688319739432623)(1.32589285714286,-0.687384945352189)(1.32603236607143,-0.686451393982972)(1.326171875,-0.685519087413444)(1.32631138392857,-0.684588027729654)(1.32645089285714,-0.683658217015216)(1.32659040178571,-0.682729657351323)(1.32672991071429,-0.681802350816725)(1.32686941964286,-0.680876299487742)(1.32700892857143,-0.67995150543824)(1.3271484375,-0.679027970739649)(1.32728794642857,-0.678105697460944)(1.32742745535714,-0.677184687668639)(1.32756696428571,-0.676264943426797)(1.32770647321429,-0.675346466797011)(1.32784598214286,-0.674429259838417)(1.32798549107143,-0.673513324607666)(1.328125,-0.672598663158947)(1.32826450892857,-0.671685277543964)(1.32840401785714,-0.670773169811934)(1.32854352678571,-0.669862342009595)(1.32868303571429,-0.668952796181185)(1.32882254464286,-0.668044534368459)(1.32896205357143,-0.667137558610657)(1.3291015625,-0.666231870944531)(1.32924107142857,-0.665327473404321)(1.32938058035714,-0.664424368021749)(1.32952008928571,-0.663522556826034)(1.32965959821429,-0.662622041843863)(1.32979910714286,-0.661722825099414)(1.32993861607143,-0.660824908614326)(1.330078125,-0.659928294407717)(1.33021763392857,-0.659032984496167)(1.33035714285714,-0.658138980893712)(1.33049665178571,-0.657246285611856)(1.33063616071429,-0.656354900659545)(1.33077566964286,-0.655464828043187)(1.33091517857143,-0.654576069766623)(1.3310546875,-0.653688627831147)(1.33119419642857,-0.652802504235488)(1.33133370535714,-0.651917700975803)(1.33147321428571,-0.651034220045689)(1.33161272321429,-0.650152063436157)(1.33175223214286,-0.649271233135657)(1.33189174107143,-0.64839173113004)(1.33203125,-0.647513559402586)(1.33217075892857,-0.64663671993398)(1.33231026785714,-0.645761214702311)(1.33244977678571,-0.644887045683078)(1.33258928571429,-0.644014214849171)(1.33272879464286,-0.643142724170887)(1.33286830357143,-0.642272575615901)(1.3330078125,-0.641403771149287)(1.33314732142857,-0.640536312733501)(1.33328683035714,-0.639670202328372)(1.33342633928571,-0.638805441891115)(1.33356584821429,-0.637942033376308)(1.33370535714286,-0.637079978735909)(1.33384486607143,-0.636219279919229)(1.333984375,-0.63535993887295)(1.33412388392857,-0.634501957541109)(1.33426339285714,-0.633645337865091)(1.33440290178571,-0.632790081783639)(1.33454241071429,-0.631936191232833)(1.33468191964286,-0.631083668146106)(1.33482142857143,-0.630232514454218)(1.3349609375,-0.629382732085274)(1.33510044642857,-0.628534322964707)(1.33523995535714,-0.627687289015272)(1.33537946428571,-0.626841632157055)(1.33551897321429,-0.625997354307453)(1.33565848214286,-0.625154457381191)(1.33579799107143,-0.624312943290294)(1.3359375,-0.623472813944105)(1.33607700892857,-0.62263407124927)(1.33621651785714,-0.62179671710973)(1.33635602678571,-0.620960753426732)(1.33649553571429,-0.620126182098808)(1.33663504464286,-0.61929300502179)(1.33677455357143,-0.618461224088786)(1.3369140625,-0.617630841190197)(1.33705357142857,-0.616801858213697)(1.33719308035714,-0.615974277044234)(1.33733258928571,-0.615148099564033)(1.33747209821429,-0.61432332765258)(1.33761160714286,-0.613499963186635)(1.33775111607143,-0.612678008040208)(1.337890625,-0.611857464084574)(1.33803013392857,-0.611038333188262)(1.33816964285714,-0.610220617217041)(1.33830915178571,-0.609404318033938)(1.33844866071429,-0.608589437499213)(1.33858816964286,-0.607775977470374)(1.33872767857143,-0.606963939802153)(1.3388671875,-0.606153326346527)(1.33900669642857,-0.605344138952692)(1.33914620535714,-0.604536379467069)(1.33928571428571,-0.603730049733305)(1.33942522321429,-0.602925151592255)(1.33956473214286,-0.602121686882002)(1.33970424107143,-0.601319657437823)(1.33984375,-0.600519065092214)(1.33998325892857,-0.59971991167487)(1.34012276785714,-0.598922199012682)(1.34026227678571,-0.598125928929744)(1.34040178571429,-0.597331103247331)(1.34054129464286,-0.596537723783922)(1.34068080357143,-0.595745792355167)(1.3408203125,-0.594955310773908)(1.34095982142857,-0.594166280850161)(1.34109933035714,-0.593378704391114)(1.34123883928571,-0.592592583201132)(1.34137834821429,-0.591807919081741)(1.34151785714286,-0.59102471383164)(1.34165736607143,-0.590242969246677)(1.341796875,-0.589462687119868)(1.34193638392857,-0.58868386924138)(1.34207589285714,-0.587906517398523)(1.34221540178571,-0.587130633375765)(1.34235491071429,-0.586356218954704)(1.34249441964286,-0.585583275914092)(1.34263392857143,-0.584811806029805)(1.3427734375,-0.584041811074861)(1.34291294642857,-0.583273292819404)(1.34305245535714,-0.5825062530307)(1.34319196428571,-0.581740693473145)(1.34333147321429,-0.580976615908245)(1.34347098214286,-0.580214022094632)(1.34361049107143,-0.57945291378804)(1.34375,-0.57869329274132)(1.34388950892857,-0.577935160704427)(1.34402901785714,-0.577178519424411)(1.34416852678571,-0.576423370645431)(1.34430803571429,-0.575669716108729)(1.34444754464286,-0.574917557552653)(1.34458705357143,-0.574166896712626)(1.3447265625,-0.573417735321166)(1.34486607142857,-0.57267007510787)(1.34500558035714,-0.571923917799407)(1.34514508928571,-0.571179265119531)(1.34528459821429,-0.570436118789057)(1.34542410714286,-0.56969448052588)(1.34556361607143,-0.568954352044947)(1.345703125,-0.568215735058279)(1.34584263392857,-0.56747863127495)(1.34598214285714,-0.566743042401083)(1.34612165178571,-0.566008970139862)(1.34626116071429,-0.565276416191512)(1.34640066964286,-0.564545382253311)(1.34654017857143,-0.563815870019567)(1.3466796875,-0.563087881181639)(1.34681919642857,-0.562361417427914)(1.34695870535714,-0.56163648044381)(1.34709821428571,-0.560913071911778)(1.34723772321429,-0.560191193511287)(1.34737723214286,-0.559470846918839)(1.34751674107143,-0.558752033807942)(1.34765625,-0.55803475584913)(1.34779575892857,-0.557319014709945)(1.34793526785714,-0.556604812054934)(1.34807477678571,-0.555892149545658)(1.34821428571429,-0.555181028840672) 
};
\addplot [
color=blue,
solid,
forget plot
]
coordinates{
 (1.34821428571429,-0.555181028840672)(1.34835379464286,-0.554471451595537)(1.34849330357143,-0.553763419462805)(1.3486328125,-0.553056934092024)(1.34877232142857,-0.552351997129733)(1.34891183035714,-0.55164861021945)(1.34905133928571,-0.550946775001686)(1.34919084821429,-0.550246493113922)(1.34933035714286,-0.549547766190626)(1.34946986607143,-0.548850595863229)(1.349609375,-0.548154983760143)(1.34974888392857,-0.547460931506741)(1.34988839285714,-0.546768440725359)(1.35002790178571,-0.546077513035299)(1.35016741071429,-0.545388150052814)(1.35030691964286,-0.544700353391121)(1.35044642857143,-0.544014124660378)(1.3505859375,-0.543329465467701)(1.35072544642857,-0.542646377417145)(1.35086495535714,-0.541964862109709)(1.35100446428571,-0.541284921143331)(1.35114397321429,-0.540606556112883)(1.35128348214286,-0.539929768610176)(1.35142299107143,-0.53925456022394)(1.3515625,-0.538580932539844)(1.35170200892857,-0.537908887140473)(1.35184151785714,-0.537238425605332)(1.35198102678571,-0.536569549510848)(1.35212053571429,-0.535902260430355)(1.35226004464286,-0.535236559934109)(1.35239955357143,-0.534572449589262)(1.3525390625,-0.53390993095988)(1.35267857142857,-0.533249005606931)(1.35281808035714,-0.532589675088274)(1.35295758928571,-0.531931940958674)(1.35309709821429,-0.531275804769779)(1.35323660714286,-0.530621268070139)(1.35337611607143,-0.529968332405176)(1.353515625,-0.529316999317211)(1.35365513392857,-0.528667270345438)(1.35379464285714,-0.528019147025927)(1.35393415178571,-0.527372630891629)(1.35407366071429,-0.526727723472361)(1.35421316964286,-0.526084426294816)(1.35435267857143,-0.525442740882544)(1.3544921875,-0.524802668755966)(1.35463169642857,-0.524164211432363)(1.35477120535714,-0.523527370425865)(1.35491071428571,-0.522892147247466)(1.35505022321429,-0.522258543405001)(1.35518973214286,-0.521626560403167)(1.35532924107143,-0.520996199743491)(1.35546875,-0.520367462924354)(1.35560825892857,-0.519740351440974)(1.35574776785714,-0.5191148667854)(1.35588727678571,-0.518491010446523)(1.35602678571429,-0.517868783910056)(1.35616629464286,-0.517248188658551)(1.35630580357143,-0.516629226171373)(1.3564453125,-0.516011897924718)(1.35658482142857,-0.5153962053916)(1.35672433035714,-0.514782150041844)(1.35686383928571,-0.514169733342095)(1.35700334821429,-0.513558956755804)(1.35714285714286,-0.512949821743235)(1.35728236607143,-0.512342329761451)(1.357421875,-0.511736482264322)(1.35756138392857,-0.511132280702517)(1.35770089285714,-0.510529726523498)(1.35784040178571,-0.509928821171524)(1.35797991071429,-0.509329566087642)(1.35811941964286,-0.508731962709693)(1.35825892857143,-0.508136012472294)(1.3583984375,-0.507541716806854)(1.35853794642857,-0.506949077141559)(1.35867745535714,-0.506358094901366)(1.35881696428571,-0.505768771508014)(1.35895647321429,-0.505181108380007)(1.35909598214286,-0.504595106932626)(1.35923549107143,-0.504010768577908)(1.359375,-0.503428094724661)(1.35951450892857,-0.50284708677845)(1.35965401785714,-0.502267746141597)(1.35979352678571,-0.501690074213181)(1.35993303571429,-0.50111407238903)(1.36007254464286,-0.500539742061729)(1.36021205357143,-0.499967084620598)(1.3603515625,-0.499396101451713)(1.36049107142857,-0.498826793937885)(1.36063058035714,-0.498259163458664)(1.36077008928571,-0.497693211390337)(1.36090959821429,-0.497128939105923)(1.36104910714286,-0.496566347975176)(1.36118861607143,-0.496005439364571)(1.361328125,-0.495446214637316)(1.36146763392857,-0.494888675153338)(1.36160714285714,-0.49433282226928)(1.36174665178571,-0.493778657338512)(1.36188616071429,-0.493226181711107)(1.36202566964286,-0.492675396733863)(1.36216517857143,-0.492126303750275)(1.3623046875,-0.491578904100554)(1.36244419642857,-0.491033199121613)(1.36258370535714,-0.490489190147064)(1.36272321428571,-0.489946878507222)(1.36286272321429,-0.489406265529092)(1.36300223214286,-0.488867352536383)(1.36314174107143,-0.488330140849486)(1.36328125,-0.487794631785486)(1.36342075892857,-0.487260826658155)(1.36356026785714,-0.486728726777942)(1.36369977678571,-0.486198333451986)(1.36383928571429,-0.485669647984095)(1.36397879464286,-0.485142671674765)(1.36411830357143,-0.484617405821152)(1.3642578125,-0.484093851717094)(1.36439732142857,-0.483572010653093)(1.36453683035714,-0.483051883916315)(1.36467633928571,-0.482533472790594)(1.36481584821429,-0.48201677855642)(1.36495535714286,-0.481501802490947)(1.36509486607143,-0.480988545867979)(1.365234375,-0.480477009957978)(1.36537388392857,-0.479967196028057)(1.36551339285714,-0.479459105341974)(1.36565290178571,-0.478952739160136)(1.36579241071429,-0.478448098739591)(1.36593191964286,-0.477945185334033)(1.36607142857143,-0.477444000193789)(1.3662109375,-0.476944544565826)(1.36635044642857,-0.476446819693746)(1.36648995535714,-0.475950826817778)(1.36662946428571,-0.475456567174784)(1.36676897321429,-0.474964041998249)(1.36690848214286,-0.474473252518287)(1.36704799107143,-0.473984199961628)(1.3671875,-0.473496885551629)(1.36732700892857,-0.473011310508258)(1.36746651785714,-0.472527476048098)(1.36760602678571,-0.472045383384348)(1.36774553571429,-0.471565033726813)(1.36788504464286,-0.47108642828191)(1.36802455357143,-0.470609568252656)(1.3681640625,-0.470134454838675)(1.36830357142857,-0.469661089236193)(1.36844308035714,-0.469189472638027)(1.36858258928571,-0.468719606233598)(1.36872209821429,-0.468251491208914)(1.36886160714286,-0.467785128746582)(1.36900111607143,-0.467320520025789)(1.369140625,-0.466857666222317)(1.36928013392857,-0.466396568508528)(1.36941964285714,-0.465937228053365)(1.36955915178571,-0.465479646022355)(1.36969866071429,-0.465023823577598)(1.36983816964286,-0.464569761877776)(1.36997767857143,-0.464117462078135)(1.3701171875,-0.463666925330501)(1.37025669642857,-0.463218152783263)(1.37039620535714,-0.462771145581376)(1.37053571428571,-0.462325904866365)(1.37067522321429,-0.461882431776308)(1.37081473214286,-0.461440727445852)(1.37095424107143,-0.461000793006193)(1.37109375,-0.46056262958509)(1.37123325892857,-0.460126238306851)(1.37137276785714,-0.459691620292334)(1.37151227678571,-0.459258776658948)(1.37165178571429,-0.458827708520647)(1.37179129464286,-0.458398416987933)(1.37193080357143,-0.457970903167843)(1.3720703125,-0.457545168163962)(1.37220982142857,-0.457121213076409)(1.37234933035714,-0.456699039001837)(1.37248883928571,-0.456278647033439)(1.37262834821429,-0.45586003826093)(1.37276785714286,-0.455443213770565)(1.37290736607143,-0.455028174645117)(1.373046875,-0.454614921963892)(1.37318638392857,-0.454203456802714)(1.37332589285714,-0.453793780233928)(1.37346540178571,-0.453385893326402)(1.37360491071429,-0.452979797145515)(1.37374441964286,-0.452575492753167)(1.37388392857143,-0.452172981207765)(1.3740234375,-0.451772263564231)(1.37416294642857,-0.451373340873994)(1.37430245535714,-0.450976214184988)(1.37444196428571,-0.450580884541655)(1.37458147321429,-0.450187352984934)(1.37472098214286,-0.449795620552271)(1.37486049107143,-0.449405688277604)(1.375,-0.449017557191373)(1.37513950892857,-0.44863122832051)(1.37527901785714,-0.448246702688438)(1.37541852678571,-0.447863981315072)(1.37555803571429,-0.447483065216815)(1.37569754464286,-0.447103955406558)(1.37583705357143,-0.446726652893673)(1.3759765625,-0.446351158684018)(1.37611607142857,-0.445977473779931)(1.37625558035714,-0.445605599180225)(1.37639508928571,-0.445235535880195)(1.37653459821429,-0.444867284871606)(1.37667410714286,-0.444500847142699)(1.37681361607143,-0.444136223678184)(1.376953125,-0.443773415459242)(1.37709263392857,-0.443412423463519)(1.37723214285714,-0.443053248665126)(1.37737165178571,-0.442695892034639)(1.37751116071429,-0.442340354539094)(1.37765066964286,-0.441986637141987)(1.37779017857143,-0.441634740803271)(1.3779296875,-0.441284666479357)(1.37806919642857,-0.440936415123107)(1.37820870535714,-0.440589987683835)(1.37834821428571,-0.440245385107307)(1.37848772321429,-0.439902608335736)(1.37862723214286,-0.439561658307785)(1.37876674107143,-0.439222535958554)(1.37890625,-0.438885242219593)(1.37904575892857,-0.438549778018892)(1.37918526785714,-0.438216144280876)(1.37932477678571,-0.437884341926412)(1.37946428571429,-0.437554371872799)(1.37960379464286,-0.437226235033773)(1.37974330357143,-0.4368999323195)(1.3798828125,-0.436575464636578)(1.38002232142857,-0.436252832888033)(1.38016183035714,-0.435932037973316)(1.38030133928571,-0.435613080788304)(1.38044084821429,-0.435295962225298)(1.38058035714286,-0.434980683173021)(1.38071986607143,-0.434667244516613)(1.380859375,-0.434355647137636)(1.38099888392857,-0.434045891914066)(1.38113839285714,-0.433737979720293)(1.38127790178571,-0.433431911427122)(1.38141741071429,-0.433127687901767)(1.38155691964286,-0.432825310007855)(1.38169642857143,-0.432524778605417)(1.3818359375,-0.432226094550892)(1.38197544642857,-0.431929258697126)(1.38211495535714,-0.431634271893362)(1.38225446428571,-0.43134113498525)(1.38239397321429,-0.431049848814835)(1.38253348214286,-0.430760414220564)(1.38267299107143,-0.430472832037278)(1.3828125,-0.430187103096214)(1.38295200892857,-0.429903228225)(1.38309151785714,-0.429621208247658)(1.38323102678571,-0.429341043984599)(1.38337053571429,-0.429062736252619)(1.38351004464286,-0.428786285864907)(1.38364955357143,-0.428511693631032)(1.3837890625,-0.428238960356949)(1.38392857142857,-0.427968086844994)(1.38406808035714,-0.427699073893883)(1.38420758928571,-0.427431922298713)(1.38434709821429,-0.427166632850954)(1.38448660714286,-0.426903206338457)(1.38462611607143,-0.426641643545443)(1.384765625,-0.426381945252509)(1.38490513392857,-0.42612411223662)(1.38504464285714,-0.425868145271114)(1.38518415178571,-0.425614045125694)(1.38532366071429,-0.42536181256643)(1.38546316964286,-0.425111448355761)(1.38560267857143,-0.424862953252484)(1.3857421875,-0.424616328011764)(1.38588169642857,-0.424371573385122)(1.38602120535714,-0.42412869012044)(1.38616071428571,-0.423887678961959)(1.38630022321429,-0.423648540650273)(1.38643973214286,-0.423411275922336)(1.38657924107143,-0.423175885511451)(1.38671875,-0.422942370147275)(1.38685825892857,-0.422710730555816)(1.38699776785714,-0.422480967459431)(1.38713727678571,-0.422253081576823)(1.38727678571429,-0.422027073623043)(1.38741629464286,-0.421802944309489)(1.38755580357143,-0.421580694343898)(1.3876953125,-0.421360324430353)(1.38783482142857,-0.421141835269277)(1.38797433035714,-0.420925227557433)(1.38811383928571,-0.42071050198792)(1.38825334821429,-0.420497659250175)(1.38839285714286,-0.420286700029973)(1.38853236607143,-0.420077625009418)(1.388671875,-0.419870434866953)(1.38881138392857,-0.419665130277347)(1.38895089285714,-0.419461711911703)(1.38909040178571,-0.419260180437451)(1.38922991071429,-0.419060536518347)(1.38936941964286,-0.418862780814479)(1.38950892857143,-0.418666913982253)(1.3896484375,-0.418472936674404)(1.38978794642857,-0.418280849539987)(1.38992745535714,-0.418090653224378)(1.39006696428571,-0.417902348369275)(1.39020647321429,-0.417715935612692)(1.39034598214286,-0.417531415588964)(1.39048549107143,-0.417348788928737)(1.390625,-0.417168056258977)(1.39076450892857,-0.416989218202962)(1.39090401785714,-0.416812275380282)(1.39104352678571,-0.416637228406839)(1.39118303571429,-0.416464077894843)(1.39132254464286,-0.416292824452818)(1.39146205357143,-0.41612346868559)(1.3916015625,-0.415956011194295)(1.39174107142857,-0.415790452576376)(1.39188058035714,-0.415626793425575)(1.39202008928571,-0.415465034331941)(1.39215959821429,-0.415305175881823)(1.39229910714286,-0.415147218657875)(1.39243861607143,-0.414991163239044)(1.392578125,-0.414837010200582)(1.39271763392857,-0.414684760114034)(1.39285714285714,-0.414534413547244)(1.39299665178571,-0.414385971064349)(1.39313616071429,-0.414239433225781)(1.39327566964286,-0.414094800588267)(1.39341517857143,-0.413952073704822)(1.3935546875,-0.413811253124757)(1.39369419642857,-0.413672339393668)(1.39383370535714,-0.413535333053442)(1.39397321428571,-0.413400234642254)(1.39411272321429,-0.413267044694565)(1.39425223214286,-0.413135763741121)(1.39439174107143,-0.413006392308953)(1.39453125,-0.412878930921378)(1.39467075892857,-0.412753380097991)(1.39481026785714,-0.412629740354671)(1.39494977678571,-0.412508012203579)(1.39508928571429,-0.412388196153152)(1.39522879464286,-0.412270292708109)(1.39536830357143,-0.412154302369444)(1.3955078125,-0.412040225634429)(1.39564732142857,-0.411928062996611)(1.39578683035714,-0.411817814945811)(1.39592633928571,-0.411709481968126)(1.39606584821429,-0.411603064545923)(1.39620535714286,-0.411498563157842)(1.39634486607143,-0.411395978278795)(1.396484375,-0.411295310379962)(1.39662388392857,-0.411196559928793)(1.39676339285714,-0.411099727389005)(1.39690290178571,-0.411004813220585)(1.39704241071429,-0.410911817879784)(1.39718191964286,-0.410820741819118)(1.39732142857143,-0.410731585487369)(1.3974609375,-0.410644349329583)(1.39760044642857,-0.410559033787067)(1.39773995535714,-0.410475639297392)(1.39787946428571,-0.410394166294389)(1.39801897321429,-0.410314615208147)(1.39815848214286,-0.41023698646502)(1.39829799107143,-0.410161280487615)(1.3984375,-0.410087497694801)(1.39857700892857,-0.410015638501701)(1.39871651785714,-0.409945703319695)(1.39885602678571,-0.409877692556419)(1.39899553571429,-0.409811606615763)(1.39913504464286,-0.409747445897872)(1.39927455357143,-0.409685210799141)(1.3994140625,-0.40962490171222)(1.39955357142857,-0.40956651902601)(1.39969308035714,-0.409510063125662)(1.39983258928571,-0.409455534392577)(1.39997209821429,-0.409402933204405)(1.40011160714286,-0.409352259935046)(1.40025111607143,-0.409303514954646)(1.400390625,-0.409256698629599)(1.40053013392857,-0.409211811322544)(1.40066964285714,-0.409168853392367)(1.40080915178571,-0.409127825194199)(1.40094866071429,-0.409088727079415)(1.40108816964286,-0.409051559395632)(1.40122767857143,-0.409016322486711)(1.4013671875,-0.408983016692757)(1.40150669642857,-0.408951642350113)(1.40164620535714,-0.408922199791365)(1.40178571428571,-0.40889468934534)(1.40192522321429,-0.408869111337102)(1.40206473214286,-0.408845466087956)(1.40220424107143,-0.408823753915444)(1.40234375,-0.408803975133348)(1.40248325892857,-0.408786130051684)(1.40262276785714,-0.408770218976707)(1.40276227678571,-0.408756242210906)(1.40290178571429,-0.408744200053006)(1.40304129464286,-0.408734092797968)(1.40318080357143,-0.408725920736986)(1.4033203125,-0.408719684157487)(1.40345982142857,-0.408715383343132)(1.40359933035714,-0.408713018573814)(1.40373883928571,-0.408712590125658)(1.40387834821429,-0.40871409827102)(1.40401785714286,-0.408717543278488)(1.40415736607143,-0.408722925412879)(1.404296875,-0.408730244935242)(1.40443638392857,-0.408739502102852)(1.40457589285714,-0.408750697169216)(1.40471540178571,-0.408763830384066)(1.40485491071429,-0.408778901993366)(1.40499441964286,-0.408795912239304)(1.40513392857143,-0.408814861360295)(1.4052734375,-0.408835749590984)(1.40541294642857,-0.408858577162236)(1.40555245535714,-0.408883344301148)(1.40569196428571,-0.408910051231037)(1.40583147321429,-0.408938698171447)(1.40597098214286,-0.408969285338146)(1.40611049107143,-0.409001812943126)(1.40625,-0.409036281194601)(1.40638950892857,-0.409072690297009)(1.40652901785714,-0.409111040451012)(1.40666852678571,-0.409151331853491)(1.40680803571429,-0.40919356469755)(1.40694754464286,-0.409237739172517)(1.40708705357143,-0.409283855463938)(1.4072265625,-0.40933191375358)(1.40736607142857,-0.409381914219431)(1.40750558035714,-0.4094338570357)(1.40764508928571,-0.409487742372814)(1.40778459821429,-0.40954357039742)(1.40792410714286,-0.409601341272384)(1.40806361607143,-0.409661055156791)(1.408203125,-0.409722712205943)(1.40834263392857,-0.409786312571362)(1.40848214285714,-0.409851856400786)(1.40862165178571,-0.409919343838173)(1.40876116071429,-0.409988775023694)(1.40890066964286,-0.410060150093742)(1.40904017857143,-0.410133469180922)(1.4091796875,-0.41020873241406)(1.40931919642857,-0.410285939918193)(1.40945870535714,-0.410365091814579)(1.40959821428571,-0.410446188220687)(1.40973772321429,-0.410529229250206)(1.40987723214286,-0.410614215013036)(1.41001674107143,-0.410701145615296)(1.41015625,-0.410790021159317)(1.41029575892857,-0.410880841743644)(1.41043526785714,-0.41097360746304)(1.41057477678571,-0.411068318408479)(1.41071428571429,-0.411164974667151)(1.41085379464286,-0.411263576322457)(1.41099330357143,-0.411364123454016)(1.4111328125,-0.411466616137657)(1.41127232142857,-0.411571054445424)(1.41141183035714,-0.411677438445575)(1.41155133928571,-0.411785768202579)(1.41169084821429,-0.41189604377712)(1.41183035714286,-0.412008265226093)(1.41196986607143,-0.412122432602608)(1.412109375,-0.412238545955986)(1.41224888392857,-0.412356605331761)(1.41238839285714,-0.412476610771681)(1.41252790178571,-0.412598562313703)(1.41266741071429,-0.412722459992001)(1.41280691964286,-0.412848303836957)(1.41294642857143,-0.412976093875168)(1.4130859375,-0.413105830129441)(1.41322544642857,-0.413237512618797)(1.41336495535714,-0.413371141358468)(1.41350446428571,-0.413506716359899)(1.41364397321429,-0.413644237630747)(1.41378348214286,-0.413783705174879)(1.41392299107143,-0.413925118992377)(1.4140625,-0.414068479079533)(1.41420200892857,-0.414213785428851)(1.41434151785714,-0.414361038029048)(1.41448102678571,-0.414510236865053)(1.41462053571429,-0.414661381918007)(1.41476004464286,-0.414814473165262)(1.41489955357143,-0.414969510580384)(1.4150390625,-0.41512649413315)(1.41517857142857,-0.41528542378955)(1.41531808035714,-0.415446299511785)(1.41545758928571,-0.415609121258269)(1.41559709821429,-0.415773888983631)(1.41573660714286,-0.415940602638709)(1.41587611607143,-0.416109262170555)(1.416015625,-0.416279867522435)(1.41615513392857,-0.416452418633825)(1.41629464285714,-0.416626915440418)(1.41643415178571,-0.416803357874116)(1.41657366071429,-0.416981745863039)(1.41671316964286,-0.417162079331514)(1.41685267857143,-0.417344358200089)(1.4169921875,-0.41752858238552)(1.41713169642857,-0.417714751800778)(1.41727120535714,-0.417902866355051)(1.41741071428571,-0.418092925953738)(1.41755022321429,-0.418284930498454)(1.41768973214286,-0.418478879887027)(1.41782924107143,-0.418674774013503)(1.41796875,-0.418872612768139)(1.41810825892857,-0.41907239603741)(1.41824776785714,-0.419274123704007)(1.41838727678571,-0.419477795646832)(1.41852678571429,-0.41968341174101)(1.41866629464286,-0.419890971857875)(1.41880580357143,-0.420100475864983)(1.4189453125,-0.420311923626104)(1.41908482142857,-0.420525315001224)(1.41922433035714,-0.42074064984655)(1.41936383928571,-0.420957928014501)(1.41950334821429,-0.421177149353719)(1.41964285714286,-0.421398313709061)(1.41978236607143,-0.421621420921604)(1.419921875,-0.421846470828642)(1.42006138392857,-0.422073463263688)(1.42020089285714,-0.422302398056478)(1.42034040178571,-0.422533275032961)(1.42047991071429,-0.422766094015313)(1.42061941964286,-0.423000854821925)(1.42075892857143,-0.423237557267412)(1.4208984375,-0.423476201162608)(1.42103794642857,-0.42371678631457)(1.42117745535714,-0.423959312526575)(1.42131696428571,-0.424203779598124)(1.42145647321429,-0.42445018732494)(1.42159598214286,-0.424698535498967)(1.42173549107143,-0.424948823908376)(1.421875,-0.425201052337559)(1.42201450892857,-0.425455220567133)(1.42215401785714,-0.425711328373939)(1.42229352678571,-0.425969375531044)(1.42243303571429,-0.426229361807741)(1.42257254464286,-0.426491286969547)(1.42271205357143,-0.426755150778207)(1.4228515625,-0.427020952991693)(1.42299107142857,-0.427288693364202)(1.42313058035714,-0.427558371646162)(1.42327008928571,-0.427829987584227)(1.42340959821429,-0.428103540921281)(1.42354910714286,-0.428379031396437)(1.42368861607143,-0.428656458745037)(1.423828125,-0.428935822698655)(1.42396763392857,-0.429217122985093)(1.42410714285714,-0.429500359328389)(1.42424665178571,-0.429785531448807)(1.42438616071429,-0.430072639062849)(1.42452566964286,-0.430361681883246)(1.42466517857143,-0.430652659618966)(1.4248046875,-0.430945571975208)(1.42494419642857,-0.431240418653405)(1.42508370535714,-0.431537199351231)(1.42522321428571,-0.431835913762588)(1.42536272321429,-0.432136561577623)(1.42550223214286,-0.432439142482712)(1.42564174107143,-0.432743656160474)(1.42578125,-0.433050102289765)(1.42592075892857,-0.433358480545677)(1.42606026785714,-0.433668790599547)(1.42619977678571,-0.433981032118947)(1.42633928571429,-0.434295204767694)(1.42647879464286,-0.434611308205843)(1.42661830357143,-0.434929342089694)(1.4267578125,-0.435249306071788)(1.42689732142857,-0.435571199800909)(1.42703683035714,-0.435895022922088)(1.42717633928571,-0.436220775076598)(1.42731584821429,-0.43654845590196)(1.42745535714286,-0.436878065031937)(1.42759486607143,-0.437209602096546)(1.427734375,-0.437543066722045)(1.42787388392857,-0.437878458530943)(1.42801339285714,-0.438215777142)(1.42815290178571,-0.438555022170222)(1.42829241071429,-0.438896193226871)(1.42843191964286,-0.439239289919454)(1.42857142857143,-0.439584311851735)(1.4287109375,-0.43993125862373)(1.42885044642857,-0.440280129831706)(1.42898995535714,-0.44063092506819)(1.42912946428571,-0.440983643921957)(1.42926897321429,-0.441338285978046)(1.42940848214286,-0.441694850817746)(1.42954799107143,-0.442053338018609)(1.4296875,-0.442413747154442)(1.42982700892857,-0.442776077795312)(1.42996651785714,-0.443140329507549)(1.43010602678571,-0.443506501853741)(1.43024553571429,-0.443874594392739)(1.43038504464286,-0.444244606679656)(1.43052455357143,-0.444616538265871)(1.4306640625,-0.444990388699024)(1.43080357142857,-0.445366157523023)(1.43094308035714,-0.445743844278042)(1.43108258928571,-0.446123448500521)(1.43122209821429,-0.446504969723171)(1.43136160714286,-0.446888407474966)(1.43150111607143,-0.447273761281157)(1.431640625,-0.447661030663261)(1.43178013392857,-0.448050215139068)(1.43191964285714,-0.448441314222642)(1.43205915178571,-0.448834327424319)(1.43219866071429,-0.449229254250712)(1.43233816964286,-0.449626094204705)(1.43247767857143,-0.450024846785464)(1.4326171875,-0.450425511488429)(1.43275669642857,-0.450828087805317)(1.43289620535714,-0.451232575224131)(1.43303571428571,-0.451638973229146)(1.43317522321429,-0.452047281300927)(1.43331473214286,-0.452457498916314)(1.43345424107143,-0.452869625548436)(1.43359375,-0.453283660666704)(1.43373325892857,-0.453699603736812)(1.43387276785714,-0.454117454220747)(1.43401227678571,-0.454537211576778)(1.43415178571429,-0.454958875259467)(1.43429129464286,-0.455382444719661)(1.43443080357143,-0.455807919404502)(1.4345703125,-0.456235298757423)(1.43470982142857,-0.456664582218147)(1.43484933035714,-0.457095769222695)(1.43498883928571,-0.457528859203381)(1.43512834821429,-0.457963851588817)(1.43526785714286,-0.45840074580391)(1.43540736607143,-0.458839541269868)(1.435546875,-0.459280237404197)(1.43568638392857,-0.459722833620703)(1.43582589285714,-0.460167329329497)(1.43596540178571,-0.460613723936989)(1.43610491071429,-0.461062016845899)(1.43624441964286,-0.461512207455245)(1.43638392857143,-0.461964295160358)(1.4365234375,-0.462418279352873)(1.43666294642857,-0.462874159420735)(1.43680245535714,-0.4633319347482)(1.43694196428571,-0.463791604715832)(1.43708147321429,-0.464253168700514)(1.43722098214286,-0.464716626075434)(1.43736049107143,-0.465181976210104)(1.4375,-0.465649218470346)(1.43763950892857,-0.4661183522183)(1.43777901785714,-0.466589376812427)(1.43791852678571,-0.467062291607507)(1.43805803571429,-0.467537095954641)(1.43819754464286,-0.46801378920125)(1.43833705357143,-0.468492370691086)(1.4384765625,-0.468972839764216)(1.43861607142857,-0.469455195757039)(1.43875558035714,-0.469939438002284)(1.43889508928571,-0.470425565829001)(1.43903459821429,-0.470913578562579)(1.43917410714286,-0.471403475524731)(1.43931361607143,-0.471895256033508)(1.439453125,-0.472388919403293)(1.43959263392857,-0.472884464944802)(1.43973214285714,-0.473381891965093)(1.43987165178571,-0.473881199767557)(1.44001116071429,-0.47438238765193)(1.44015066964286,-0.474885454914283)(1.44029017857143,-0.475390400847034)(1.4404296875,-0.475897224738941)(1.44056919642857,-0.476405925875108)(1.44070870535714,-0.476916503536988)(1.44084821428571,-0.477428957002378)(1.44098772321429,-0.477943285545427)(1.44112723214286,-0.478459488436632)(1.44126674107143,-0.478977564942845)(1.44140625,-0.479497514327269)(1.44154575892857,-0.480019335849461)(1.44168526785714,-0.480543028765339)(1.44182477678571,-0.481068592327174)(1.44196428571429,-0.4815960257836)(1.44210379464286,-0.482125328379606)(1.44224330357143,-0.48265649935655)(1.4423828125,-0.48318953795215)(1.44252232142857,-0.483724443400487)(1.44266183035714,-0.484261214932015)(1.44280133928571,-0.484799851773548)(1.44294084821429,-0.485340353148278)(1.44308035714286,-0.48588271827576)(1.44321986607143,-0.48642694637193)(1.443359375,-0.48697303664909)(1.44349888392857,-0.487520988315921)(1.44363839285714,-0.488070800577485)(1.44377790178571,-0.488622472635215)(1.44391741071429,-0.489176003686934)(1.44405691964286,-0.489731392926837)(1.44419642857143,-0.49028863954551)(1.4443359375,-0.49084774272992)(1.44447544642857,-0.491408701663419)(1.44461495535714,-0.491971515525755)(1.44475446428571,-0.492536183493057)(1.44489397321429,-0.493102704737851)(1.44503348214286,-0.493671078429052)(1.44517299107143,-0.494241303731976)(1.4453125,-0.494813379808326)(1.44545200892857,-0.49538730581621)(1.44559151785714,-0.495963080910134)(1.44573102678571,-0.496540704241003)(1.44587053571429,-0.497120174956129)(1.44601004464286,-0.497701492199223)(1.44614955357143,-0.498284655110408)(1.4462890625,-0.49886966282621)(1.44642857142857,-0.499456514479565)(1.44656808035714,-0.500045209199825)(1.44670758928571,-0.500635746112747)(1.44684709821429,-0.501228124340513)(1.44698660714286,-0.50182234300171)(1.44712611607143,-0.502418401211353)(1.447265625,-0.503016298080868)(1.44740513392857,-0.503616032718107)(1.44754464285714,-0.504217604227347)(1.44768415178571,-0.504821011709286)(1.44782366071429,-0.505426254261052)(1.44796316964286,-0.506033330976198)(1.44810267857143,-0.506642240944711)(1.4482421875,-0.507252983253007)(1.44838169642857,-0.507865556983935)(1.44852120535714,-0.508479961216785)(1.44866071428571,-0.509096195027277)(1.44880022321429,-0.509714257487578)(1.44893973214286,-0.510334147666287)(1.44907924107143,-0.510955864628456)(1.44921875,-0.511579407435572)(1.44935825892857,-0.512204775145572)(1.44949776785714,-0.512831966812845)(1.44963727678571,-0.513460981488225)(1.44977678571429,-0.514091818219002)(1.44991629464286,-0.514724476048913)(1.45005580357143,-0.51535895401816)(1.4501953125,-0.515995251163396)(1.45033482142857,-0.516633366517733)(1.45047433035714,-0.517273299110749)(1.45061383928571,-0.517915047968481)(1.45075334821429,-0.518558612113435)(1.45089285714286,-0.519203990564578)(1.45103236607143,-0.519851182337354)(1.451171875,-0.520500186443669)(1.45131138392857,-0.521151001891906)(1.45145089285714,-0.521803627686925)(1.45159040178571,-0.522458062830058)(1.45172991071429,-0.52311430631912)(1.45186941964286,-0.523772357148401)(1.45200892857143,-0.52443221430868)(1.4521484375,-0.525093876787215)(1.45228794642857,-0.52575734356775)(1.45242745535714,-0.526422613630523)(1.45256696428571,-0.527089685952256)(1.45270647321429,-0.527758559506171)(1.45284598214286,-0.528429233261973)(1.45298549107143,-0.529101706185876)(1.453125,-0.529775977240583)(1.45326450892857,-0.530452045385298)(1.45340401785714,-0.531129909575735)(1.45354352678571,-0.531809568764103)(1.45368303571429,-0.532491021899126)(1.45382254464286,-0.533174267926028)(1.45396205357143,-0.533859305786551)(1.4541015625,-0.534546134418944)(1.45424107142857,-0.535234752757972)(1.45438058035714,-0.535925159734922)(1.45452008928571,-0.536617354277591)(1.45465959821429,-0.537311335310306)(1.45479910714286,-0.538007101753908)(1.45493861607143,-0.538704652525772)(1.455078125,-0.539403986539792)(1.45521763392857,-0.540105102706396)(1.45535714285714,-0.540807999932544)(1.45549665178571,-0.541512677121725)(1.45563616071429,-0.542219133173972)(1.45577566964286,-0.542927366985846)(1.45591517857143,-0.543637377450456)(1.4560546875,-0.544349163457447)(1.45619419642857,-0.545062723893011)(1.45633370535714,-0.54577805763989)(1.45647321428571,-0.546495163577366)(1.45661272321429,-0.547214040581284)(1.45675223214286,-0.547934687524028)(1.45689174107143,-0.548657103274551)(1.45703125,-0.549381286698352)(1.45717075892857,-0.550107236657493)(1.45731026785714,-0.550834952010605)(1.45744977678571,-0.551564431612872)(1.45758928571429,-0.552295674316054)(1.45772879464286,-0.553028678968469)(1.45786830357143,-0.55376344441502)(1.4580078125,-0.554499969497168)(1.45814732142857,-0.555238253052957)(1.45828683035714,-0.555978293917012)(1.45842633928571,-0.556720090920527)(1.45856584821429,-0.557463642891292)(1.45870535714286,-0.558208948653666)(1.45884486607143,-0.55895600702861)(1.458984375,-0.559704816833662)(1.45912388392857,-0.560455376882954)(1.45926339285714,-0.561207685987217)(1.45940290178571,-0.561961742953772)(1.45954241071429,-0.562717546586543)(1.45968191964286,-0.563475095686047)(1.45982142857143,-0.564234389049415)(1.4599609375,-0.564995425470371)(1.46010044642857,-0.565758203739254)(1.46023995535714,-0.566522722643012)(1.46037946428571,-0.567288980965203)(1.46051897321429,-0.568056977486005)(1.46065848214286,-0.568826710982204)(1.46079799107143,-0.569598180227215)(1.4609375,-0.570371383991067)(1.46107700892857,-0.571146321040416)(1.46121651785714,-0.571922990138547)(1.46135602678571,-0.57270139004537)(1.46149553571429,-0.573481519517432)(1.46163504464286,-0.574263377307904)(1.46177455357143,-0.575046962166605)(1.4619140625,-0.575832272839984)(1.46205357142857,-0.576619308071132)(1.46219308035714,-0.577408066599791)(1.46233258928571,-0.578198547162338)(1.46247209821429,-0.578990748491809)(1.46261160714286,-0.57978466931788)(1.46275111607143,-0.580580308366893)(1.462890625,-0.581377664361834)(1.46303013392857,-0.582176736022353)(1.46316964285714,-0.582977522064763)(1.46330915178571,-0.583780021202035)(1.46344866071429,-0.584584232143812)(1.46358816964286,-0.585390153596396)(1.46372767857143,-0.586197784262773)(1.4638671875,-0.587007122842591)(1.46400669642857,-0.587818168032176)(1.46414620535714,-0.58863091852454)(1.46428571428571,-0.589445373009365)(1.46442522321429,-0.590261530173028)(1.46456473214286,-0.59107938869858)(1.46470424107143,-0.591898947265774)(1.46484375,-0.592720204551042)(1.46498325892857,-0.593543159227514)(1.46512276785714,-0.594367809965022)(1.46526227678571,-0.595194155430088)(1.46540178571429,-0.596022194285946)(1.46554129464286,-0.596851925192521)(1.46568080357143,-0.597683346806459)(1.4658203125,-0.598516457781104)(1.46595982142857,-0.599351256766516)(1.46609933035714,-0.600187742409473)(1.46623883928571,-0.601025913353465)(1.46637834821429,-0.601865768238708)(1.46651785714286,-0.602707305702132)(1.46665736607143,-0.603550524377403)(1.466796875,-0.604395422894905)(1.46693638392857,-0.605241999881756)(1.46707589285714,-0.606090253961812)(1.46721540178571,-0.606940183755656)(1.46735491071429,-0.60779178788062)(1.46749441964286,-0.608645064950767)(1.46763392857143,-0.609500013576913)(1.4677734375,-0.610356632366614)(1.46791294642857,-0.611214919924175)(1.46805245535714,-0.612074874850661)(1.46819196428571,-0.612936495743884)(1.46833147321429,-0.613799781198419)(1.46847098214286,-0.614664729805595)(1.46861049107143,-0.615531340153512)(1.46875,-0.616399610827029)(1.46888950892857,-0.617269540407773)(1.46902901785714,-0.618141127474152)(1.46916852678571,-0.619014370601336)(1.46930803571429,-0.619889268361283)(1.46944754464286,-0.620765819322718)(1.46958705357143,-0.621644022051163)(1.4697265625,-0.622523875108912)(1.46986607142857,-0.623405377055053)(1.47000558035714,-0.624288526445467)(1.47014508928571,-0.625173321832823)(1.47028459821429,-0.626059761766594)(1.47042410714286,-0.626947844793042)(1.47056361607143,-0.627837569455243)(1.470703125,-0.628728934293068)(1.47084263392857,-0.629621937843198)(1.47098214285714,-0.63051657863913)(1.47112165178571,-0.631412855211168)(1.47126116071429,-0.632310766086438)(1.47140066964286,-0.633210309788877)(1.47154017857143,-0.634111484839256)(1.4716796875,-0.635014289755159)(1.47181919642857,-0.635918723051003)(1.47195870535714,-0.636824783238039)(1.47209821428571,-0.637732468824344)(1.47223772321429,-0.638641778314842)(1.47237723214286,-0.639552710211284)(1.47251674107143,-0.640465263012276)(1.47265625,-0.641379435213258)(1.47279575892857,-0.642295225306523)(1.47293526785714,-0.643212631781219)(1.47307477678571,-0.64413165312334)(1.47321428571429,-0.645052287815748)(1.47335379464286,-0.64597453433815)(1.47349330357143,-0.646898391167132)(1.4736328125,-0.647823856776134)(1.47377232142857,-0.648750929635467)(1.47391183035714,-0.649679608212321)(1.47405133928571,-0.650609890970752)(1.47419084821429,-0.651541776371703)(1.47433035714286,-0.652475262872984)(1.47446986607143,-0.653410348929306)(1.474609375,-0.654347032992253)(1.47474888392857,-0.655285313510303)(1.47488839285714,-0.656225188928831)(1.47502790178571,-0.657166657690102)(1.47516741071429,-0.658109718233287)(1.47530691964286,-0.659054368994448)(1.47544642857143,-0.660000608406565)(1.4755859375,-0.660948434899517)(1.47572544642857,-0.661897846900093)(1.47586495535714,-0.662848842832006)(1.47600446428571,-0.663801421115874)(1.47614397321429,-0.664755580169249)(1.47628348214286,-0.665711318406591)(1.47642299107143,-0.666668634239301)(1.4765625,-0.6676275260757)(1.47670200892857,-0.66858799232104)(1.47684151785714,-0.669550031377521)(1.47698102678571,-0.670513641644268)(1.47712053571429,-0.67147882151736)(1.47726004464286,-0.672445569389809)(1.47739955357143,-0.673413883651588)(1.4775390625,-0.67438376268961)(1.47767857142857,-0.675355204887747)(1.47781808035714,-0.676328208626833)(1.47795758928571,-0.677302772284655)(1.47809709821429,-0.678278894235973)(1.47823660714286,-0.679256572852502)(1.47837611607143,-0.680235806502941)(1.478515625,-0.681216593552953)(1.47865513392857,-0.682198932365177)(1.47879464285714,-0.683182821299241)(1.47893415178571,-0.684168258711745)(1.47907366071429,-0.685155242956288)(1.47921316964286,-0.686143772383442)(1.47935267857143,-0.687133845340789)(1.4794921875,-0.688125460172893)(1.47963169642857,-0.689118615221322)(1.47977120535714,-0.690113308824652)(1.47991071428571,-0.691109539318453)(1.48005022321429,-0.692107305035318)(1.48018973214286,-0.693106604304837)(1.48032924107143,-0.694107435453629)(1.48046875,-0.69510979680532)(1.48060825892857,-0.696113686680563)(1.48074776785714,-0.69711910339704)(1.48088727678571,-0.698126045269453)(1.48102678571429,-0.699134510609545)(1.48116629464286,-0.700144497726083)(1.48130580357143,-0.701156004924885)(1.4814453125,-0.702169030508798)(1.48158482142857,-0.703183572777719)(1.48172433035714,-0.704199630028598)(1.48186383928571,-0.705217200555427)(1.48200334821429,-0.706236282649265)(1.48214285714286,-0.707256874598213)(1.48228236607143,-0.708278974687449)(1.482421875,-0.709302581199207)(1.48256138392857,-0.710327692412786)(1.48270089285714,-0.71135430660457)(1.48284040178571,-0.712382422048002)(1.48297991071429,-0.713412037013617)(1.48311941964286,-0.714443149769018)(1.48325892857143,-0.715475758578907)(1.4833984375,-0.716509861705062)(1.48353794642857,-0.717545457406357)(1.48367745535714,-0.718582543938766)(1.48381696428571,-0.719621119555353)(1.48395647321429,-0.720661182506294)(1.48409598214286,-0.721702731038857)(1.48423549107143,-0.722745763397433)(1.484375,-0.723790277823513)(1.48451450892857,-0.724836272555708)(1.48465401785714,-0.725883745829753)(1.48479352678571,-0.726932695878495)(1.48493303571429,-0.727983120931921)(1.48507254464286,-0.729035019217128)(1.48521205357143,-0.730088388958367)(1.4853515625,-0.731143228377009)(1.48549107142857,-0.732199535691568)(1.48563058035714,-0.733257309117709)(1.48577008928571,-0.734316546868232)(1.48590959821429,-0.7353772471531)(1.48604910714286,-0.736439408179416)(1.48618861607143,-0.737503028151451)(1.486328125,-0.73856810527063)(1.48646763392857,-0.73963463773554)(1.48660714285714,-0.740702623741947)(1.48674665178571,-0.741772061482772)(1.48688616071429,-0.742842949148129)(1.48702566964286,-0.743915284925289)(1.48716517857143,-0.744989066998726)(1.4873046875,-0.746064293550081)(1.48744419642857,-0.747140962758192)(1.48758370535714,-0.748219072799093)(1.48772321428571,-0.749298621846004)(1.48786272321429,-0.750379608069356)(1.48800223214286,-0.751462029636768)(1.48814174107143,-0.752545884713083)(1.48828125,-0.753631171460339)(1.48842075892857,-0.754717888037791)(1.48856026785714,-0.755806032601921)(1.48869977678571,-0.756895603306417)(1.48883928571429,-0.757986598302207)(1.48897879464286,-0.759079015737429)(1.48911830357143,-0.760172853757471)(1.4892578125,-0.761268110504942)(1.48939732142857,-0.762364784119694)(1.48953683035714,-0.763462872738827)(1.48967633928571,-0.764562374496678)(1.48981584821429,-0.765663287524843)(1.48995535714286,-0.766765609952161)(1.49009486607143,-0.767869339904739)(1.490234375,-0.768974475505936)(1.49037388392857,-0.770081014876374)(1.49051339285714,-0.771188956133954)(1.49065290178571,-0.772298297393836)(1.49079241071429,-0.773409036768467)(1.49093191964286,-0.774521172367559)(1.49107142857143,-0.775634702298122)(1.4912109375,-0.776749624664439)(1.49135044642857,-0.777865937568087)(1.49148995535714,-0.778983639107944)(1.49162946428571,-0.780102727380174)(1.49176897321429,-0.781223200478253)(1.49190848214286,-0.782345056492949)(1.49204799107143,-0.783468293512353)(1.4921875,-0.784592909621857)(1.49232700892857,-0.785718902904169)(1.49246651785714,-0.786846271439328)(1.49260602678571,-0.787975013304681)(1.49274553571429,-0.789105126574919)(1.49288504464286,-0.790236609322045)(1.49302455357143,-0.791369459615414)(1.4931640625,-0.792503675521709)(1.49330357142857,-0.793639255104954)(1.49344308035714,-0.794776196426529)(1.49358258928571,-0.795914497545152)(1.49372209821429,-0.797054156516906)(1.49386160714286,-0.798195171395219)(1.49400111607143,-0.799337540230892)(1.494140625,-0.800481261072082)(1.49428013392857,-0.801626331964314)(1.49441964285714,-0.802772750950496)(1.49455915178571,-0.8039205160709)(1.49469866071429,-0.805069625363189)(1.49483816964286,-0.806220076862396)(1.49497767857143,-0.80737186860096)(1.4951171875,-0.808524998608695)(1.49525669642857,-0.809679464912815)(1.49539620535714,-0.810835265537941)(1.49553571428571,-0.811992398506086)(1.49567522321429,-0.813150861836681)(1.49581473214286,-0.814310653546554)(1.49595424107143,-0.815471771649963)(1.49609375,-0.816634214158572)(1.49623325892857,-0.817797979081471)(1.49637276785714,-0.818963064425183)(1.49651227678571,-0.820129468193649)(1.49665178571429,-0.821297188388257)(1.49679129464286,-0.82246622300782)(1.49693080357143,-0.823636570048606)(1.4970703125,-0.824808227504317)(1.49720982142857,-0.825981193366108)(1.49734933035714,-0.827155465622593)(1.49748883928571,-0.828331042259836)(1.49762834821429,-0.82950792126137)(1.49776785714286,-0.830686100608182)(1.49790736607143,-0.831865578278743)(1.498046875,-0.833046352248984)(1.49818638392857,-0.834228420492316)(1.49832589285714,-0.835411780979639)(1.49846540178571,-0.836596431679326)(1.49860491071429,-0.837782370557253)(1.49874441964286,-0.838969595576771)(1.49888392857143,-0.840158104698747)(1.4990234375,-0.841347895881534)(1.49916294642857,-0.842538967080994)(1.49930245535714,-0.843731316250503)(1.49944196428571,-0.844924941340941)(1.49958147321429,-0.846119840300716)(1.49972098214286,-0.847316011075743)(1.49986049107143,-0.848513451609475)(1.5,-0.849712159842883)(1.50013950892857,-0.850912133714474)(1.50027901785714,-0.852113371160298)(1.50041852678571,-0.853315870113934)(1.50055803571429,-0.854519628506522)(1.50069754464286,-0.85572464426673)(1.50083705357143,-0.8569309153208)(1.5009765625,-0.858138439592515)(1.50111607142857,-0.859347215003223)(1.50125558035714,-0.860557239471844)(1.50139508928571,-0.861768510914857)(1.50153459821429,-0.862981027246325)(1.50167410714286,-0.864194786377873)(1.50181361607143,-0.865409786218725)(1.501953125,-0.866626024675676)(1.50209263392857,-0.867843499653113)(1.50223214285714,-0.869062209053026)(1.50237165178571,-0.870282150774988)(1.50251116071429,-0.87150332271619)(1.50265066964286,-0.872725722771409)(1.50279017857143,-0.873949348833053)(1.5029296875,-0.875174198791126)(1.50306919642857,-0.876400270533257)(1.50320870535714,-0.877627561944701)(1.50334821428571,-0.878856070908332)(1.50348772321429,-0.880085795304663)(1.50362723214286,-0.881316733011829)(1.50376674107143,-0.882548881905619)(1.50390625,-0.883782239859451)(1.50404575892857,-0.885016804744393)(1.50418526785714,-0.886252574429172)(1.50432477678571,-0.88748954678016)(1.50446428571429,-0.888727719661399)(1.50460379464286,-0.889967090934579)(1.50474330357143,-0.891207658459076)(1.5048828125,-0.892449420091923)(1.50502232142857,-0.893692373687832)(1.50516183035714,-0.894936517099205)(1.50530133928571,-0.896181848176113)(1.50544084821429,-0.897428364766333)(1.50558035714286,-0.898676064715314)(1.50571986607143,-0.899924945866222)(1.505859375,-0.901175006059911)(1.50599888392857,-0.90242624313494)(1.50613839285714,-0.903678654927589)(1.50627790178571,-0.904932239271839)(1.50641741071429,-0.9061869939994)(1.50655691964286,-0.90744291693969)(1.50669642857143,-0.908700005919872)(1.5068359375,-0.909958258764821)(1.50697544642857,-0.911217673297155)(1.50711495535714,-0.912478247337237)(1.50725446428571,-0.91373997870316)(1.50739397321429,-0.915002865210782)(1.50753348214286,-0.916266904673691)(1.50767299107143,-0.917532094903252)(1.5078125,-0.918798433708577)(1.50795200892857,-0.920065918896543)(1.50809151785714,-0.921334548271807)(1.50823102678571,-0.922604319636783)(1.50837053571429,-0.923875230791681)(1.50851004464286,-0.925147279534472)(1.50864955357143,-0.926420463660933)(1.5087890625,-0.927694780964616)(1.50892857142857,-0.92897022923687)(1.50906808035714,-0.930246806266854)(1.50920758928571,-0.931524509841515)(1.50934709821429,-0.932803337745622)(1.50948660714286,-0.93408328776174)(1.50962611607143,-0.935364357670268)(1.509765625,-0.936646545249408)(1.50990513392857,-0.937929848275194)(1.51004464285714,-0.939214264521496)(1.51018415178571,-0.940499791760003)(1.51032366071429,-0.94178642776026)(1.51046316964286,-0.943074170289632)(1.51060267857143,-0.944363017113352)(1.5107421875,-0.945652965994488)(1.51088169642857,-0.946944014693967)(1.51102120535714,-0.948236160970584)(1.51116071428571,-0.949529402580984)(1.51130022321429,-0.950823737279696)(1.51143973214286,-0.952119162819103)(1.51157924107143,-0.953415676949484)(1.51171875,-0.954713277418984)(1.51185825892857,-0.956011961973639)(1.51199776785714,-0.957311728357381)(1.51213727678571,-0.958612574312025)(1.51227678571429,-0.9599144975773)(1.51241629464286,-0.96121749589082)(1.51255580357143,-0.962521566988124)(1.5126953125,-0.96382670860265)(1.51283482142857,-0.965132918465757)(1.51297433035714,-0.96644019430673)(1.51311383928571,-0.967748533852769)(1.51325334821429,-0.969057934829018)(1.51339285714286,-0.970368394958538)(1.51353236607143,-0.971679911962345)(1.513671875,-0.972992483559385)(1.51381138392857,-0.974306107466556)(1.51395089285714,-0.975620781398714)(1.51409040178571,-0.97693650306866)(1.51422991071429,-0.978253270187171)(1.51436941964286,-0.979571080462969)(1.51450892857143,-0.980889931602767)(1.5146484375,-0.982209821311237)(1.51478794642857,-0.983530747291032)(1.51492745535714,-0.984852707242797)(1.51506696428571,-0.986175698865153)(1.51520647321429,-0.987499719854727)(1.51534598214286,-0.988824767906124)(1.51548549107143,-0.99015084071197)(1.515625,-0.99147793596288)(1.51576450892857,-0.992806051347485)(1.51590401785714,-0.994135184552437)(1.51604352678571,-0.995465333262395)(1.51618303571429,-0.996796495160057)(1.51632254464286,-0.99812866792613)(1.51646205357143,-0.999461849239373)(1.5166015625,-1.00079603677657)(1.51674107142857,-1.00213122821254)(1.51688058035714,-1.00346742122017)(1.51702008928571,-1.00480461347037)(1.51715959821429,-1.00614280263214)(1.51729910714286,-1.00748198637251)(1.51743861607143,-1.00882216235658)(1.517578125,-1.01016332824753)(1.51771763392857,-1.0115054817066)(1.51785714285714,-1.01284862039313)(1.51799665178571,-1.01419274196451)(1.51813616071429,-1.01553784407624)(1.51827566964286,-1.0168839243819)(1.51841517857143,-1.01823098053318)(1.5185546875,-1.01957901017987)(1.51869419642857,-1.02092801096983)(1.51883370535714,-1.02227798054909)(1.51897321428571,-1.02362891656174)(1.51911272321429,-1.02498081665002)(1.51925223214286,-1.02633367845428)(1.51939174107143,-1.02768749961301)(1.51953125,-1.02904227776282)(1.51967075892857,-1.03039801053846)(1.51981026785714,-1.03175469557284)(1.51994977678571,-1.03311233049699)(1.52008928571429,-1.03447091294011)(1.52022879464286,-1.03583044052954)(1.52036830357143,-1.0371909108908)(1.5205078125,-1.03855232164756)(1.52064732142857,-1.03991467042166)(1.52078683035714,-1.04127795483312)(1.52092633928571,-1.04264217250014)(1.52106584821429,-1.04400732103912)(1.52120535714286,-1.04537339806459)(1.52134486607143,-1.04674040118935)(1.521484375,-1.04810832802434)(1.52162388392857,-1.04947717617873)(1.52176339285714,-1.05084694325988)(1.52190290178571,-1.05221762687338)(1.52204241071429,-1.05358922462303)(1.52218191964286,-1.05496173411084)(1.52232142857143,-1.05633515293706)(1.5224609375,-1.05770947870015)(1.52260044642857,-1.05908470899683)(1.52273995535714,-1.06046084142205)(1.52287946428571,-1.06183787356899)(1.52301897321429,-1.06321580302911)(1.52315848214286,-1.06459462739208)(1.52329799107143,-1.06597434424587)(1.5234375,-1.0673549511767)(1.52357700892857,-1.06873644576904)(1.52371651785714,-1.07011882560565)(1.52385602678571,-1.07150208826756)(1.52399553571429,-1.07288623133409)(1.52413504464286,-1.07427125238284)(1.52427455357143,-1.07565714898971)(1.5244140625,-1.07704391872887)(1.52455357142857,-1.07843155917283)(1.52469308035714,-1.07982006789237)(1.52483258928571,-1.08120944245659)(1.52497209821429,-1.08259968043293)(1.52511160714286,-1.0839907793871)(1.52525111607143,-1.08538273688319)(1.525390625,-1.08677555048359)(1.52553013392857,-1.088169217749)(1.52566964285714,-1.08956373623852)(1.52580915178571,-1.09095910350952)(1.52594866071429,-1.09235531711779)(1.52608816964286,-1.09375237461742)(1.52622767857143,-1.09515027356088)(1.5263671875,-1.096549011499)(1.52650669642857,-1.09794858598098)(1.52664620535714,-1.09934899455437)(1.52678571428571,-1.10075023476513)(1.52692522321429,-1.10215230415759)(1.52706473214286,-1.10355520027444)(1.52720424107143,-1.10495892065681)(1.52734375,-1.10636346284418)(1.52748325892857,-1.10776882437444)(1.52762276785714,-1.10917500278392)(1.52776227678571,-1.11058199560732)(1.52790178571429,-1.11198980037777)(1.52804129464286,-1.11339841462682)(1.52818080357143,-1.11480783588444)(1.5283203125,-1.11621806167904)(1.52845982142857,-1.11762908953745)(1.52859933035714,-1.11904091698495)(1.52873883928571,-1.12045354154526)(1.52887834821429,-1.12186696074055)(1.52901785714286,-1.12328117209144)(1.52915736607143,-1.12469617311701)(1.529296875,-1.1261119613348)(1.52943638392857,-1.12752853426082)(1.52957589285714,-1.12894588940957)(1.52971540178571,-1.13036402429399)(1.52985491071429,-1.13178293642553)(1.52999441964286,-1.13320262331413)(1.53013392857143,-1.13462308246821)(1.5302734375,-1.13604431139468)(1.53041294642857,-1.13746630759897)(1.53055245535714,-1.13888906858501)(1.53069196428571,-1.14031259185523)(1.53083147321429,-1.14173687491061)(1.53097098214286,-1.1431619152506)(1.53111049107143,-1.14458771037322)(1.53125,-1.146014257775)(1.53138950892857,-1.14744155495101)(1.53152901785714,-1.14886959939487)(1.53166852678571,-1.15029838859872)(1.53180803571429,-1.15172792005329)(1.53194754464286,-1.15315819124782)(1.53208705357143,-1.15458919967015)(1.5322265625,-1.15602094280666)(1.53236607142857,-1.15745341814231)(1.53250558035714,-1.15888662316064)(1.53264508928571,-1.16032055534375)(1.53278459821429,-1.16175521217235)(1.53292410714286,-1.16319059112571)(1.53306361607143,-1.16462668968174)(1.533203125,-1.16606350531691)(1.53334263392857,-1.16750103550629)(1.53348214285714,-1.16893927772359)(1.53362165178571,-1.17037822944112)(1.53376116071429,-1.17181788812981)(1.53390066964286,-1.1732582512592)(1.53404017857143,-1.17469931629749)(1.5341796875,-1.17614108071148)(1.53431919642857,-1.17758354196663)(1.53445870535714,-1.17902669752703)(1.53459821428571,-1.18047054485544)(1.53473772321429,-1.18191508141326)(1.53487723214286,-1.18336030466054)(1.53501674107143,-1.18480621205601)(1.53515625,-1.18625280105705)(1.53529575892857,-1.18770006911973)(1.53543526785714,-1.1891480136988)(1.53557477678571,-1.19059663224767)(1.53571428571429,-1.19204592221848)(1.53585379464286,-1.19349588106201)(1.53599330357143,-1.19494650622778)(1.5361328125,-1.196397795164)(1.53627232142857,-1.19784974531757)(1.53641183035714,-1.19930235413415)(1.53655133928571,-1.20075561905806)(1.53669084821429,-1.20220953753239)(1.53683035714286,-1.20366410699892)(1.53696986607143,-1.2051193248982)(1.537109375,-1.2065751886695)(1.53724888392857,-1.20803169575082)(1.53738839285714,-1.20948884357893)(1.53752790178571,-1.21094662958933)(1.53766741071429,-1.21240505121632)(1.53780691964286,-1.21386410589291)(1.53794642857143,-1.21532379105091)(1.5380859375,-1.2167841041209)(1.53822544642857,-1.21824504253223)(1.53836495535714,-1.21970660371304)(1.53850446428571,-1.22116878509024)(1.53864397321429,-1.22263158408957)(1.53878348214286,-1.22409499813552)(1.53892299107143,-1.22555902465143)(1.5390625,-1.22702366105942)(1.53920200892857,-1.22848890478042)(1.53934151785714,-1.2299547532342)(1.53948102678571,-1.23142120383933)(1.53962053571429,-1.23288825401322)(1.53976004464286,-1.23435590117212)(1.53989955357143,-1.2358241427311)(1.5400390625,-1.23729297610408)(1.54017857142857,-1.23876239870384)(1.54031808035714,-1.240232407942)(1.54045758928571,-1.24170300122904)(1.54059709821429,-1.24317417597431)(1.54073660714286,-1.24464592958602)(1.54087611607143,-1.24611825947126)(1.541015625,-1.24759116303599)(1.54115513392857,-1.24906463768506)(1.54129464285714,-1.25053868082222)(1.54143415178571,-1.25201328985008)(1.54157366071429,-1.25348846217019)(1.54171316964286,-1.25496419518296)(1.54185267857143,-1.25644048628776)(1.5419921875,-1.25791733288283)(1.54213169642857,-1.25939473236534)(1.54227120535714,-1.2608726821314)(1.54241071428571,-1.26235117957603)(1.54255022321429,-1.2638302220932)(1.54268973214286,-1.2653098070758)(1.54282924107143,-1.26678993191568)(1.54296875,-1.26827059400364)(1.54310825892857,-1.26975179072942)(1.54324776785714,-1.27123351948173)(1.54338727678571,-1.27271577764825)(1.54352678571429,-1.27419856261562)(1.54366629464286,-1.27568187176944)(1.54380580357143,-1.27716570249434)(1.5439453125,-1.27865005217387)(1.54408482142857,-1.28013491819061)(1.54422433035714,-1.28162029792614)(1.54436383928571,-1.28310618876102)(1.54450334821429,-1.28459258807482)(1.54464285714286,-1.28607949324611)(1.54478236607143,-1.28756690165251)(1.544921875,-1.28905481067063)(1.54506138392857,-1.29054321767611)(1.54520089285714,-1.29203212004363)(1.54534040178571,-1.2935215151469)(1.54547991071429,-1.29501140035867)(1.54561941964286,-1.29650177305074)(1.54575892857143,-1.29799263059396)(1.5458984375,-1.29948397035824)(1.54603794642857,-1.30097578971253)(1.54617745535714,-1.30246808602488)(1.54631696428571,-1.30396085666239)(1.54645647321429,-1.30545409899124)(1.54659598214286,-1.3069478103767)(1.54673549107143,-1.30844198818312)(1.546875,-1.30993662977393)(1.54701450892857,-1.31143173251169)(1.54715401785714,-1.31292729375804)(1.54729352678571,-1.31442331087372)(1.54743303571429,-1.3159197812186)(1.54757254464286,-1.31741670215166)(1.54771205357143,-1.31891407103101)(1.5478515625,-1.32041188521388)(1.54799107142857,-1.32191014205663)(1.54813058035714,-1.32340883891477)(1.54827008928571,-1.32490797314295)(1.54840959821429,-1.32640754209496)(1.54854910714286,-1.32790754312376)(1.54868861607143,-1.32940797358145)(1.548828125,-1.33090883081932)(1.54896763392857,-1.33241011218779)(1.54910714285714,-1.3339118150365)(1.54924665178571,-1.33541393671423)(1.54938616071429,-1.33691647456898)(1.54952566964286,-1.3384194259479)(1.54966517857143,-1.33992278819739)(1.5498046875,-1.34142655866299)(1.54994419642857,-1.34293073468948)(1.55008370535714,-1.34443531362084)(1.55022321428571,-1.34594029280029)(1.55036272321429,-1.34744566957023)(1.55050223214286,-1.34895144127232)(1.55064174107143,-1.35045760524743)(1.55078125,-1.35196415883568)(1.55092075892857,-1.35347109937642)(1.55106026785714,-1.35497842420827)(1.55119977678571,-1.35648613066906)(1.55133928571429,-1.35799421609593)(1.55147879464286,-1.35950267782524)(1.55161830357143,-1.36101151319264)(1.5517578125,-1.36252071953304)(1.55189732142857,-1.36403029418063)(1.55203683035714,-1.3655402344689)(1.55217633928571,-1.3670505377306)(1.55231584821429,-1.36856120129781)(1.55245535714286,-1.37007222250186)(1.55259486607143,-1.37158359867344)(1.552734375,-1.3730953271425)(1.55287388392857,-1.37460740523833)(1.55301339285714,-1.37611983028954)(1.55315290178571,-1.37763259962406)(1.55329241071429,-1.37914571056914)(1.55343191964286,-1.38065916045138)(1.55357142857143,-1.38217294659672)(1.5537109375,-1.38368706633043)(1.55385044642857,-1.38520151697714)(1.55398995535714,-1.38671629586085)(1.55412946428571,-1.3882314003049)(1.55426897321429,-1.38974682763201)(1.55440848214286,-1.39126257516425)(1.55454799107143,-1.39277864022311)(1.5546875,-1.39429502012941)(1.55482700892857,-1.39581171220338)(1.55496651785714,-1.39732871376467)(1.55510602678571,-1.39884602213228)(1.55524553571429,-1.40036363462465)(1.55538504464286,-1.40188154855961)(1.55552455357143,-1.40339976125439)(1.5556640625,-1.40491827002568)(1.55580357142857,-1.40643707218955)(1.55594308035714,-1.40795616506152)(1.55608258928571,-1.40947554595655)(1.55622209821429,-1.41099521218903)(1.55636160714286,-1.41251516107279)(1.55650111607143,-1.41403538992112)(1.556640625,-1.41555589604676)(1.55678013392857,-1.41707667676191)(1.55691964285714,-1.41859772937824)(1.55705915178571,-1.42011905120688)(1.55719866071429,-1.42164063955846)(1.55733816964286,-1.42316249174306)(1.55747767857143,-1.42468460507027)(1.5576171875,-1.42620697684915)(1.55775669642857,-1.42772960438829)(1.55789620535714,-1.42925248499575)(1.55803571428571,-1.43077561597911)(1.55817522321429,-1.43229899464547)(1.55831473214286,-1.43382261830142)(1.55845424107143,-1.43534648425313)(1.55859375,-1.43687058980623)(1.55873325892857,-1.43839493226593)(1.55887276785714,-1.43991950893696)(1.55901227678571,-1.44144431712361)(1.55915178571429,-1.44296935412969)(1.55929129464286,-1.4444946172586)(1.55943080357143,-1.44602010381327)(1.5595703125,-1.44754581109622)(1.55970982142857,-1.44907173640951)(1.55984933035714,-1.45059787705481)(1.55998883928571,-1.45212423033336)(1.56012834821429,-1.45365079354596)(1.56026785714286,-1.45517756399304)(1.56040736607143,-1.45670453897461)(1.560546875,-1.45823171579027)(1.56068638392857,-1.45975909173925)(1.56082589285714,-1.46128666412037)(1.56096540178571,-1.46281443023207)(1.56110491071429,-1.46434238737245)(1.56124441964286,-1.46587053283919)(1.56138392857143,-1.46739886392963)(1.5615234375,-1.46892737794074)(1.56166294642857,-1.47045607216912)(1.56180245535714,-1.47198494391107)(1.56194196428571,-1.47351399046247)(1.56208147321429,-1.47504320911893)(1.56222098214286,-1.47657259717568)(1.56236049107143,-1.47810215192764)(1.5625,-1.4796318706694)(1.56263950892857,-1.48116175069522)(1.56277901785714,-1.48269178929908)(1.56291852678571,-1.4842219837746)(1.56305803571429,-1.48575233141515)(1.56319754464286,-1.48728282951376)(1.56333705357143,-1.4888134753632)(1.5634765625,-1.49034426625592)(1.56361607142857,-1.49187519948411)(1.56375558035714,-1.49340627233968)(1.56389508928571,-1.49493748211427)(1.56403459821429,-1.49646882609924)(1.56417410714286,-1.49800030158571)(1.56431361607143,-1.49953190586452)(1.564453125,-1.50106363622627)(1.56459263392857,-1.50259548996132)(1.56473214285714,-1.5041274643598)(1.56487165178571,-1.50565955671156)(1.56501116071429,-1.50719176430628)(1.56515066964286,-1.50872408443338)(1.56529017857143,-1.51025651438206)(1.5654296875,-1.51178905144132)(1.56556919642857,-1.51332169289994)(1.56570870535714,-1.51485443604651)(1.56584821428571,-1.51638727816941)(1.56598772321429,-1.51792021655684)(1.56612723214286,-1.51945324849679)(1.56626674107143,-1.5209863712771)(1.56640625,-1.5225195821854)(1.56654575892857,-1.52405287850917)(1.56668526785714,-1.52558625753573)(1.56682477678571,-1.52711971655221)(1.56696428571429,-1.52865325284561)(1.56710379464286,-1.53018686370277)(1.56724330357143,-1.53172054641039)(1.5673828125,-1.53325429825502)(1.56752232142857,-1.53478811652307)(1.56766183035714,-1.53632199850086)(1.56780133928571,-1.53785594147453)(1.56794084821429,-1.53938994273016)(1.56808035714286,-1.54092399955366)(1.56821986607143,-1.54245810923087)(1.568359375,-1.54399226904751)(1.56849888392857,-1.54552647628921)(1.56863839285714,-1.54706072824151)(1.56877790178571,-1.54859502218984)(1.56891741071429,-1.5501293554196)(1.56905691964286,-1.55166372521604)(1.56919642857143,-1.55319812886441)(1.5693359375,-1.55473256364985)(1.56947544642857,-1.55626702685745)(1.56961495535714,-1.55780151577226)(1.56975446428571,-1.55933602767925)(1.56989397321429,-1.56087055986337)(1.57003348214286,-1.56240510960952)(1.57017299107143,-1.56393967420258)(1.5703125,-1.56547425092738)(1.57045200892857,-1.56700883706873)(1.57059151785714,-1.56854342991145)(1.57073102678571,-1.57007802674029)(1.57087053571429,-1.57161262484006)(1.57101004464286,-1.57314722149552)(1.57114955357143,-1.57468181399145)(1.5712890625,-1.57621639961263)(1.57142857142857,-1.57775097564386)(1.57156808035714,-1.57928553936997)(1.57170758928571,-1.58082008807578)(1.57184709821429,-1.58235461904619)(1.57198660714286,-1.58388912956607)(1.57212611607143,-1.58542361692039)(1.572265625,-1.58695807839413)(1.57240513392857,-1.58849251127233)(1.57254464285714,-1.5900269128401)(1.57268415178571,-1.59156128038257)(1.57282366071429,-1.593095611185)(1.57296316964286,-1.59462990253264)(1.57310267857143,-1.59616415171091)(1.5732421875,-1.59769835600522)(1.57338169642857,-1.59923251270114)(1.57352120535714,-1.60076661908429)(1.57366071428571,-1.6023006724404)(1.57380022321429,-1.60383467005532)(1.57393973214286,-1.60536860921497)(1.57407924107143,-1.60690248720544)(1.57421875,-1.60843630131287)(1.57435825892857,-1.60997004882358)(1.57449776785714,-1.611503727024)(1.57463727678571,-1.61303733320068)(1.57477678571429,-1.61457086464035)(1.57491629464286,-1.61610431862984)(1.57505580357143,-1.61763769245616)(1.5751953125,-1.61917098340647)(1.57533482142857,-1.62070418876807)(1.57547433035714,-1.62223730582847)(1.57561383928571,-1.62377033187531)(1.57575334821429,-1.62530326419644)(1.57589285714286,-1.62683610007986)(1.57603236607143,-1.62836883681378)(1.576171875,-1.6299014716866)(1.57631138392857,-1.63143400198691)(1.57645089285714,-1.63296642500352)(1.57659040178571,-1.63449873802543)(1.57672991071429,-1.63603093834186)(1.57686941964286,-1.63756302324225)(1.57700892857143,-1.63909499001627)(1.5771484375,-1.64062683595382)(1.57728794642857,-1.64215855834502)(1.57742745535714,-1.64369015448026)(1.57756696428571,-1.64522162165013)(1.57770647321429,-1.64675295714552)(1.57784598214286,-1.64828415825755)(1.57798549107143,-1.6498152222776)(1.578125,-1.65134614649733)(1.57826450892857,-1.65287692820865)(1.57840401785714,-1.65440756470379)(1.57854352678571,-1.6559380532752)(1.57868303571429,-1.65746839121568)(1.57882254464286,-1.65899857581829)(1.57896205357143,-1.66052860437639)(1.5791015625,-1.66205847418365)(1.57924107142857,-1.66358818253404)(1.57938058035714,-1.66511772672186)(1.57952008928571,-1.66664710404172)(1.57965959821429,-1.66817631178855)(1.57979910714286,-1.66970534725762)(1.57993861607143,-1.67123420774454)(1.580078125,-1.67276289054523)(1.58021763392857,-1.67429139295599)(1.58035714285714,-1.67581971227347)(1.58049665178571,-1.67734784579464)(1.58063616071429,-1.67887579081687)(1.58077566964286,-1.68040354463789)(1.58091517857143,-1.68193110455578)(1.5810546875,-1.68345846786902)(1.58119419642857,-1.68498563187647)(1.58133370535714,-1.68651259387737)(1.58147321428571,-1.68803935117134)(1.58161272321429,-1.68956590105845)(1.58175223214286,-1.6910922408391)(1.58189174107143,-1.69261836781417)(1.58203125,-1.6941442792849)(1.58217075892857,-1.69566997255298)(1.58231026785714,-1.69719544492052)(1.58244977678571,-1.69872069369004)(1.58258928571429,-1.70024571616453)(1.58272879464286,-1.70177050964739)(1.58286830357143,-1.70329507144248)(1.5830078125,-1.7048193988541)(1.58314732142857,-1.70634348918702)(1.58328683035714,-1.70786733974647)(1.58342633928571,-1.70939094783814)(1.58356584821429,-1.7109143107682)(1.58370535714286,-1.71243742584327)(1.58384486607143,-1.7139602903705)(1.583984375,-1.71548290165748)(1.58412388392857,-1.71700525701233)(1.58426339285714,-1.71852735374364)(1.58440290178571,-1.72004918916052)(1.58454241071429,-1.72157076057259)(1.58468191964286,-1.72309206528996)(1.58482142857143,-1.72461310062331)(1.5849609375,-1.72613386388378)(1.58510044642857,-1.72765435238308)(1.58523995535714,-1.72917456343345)(1.58537946428571,-1.73069449434766)(1.58551897321429,-1.73221414243904)(1.58565848214286,-1.73373350502146)(1.58579799107143,-1.73525257940933)(1.5859375,-1.73677136291766)(1.58607700892857,-1.73828985286199)(1.58621651785714,-1.73980804655846)(1.58635602678571,-1.74132594132375)(1.58649553571429,-1.74284353447516)(1.58663504464286,-1.74436082333055)(1.58677455357143,-1.7458778052084)(1.5869140625,-1.74739447742776)(1.58705357142857,-1.74891083730828)(1.58719308035714,-1.75042688217026)(1.58733258928571,-1.75194260933455)(1.58747209821429,-1.75345801612269)(1.58761160714286,-1.75497309985677)(1.58775111607143,-1.75648785785958)(1.587890625,-1.75800228745448)(1.58803013392857,-1.7595163859655)(1.58816964285714,-1.76103015071734)(1.58830915178571,-1.76254357903528)(1.58844866071429,-1.76405666824534)(1.58858816964286,-1.76556941567412)(1.58872767857143,-1.76708181864895)(1.5888671875,-1.76859387449779)(1.58900669642857,-1.77010558054928)(1.58914620535714,-1.77161693413276)(1.58928571428571,-1.77312793257824)(1.58942522321429,-1.77463857321643)(1.58956473214286,-1.77614885337873)(1.58970424107143,-1.77765877039724)(1.58984375,-1.77916832160478)(1.58998325892857,-1.78067750433486)(1.59012276785714,-1.78218631592173)(1.59026227678571,-1.78369475370035)(1.59040178571429,-1.7852028150064)(1.59054129464286,-1.78671049717632)(1.59068080357143,-1.78821779754726)(1.5908203125,-1.78972471345712)(1.59095982142857,-1.79123124224456)(1.59109933035714,-1.79273738124898)(1.59123883928571,-1.79424312781054)(1.59137834821429,-1.79574847927017)(1.59151785714286,-1.79725343296958)(1.59165736607143,-1.79875798625122)(1.591796875,-1.80026213645835)(1.59193638392857,-1.80176588093501)(1.59207589285714,-1.80326921702601)(1.59221540178571,-1.80477214207698)(1.59235491071429,-1.80627465343434)(1.59249441964286,-1.8077767484453)(1.59263392857143,-1.80927842445791)(1.5927734375,-1.81077967882101)(1.59291294642857,-1.81228050888428)(1.59305245535714,-1.81378091199821)(1.59319196428571,-1.81528088551412)(1.59333147321429,-1.81678042678419)(1.59347098214286,-1.81827953316141)(1.59361049107143,-1.81977820199964)(1.59375,-1.82127643065358)(1.59388950892857,-1.82277421647879)(1.59402901785714,-1.82427155683169)(1.59416852678571,-1.82576844906957)(1.59430803571429,-1.82726489055059)(1.59444754464286,-1.82876087863378)(1.59458705357143,-1.83025641067908)(1.5947265625,-1.83175148404728)(1.59486607142857,-1.83324609610008)(1.59500558035714,-1.83474024420009)(1.59514508928571,-1.83623392571079)(1.59528459821429,-1.83772713799662)(1.59542410714286,-1.83921987842288)(1.59556361607143,-1.84071214435583)(1.595703125,-1.84220393316261)(1.59584263392857,-1.84369524221133)(1.59598214285714,-1.84518606887101)(1.59612165178571,-1.84667641051162)(1.59626116071429,-1.84816626450407)(1.59640066964286,-1.84965562822021)(1.59654017857143,-1.85114449903286)(1.5966796875,-1.8526328743158)(1.59681919642857,-1.85412075144375)(1.59695870535714,-1.85560812779243)(1.59709821428571,-1.85709500073851)(1.59723772321429,-1.85858136765967)(1.59737723214286,-1.86006722593453)(1.59751674107143,-1.86155257294274)(1.59765625,-1.86303740606493)(1.59779575892857,-1.86452172268273)(1.59793526785714,-1.86600552017877)(1.59807477678571,-1.8674887959367)(1.59821428571429,-1.86897154734119)(1.59835379464286,-1.87045377177792)(1.59849330357143,-1.87193546663359)(1.5986328125,-1.87341662929594)(1.59877232142857,-1.87489725715374)(1.59891183035714,-1.87637734759682)(1.59905133928571,-1.87785689801603)(1.59919084821429,-1.87933590580329)(1.59933035714286,-1.88081436835156)(1.59946986607143,-1.88229228305488)(1.599609375,-1.88376964730834)(1.59974888392857,-1.8852464585081)(1.59988839285714,-1.88672271405142)(1.60002790178571,-1.88819841133661)(1.60016741071429,-1.88967354776308)(1.60030691964286,-1.89114812073135)(1.60044642857143,-1.89262212764301)(1.6005859375,-1.89409556590077)(1.60072544642857,-1.89556843290842)(1.60086495535714,-1.89704072607089)(1.60100446428571,-1.89851244279422)(1.60114397321429,-1.89998358048556)(1.60128348214286,-1.90145413655321)(1.60142299107143,-1.90292410840658)(1.6015625,-1.90439349345622)(1.60170200892857,-1.90586228911384)(1.60184151785714,-1.90733049279227)(1.60198102678571,-1.90879810190551)(1.60212053571429,-1.91026511386872)(1.60226004464286,-1.91173152609821)(1.60239955357143,-1.91319733601146)(1.6025390625,-1.91466254102713)(1.60267857142857,-1.91612713856506)(1.60281808035714,-1.91759112604625)(1.60295758928571,-1.9190545008929)(1.60309709821429,-1.92051726052842)(1.60323660714286,-1.9219794023774)(1.60337611607143,-1.92344092386563)(1.603515625,-1.9249018224201)(1.60365513392857,-1.92636209546904)(1.60379464285714,-1.92782174044188)(1.60393415178571,-1.92928075476926)(1.60407366071429,-1.93073913588308)(1.60421316964286,-1.93219688121644)(1.60435267857143,-1.9336539882037)(1.6044921875,-1.93511045428044)(1.60463169642857,-1.93656627688351)(1.60477120535714,-1.93802145345101)(1.60491071428571,-1.93947598142228)(1.60505022321429,-1.94092985823794)(1.60518973214286,-1.94238308133987)(1.60532924107143,-1.94383564817121)(1.60546875,-1.9452875561764)(1.60560825892857,-1.94673880280115)(1.60574776785714,-1.94818938549247)(1.60588727678571,-1.94963930169863)(1.60602678571429,-1.95108854886924)(1.60616629464286,-1.95253712445517)(1.60630580357143,-1.95398502590864)(1.6064453125,-1.95543225068315)(1.60658482142857,-1.95687879623352)(1.60672433035714,-1.9583246600159)(1.60686383928571,-1.95976983948778)(1.60700334821429,-1.96121433210795)(1.60714285714286,-1.96265813533657)(1.60728236607143,-1.96410124663511)(1.607421875,-1.96554366346642)(1.60756138392857,-1.96698538329467)(1.60770089285714,-1.96842640358541)(1.60784040178571,-1.96986672180554)(1.60797991071429,-1.97130633542333)(1.60811941964286,-1.97274524190842)(1.60825892857143,-1.97418343873182)(1.6083984375,-1.97562092336594)(1.60853794642857,-1.97705769328455)(1.60867745535714,-1.97849374596284)(1.60881696428571,-1.97992907887736)(1.60895647321429,-1.98136368950611)(1.60909598214286,-1.98279757532844)(1.60923549107143,-1.98423073382516)(1.609375,-1.98566316247846)(1.60951450892857,-1.98709485877197)(1.60965401785714,-1.98852582019073)(1.60979352678571,-1.98995604422124)(1.60993303571429,-1.99138552835139)(1.61007254464286,-1.99281427007056)(1.61021205357143,-1.99424226686953)(1.6103515625,-1.99566951624056)(1.61049107142857,-1.99709601567735)(1.61063058035714,-1.99852176267507)(1.61077008928571,-1.99994675473034)(1.61090959821429,-2.00137098934127)(1.61104910714286,-2.00279446400742)(1.61118861607143,-2.00421717622985)(1.611328125,-2.00563912351109)(1.61146763392857,-2.00706030335516)(1.61160714285714,-2.00848071326758)(1.61174665178571,-2.00990035075537)(1.61188616071429,-2.01131921332706)(1.61202566964286,-2.01273729849265)(1.61216517857143,-2.0141546037637)(1.6123046875,-2.01557112665327)(1.61244419642857,-2.01698686467593)(1.61258370535714,-2.01840181534779)(1.61272321428571,-2.0198159761865)(1.61286272321429,-2.02122934471123)(1.61300223214286,-2.02264191844271)(1.61314174107143,-2.0240536949032)(1.61328125,-2.02546467161652)(1.61342075892857,-2.02687484610804)(1.61356026785714,-2.02828421590471)(1.61369977678571,-2.02969277853502)(1.61383928571429,-2.03110053152906)(1.61397879464286,-2.03250747241846)(1.61411830357143,-2.03391359873647)(1.6142578125,-2.03531890801789)(1.61439732142857,-2.03672339779913)(1.61453683035714,-2.03812706561819)(1.61467633928571,-2.03952990901468)(1.61481584821429,-2.04093192552979)(1.61495535714286,-2.04233311270635)(1.61509486607143,-2.04373346808879)(1.615234375,-2.04513298922314)(1.61537388392857,-2.04653167365708)(1.61551339285714,-2.04792951893991)(1.61565290178571,-2.04932652262256)(1.61579241071429,-2.05072268225761)(1.61593191964286,-2.05211799539927)(1.61607142857143,-2.0535124596034)(1.6162109375,-2.05490607242751)(1.61635044642857,-2.05629883143076)(1.61648995535714,-2.05769073417401)(1.61662946428571,-2.05908177821974)(1.61676897321429,-2.06047196113213)(1.61690848214286,-2.06186128047701)(1.61704799107143,-2.06324973382192)(1.6171875,-2.06463731873607)(1.61732700892857,-2.06602403279036)(1.61746651785714,-2.06740987355739)(1.61760602678571,-2.06879483861146)(1.61774553571429,-2.07017892552856)(1.61788504464286,-2.07156213188641)(1.61802455357143,-2.07294445526443)(1.6181640625,-2.07432589324377)(1.61830357142857,-2.07570644340728)(1.61844308035714,-2.07708610333957)(1.61858258928571,-2.07846487062696)(1.61872209821429,-2.07984274285752)(1.61886160714286,-2.08121971762105)(1.61900111607143,-2.0825957925091)(1.619140625,-2.08397096511498)(1.61928013392857,-2.08534523303375)(1.61941964285714,-2.08671859386222)(1.61955915178571,-2.08809104519899)(1.61969866071429,-2.08946258464442)(1.61983816964286,-2.09083320980062)(1.61997767857143,-2.09220291827152)(1.6201171875,-2.09357170766281)(1.62025669642857,-2.09493957558197)(1.62039620535714,-2.09630651963828)(1.62053571428571,-2.09767253744281)(1.62067522321429,-2.09903762660845)(1.62081473214286,-2.10040178474987)(1.62095424107143,-2.10176500948358)(1.62109375,-2.10312729842789)(1.62123325892857,-2.10448864920293)(1.62137276785714,-2.10584905943066)(1.62151227678571,-2.10720852673489)(1.62165178571429,-2.10856704874123)(1.62179129464286,-2.10992462307715)(1.62193080357143,-2.11128124737196)(1.6220703125,-2.11263691925682)(1.62220982142857,-2.11399163636475)(1.62234933035714,-2.11534539633062)(1.62248883928571,-2.11669819679116)(1.62262834821429,-2.11805003538498)(1.62276785714286,-2.11940090975255)(1.62290736607143,-2.12075081753622)(1.623046875,-2.12209975638023)(1.62318638392857,-2.12344772393069)(1.62332589285714,-2.12479471783561)(1.62346540178571,-2.12614073574489)(1.62360491071429,-2.12748577531034)(1.62374441964286,-2.12882983418567)(1.62388392857143,-2.13017291002649)(1.6240234375,-2.13151500049032)(1.62416294642857,-2.13285610323663)(1.62430245535714,-2.13419621592677)(1.62444196428571,-2.13553533622405)(1.62458147321429,-2.13687346179368)(1.62472098214286,-2.13821059030285)(1.62486049107143,-2.13954671942064)(1.625,-2.14088184681811)(1.62513950892857,-2.14221597016826)(1.62527901785714,-2.14354908714604)(1.62541852678571,-2.14488119542836)(1.62555803571429,-2.1462122926941)(1.62569754464286,-2.1475423766241)(1.62583705357143,-2.14887144490118)(1.6259765625,-2.15019949521013)(1.62611607142857,-2.15152652523771)(1.62625558035714,-2.1528525326727)(1.62639508928571,-2.15417751520583)(1.62653459821429,-2.15550147052986)(1.62667410714286,-2.15682439633952)(1.62681361607143,-2.15814629033157)(1.626953125,-2.15946715020476)(1.62709263392857,-2.16078697365984)(1.62723214285714,-2.16210575839962)(1.62737165178571,-2.16342350212889)(1.62751116071429,-2.1647402025545)(1.62765066964286,-2.16605585738529)(1.62779017857143,-2.16737046433217)(1.6279296875,-2.16868402110809)(1.62806919642857,-2.169996525428)(1.62820870535714,-2.17130797500896)(1.62834821428571,-2.17261836757004)(1.62848772321429,-2.17392770083238)(1.62862723214286,-2.17523597251918)(1.62876674107143,-2.17654318035571)(1.62890625,-2.17784932206932)(1.62904575892857,-2.1791543953894)(1.62918526785714,-2.18045839804745)(1.62932477678571,-2.18176132777706)(1.62946428571429,-2.18306318231388)(1.62960379464286,-2.18436395939567)(1.62974330357143,-2.18566365676229)(1.6298828125,-2.18696227215569)(1.63002232142857,-2.18825980331994)(1.63016183035714,-2.1895562480012)(1.63030133928571,-2.19085160394776)(1.63044084821429,-2.19214586891003)(1.63058035714286,-2.19343904064053)(1.63071986607143,-2.19473111689393)(1.630859375,-2.196022095427)(1.63099888392857,-2.19731197399868)(1.63113839285714,-2.19860075037003)(1.63127790178571,-2.19988842230426)(1.63141741071429,-2.20117498756673)(1.63155691964286,-2.20246044392496)(1.63169642857143,-2.20374478914862)(1.6318359375,-2.20502802100955)(1.63197544642857,-2.20631013728173)(1.63211495535714,-2.20759113574136)(1.63225446428571,-2.20887101416677)(1.63239397321429,-2.2101497703385)(1.63253348214286,-2.21142740203927)(1.63267299107143,-2.21270390705396)(1.6328125,-2.21397928316968)(1.63295200892857,-2.21525352817571)(1.63309151785714,-2.21652663986355)(1.63323102678571,-2.2177986160269)(1.63337053571429,-2.21906945446167)(1.63351004464286,-2.22033915296598)(1.63364955357143,-2.22160770934018)(1.6337890625,-2.22287512138681)(1.63392857142857,-2.22414138691069)(1.63406808035714,-2.22540650371882)(1.63420758928571,-2.22667046962048)(1.63434709821429,-2.22793328242715)(1.63448660714286,-2.22919493995259)(1.63462611607143,-2.23045544001278)(1.634765625,-2.23171478042597)(1.63490513392857,-2.23297295901266)(1.63504464285714,-2.23422997359561)(1.63518415178571,-2.23548582199985)(1.63532366071429,-2.23674050205267)(1.63546316964286,-2.23799401158366)(1.63560267857143,-2.23924634842466)(1.6357421875,-2.24049751040979)(1.63588169642857,-2.24174749537549)(1.63602120535714,-2.24299630116045)(1.63616071428571,-2.24424392560569)(1.63630022321429,-2.24549036655451)(1.63643973214286,-2.24673562185251)(1.63657924107143,-2.24797968934762)(1.63671875,-2.24922256689005)(1.63685825892857,-2.25046425233234)(1.63699776785714,-2.25170474352937)(1.63713727678571,-2.25294403833832)(1.63727678571429,-2.25418213461871)(1.63741629464286,-2.25541903023237)(1.63755580357143,-2.25665472304351)(1.6376953125,-2.25788921091864)(1.63783482142857,-2.25912249172663)(1.63797433035714,-2.26035456333871)(1.63811383928571,-2.26158542362844)(1.63825334821429,-2.26281507047177)(1.63839285714286,-2.26404350174698)(1.63853236607143,-2.26527071533474)(1.638671875,-2.26649670911807)(1.63881138392857,-2.26772148098236)(1.63895089285714,-2.26894502881542)(1.63909040178571,-2.2701673505074)(1.63922991071429,-2.27138844395084)(1.63936941964286,-2.2726083070407)(1.63950892857143,-2.2738269376743)(1.6396484375,-2.27504433375138)(1.63978794642857,-2.27626049317407)(1.63992745535714,-2.27747541384692)(1.64006696428571,-2.27868909367688)(1.64020647321429,-2.27990153057333)(1.64034598214286,-2.28111272244805)(1.64048549107143,-2.28232266721525)(1.640625,-2.28353136279157)(1.64076450892857,-2.28473880709607)(1.64090401785714,-2.28594499805028)(1.64104352678571,-2.28714993357812)(1.64118303571429,-2.288353611606)(1.64132254464286,-2.28955603006274)(1.64146205357143,-2.29075718687963)(1.6416015625,-2.29195707999041)(1.64174107142857,-2.29315570733129)(1.64188058035714,-2.29435306684093)(1.64202008928571,-2.29554915646047)(1.64215959821429,-2.2967439741335)(1.64229910714286,-2.29793751780612)(1.64243861607143,-2.29912978542687)(1.642578125,-2.30032077494681)(1.64271763392857,-2.30151048431946)(1.64285714285714,-2.30269891150085)(1.64299665178571,-2.30388605444949)(1.64313616071429,-2.3050719111264)(1.64327566964286,-2.30625647949511)(1.64341517857143,-2.30743975752163)(1.6435546875,-2.30862174317452)(1.64369419642857,-2.30980243442481)(1.64383370535714,-2.3109818292461)(1.64397321428571,-2.31215992561447)(1.64411272321429,-2.31333672150856)(1.64425223214286,-2.31451221490951)(1.64439174107143,-2.31568640380102)(1.64453125,-2.31685928616931)(1.64467075892857,-2.31803086000316)(1.64481026785714,-2.31920112329389)(1.64494977678571,-2.32037007403537)(1.64508928571429,-2.32153771022402)(1.64522879464286,-2.32270402985882)(1.64536830357143,-2.32386903094133)(1.6455078125,-2.32503271147564)(1.64564732142857,-2.32619506946845)(1.64578683035714,-2.32735610292901)(1.64592633928571,-2.32851580986914)(1.64606584821429,-2.32967418830328)(1.64620535714286,-2.33083123624842)(1.64634486607143,-2.33198695172414)(1.646484375,-2.33314133275264)(1.64662388392857,-2.33429437735869)(1.64676339285714,-2.33544608356968)(1.64690290178571,-2.33659644941559)(1.64704241071429,-2.33774547292902)(1.64718191964286,-2.33889315214517)(1.64732142857143,-2.34003948510188)(1.6474609375,-2.34118446983959)(1.64760044642857,-2.34232810440135)(1.64773995535714,-2.34347038683288)(1.64787946428571,-2.3446113151825)(1.64801897321429,-2.34575088750117)(1.64815848214286,-2.3468891018425)(1.64829799107143,-2.34802595626274)(1.6484375,-2.34916144882076)(1.64857700892857,-2.35029557757812)(1.64871651785714,-2.35142834059902)(1.64885602678571,-2.35255973595031)(1.64899553571429,-2.35368976170151)(1.64913504464286,-2.3548184159248)(1.64927455357143,-2.35594569669503)(1.6494140625,-2.35707160208974)(1.64955357142857,-2.35819613018911)(1.64969308035714,-2.35931927907604)(1.64983258928571,-2.36044104683609)(1.64997209821429,-2.36156143155751)(1.65011160714286,-2.36268043133127)(1.65025111607143,-2.36379804425099)(1.650390625,-2.36491426841302)(1.65053013392857,-2.3660291019164)(1.65066964285714,-2.36714254286289)(1.65080915178571,-2.36825458935694)(1.65094866071429,-2.36936523950574)(1.65108816964286,-2.37047449141917)(1.65122767857143,-2.37158234320985)(1.6513671875,-2.37268879299313)(1.65150669642857,-2.37379383888705)(1.65164620535714,-2.37489747901244)(1.65178571428571,-2.37599971149283)(1.65192522321429,-2.37710053445449)(1.65206473214286,-2.37819994602643)(1.65220424107143,-2.37929794434044)(1.65234375,-2.38039452753103)(1.65248325892857,-2.38148969373545)(1.65262276785714,-2.38258344109376)(1.65276227678571,-2.38367576774873)(1.65290178571429,-2.38476667184593)(1.65304129464286,-2.38585615153367)(1.65318080357143,-2.38694420496306)(1.6533203125,-2.38803083028797)(1.65345982142857,-2.38911602566504)(1.65359933035714,-2.39019978925372)(1.65373883928571,-2.39128211921622)(1.65387834821429,-2.39236301371756)(1.65401785714286,-2.39344247092555)(1.65415736607143,-2.39452048901079)(1.654296875,-2.39559706614667)(1.65443638392857,-2.39667220050942)(1.65457589285714,-2.39774589027804)(1.65471540178571,-2.39881813363437)(1.65485491071429,-2.39988892876305)(1.65499441964286,-2.40095827385154)(1.65513392857143,-2.40202616709012)(1.6552734375,-2.40309260667191)(1.65541294642857,-2.40415759079284)(1.65555245535714,-2.40522111765169)(1.65569196428571,-2.40628318545007)(1.65583147321429,-2.40734379239241)(1.65597098214286,-2.40840293668602)(1.65611049107143,-2.40946061654104)(1.65625,-2.41051683017044)(1.65638950892857,-2.41157157579006)(1.65652901785714,-2.41262485161862)(1.65666852678571,-2.41367665587767)(1.65680803571429,-2.41472698679163)(1.65694754464286,-2.4157758425878)(1.65708705357143,-2.41682322149633)(1.6572265625,-2.41786912175026)(1.65736607142857,-2.41891354158551)(1.65750558035714,-2.41995647924087)(1.65764508928571,-2.42099793295802)(1.65778459821429,-2.42203790098153)(1.65792410714286,-2.42307638155886)(1.65806361607143,-2.42411337294037)(1.658203125,-2.42514887337931)(1.65834263392857,-2.42618288113183)(1.65848214285714,-2.427215394457)(1.65862165178571,-2.42824641161678)(1.65876116071429,-2.42927593087605)(1.65890066964286,-2.43030395050262)(1.65904017857143,-2.43133046876719)(1.6591796875,-2.43235548394341)(1.65931919642857,-2.43337899430782)(1.65945870535714,-2.43440099813994)(1.65959821428571,-2.43542149372217)(1.65973772321429,-2.43644047933989)(1.65987723214286,-2.43745795328138)(1.66001674107143,-2.43847391383788)(1.66015625,-2.43948835930359)(1.66029575892857,-2.44050128797562)(1.66043526785714,-2.44151269815409)(1.66057477678571,-2.44252258814201)(1.66071428571429,-2.44353095624539)(1.66085379464286,-2.4445378007732)(1.66099330357143,-2.44554312003736)(1.6611328125,-2.44654691235277)(1.66127232142857,-2.44754917603729)(1.66141183035714,-2.44854990941178)(1.66155133928571,-2.44954911080004)(1.66169084821429,-2.45054677852889)(1.66183035714286,-2.4515429109281)(1.66196986607143,-2.45253750633047)(1.662109375,-2.45353056307175)(1.66224888392857,-2.45452207949071)(1.66238839285714,-2.4555120539291)(1.66252790178571,-2.45650048473169)(1.66266741071429,-2.45748737024625)(1.66280691964286,-2.45847270882355)(1.66294642857143,-2.45945649881737)(1.6630859375,-2.4604387385845)(1.66322544642857,-2.46141942648477)(1.66336495535714,-2.46239856088101)(1.66350446428571,-2.46337614013908)(1.66364397321429,-2.46435216262786)(1.66378348214286,-2.46532662671928)(1.66392299107143,-2.46629953078827)(1.6640625,-2.46727087321283)(1.66420200892857,-2.46824065237397)(1.66434151785714,-2.46920886665577)(1.66448102678571,-2.47017551444533)(1.66462053571429,-2.47114059413282)(1.66476004464286,-2.47210410411146)(1.66489955357143,-2.4730660427775)(1.6650390625,-2.47402640853029)(1.66517857142857,-2.47498519977219)(1.66531808035714,-2.47594241490867)(1.66545758928571,-2.47689805234825)(1.66559709821429,-2.47785211050251)(1.66573660714286,-2.47880458778613)(1.66587611607143,-2.47975548261684)(1.666015625,-2.48070479341546)(1.66615513392857,-2.4816525186059)(1.66629464285714,-2.48259865661514)(1.66643415178571,-2.48354320587327)(1.66657366071429,-2.48448616481346)(1.66671316964286,-2.48542753187197)(1.66685267857143,-2.48636730548817)(1.6669921875,-2.48730548410452)(1.66713169642857,-2.4882420661666)(1.66727120535714,-2.48917705012307)(1.66741071428571,-2.49011043442573)(1.66755022321429,-2.49104221752949)(1.66768973214286,-2.49197239789235)(1.66782924107143,-2.49290097397546)(1.66796875,-2.49382794424307)(1.66810825892857,-2.49475330716256)(1.66824776785714,-2.49567706120446)(1.66838727678571,-2.49659920484241)(1.66852678571429,-2.49751973655318)(1.66866629464286,-2.49843865481669)(1.66880580357143,-2.49935595811599)(1.6689453125,-2.50027164493729)(1.66908482142857,-2.50118571376993)(1.66922433035714,-2.5020981631064)(1.66936383928571,-2.50300899144236)(1.66950334821429,-2.5039181972766)(1.66964285714286,-2.50482577911107)(1.66978236607143,-2.50573173545092)(1.669921875,-2.5066360648044)(1.67006138392857,-2.50753876568298)(1.67020089285714,-2.50843983660127)(1.67034040178571,-2.50933927607707)(1.67047991071429,-2.51023708263135)(1.67061941964286,-2.51113325478824)(1.67075892857143,-2.51202779107508)(1.6708984375,-2.51292069002237)(1.67103794642857,-2.51381195016381)(1.67117745535714,-2.51470157003629)(1.67131696428571,-2.51558954817988)(1.67145647321429,-2.51647588313787)(1.67159598214286,-2.51736057345672)(1.67173549107143,-2.51824361768611)(1.671875,-2.51912501437891)(1.67201450892857,-2.5200047620912)(1.67215401785714,-2.52088285938228)(1.67229352678571,-2.52175930481465)(1.67243303571429,-2.52263409695404)(1.67257254464286,-2.52350723436937)(1.67271205357143,-2.52437871563282)(1.6728515625,-2.52524853931975)(1.67299107142857,-2.52611670400878)(1.67313058035714,-2.52698320828175)(1.67327008928571,-2.52784805072371)(1.67340959821429,-2.52871122992299)(1.67354910714286,-2.5295727444711)(1.67368861607143,-2.53043259296284)(1.673828125,-2.53129077399622)(1.67396763392857,-2.53214728617251)(1.67410714285714,-2.53300212809624)(1.67424665178571,-2.53385529837515)(1.67438616071429,-2.53470679562028)(1.67452566964286,-2.5355566184459)(1.67466517857143,-2.53640476546954)(1.6748046875,-2.53725123531201)(1.67494419642857,-2.53809602659735)(1.67508370535714,-2.5389391379529)(1.67522321428571,-2.53978056800926)(1.67536272321429,-2.54062031540029)(1.67550223214286,-2.54145837876314)(1.67564174107143,-2.54229475673823)(1.67578125,-2.54312944796926)(1.67592075892857,-2.54396245110322)(1.67606026785714,-2.54479376479037)(1.67619977678571,-2.54562338768428)(1.67633928571429,-2.5464513184418)(1.67647879464286,-2.54727755572307)(1.67661830357143,-2.54810209819152)(1.6767578125,-2.5489249445139)(1.67689732142857,-2.54974609336024)(1.67703683035714,-2.55056554340388)(1.67717633928571,-2.55138329332148)(1.67731584821429,-2.55219934179299)(1.67745535714286,-2.55301368750167)(1.67759486607143,-2.55382632913412)(1.677734375,-2.55463726538023)(1.67787388392857,-2.55544649493322)(1.67801339285714,-2.55625401648962)(1.67815290178571,-2.55705982874932)(1.67829241071429,-2.55786393041548)(1.67843191964286,-2.55866632019465)(1.67857142857143,-2.55946699679666)(1.6787109375,-2.56026595893471)(1.67885044642857,-2.56106320532532)(1.67898995535714,-2.56185873468835)(1.67912946428571,-2.56265254574701)(1.67926897321429,-2.56344463722784)(1.67940848214286,-2.56423500786074)(1.67954799107143,-2.56502365637896)(1.6796875,-2.5658105815191)(1.67982700892857,-2.56659578202109)(1.67996651785714,-2.56737925662826)(1.68010602678571,-2.56816100408726)(1.68024553571429,-2.56894102314813)(1.68038504464286,-2.56971931256425)(1.68052455357143,-2.57049587109239)(1.6806640625,-2.57127069749268)(1.68080357142857,-2.5720437905286)(1.68094308035714,-2.57281514896704)(1.68108258928571,-2.57358477157823)(1.68122209821429,-2.57435265713582)(1.68136160714286,-2.57511880441679)(1.68150111607143,-2.57588321220155)(1.681640625,-2.57664587927388)(1.68178013392857,-2.57740680442092)(1.68191964285714,-2.57816598643324)(1.68205915178571,-2.57892342410478)(1.68219866071429,-2.57967911623289)(1.68233816964286,-2.58043306161831)(1.68247767857143,-2.58118525906516)(1.6826171875,-2.581935707381)(1.68275669642857,-2.58268440537677)(1.68289620535714,-2.58343135186681)(1.68303571428571,-2.58417654566889)(1.68317522321429,-2.58491998560419)(1.68331473214286,-2.58566167049728)(1.68345424107143,-2.58640159917616)(1.68359375,-2.58713977047226)(1.68373325892857,-2.58787618322042)(1.68387276785714,-2.58861083625889)(1.68401227678571,-2.58934372842935)(1.68415178571429,-2.59007485857694)(1.68429129464286,-2.59080422555019)(1.68443080357143,-2.59153182820107)(1.6845703125,-2.59225766538499)(1.68470982142857,-2.5929817359608)(1.68484933035714,-2.59370403879078)(1.68498883928571,-2.59442457274065)(1.68512834821429,-2.59514333667958)(1.68526785714286,-2.59586032948018)(1.68540736607143,-2.59657555001852)(1.685546875,-2.59728899717408)(1.68568638392857,-2.59800066982985)(1.68582589285714,-2.59871056687222)(1.68596540178571,-2.59941868719107)(1.68610491071429,-2.60012502967972)(1.68624441964286,-2.60082959323497)(1.68638392857143,-2.60153237675705)(1.6865234375,-2.60223337914969)(1.68666294642857,-2.60293259932005)(1.68680245535714,-2.6036300361788)(1.68694196428571,-2.60432568864005)(1.68708147321429,-2.6050195556214)(1.68722098214286,-2.6057116360439)(1.68736049107143,-2.60640192883211)(1.6875,-2.60709043291405)(1.68763950892857,-2.60777714722122)(1.68777901785714,-2.60846207068862)(1.68791852678571,-2.60914520225471)(1.68805803571429,-2.60982654086146)(1.68819754464286,-2.61050608545432)(1.68833705357143,-2.61118383498223)(1.6884765625,-2.61185978839764)(1.68861607142857,-2.61253394465645)(1.68875558035714,-2.61320630271813)(1.68889508928571,-2.61387686154558)(1.68903459821429,-2.61454562010524)(1.68917410714286,-2.61521257736705)(1.68931361607143,-2.61587773230445)(1.689453125,-2.61654108389439)(1.68959263392857,-2.61720263111732)(1.68973214285714,-2.61786237295721)(1.68987165178571,-2.61852030840156)(1.69001116071429,-2.61917643644136)(1.69015066964286,-2.61983075607113)(1.69029017857143,-2.6204832662889)(1.6904296875,-2.62113396609624)(1.69056919642857,-2.62178285449823)(1.69070870535714,-2.62242993050347)(1.69084821428571,-2.6230751931241)(1.69098772321429,-2.62371864137579)(1.69112723214286,-2.62436027427773)(1.69126674107143,-2.62500009085265)(1.69140625,-2.62563809012682)(1.69154575892857,-2.62627427113003)(1.69168526785714,-2.62690863289563)(1.69182477678571,-2.6275411744605)(1.69196428571429,-2.62817189486506)(1.69210379464286,-2.62880079315329)(1.69224330357143,-2.62942786837268)(1.6923828125,-2.63005311957431)(1.69252232142857,-2.63067654581279)(1.69266183035714,-2.63129814614628)(1.69280133928571,-2.63191791963651)(1.69294084821429,-2.63253586534874)(1.69308035714286,-2.6331519823518)(1.69321986607143,-2.63376626971809)(1.693359375,-2.63437872652356)(1.69349888392857,-2.63498935184772)(1.69363839285714,-2.63559814477365)(1.69377790178571,-2.63620510438799)(1.69391741071429,-2.63681022978097)(1.69405691964286,-2.63741352004635)(1.69419642857143,-2.6380149742815)(1.6943359375,-2.63861459158734)(1.69447544642857,-2.63921237106838)(1.69461495535714,-2.63980831183269)(1.69475446428571,-2.64040241299194)(1.69489397321429,-2.64099467366136)(1.69503348214286,-2.64158509295978)(1.69517299107143,-2.6421736700096)(1.6953125,-2.6427604039368)(1.69545200892857,-2.64334529387098)(1.69559151785714,-2.64392833894529)(1.69573102678571,-2.64450953829649)(1.69587053571429,-2.64508889106494)(1.69601004464286,-2.64566639639457)(1.69614955357143,-2.64624205343292)(1.6962890625,-2.64681586133114)(1.69642857142857,-2.64738781924395)(1.69656808035714,-2.64795792632969)(1.69670758928571,-2.6485261817503)(1.69684709821429,-2.64909258467132)(1.69698660714286,-2.6496571342619)(1.69712611607143,-2.6502198296948)(1.697265625,-2.65078067014637)(1.69740513392857,-2.65133965479659)(1.69754464285714,-2.65189678282905)(1.69768415178571,-2.65245205343094)(1.69782366071429,-2.65300546579309)(1.69796316964286,-2.65355701910991)(1.69810267857143,-2.65410671257946)(1.6982421875,-2.6546545454034)(1.69838169642857,-2.65520051678703)(1.69852120535714,-2.65574462593926)(1.69866071428571,-2.65628687207263)(1.69880022321429,-2.65682725440329)(1.69893973214286,-2.65736577215105)(1.69907924107143,-2.65790242453931)(1.69921875,-2.65843721079514)(1.69935825892857,-2.65897013014921)(1.69949776785714,-2.65950118183583)(1.69963727678571,-2.66003036509296)(1.69977678571429,-2.66055767916219)(1.69991629464286,-2.66108312328874)(1.70005580357143,-2.66160669672147)(1.7001953125,-2.66212839871289)(1.70033482142857,-2.66264822851915)(1.70047433035714,-2.66316618540004)(1.70061383928571,-2.66368226861899)(1.70075334821429,-2.6641964774431)(1.70089285714286,-2.66470881114308)(1.70103236607143,-2.66521926899332)(1.701171875,-2.66572785027186)(1.70131138392857,-2.66623455426037)(1.70145089285714,-2.6667393802442)(1.70159040178571,-2.66724232751234)(1.70172991071429,-2.66774339535744)(1.70186941964286,-2.6682425830758)(1.70200892857143,-2.66873988996739)(1.7021484375,-2.66923531533584)(1.70228794642857,-2.66972885848844)(1.70242745535714,-2.67022051873614)(1.70256696428571,-2.67071029539355)(1.70270647321429,-2.67119818777897)(1.70284598214286,-2.67168419521434)(1.70298549107143,-2.67216831702528)(1.703125,-2.67265055254107)(1.70326450892857,-2.67313090109469)(1.70340401785714,-2.67360936202276)(1.70354352678571,-2.67408593466558)(1.70368303571429,-2.67456061836715)(1.70382254464286,-2.67503341247511)(1.70396205357143,-2.67550431634081)(1.7041015625,-2.67597332931926)(1.70424107142857,-2.67644045076915)(1.70438058035714,-2.67690568005287)(1.70452008928571,-2.67736901653646)(1.70465959821429,-2.67783045958967)(1.70479910714286,-2.67829000858593)(1.70493861607143,-2.67874766290236)(1.705078125,-2.67920342191976)(1.70521763392857,-2.67965728502261)(1.70535714285714,-2.6801092515991)(1.70549665178571,-2.6805593210411)(1.70563616071429,-2.68100749274417)(1.70577566964286,-2.68145376610758)(1.70591517857143,-2.68189814053427)(1.7060546875,-2.68234061543089)(1.70619419642857,-2.68278119020779)(1.70633370535714,-2.68321986427901)(1.70647321428571,-2.68365663706229)(1.70661272321429,-2.68409150797908)(1.70675223214286,-2.68452447645453)(1.70689174107143,-2.68495554191747)(1.70703125,-2.68538470380046)(1.70717075892857,-2.68581196153976)(1.70731026785714,-2.68623731457533)(1.70744977678571,-2.68666076235083)(1.70758928571429,-2.68708230431365)(1.70772879464286,-2.68750193991487)(1.70786830357143,-2.68791966860929)(1.7080078125,-2.68833548985541)(1.70814732142857,-2.68874940311546)(1.70828683035714,-2.68916140785537)(1.70842633928571,-2.68957150354479)(1.70856584821429,-2.68997968965709)(1.70870535714286,-2.69038596566934)(1.70884486607143,-2.69079033106234)(1.708984375,-2.69119278532061)(1.70912388392857,-2.69159332793238)(1.70926339285714,-2.69199195838962)(1.70940290178571,-2.692388676188)(1.70954241071429,-2.69278348082692)(1.70968191964286,-2.6931763718095)(1.70982142857143,-2.69356734864261)(1.7099609375,-2.6939564108368)(1.71010044642857,-2.69434355790639)(1.71023995535714,-2.69472878936939)(1.71037946428571,-2.69511210474757)(1.71051897321429,-2.69549350356641)(1.71065848214286,-2.69587298535514)(1.71079799107143,-2.69625054964668)(1.7109375,-2.69662619597774)(1.71107700892857,-2.6969999238887)(1.71121651785714,-2.69737173292374)(1.71135602678571,-2.69774162263071)(1.71149553571429,-2.69810959256124)(1.71163504464286,-2.69847564227068)(1.71177455357143,-2.69883977131813)(1.7119140625,-2.6992019792664)(1.71205357142857,-2.69956226568207)(1.71219308035714,-2.69992063013544)(1.71233258928571,-2.70027707220056)(1.71247209821429,-2.70063159145521)(1.71261160714286,-2.70098418748094)(1.71275111607143,-2.701334859863)(1.712890625,-2.70168360819043)(1.71303013392857,-2.70203043205598)(1.71316964285714,-2.70237533105617)(1.71330915178571,-2.70271830479124)(1.71344866071429,-2.70305935286519)(1.71358816964286,-2.70339847488579)(1.71372767857143,-2.70373567046453)(1.7138671875,-2.70407093921664)(1.71400669642857,-2.70440428076114)(1.71414620535714,-2.70473569472077)(1.71428571428571,-2.70506518072203)(1.71442522321429,-2.70539273839518)(1.71456473214286,-2.70571836737421)(1.71470424107143,-2.7060420672969)(1.71484375,-2.70636383780474)(1.71498325892857,-2.70668367854302)(1.71512276785714,-2.70700158916075)(1.71526227678571,-2.70731756931072)(1.71540178571429,-2.70763161864947)(1.71554129464286,-2.70794373683729)(1.71568080357143,-2.70825392353824)(1.7158203125,-2.70856217842013)(1.71595982142857,-2.70886850115453)(1.71609933035714,-2.70917289141679)(1.71623883928571,-2.709475348886)(1.71637834821429,-2.70977587324501)(1.71651785714286,-2.71007446418045)(1.71665736607143,-2.71037112138269)(1.716796875,-2.71066584454589)(1.71693638392857,-2.71095863336795)(1.71707589285714,-2.71124948755055)(1.71721540178571,-2.71153840679913)(1.71735491071429,-2.7118253908229)(1.71749441964286,-2.71211043933483)(1.71763392857143,-2.71239355205166)(1.7177734375,-2.7126747286939)(1.71791294642857,-2.71295396898582)(1.71805245535714,-2.71323127265547)(1.71819196428571,-2.71350663943466)(1.71833147321429,-2.71378006905898)(1.71847098214286,-2.71405156126777)(1.71861049107143,-2.71432111580417)(1.71875,-2.71458873241507)(1.71888950892857,-2.71485441085114)(1.71902901785714,-2.71511815086681)(1.71916852678571,-2.7153799522203)(1.71930803571429,-2.7156398146736)(1.71944754464286,-2.71589773799246)(1.71958705357143,-2.71615372194641)(1.7197265625,-2.71640776630877)(1.71986607142857,-2.71665987085661)(1.72000558035714,-2.7169100353708)(1.72014508928571,-2.71715825963597)(1.72028459821429,-2.71740454344053)(1.72042410714286,-2.71764888657666)(1.72056361607143,-2.71789128884033)(1.720703125,-2.71813175003129)(1.72084263392857,-2.71837026995304)(1.72098214285714,-2.71860684841289)(1.72112165178571,-2.71884148522192)(1.72126116071429,-2.71907418019497)(1.72140066964286,-2.71930493315069)(1.72154017857143,-2.71953374391149)(1.7216796875,-2.71976061230355)(1.72181919642857,-2.71998553815687)(1.72195870535714,-2.72020852130518)(1.72209821428571,-2.72042956158603)(1.72223772321429,-2.72064865884074)(1.72237723214286,-2.7208658129144)(1.72251674107143,-2.72108102365591)(1.72265625,-2.7212942909179)(1.72279575892857,-2.72150561455685)(1.72293526785714,-2.72171499443297)(1.72307477678571,-2.72192243041028)(1.72321428571429,-2.72212792235656)(1.72335379464286,-2.72233147014341)(1.72349330357143,-2.72253307364619)(1.7236328125,-2.72273273274403)(1.72377232142857,-2.72293044731988)(1.72391183035714,-2.72312621726045)(1.72405133928571,-2.72332004245623)(1.72419084821429,-2.72351192280152)(1.72433035714286,-2.72370185819438)(1.72446986607143,-2.72388984853668)(1.724609375,-2.72407589373404)(1.72474888392857,-2.72425999369591)(1.72488839285714,-2.7244421483355)(1.72502790178571,-2.7246223575698)(1.72516741071429,-2.7248006213196)(1.72530691964286,-2.72497693950947)(1.72544642857143,-2.72515131206777)(1.7255859375,-2.72532373892666)(1.72572544642857,-2.72549422002205)(1.72586495535714,-2.72566275529368)(1.72600446428571,-2.72582934468504)(1.72614397321429,-2.72599398814343)(1.72628348214286,-2.72615668561994)(1.72642299107143,-2.72631743706942)(1.7265625,-2.72647624245055)(1.72670200892857,-2.72663310172575)(1.72684151785714,-2.72678801486127)(1.72698102678571,-2.72694098182712)(1.72712053571429,-2.72709200259711)(1.72726004464286,-2.72724107714883)(1.72739955357143,-2.72738820546367)(1.7275390625,-2.7275333875268)(1.72767857142857,-2.72767662332717)(1.72781808035714,-2.72781791285754)(1.72795758928571,-2.72795725611443)(1.72809709821429,-2.72809465309817)(1.72823660714286,-2.72823010381287)(1.72837611607143,-2.72836360826642)(1.728515625,-2.72849516647053)(1.72865513392857,-2.72862477844065)(1.72879464285714,-2.72875244419605)(1.72893415178571,-2.72887816375978)(1.72907366071429,-2.72900193715868)(1.72921316964286,-2.72912376442337)(1.72935267857143,-2.72924364558827)(1.7294921875,-2.72936158069158)(1.72963169642857,-2.72947756977529)(1.72977120535714,-2.72959161288517)(1.72991071428571,-2.72970371007079)(1.73005022321429,-2.72981386138549)(1.73018973214286,-2.72992206688642)(1.73032924107143,-2.7300283266345)(1.73046875,-2.73013264069445)(1.73060825892857,-2.73023500913475)(1.73074776785714,-2.7303354320277)(1.73088727678571,-2.73043390944937)(1.73102678571429,-2.73053044147962)(1.73116629464286,-2.73062502820209)(1.73130580357143,-2.73071766970421)(1.7314453125,-2.73080836607721)(1.73158482142857,-2.73089711741607)(1.73172433035714,-2.7309839238196)(1.73186383928571,-2.73106878539036)(1.73200334821429,-2.73115170223471)(1.73214285714286,-2.7312326744628)(1.73228236607143,-2.73131170218856)(1.732421875,-2.73138878552969)(1.73256138392857,-2.7314639246077)(1.73270089285714,-2.73153711954787)(1.73284040178571,-2.73160837047926)(1.73297991071429,-2.73167767753472)(1.73311941964286,-2.73174504085088)(1.73325892857143,-2.73181046056816)(1.7333984375,-2.73187393683076)(1.73353794642857,-2.73193546978665)(1.73367745535714,-2.73199505958761)(1.73381696428571,-2.73205270638916)(1.73395647321429,-2.73210841035064)(1.73409598214286,-2.73216217163517)(1.73423549107143,-2.73221399040961)(1.734375,-2.73226386684465)(1.73451450892857,-2.73231180111474)(1.73465401785714,-2.7323577933981)(1.73479352678571,-2.73240184387674)(1.73493303571429,-2.73244395273645)(1.73507254464286,-2.73248412016679)(1.73521205357143,-2.73252234636112)(1.7353515625,-2.73255863151656)(1.73549107142857,-2.73259297583401)(1.73563058035714,-2.73262537951814)(1.73577008928571,-2.73265584277742)(1.73590959821429,-2.73268436582408)(1.73604910714286,-2.73271094887412)(1.73618861607143,-2.73273559214733)(1.736328125,-2.73275829586728)(1.73646763392857,-2.73277906026129)(1.73660714285714,-2.73279788556047)(1.73674665178571,-2.73281477199971)(1.73688616071429,-2.73282971981767)(1.73702566964286,-2.73284272925677)(1.73716517857143,-2.73285380056322)(1.7373046875,-2.73286293398699)(1.73744419642857,-2.73287012978183)(1.73758370535714,-2.73287538820526)(1.73772321428571,-2.73287870951857)(1.73786272321429,-2.7328800939868)(1.73800223214286,-2.7328795418788)(1.73814174107143,-2.73287705346715)(1.73828125,-2.73287262902824)(1.73842075892857,-2.73286626884217)(1.73856026785714,-2.73285797319287)(1.73869977678571,-2.73284774236799)(1.73883928571429,-2.73283557665898)(1.73897879464286,-2.73282147636103)(1.73911830357143,-2.7328054417731)(1.7392578125,-2.73278747319794)(1.73939732142857,-2.73276757094202)(1.73953683035714,-2.73274573531562)(1.73967633928571,-2.73272196663275)(1.73981584821429,-2.73269626521119)(1.73995535714286,-2.73266863137249)(1.74009486607143,-2.73263906544195)(1.740234375,-2.73260756774865)(1.74037388392857,-2.7325741386254)(1.74051339285714,-2.73253877840879)(1.74065290178571,-2.73250148743916)(1.74079241071429,-2.73246226606062)(1.74093191964286,-2.73242111462102)(1.74107142857143,-2.73237803347198)(1.7412109375,-2.73233302296887)(1.74135044642857,-2.7322860834708)(1.74148995535714,-2.73223721534068)(1.74162946428571,-2.73218641894511)(1.74176897321429,-2.7321336946545)(1.74190848214286,-2.73207904284299)(1.74204799107143,-2.73202246388845)(1.7421875,-2.73196395817253)(1.74232700892857,-2.73190352608063)(1.74246651785714,-2.73184116800187)(1.74260602678571,-2.73177688432916)(1.74274553571429,-2.73171067545913)(1.74288504464286,-2.73164254179215)(1.74302455357143,-2.73157248373236)(1.7431640625,-2.73150050168764)(1.74330357142857,-2.7314265960696)(1.74344308035714,-2.73135076729361)(1.74358258928571,-2.73127301577877)(1.74372209821429,-2.73119334194793)(1.74386160714286,-2.73111174622769)(1.74400111607143,-2.73102822904836)(1.744140625,-2.73094279084402)(1.74428013392857,-2.73085543205249)(1.74441964285714,-2.7307661531153)(1.74455915178571,-2.73067495447773)(1.74469866071429,-2.73058183658882)(1.74483816964286,-2.73048679990132)(1.74497767857143,-2.73038984487171)(1.7451171875,-2.73029097196022)(1.74525669642857,-2.7301901816308)(1.74539620535714,-2.73008747435115)(1.74553571428571,-2.72998285059268)(1.74567522321429,-2.72987631083054)(1.74581473214286,-2.72976785554361)(1.74595424107143,-2.7296574852145)(1.74609375,-2.72954520032954)(1.74623325892857,-2.72943100137879)(1.74637276785714,-2.72931488885603)(1.74651227678571,-2.72919686325877)(1.74665178571429,-2.72907692508825)(1.74679129464286,-2.72895507484942)(1.74693080357143,-2.72883131305096)(1.7470703125,-2.72870564020527)(1.74720982142857,-2.72857805682845)(1.74734933035714,-2.72844856344035)(1.74748883928571,-2.72831716056452)(1.74762834821429,-2.72818384872823)(1.74776785714286,-2.72804862846246)(1.74790736607143,-2.7279115003019)(1.748046875,-2.72777246478498)(1.74818638392857,-2.72763152245381)(1.74832589285714,-2.72748867385422)(1.74846540178571,-2.72734391953578)(1.74860491071429,-2.72719726005172)(1.74874441964286,-2.72704869595901)(1.74888392857143,-2.72689822781832)(1.7490234375,-2.72674585619402)(1.74916294642857,-2.7265915816542)(1.74930245535714,-2.72643540477063)(1.74944196428571,-2.7262773261188)(1.74958147321429,-2.72611734627791)(1.74972098214286,-2.72595546583083)(1.74986049107143,-2.72579168536415)(1.75,-2.72562600546816)(1.75013950892857,-2.72545842673683)(1.75027901785714,-2.72528894976785)(1.75041852678571,-2.72511757516259)(1.75055803571429,-2.72494430352611)(1.75069754464286,-2.72476913546716)(1.75083705357143,-2.72459207159819)(1.7509765625,-2.72441311253535)(1.75111607142857,-2.72423225889845)(1.75125558035714,-2.72404951131102)(1.75139508928571,-2.72386487040025)(1.75153459821429,-2.72367833679701)(1.75167410714286,-2.7234899111359)(1.75181361607143,-2.72329959405514)(1.751953125,-2.72310738619668)(1.75209263392857,-2.72291328820612)(1.75223214285714,-2.72271730073276)(1.75237165178571,-2.72251942442957)(1.75251116071429,-2.72231965995319)(1.75265066964286,-2.72211800796393)(1.75279017857143,-2.72191446912579)(1.7529296875,-2.72170904410644)(1.75306919642857,-2.7215017335772)(1.75320870535714,-2.7212925382131)(1.75334821428571,-2.72108145869279)(1.75348772321429,-2.72086849569863)(1.75362723214286,-2.72065364991661)(1.75376674107143,-2.72043692203642)(1.75390625,-2.72021831275138)(1.75404575892857,-2.71999782275848)(1.75418526785714,-2.7197754527584)(1.75432477678571,-2.71955120345544)(1.75446428571429,-2.71932507555757)(1.75460379464286,-2.71909706977643)(1.75474330357143,-2.71886718682729)(1.7548828125,-2.7186354274291)(1.75502232142857,-2.71840179230445)(1.75516183035714,-2.71816628217958)(1.75530133928571,-2.71792889778437)(1.75544084821429,-2.71768963985237)(1.75558035714286,-2.71744850912075)(1.75571986607143,-2.71720550633036)(1.755859375,-2.71696063222565)(1.75599888392857,-2.71671388755474)(1.75613839285714,-2.7164652730694)(1.75627790178571,-2.71621478952501)(1.75641741071429,-2.71596243768061)(1.75655691964286,-2.71570821829886)(1.75669642857143,-2.71545213214607)(1.7568359375,-2.71519417999217)(1.75697544642857,-2.71493436261074)(1.75711495535714,-2.71467268077896)(1.75725446428571,-2.71440913527768)(1.75739397321429,-2.71414372689133)(1.75753348214286,-2.71387645640801)(1.75767299107143,-2.71360732461941)(1.7578125,-2.71333633232087)(1.75795200892857,-2.71306348031132)(1.75809151785714,-2.71278876939335)(1.75823102678571,-2.71251220037312)(1.75837053571429,-2.71223377406045)(1.75851004464286,-2.71195349126875)(1.75864955357143,-2.71167135281506)(1.7587890625,-2.71138735952)(1.75892857142857,-2.71110151220784)(1.75906808035714,-2.71081381170643)(1.75920758928571,-2.71052425884724)(1.75934709821429,-2.71023285446534)(1.75948660714286,-2.70993959939941)(1.75962611607143,-2.70964449449172)(1.759765625,-2.70934754058815)(1.75990513392857,-2.70904873853818)(1.76004464285714,-2.70874808919488)(1.76018415178571,-2.70844559341492)(1.76032366071429,-2.70814125205856)(1.76046316964286,-2.70783506598965)(1.76060267857143,-2.70752703607564)(1.7607421875,-2.70721716318757)(1.76088169642857,-2.70690544820005)(1.76102120535714,-2.70659189199129)(1.76116071428571,-2.70627649544308)(1.76130022321429,-2.70595925944078)(1.76143973214286,-2.70564018487335)(1.76157924107143,-2.70531927263332)(1.76171875,-2.70499652361679)(1.76185825892857,-2.70467193872345)(1.76199776785714,-2.70434551885654)(1.76213727678571,-2.7040172649229)(1.76227678571429,-2.70368717783291)(1.76241629464286,-2.70335525850054)(1.76255580357143,-2.70302150784332)(1.7626953125,-2.70268592678234)(1.76283482142857,-2.70234851624226)(1.76297433035714,-2.70200927715128)(1.76311383928571,-2.7016682104412)(1.76325334821429,-2.70132531704733)(1.76339285714286,-2.70098059790857)(1.76353236607143,-2.70063405396735)(1.763671875,-2.70028568616967)(1.76381138392857,-2.69993549546507)(1.76395089285714,-2.69958348280663)(1.76409040178571,-2.69922964915101)(1.76422991071429,-2.69887399545836)(1.76436941964286,-2.69851652269242)(1.76450892857143,-2.69815723182045)(1.7646484375,-2.69779612381325)(1.76478794642857,-2.69743319964516)(1.76492745535714,-2.69706846029404)(1.76506696428571,-2.69670190674132)(1.76520647321429,-2.69633353997191)(1.76534598214286,-2.69596336097428)(1.76548549107143,-2.69559137074044)(1.765625,-2.69521757026589)(1.76576450892857,-2.69484196054968)(1.76590401785714,-2.69446454259437)(1.76604352678571,-2.69408531740603)(1.76618303571429,-2.69370428599428)(1.76632254464286,-2.69332144937221)(1.76646205357143,-2.69293680855647)(1.7666015625,-2.69255036456718)(1.76674107142857,-2.692162118428)(1.76688058035714,-2.69177207116609)(1.76702008928571,-2.6913802238121)(1.76715959821429,-2.69098657740019)(1.76729910714286,-2.69059113296804)(1.76743861607143,-2.69019389155682)(1.767578125,-2.68979485421117)(1.76771763392857,-2.68939402197927)(1.76785714285714,-2.68899139591277)(1.76799665178571,-2.6885869770668)(1.76813616071429,-2.6881807665)(1.76827566964286,-2.68777276527449)(1.76841517857143,-2.68736297445587)(1.7685546875,-2.68695139511324)(1.76869419642857,-2.68653802831916)(1.76883370535714,-2.68612287514967)(1.76897321428571,-2.68570593668431)(1.76911272321429,-2.68528721400607)(1.76925223214286,-2.68486670820142)(1.76939174107143,-2.68444442036031)(1.76953125,-2.68402035157614)(1.76967075892857,-2.68359450294579)(1.76981026785714,-2.68316687556961)(1.76994977678571,-2.68273747055139)(1.77008928571429,-2.6823062889984)(1.77022879464286,-2.68187333202135)(1.77036830357143,-2.68143860073443)(1.7705078125,-2.68100209625526)(1.77064732142857,-2.68056381970492)(1.77078683035714,-2.68012377220793)(1.77092633928571,-2.67968195489229)(1.77106584821429,-2.6792383688894)(1.77120535714286,-2.67879301533413)(1.77134486607143,-2.67834589536477)(1.771484375,-2.67789701012308)(1.77162388392857,-2.67744636075423)(1.77176339285714,-2.67699394840682)(1.77190290178571,-2.6765397742329)(1.77204241071429,-2.67608383938793)(1.77218191964286,-2.67562614503082)(1.77232142857143,-2.67516669232388)(1.7724609375,-2.67470548243287)(1.77260044642857,-2.67424251652693)(1.77273995535714,-2.67377779577866)(1.77287946428571,-2.67331132136404)(1.77301897321429,-2.6728430944625)(1.77315848214286,-2.67237311625684)(1.77329799107143,-2.6719013879333)(1.7734375,-2.67142791068151)(1.77357700892857,-2.67095268569451)(1.77371651785714,-2.67047571416875)(1.77385602678571,-2.66999699730406)(1.77399553571429,-2.66951653630367)(1.77413504464286,-2.66903433237423)(1.77427455357143,-2.66855038672575)(1.7744140625,-2.66806470057164)(1.77455357142857,-2.66757727512872)(1.77469308035714,-2.66708811161715)(1.77483258928571,-2.66659721126052)(1.77497209821429,-2.66610457528577)(1.77511160714286,-2.66561020492323)(1.77525111607143,-2.66511410140659)(1.775390625,-2.66461626597295)(1.77553013392857,-2.66411669986274)(1.77566964285714,-2.66361540431977)(1.77580915178571,-2.66311238059124)(1.77594866071429,-2.66260762992767)(1.77608816964286,-2.66210115358299)(1.77622767857143,-2.66159295281445)(1.7763671875,-2.66108302888268)(1.77650669642857,-2.66057138305164)(1.77664620535714,-2.66005801658866)(1.77678571428571,-2.65954293076443)(1.77692522321429,-2.65902612685295)(1.77706473214286,-2.65850760613159)(1.77720424107143,-2.65798736988107)(1.77734375,-2.65746541938541)(1.77748325892857,-2.65694175593202)(1.77762276785714,-2.65641638081159)(1.77776227678571,-2.65588929531817)(1.77790178571429,-2.65536050074915)(1.77804129464286,-2.65482999840522)(1.77818080357143,-2.65429778959041)(1.7783203125,-2.65376387561207)(1.77845982142857,-2.65322825778085)(1.77859933035714,-2.65269093741075)(1.77873883928571,-2.65215191581905)(1.77887834821429,-2.65161119432637)(1.77901785714286,-2.65106877425661)(1.77915736607143,-2.650524656937)(1.779296875,-2.64997884369806)(1.77943638392857,-2.64943133587363)(1.77957589285714,-2.6488821348008)(1.77971540178571,-2.64833124182002)(1.77985491071429,-2.64777865827499)(1.77999441964286,-2.6472243855127)(1.78013392857143,-2.64666842488346)(1.7802734375,-2.64611077774083)(1.78041294642857,-2.64555144544167)(1.78055245535714,-2.64499042934612)(1.78069196428571,-2.64442773081758)(1.78083147321429,-2.64386335122276)(1.78097098214286,-2.6432972919316)(1.78111049107143,-2.64272955431734)(1.78125,-2.64216013975647)(1.78138950892857,-2.64158904962875)(1.78152901785714,-2.6410162853172)(1.78166852678571,-2.64044184820811)(1.78180803571429,-2.63986573969099)(1.78194754464286,-2.63928796115865)(1.78208705357143,-2.63870851400712)(1.7822265625,-2.63812739963569)(1.78236607142857,-2.63754461944688)(1.78250558035714,-2.63696017484647)(1.78264508928571,-2.63637406724348)(1.78278459821429,-2.63578629805013)(1.78292410714286,-2.63519686868193)(1.78306361607143,-2.63460578055759)(1.783203125,-2.63401303509904)(1.78334263392857,-2.63341863373145)(1.78348214285714,-2.63282257788322)(1.78362165178571,-2.63222486898595)(1.78376116071429,-2.63162550847447)(1.78390066964286,-2.63102449778683)(1.78404017857143,-2.63042183836428)(1.7841796875,-2.62981753165128)(1.78431919642857,-2.6292115790955)(1.78445870535714,-2.62860398214781)(1.78459821428571,-2.62799474226228)(1.78473772321429,-2.62738386089619)(1.78487723214286,-2.62677133951)(1.78501674107143,-2.62615717956737)(1.78515625,-2.62554138253514)(1.78529575892857,-2.62492394988335)(1.78543526785714,-2.6243048830852)(1.78557477678571,-2.62368418361711)(1.78571428571429,-2.62306185295863)(1.78585379464286,-2.62243789259252)(1.78599330357143,-2.6218123040047)(1.7861328125,-2.62118508868426)(1.78627232142857,-2.62055624812346)(1.78641183035714,-2.6199257838177)(1.78655133928571,-2.61929369726558)(1.78669084821429,-2.61865998996882)(1.78683035714286,-2.61802466343232)(1.78696986607143,-2.61738771916412)(1.787109375,-2.61674915867541)(1.78724888392857,-2.61610898348053)(1.78738839285714,-2.61546719509695)(1.78752790178571,-2.6148237950453)(1.78766741071429,-2.61417878484932)(1.78780691964286,-2.61353216603591)(1.78794642857143,-2.61288394013508)(1.7880859375,-2.61223410868)(1.78822544642857,-2.61158267320692)(1.78836495535714,-2.61092963525524)(1.78850446428571,-2.61027499636748)(1.78864397321429,-2.60961875808926)(1.78878348214286,-2.60896092196933)(1.78892299107143,-2.60830148955954)(1.7890625,-2.60764046241484)(1.78920200892857,-2.6069778420933)(1.78934151785714,-2.60631363015608)(1.78948102678571,-2.60564782816745)(1.78962053571429,-2.60498043769476)(1.78976004464286,-2.60431146030845)(1.78989955357143,-2.60364089758206)(1.7900390625,-2.60296875109221)(1.79017857142857,-2.60229502241861)(1.79031808035714,-2.60161971314404)(1.79045758928571,-2.60094282485436)(1.79059709821429,-2.6002643591385)(1.79073660714286,-2.59958431758847)(1.79087611607143,-2.59890270179932)(1.791015625,-2.59821951336921)(1.79115513392857,-2.59753475389931)(1.79129464285714,-2.59684842499389)(1.79143415178571,-2.59616052826025)(1.79157366071429,-2.59547106530875)(1.79171316964286,-2.59478003775279)(1.79185267857143,-2.59408744720882)(1.7919921875,-2.59339329529635)(1.79213169642857,-2.5926975836379)(1.79227120535714,-2.59200031385903)(1.79241071428571,-2.59130148758836)(1.79255022321429,-2.5906011064575)(1.79268973214286,-2.58989917210112)(1.79282924107143,-2.58919568615689)(1.79296875,-2.58849065026551)(1.79310825892857,-2.58778406607071)(1.79324776785714,-2.5870759352192)(1.79338727678571,-2.58636625936073)(1.79352678571429,-2.58565504014804)(1.79366629464286,-2.58494227923689)(1.79380580357143,-2.58422797828601)(1.7939453125,-2.58351213895717)(1.79408482142857,-2.5827947629151)(1.79422433035714,-2.58207585182754)(1.79436383928571,-2.5813554073652)(1.79450334821429,-2.5806334312018)(1.79464285714286,-2.579909925014)(1.79478236607143,-2.57918489048149)(1.794921875,-2.5784583292869)(1.79506138392857,-2.57773024311584)(1.79520089285714,-2.57700063365688)(1.79534040178571,-2.57626950260157)(1.79547991071429,-2.57553685164441)(1.79561941964286,-2.57480268248287)(1.79575892857143,-2.57406699681735)(1.7958984375,-2.57332979635124)(1.79603794642857,-2.57259108279084)(1.79617745535714,-2.57185085784542)(1.79631696428571,-2.57110912322718)(1.79645647321429,-2.57036588065125)(1.79659598214286,-2.56962113183573)(1.79673549107143,-2.56887487850161)(1.796875,-2.56812712237283)(1.79701450892857,-2.56737786517625)(1.79715401785714,-2.56662710864166)(1.79729352678571,-2.56587485450175)(1.79743303571429,-2.56512110449214)(1.79757254464286,-2.56436586035135)(1.79771205357143,-2.56360912382083)(1.7978515625,-2.5628508966449)(1.79799107142857,-2.56209118057082)(1.79813058035714,-2.56132997734871)(1.79827008928571,-2.56056728873161)(1.79840959821429,-2.55980311647544)(1.79854910714286,-2.55903746233902)(1.79868861607143,-2.55827032808404)(1.798828125,-2.55750171547507)(1.79896763392857,-2.55673162627957)(1.79910714285714,-2.55596006226787)(1.79924665178571,-2.55518702521316)(1.79938616071429,-2.55441251689151)(1.79952566964286,-2.55363653908184)(1.79966517857143,-2.55285909356594)(1.7998046875,-2.55208018212847)(1.79994419642857,-2.5512998065569)(1.80008370535714,-2.5505179686416)(1.80022321428571,-2.54973467017576)(1.80036272321429,-2.5489499129554)(1.80050223214286,-2.54816369877942)(1.80064174107143,-2.5473760294495)(1.80078125,-2.54658690677021)(1.80092075892857,-2.54579633254891)(1.80106026785714,-2.5450043085958)(1.80119977678571,-2.54421083672389)(1.80133928571429,-2.54341591874902)(1.80147879464286,-2.54261955648984)(1.80161830357143,-2.54182175176782)(1.8017578125,-2.54102250640722)(1.80189732142857,-2.54022182223512)(1.80203683035714,-2.53941970108138)(1.80217633928571,-2.53861614477868)(1.80231584821429,-2.53781115516249)(1.80245535714286,-2.53700473407105)(1.80259486607143,-2.5361968833454)(1.802734375,-2.53538760482936)(1.80287388392857,-2.53457690036955)(1.80301339285714,-2.53376477181532)(1.80315290178571,-2.53295122101883)(1.80329241071429,-2.53213624983499)(1.80343191964286,-2.53131986012148)(1.80357142857143,-2.53050205373875)(1.8037109375,-2.52968283254998)(1.80385044642857,-2.52886219842114)(1.80398995535714,-2.52804015322092)(1.80412946428571,-2.52721669882077)(1.80426897321429,-2.52639183709488)(1.80440848214286,-2.52556556992018)(1.80454799107143,-2.52473789917634)(1.8046875,-2.52390882674575)(1.80482700892857,-2.52307835451353)(1.80496651785714,-2.52224648436754)(1.80510602678571,-2.52141321819834)(1.80524553571429,-2.52057855789923)(1.80538504464286,-2.51974250536619)(1.80552455357143,-2.51890506249794)(1.8056640625,-2.5180662311959)(1.80580357142857,-2.51722601336419)(1.80594308035714,-2.51638441090962)(1.80608258928571,-2.5155414257417)(1.80622209821429,-2.51469705977263)(1.80636160714286,-2.51385131491731)(1.80650111607143,-2.5130041930933)(1.806640625,-2.51215569622087)(1.80678013392857,-2.51130582622295)(1.80691964285714,-2.51045458502512)(1.80705915178571,-2.50960197455567)(1.80719866071429,-2.50874799674553)(1.80733816964286,-2.5078926535283)(1.80747767857143,-2.50703594684023)(1.8076171875,-2.50617787862023)(1.80775669642857,-2.50531845080987)(1.80789620535714,-2.50445766535333)(1.80803571428571,-2.50359552419748)(1.80817522321429,-2.5027320292918)(1.80831473214286,-2.50186718258839)(1.80845424107143,-2.50100098604203)(1.80859375,-2.50013344161008)(1.80873325892857,-2.49926455125255)(1.80887276785714,-2.49839431693206)(1.80901227678571,-2.49752274061384)(1.80915178571429,-2.49664982426575)(1.80929129464286,-2.49577556985823)(1.80943080357143,-2.49489997936436)(1.8095703125,-2.49402305475979)(1.80970982142857,-2.49314479802278)(1.80984933035714,-2.49226521113418)(1.80998883928571,-2.49138429607744)(1.81012834821429,-2.49050205483857)(1.81026785714286,-2.48961848940618)(1.81040736607143,-2.48873360177145)(1.810546875,-2.48784739392814)(1.81068638392857,-2.48695986787258)(1.81082589285714,-2.48607102560366)(1.81096540178571,-2.48518086912283)(1.81110491071429,-2.48428940043409)(1.81124441964286,-2.48339662154402)(1.81138392857143,-2.48250253446174)(1.8115234375,-2.48160714119889)(1.81166294642857,-2.48071044376969)(1.81180245535714,-2.47981244419087)(1.81194196428571,-2.47891314448172)(1.81208147321429,-2.47801254666403)(1.81222098214286,-2.47711065276213)(1.81236049107143,-2.4762074648029)(1.8125,-2.47530298481569)(1.81263950892857,-2.4743972148324)(1.81277901785714,-2.47349015688743)(1.81291852678571,-2.47258181301767)(1.81305803571429,-2.47167218526255)(1.81319754464286,-2.47076127566396)(1.81333705357143,-2.46984908626631)(1.8134765625,-2.46893561911649)(1.81361607142857,-2.46802087626389)(1.81375558035714,-2.46710485976036)(1.81389508928571,-2.46618757166024)(1.81403459821429,-2.46526901402036)(1.81417410714286,-2.46434918889999)(1.81431361607143,-2.4634280983609)(1.814453125,-2.4625057444673)(1.81459263392857,-2.46158212928587)(1.81473214285714,-2.46065725488573)(1.81487165178571,-2.45973112333847)(1.81501116071429,-2.45880373671811)(1.81515066964286,-2.45787509710114)(1.81529017857143,-2.45694520656645)(1.8154296875,-2.45601406719538)(1.81556919642857,-2.45508168107173)(1.81570870535714,-2.45414805028168)(1.81584821428571,-2.45321317691387)(1.81598772321429,-2.45227706305932)(1.81612723214286,-2.45133971081151)(1.81626674107143,-2.45040112226629)(1.81640625,-2.44946129952193)(1.81654575892857,-2.44852024467913)(1.81668526785714,-2.44757795984094)(1.81682477678571,-2.44663444711283)(1.81696428571429,-2.44568970860266)(1.81710379464286,-2.44474374642067)(1.81724330357143,-2.44379656267949)(1.8173828125,-2.44284815949412)(1.81752232142857,-2.44189853898194)(1.81766183035714,-2.44094770326267)(1.81780133928571,-2.43999565445845)(1.81794084821429,-2.43904239469373)(1.81808035714286,-2.43808792609534)(1.81821986607143,-2.43713225079247)(1.818359375,-2.43617537091663)(1.81849888392857,-2.43521728860172)(1.81863839285714,-2.43425800598393)(1.81877790178571,-2.43329752520181)(1.81891741071429,-2.43233584839626)(1.81905691964286,-2.43137297771047)(1.81919642857143,-2.43040891528998)(1.8193359375,-2.42944366328264)(1.81947544642857,-2.42847722383862)(1.81961495535714,-2.42750959911039)(1.81975446428571,-2.42654079125275)(1.81989397321429,-2.42557080242278)(1.82003348214286,-2.42459963477987)(1.82017299107143,-2.42362729048569)(1.8203125,-2.42265377170422)(1.82045200892857,-2.42167908060171)(1.82059151785714,-2.42070321934671)(1.82073102678571,-2.41972619011002)(1.82087053571429,-2.41874799506475)(1.82101004464286,-2.41776863638624)(1.82114955357143,-2.41678811625211)(1.8212890625,-2.41580643684226)(1.82142857142857,-2.41482360033882)(1.82156808035714,-2.41383960892618)(1.82170758928571,-2.41285446479099)(1.82184709821429,-2.41186817012212)(1.82198660714286,-2.41088072711069)(1.82212611607143,-2.40989213795007)(1.822265625,-2.40890240483584)(1.82240513392857,-2.40791152996582)(1.82254464285714,-2.40691951554003)(1.82268415178571,-2.40592636376075)(1.82282366071429,-2.40493207683243)(1.82296316964286,-2.40393665696174)(1.82310267857143,-2.40294010635759)(1.8232421875,-2.40194242723104)(1.82338169642857,-2.40094362179537)(1.82352120535714,-2.39994369226606)(1.82366071428571,-2.39894264086077)(1.82380022321429,-2.39794046979934)(1.82393973214286,-2.39693718130379)(1.82407924107143,-2.39593277759831)(1.82421875,-2.39492726090928)(1.82435825892857,-2.39392063346523)(1.82449776785714,-2.39291289749685)(1.82463727678571,-2.391904055237)(1.82477678571429,-2.39089410892068)(1.82491629464286,-2.38988306078505)(1.82505580357143,-2.3888709130694)(1.8251953125,-2.38785766801517)(1.82533482142857,-2.38684332786595)(1.82547433035714,-2.38582789486743)(1.82561383928571,-2.38481137126746)(1.82575334821429,-2.38379375931597)(1.82589285714286,-2.38277506126505)(1.82603236607143,-2.38175527936889)(1.826171875,-2.38073441588379)(1.82631138392857,-2.37971247306814)(1.82645089285714,-2.37868945318245)(1.82659040178571,-2.37766535848932)(1.82672991071429,-2.37664019125344)(1.82686941964286,-2.37561395374159)(1.82700892857143,-2.37458664822263)(1.8271484375,-2.3735582769675)(1.82728794642857,-2.37252884224922)(1.82742745535714,-2.37149834634286)(1.82756696428571,-2.37046679152559)(1.82770647321429,-2.36943418007661)(1.82784598214286,-2.36840051427718)(1.82798549107143,-2.36736579641063)(1.828125,-2.36633002876233)(1.82826450892857,-2.36529321361968)(1.82840401785714,-2.36425535327213)(1.82854352678571,-2.36321645001117)(1.82868303571429,-2.36217650613031)(1.82882254464286,-2.36113552392508)(1.82896205357143,-2.36009350569304)(1.8291015625,-2.35905045373378)(1.82924107142857,-2.35800637034888)(1.82938058035714,-2.35696125784193)(1.82952008928571,-2.35591511851853)(1.82965959821429,-2.35486795468627)(1.82979910714286,-2.35381976865475)(1.82993861607143,-2.35277056273555)(1.830078125,-2.35172033924223)(1.83021763392857,-2.35066910049033)(1.83035714285714,-2.34961684879738)(1.83049665178571,-2.34856358648288)(1.83063616071429,-2.34750931586826)(1.83077566964286,-2.34645403927697)(1.83091517857143,-2.34539775903438)(1.8310546875,-2.34434047746782)(1.83119419642857,-2.34328219690658)(1.83133370535714,-2.34222291968188)(1.83147321428571,-2.34116264812689)(1.83161272321429,-2.34010138457671)(1.83175223214286,-2.33903913136837)(1.83189174107143,-2.33797589084082)(1.83203125,-2.33691166533496)(1.83217075892857,-2.33584645719357)(1.83231026785714,-2.33478026876137)(1.83244977678571,-2.33371310238497)(1.83258928571429,-2.33264496041289)(1.83272879464286,-2.33157584519555)(1.83286830357143,-2.33050575908527)(1.8330078125,-2.32943470443625)(1.83314732142857,-2.32836268360458)(1.83328683035714,-2.32728969894822)(1.83342633928571,-2.32621575282702)(1.83356584821429,-2.3251408476027)(1.83370535714286,-2.32406498563883)(1.83384486607143,-2.32298816930087)(1.833984375,-2.32191040095611)(1.83412388392857,-2.32083168297371)(1.83426339285714,-2.31975201772468)(1.83440290178571,-2.31867140758185)(1.83454241071429,-2.31758985491992)(1.83468191964286,-2.3165073621154)(1.83482142857143,-2.31542393154665)(1.8349609375,-2.31433956559384)(1.83510044642857,-2.31325426663897)(1.83523995535714,-2.31216803706585)(1.83537946428571,-2.31108087926009)(1.83551897321429,-2.30999279560914)(1.83565848214286,-2.30890378850221)(1.83579799107143,-2.30781386033034)(1.8359375,-2.30672301348636)(1.83607700892857,-2.30563125036487)(1.83621651785714,-2.30453857336225)(1.83635602678571,-2.3034449848767)(1.83649553571429,-2.30235048730814)(1.83663504464286,-2.30125508305829)(1.83677455357143,-2.30015877453064)(1.8369140625,-2.29906156413041)(1.83705357142857,-2.29796345426462)(1.83719308035714,-2.296864447342)(1.83733258928571,-2.29576454577303)(1.83747209821429,-2.29466375196997)(1.83761160714286,-2.29356206834676)(1.83775111607143,-2.29245949731912)(1.837890625,-2.29135604130447)(1.83803013392857,-2.29025170272196)(1.83816964285714,-2.28914648399246)(1.83830915178571,-2.28804038753855)(1.83844866071429,-2.28693341578452)(1.83858816964286,-2.28582557115636)(1.83872767857143,-2.28471685608176)(1.8388671875,-2.28360727299011)(1.83900669642857,-2.28249682431249)(1.83914620535714,-2.28138551248165)(1.83928571428571,-2.28027333993204)(1.83942522321429,-2.27916030909976)(1.83956473214286,-2.2780464224226)(1.83970424107143,-2.27693168234002)(1.83984375,-2.27581609129313)(1.83998325892857,-2.27469965172469)(1.84012276785714,-2.27358236607913)(1.84026227678571,-2.27246423680251)(1.84040178571429,-2.27134526634254)(1.84054129464286,-2.27022545714856)(1.84068080357143,-2.26910481167155)(1.8408203125,-2.26798333236412)(1.84095982142857,-2.2668610216805)(1.84109933035714,-2.26573788207653)(1.84123883928571,-2.26461391600968)(1.84137834821429,-2.263489125939)(1.84151785714286,-2.26236351432518)(1.84165736607143,-2.26123708363048)(1.841796875,-2.26010983631877)(1.84193638392857,-2.2589817748555)(1.84207589285714,-2.25785290170772)(1.84221540178571,-2.25672321934404)(1.84235491071429,-2.25559273023465)(1.84249441964286,-2.25446143685132)(1.84263392857143,-2.25332934166738)(1.8427734375,-2.25219644715771)(1.84291294642857,-2.25106275579877)(1.84305245535714,-2.24992827006855)(1.84319196428571,-2.2487929924466)(1.84333147321429,-2.24765692541399)(1.84347098214286,-2.24652007145334)(1.84361049107143,-2.24538243304881)(1.84375,-2.24424401268608)(1.84388950892857,-2.24310481285234)(1.84402901785714,-2.24196483603632)(1.84416852678571,-2.24082408472824)(1.84430803571429,-2.23968256141985)(1.84444754464286,-2.23854026860436)(1.84458705357143,-2.23739720877654)(1.8447265625,-2.2362533844326)(1.84486607142857,-2.23510879807028)(1.84500558035714,-2.23396345218875)(1.84514508928571,-2.23281734928871)(1.84528459821429,-2.23167049187231)(1.84542410714286,-2.23052288244316)(1.84556361607143,-2.22937452350636)(1.845703125,-2.22822541756845)(1.84584263392857,-2.22707556713742)(1.84598214285714,-2.22592497472272)(1.84612165178571,-2.22477364283525)(1.84626116071429,-2.22362157398732)(1.84640066964286,-2.2224687706927)(1.84654017857143,-2.2213152354666)(1.8466796875,-2.22016097082562)(1.84681919642857,-2.21900597928781)(1.84695870535714,-2.21785026337261)(1.84709821428571,-2.21669382560089)(1.84723772321429,-2.21553666849492)(1.84737723214286,-2.21437879457835)(1.84751674107143,-2.21322020637626)(1.84765625,-2.2120609064151)(1.84779575892857,-2.21090089722271)(1.84793526785714,-2.20974018132829)(1.84807477678571,-2.20857876126246)(1.84821428571429,-2.20741663955716)(1.84835379464286,-2.20625381874574)(1.84849330357143,-2.20509030136287)(1.8486328125,-2.20392608994461)(1.84877232142857,-2.20276118702835)(1.84891183035714,-2.20159559515284)(1.84905133928571,-2.20042931685816)(1.84919084821429,-2.19926235468573)(1.84933035714286,-2.19809471117829)(1.84946986607143,-2.19692638887993)(1.849609375,-2.19575739033605)(1.84974888392857,-2.19458771809335)(1.84988839285714,-2.19341737469986)(1.85002790178571,-2.19224636270492)(1.85016741071429,-2.19107468465915)(1.85030691964286,-2.18990234311449)(1.85044642857143,-2.18872934062415)(1.8505859375,-2.18755567974264)(1.85072544642857,-2.18638136302576)(1.85086495535714,-2.18520639303056)(1.85100446428571,-2.18403077231537)(1.85114397321429,-2.18285450343981)(1.85128348214286,-2.18167758896474)(1.85142299107143,-2.18050003145227)(1.8515625,-2.17932183346578)(1.85170200892857,-2.17814299756988)(1.85184151785714,-2.17696352633045)(1.85198102678571,-2.17578342231457)(1.85212053571429,-2.17460268809056)(1.85226004464286,-2.173421326228)(1.85239955357143,-2.17223933929764)(1.8525390625,-2.17105672987149)(1.85267857142857,-2.16987350052275)(1.85281808035714,-2.16868965382583)(1.85295758928571,-2.16750519235635)(1.85309709821429,-2.1663201186911)(1.85323660714286,-2.1651344354081)(1.85337611607143,-2.16394814508652)(1.853515625,-2.16276125030674)(1.85365513392857,-2.16157375365031)(1.85379464285714,-2.16038565769993)(1.85393415178571,-2.1591969650395)(1.85407366071429,-2.15800767825404)(1.85421316964286,-2.15681779992977)(1.85435267857143,-2.15562733265404)(1.8544921875,-2.15443627901535)(1.85463169642857,-2.15324464160332)(1.85477120535714,-2.15205242300875)(1.85491071428571,-2.15085962582353)(1.85505022321429,-2.1496662526407)(1.85518973214286,-2.14847230605441)(1.85532924107143,-2.14727778865992)(1.85546875,-2.14608270305363)(1.85560825892857,-2.14488705183302)(1.85574776785714,-2.14369083759668)(1.85588727678571,-2.14249406294428)(1.85602678571429,-2.1412967304766)(1.85616629464286,-2.1400988427955)(1.85630580357143,-2.13890040250392)(1.8564453125,-2.13770141220588)(1.85658482142857,-2.13650187450646)(1.85672433035714,-2.13530179201182)(1.85686383928571,-2.13410116732916)(1.85700334821429,-2.13290000306675)(1.85714285714286,-2.13169830183391)(1.85728236607143,-2.130496066241)(1.857421875,-2.12929329889943)(1.85756138392857,-2.12809000242163)(1.85770089285714,-2.12688617942107)(1.85784040178571,-2.12568183251223)(1.85797991071429,-2.12447696431064)(1.85811941964286,-2.12327157743281)(1.85825892857143,-2.12206567449628)(1.8583984375,-2.1208592581196)(1.85853794642857,-2.1196523309223)(1.85867745535714,-2.11844489552492)(1.85881696428571,-2.11723695454897)(1.85895647321429,-2.11602851061696)(1.85909598214286,-2.11481956635239)(1.85923549107143,-2.1136101243797)(1.859375,-2.11240018732433)(1.85951450892857,-2.11118975781266)(1.85965401785714,-2.10997883847206)(1.85979352678571,-2.10876743193081)(1.85993303571429,-2.10755554081817)(1.86007254464286,-2.10634316776434)(1.86021205357143,-2.10513031540045)(1.8603515625,-2.10391698635856)(1.86049107142857,-2.10270318327168)(1.86063058035714,-2.1014889087737)(1.86077008928571,-2.10027416549947)(1.86090959821429,-2.09905895608473)(1.86104910714286,-2.09784328316613)(1.86118861607143,-2.09662714938124)(1.861328125,-2.09541055736849)(1.86146763392857,-2.09419350976725)(1.86160714285714,-2.09297600921772)(1.86174665178571,-2.09175805836104)(1.86188616071429,-2.09053965983918)(1.86202566964286,-2.089320816295)(1.86216517857143,-2.08810153037224)(1.8623046875,-2.08688180471547)(1.86244419642857,-2.08566164197015)(1.86258370535714,-2.08444104478256)(1.86272321428571,-2.08322001579984)(1.86286272321429,-2.08199855766997)(1.86300223214286,-2.08077667304177)(1.86314174107143,-2.07955436456489)(1.86328125,-2.07833163488978)(1.86342075892857,-2.07710848666775)(1.86356026785714,-2.07588492255088)(1.86369977678571,-2.07466094519211)(1.86383928571429,-2.07343655724514)(1.86397879464286,-2.0722117613645)(1.86411830357143,-2.07098656020548)(1.8642578125,-2.0697609564242)(1.86439732142857,-2.06853495267754)(1.86453683035714,-2.06730855162315)(1.86467633928571,-2.06608175591947)(1.86481584821429,-2.06485456822571)(1.86495535714286,-2.06362699120182)(1.86509486607143,-2.06239902750854)(1.865234375,-2.06117067980734)(1.86537388392857,-2.05994195076045)(1.86551339285714,-2.05871284303082)(1.86565290178571,-2.05748335928217)(1.86579241071429,-2.05625350217892)(1.86593191964286,-2.05502327438624)(1.86607142857143,-2.05379267857001)(1.8662109375,-2.05256171739683)(1.86635044642857,-2.05133039353401)(1.86648995535714,-2.05009870964957)(1.86662946428571,-2.04886666841222)(1.86676897321429,-2.04763427249137)(1.86690848214286,-2.04640152455713)(1.86704799107143,-2.04516842728028)(1.8671875,-2.04393498333229)(1.86732700892857,-2.04270119538532)(1.86746651785714,-2.04146706611215)(1.86760602678571,-2.04023259818628)(1.86774553571429,-2.03899779428183)(1.86788504464286,-2.0377626570736)(1.86802455357143,-2.03652718923702)(1.8681640625,-2.03529139344818)(1.86830357142857,-2.0340552723838)(1.86844308035714,-2.03281882872122)(1.86858258928571,-2.03158206513843)(1.86872209821429,-2.03034498431403)(1.86886160714286,-2.02910758892724)(1.86900111607143,-2.02786988165789)(1.869140625,-2.02663186518643)(1.86928013392857,-2.02539354219389)(1.86941964285714,-2.02415491536192)(1.86955915178571,-2.02291598737275)(1.86969866071429,-2.02167676090917)(1.86983816964286,-2.02043723865461)(1.86997767857143,-2.01919742329302)(1.8701171875,-2.01795731750896)(1.87025669642857,-2.01671692398752)(1.87039620535714,-2.01547624541438)(1.87053571428571,-2.01423528447576)(1.87067522321429,-2.01299404385843)(1.87081473214286,-2.0117525262497)(1.87095424107143,-2.01051073433745)(1.87109375,-2.00926867081004)(1.87123325892857,-2.00802633835641)(1.87137276785714,-2.00678373966599)(1.87151227678571,-2.00554087742875)(1.87165178571429,-2.00429775433515)(1.87179129464286,-2.00305437307617)(1.87193080357143,-2.00181073634331)(1.8720703125,-2.00056684682853)(1.87220982142857,-1.99932270722432)(1.87234933035714,-1.99807832022363)(1.87248883928571,-1.99683368851991)(1.87262834821429,-1.99558881480705)(1.87276785714286,-1.99434370177947)(1.87290736607143,-1.993098352132)(1.873046875,-1.99185276855996)(1.87318638392857,-1.99060695375913)(1.87332589285714,-1.98936091042572)(1.87346540178571,-1.9881146412564)(1.87360491071429,-1.98686814894826)(1.87374441964286,-1.98562143619885)(1.87388392857143,-1.98437450570613)(1.8740234375,-1.9831273601685)(1.87416294642857,-1.98188000228477)(1.87430245535714,-1.98063243475415)(1.87444196428571,-1.97938466027628)(1.87458147321429,-1.97813668155118)(1.87472098214286,-1.9768885012793)(1.87486049107143,-1.97564012216145)(1.875,-1.97439154689885)(1.87513950892857,-1.97314277819308)(1.87527901785714,-1.97189381874612)(1.87541852678571,-1.97064467126031)(1.87555803571429,-1.96939533843834)(1.87569754464286,-1.96814582298329)(1.87583705357143,-1.96689612759857)(1.8759765625,-1.96564625498797)(1.87611607142857,-1.96439620785559)(1.87625558035714,-1.9631459889059)(1.87639508928571,-1.96189560084369)(1.87653459821429,-1.96064504637406)(1.87667410714286,-1.95939432820247)(1.87681361607143,-1.95814344903469)(1.876953125,-1.95689241157678)(1.87709263392857,-1.95564121853514)(1.87723214285714,-1.95438987261644)(1.87737165178571,-1.95313837652767)(1.87751116071429,-1.95188673297612)(1.87765066964286,-1.95063494466933)(1.87779017857143,-1.94938301431516)(1.8779296875,-1.94813094462173)(1.87806919642857,-1.94687873829744)(1.87820870535714,-1.94562639805094)(1.87834821428571,-1.94437392659115)(1.87848772321429,-1.94312132662724)(1.87862723214286,-1.94186860086864)(1.87876674107143,-1.94061575202503)(1.87890625,-1.9393627828063)(1.87904575892857,-1.93810969592261)(1.87918526785714,-1.93685649408433)(1.87932477678571,-1.93560318000205)(1.87946428571429,-1.93434975638659)(1.87960379464286,-1.93309622594898)(1.87974330357143,-1.93184259140044)(1.8798828125,-1.93058885545243)(1.88002232142857,-1.92933502081658)(1.88016183035714,-1.92808109020471)(1.88030133928571,-1.92682706632884)(1.88044084821429,-1.92557295190117)(1.88058035714286,-1.92431874963407)(1.88071986607143,-1.92306446224007)(1.880859375,-1.92181009243191)(1.88099888392857,-1.92055564292243)(1.88113839285714,-1.91930111642467)(1.88127790178571,-1.9180465156518)(1.88141741071429,-1.91679184331714)(1.88155691964286,-1.91553710213415)(1.88169642857143,-1.91428229481642)(1.8818359375,-1.91302742407767)(1.88197544642857,-1.91177249263175)(1.88211495535714,-1.91051750319262)(1.88225446428571,-1.90926245847436)(1.88239397321429,-1.90800736119114)(1.88253348214286,-1.90675221405726)(1.88267299107143,-1.90549701978709)(1.8828125,-1.90424178109512)(1.88295200892857,-1.90298650069589)(1.88309151785714,-1.90173118130406)(1.88323102678571,-1.90047582563435)(1.88337053571429,-1.89922043640153)(1.88351004464286,-1.89796501632046)(1.88364955357143,-1.89670956810606)(1.8837890625,-1.8954540944733)(1.88392857142857,-1.89419859813719)(1.88406808035714,-1.8929430818128)(1.88420758928571,-1.89168754821523)(1.88434709821429,-1.89043200005961)(1.88448660714286,-1.8891764400611)(1.88462611607143,-1.88792087093489)(1.884765625,-1.88666529539619)(1.88490513392857,-1.88540971616021)(1.88504464285714,-1.88415413594217)(1.88518415178571,-1.88289855745731)(1.88532366071429,-1.88164298342083)(1.88546316964286,-1.88038741654797)(1.88560267857143,-1.8791318595539)(1.8857421875,-1.87787631515383)(1.88588169642857,-1.87662078606289)(1.88602120535714,-1.87536527499622)(1.88616071428571,-1.87410978466892)(1.88630022321429,-1.87285431779601)(1.88643973214286,-1.87159887709252)(1.88657924107143,-1.8703434652734)(1.88671875,-1.86908808505355)(1.88685825892857,-1.86783273914779)(1.88699776785714,-1.86657743027091)(1.88713727678571,-1.8653221611376)(1.88727678571429,-1.86406693446247)(1.88741629464286,-1.86281175296006)(1.88755580357143,-1.86155661934484)(1.8876953125,-1.86030153633114)(1.88783482142857,-1.85904650663323)(1.88797433035714,-1.85779153296527)(1.88811383928571,-1.85653661804129)(1.88825334821429,-1.85528176457523)(1.88839285714286,-1.8540269752809)(1.88853236607143,-1.85277225287199)(1.888671875,-1.85151760006205)(1.88881138392857,-1.8502630195645)(1.88895089285714,-1.84900851409262)(1.88909040178571,-1.84775408635955)(1.88922991071429,-1.84649973907826)(1.88936941964286,-1.84524547496159)(1.88950892857143,-1.84399129672219)(1.8896484375,-1.84273720707257)(1.88978794642857,-1.84148320872505)(1.88992745535714,-1.84022930439177)(1.89006696428571,-1.8389754967847)(1.89020647321429,-1.83772178861562)(1.89034598214286,-1.8364681825961)(1.89048549107143,-1.83521468143754)(1.890625,-1.8339612878511)(1.89076450892857,-1.83270800454776)(1.89090401785714,-1.83145483423827)(1.89104352678571,-1.83020177963318)(1.89118303571429,-1.82894884344278)(1.89132254464286,-1.82769602837715)(1.89146205357143,-1.82644333714615)(1.8916015625,-1.82519077245937)(1.89174107142857,-1.82393833702618)(1.89188058035714,-1.82268603355567)(1.89202008928571,-1.8214338647567)(1.89215959821429,-1.82018183333784)(1.89229910714286,-1.81892994200742)(1.89243861607143,-1.81767819347348)(1.892578125,-1.8164265904438)(1.89271763392857,-1.81517513562585)(1.89285714285714,-1.81392383172684)(1.89299665178571,-1.81267268145366)(1.89313616071429,-1.81142168751292)(1.89327566964286,-1.81017085261092)(1.89341517857143,-1.80892017945364)(1.8935546875,-1.80766967074677)(1.89369419642857,-1.80641932919566)(1.89383370535714,-1.80516915750534)(1.89397321428571,-1.8039191583805)(1.89411272321429,-1.80266933452551)(1.89425223214286,-1.80141968864439)(1.89439174107143,-1.80017022344082)(1.89453125,-1.79892094161812)(1.89467075892857,-1.79767184587926)(1.89481026785714,-1.79642293892684)(1.89494977678571,-1.7951742234631)(1.89508928571429,-1.79392570218991)(1.89522879464286,-1.79267737780875)(1.89536830357143,-1.79142925302072)(1.8955078125,-1.79018133052655)(1.89564732142857,-1.78893361302655)(1.89578683035714,-1.78768610322064)(1.89592633928571,-1.78643880380835)(1.89606584821429,-1.78519171748877)(1.89620535714286,-1.78394484696062)(1.89634486607143,-1.78269819492215)(1.896484375,-1.78145176407122)(1.89662388392857,-1.78020555710525)(1.89676339285714,-1.77895957672121)(1.89690290178571,-1.77771382561567)(1.89704241071429,-1.77646830648469)(1.89718191964286,-1.77522302202393)(1.89732142857143,-1.77397797492858)(1.8974609375,-1.77273316789336)(1.89760044642857,-1.77148860361254)(1.89773995535714,-1.77024428477989)(1.89787946428571,-1.76900021408872)(1.89801897321429,-1.76775639423186)(1.89815848214286,-1.76651282790164)(1.89829799107143,-1.76526951778992)(1.8984375,-1.76402646658802)(1.89857700892857,-1.7627836769868)(1.89871651785714,-1.76154115167658)(1.89885602678571,-1.76029889334718)(1.89899553571429,-1.7590569046879)(1.89913504464286,-1.75781518838751)(1.89927455357143,-1.75657374713426)(1.8994140625,-1.75533258361584)(1.89955357142857,-1.75409170051944)(1.89969308035714,-1.75285110053166)(1.89983258928571,-1.75161078633858)(1.89997209821429,-1.75037076062571)(1.90011160714286,-1.74913102607801)(1.90025111607143,-1.74789158537985)(1.900390625,-1.74665244121506)(1.90053013392857,-1.74541359626686)(1.90066964285714,-1.74417505321792)(1.90080915178571,-1.7429368147503)(1.90094866071429,-1.74169888354547)(1.90108816964286,-1.74046126228431)(1.90122767857143,-1.73922395364709)(1.9013671875,-1.73798696031348)(1.90150669642857,-1.73675028496253)(1.90164620535714,-1.73551393027265)(1.90178571428571,-1.73427789892168)(1.90192522321429,-1.73304219358678)(1.90206473214286,-1.73180681694449)(1.90220424107143,-1.73057177167072)(1.90234375,-1.72933706044073)(1.90248325892857,-1.72810268592913)(1.90262276785714,-1.72686865080986)(1.90276227678571,-1.72563495775623)(1.90290178571429,-1.72440160944086)(1.90304129464286,-1.72316860853571)(1.90318080357143,-1.72193595771205)(1.9033203125,-1.72070365964049)(1.90345982142857,-1.71947171699093)(1.90359933035714,-1.71824013243261)(1.90373883928571,-1.71700890863403)(1.90387834821429,-1.71577804826302)(1.90401785714286,-1.7145475539867)(1.90415736607143,-1.71331742847147)(1.904296875,-1.712087674383)(1.90443638392857,-1.71085829438627)(1.90457589285714,-1.7096292911455)(1.90471540178571,-1.7084006673242)(1.90485491071429,-1.70717242558512)(1.90499441964286,-1.70594456859027)(1.90513392857143,-1.70471709900094)(1.9052734375,-1.70349001947764)(1.90541294642857,-1.70226333268011)(1.90555245535714,-1.70103704126736)(1.90569196428571,-1.6998111478976)(1.90583147321429,-1.69858565522827)(1.90597098214286,-1.69736056591605)(1.90611049107143,-1.69613588261682)(1.90625,-1.69491160798566) 
};
\addplot [
color=blue,
solid
]
coordinates{
 (1.90625,-1.69491160798566)(1.90638950892857,-1.69368774467689)(1.90652901785714,-1.69246429534398)(1.90666852678571,-1.69124126263964)(1.90680803571429,-1.69001864921573)(1.90694754464286,-1.68879645772333)(1.90708705357143,-1.68757469081267)(1.9072265625,-1.68635335113317)(1.90736607142857,-1.68513244133342)(1.90750558035714,-1.68391196406115)(1.90764508928571,-1.68269192196329)(1.90778459821429,-1.68147231768588)(1.90792410714286,-1.68025315387413)(1.90806361607143,-1.6790344331724)(1.908203125,-1.67781615822416)(1.90834263392857,-1.67659833167204)(1.90848214285714,-1.67538095615778)(1.90862165178571,-1.67416403432226)(1.90876116071429,-1.67294756880545)(1.90890066964286,-1.67173156224646)(1.90904017857143,-1.67051601728349)(1.9091796875,-1.66930093655384)(1.90931919642857,-1.66808632269392)(1.90945870535714,-1.66687217833923)(1.90959821428571,-1.66565850612433)(1.90973772321429,-1.6644453086829)(1.90987723214286,-1.66323258864767)(1.91001674107143,-1.66202034865044)(1.91015625,-1.6608085913221)(1.91029575892857,-1.65959731929257)(1.91043526785714,-1.65838653519084)(1.91057477678571,-1.65717624164497)(1.91071428571429,-1.65596644128202)(1.91085379464286,-1.65475713672812)(1.91099330357143,-1.65354833060844)(1.9111328125,-1.65234002554716)(1.91127232142857,-1.65113222416751)(1.91141183035714,-1.64992492909172)(1.91155133928571,-1.64871814294102)(1.91169084821429,-1.6475118683357)(1.91183035714286,-1.64630610789499)(1.91196986607143,-1.64510086423718)(1.912109375,-1.64389613997951)(1.91224888392857,-1.64269193773824)(1.91238839285714,-1.64148826012858)(1.91252790178571,-1.64028510976476)(1.91266741071429,-1.63908248925995)(1.91280691964286,-1.63788040122631)(1.91294642857143,-1.63667884827494)(1.9130859375,-1.63547783301593)(1.91322544642857,-1.6342773580583)(1.91336495535714,-1.63307742601003)(1.91350446428571,-1.63187803947803)(1.91364397321429,-1.63067920106816)(1.91378348214286,-1.62948091338521)(1.91392299107143,-1.6282831790329)(1.9140625,-1.62708600061388)(1.91420200892857,-1.6258893807297)(1.91434151785714,-1.62469332198083)(1.91448102678571,-1.62349782696666)(1.91462053571429,-1.62230289828546)(1.91476004464286,-1.62110853853442)(1.91489955357143,-1.61991475030962)(1.9150390625,-1.61872153620601)(1.91517857142857,-1.61752889881744)(1.91531808035714,-1.61633684073662)(1.91545758928571,-1.61514536455517)(1.91559709821429,-1.61395447286352)(1.91573660714286,-1.61276416825101)(1.91587611607143,-1.61157445330582)(1.916015625,-1.61038533061499)(1.91615513392857,-1.60919680276439)(1.91629464285714,-1.60800887233875)(1.91643415178571,-1.60682154192163)(1.91657366071429,-1.60563481409541)(1.91671316964286,-1.60444869144133)(1.91685267857143,-1.60326317653942)(1.9169921875,-1.60207827196854)(1.91713169642857,-1.60089398030636)(1.91727120535714,-1.59971030412937)(1.91741071428571,-1.59852724601284)(1.91755022321429,-1.59734480853084)(1.91768973214286,-1.59616299425626)(1.91782924107143,-1.59498180576074)(1.91796875,-1.59380124561472)(1.91810825892857,-1.59262131638743)(1.91824776785714,-1.59144202064684)(1.91838727678571,-1.59026336095972)(1.91852678571429,-1.58908533989158)(1.91866629464286,-1.58790796000669)(1.91880580357143,-1.58673122386808)(1.9189453125,-1.58555513403753)(1.91908482142857,-1.58437969307555)(1.91922433035714,-1.58320490354139)(1.91936383928571,-1.58203076799303)(1.91950334821429,-1.5808572889872)(1.91964285714286,-1.57968446907931)(1.91978236607143,-1.57851231082353)(1.919921875,-1.57734081677271)(1.92006138392857,-1.57616998947844)(1.92020089285714,-1.57499983149097)(1.92034040178571,-1.57383034535929)(1.92047991071429,-1.57266153363105)(1.92061941964286,-1.57149339885261)(1.92075892857143,-1.57032594356899)(1.9208984375,-1.5691591703239)(1.92103794642857,-1.56799308165975)(1.92117745535714,-1.56682768011755)(1.92131696428571,-1.56566296823705)(1.92145647321429,-1.56449894855659)(1.92159598214286,-1.56333562361322)(1.92173549107143,-1.56217299594259)(1.921875,-1.56101106807903)(1.92201450892857,-1.55984984255548)(1.92215401785714,-1.55868932190353)(1.92229352678571,-1.5575295086534)(1.92243303571429,-1.55637040533392)(1.92257254464286,-1.55521201447255)(1.92271205357143,-1.55405433859535)(1.9228515625,-1.552897380227)(1.92299107142857,-1.55174114189079)(1.92313058035714,-1.55058562610859)(1.92327008928571,-1.54943083540087)(1.92340959821429,-1.54827677228669)(1.92354910714286,-1.54712343928371)(1.92368861607143,-1.54597083890813)(1.923828125,-1.54481897367478)(1.92396763392857,-1.543667846097)(1.92410714285714,-1.54251745868674)(1.92424665178571,-1.54136781395448)(1.92438616071429,-1.54021891440928)(1.92452566964286,-1.53907076255871)(1.92466517857143,-1.53792336090894)(1.9248046875,-1.53677671196463)(1.92494419642857,-1.535630818229)(1.92508370535714,-1.5344856822038)(1.92522321428571,-1.53334130638931)(1.92536272321429,-1.53219769328429)(1.92550223214286,-1.53105484538608)(1.92564174107143,-1.52991276519048)(1.92578125,-1.52877145519182)(1.92592075892857,-1.52763091788292)(1.92606026785714,-1.5264911557551)(1.92619977678571,-1.52535217129818)(1.92633928571429,-1.52421396700045)(1.92647879464286,-1.52307654534868)(1.92661830357143,-1.52193990882815)(1.9267578125,-1.52080405992258)(1.92689732142857,-1.51966900111416)(1.92703683035714,-1.51853473488354)(1.92717633928571,-1.51740126370986)(1.92731584821429,-1.51626859007067)(1.92745535714286,-1.51513671644198)(1.92759486607143,-1.51400564529826)(1.927734375,-1.51287537911241)(1.92787388392857,-1.51174592035575)(1.92801339285714,-1.51061727149805)(1.92815290178571,-1.50948943500748)(1.92829241071429,-1.50836241335064)(1.92843191964286,-1.50723620899256)(1.92857142857143,-1.50611082439665)(1.9287109375,-1.50498626202475)(1.92885044642857,-1.50386252433708)(1.92898995535714,-1.50273961379227)(1.92912946428571,-1.50161753284733)(1.92926897321429,-1.50049628395766)(1.92940848214286,-1.49937586957705)(1.92954799107143,-1.49825629215764)(1.9296875,-1.49713755414996)(1.92982700892857,-1.49601965800291)(1.92996651785714,-1.49490260616374)(1.93010602678571,-1.49378640107807)(1.93024553571429,-1.49267104518985)(1.93038504464286,-1.49155654094139)(1.93052455357143,-1.49044289077336)(1.9306640625,-1.48933009712473)(1.93080357142857,-1.48821816243284)(1.93094308035714,-1.48710708913334)(1.93108258928571,-1.4859968796602)(1.93122209821429,-1.48488753644571)(1.93136160714286,-1.48377906192048)(1.93150111607143,-1.48267145851343)(1.931640625,-1.48156472865179)(1.93178013392857,-1.48045887476107)(1.93191964285714,-1.4793538992651)(1.93205915178571,-1.47824980458597)(1.93219866071429,-1.47714659314409)(1.93233816964286,-1.47604426735812)(1.93247767857143,-1.47494282964502)(1.9326171875,-1.47384228242001)(1.93275669642857,-1.47274262809657)(1.93289620535714,-1.47164386908647)(1.93303571428571,-1.4705460077997)(1.93317522321429,-1.46944904664453)(1.93331473214286,-1.46835298802746)(1.93345424107143,-1.46725783435326)(1.93359375,-1.46616358802491)(1.93373325892857,-1.46507025144364)(1.93387276785714,-1.4639778270089)(1.93401227678571,-1.46288631711837)(1.93415178571429,-1.46179572416795)(1.93429129464286,-1.46070605055176)(1.93443080357143,-1.45961729866212)(1.9345703125,-1.45852947088956)(1.93470982142857,-1.45744256962283)(1.93484933035714,-1.45635659724884)(1.93498883928571,-1.45527155615272)(1.93512834821429,-1.45418744871778)(1.93526785714286,-1.45310427732551)(1.93540736607143,-1.45202204435558)(1.935546875,-1.45094075218584)(1.93568638392857,-1.4498604031923)(1.93582589285714,-1.44878099974914)(1.93596540178571,-1.4477025442287)(1.93610491071429,-1.44662503900145)(1.93624441964286,-1.44554848643606)(1.93638392857143,-1.44447288889931)(1.9365234375,-1.44339824875612)(1.93666294642857,-1.44232456836957)(1.93680245535714,-1.44125185010086)(1.93694196428571,-1.4401800963093)(1.93708147321429,-1.43910930935235)(1.93722098214286,-1.43803949158558)(1.93736049107143,-1.43697064536266)(1.9375,-1.43590277303539)(1.93763950892857,-1.43483587695366)(1.93777901785714,-1.43376995946546)(1.93791852678571,-1.43270502291689)(1.93805803571429,-1.43164106965213)(1.93819754464286,-1.43057810201343)(1.93833705357143,-1.42951612234116)(1.9384765625,-1.42845513297374)(1.93861607142857,-1.42739513624766)(1.93875558035714,-1.4263361344975)(1.93889508928571,-1.42527813005587)(1.93903459821429,-1.42422112525348)(1.93917410714286,-1.42316512241906)(1.93931361607143,-1.4221101238794)(1.939453125,-1.42105613195936)(1.93959263392857,-1.4200031489818)(1.93973214285714,-1.41895117726764)(1.93987165178571,-1.41790021913583)(1.94001116071429,-1.41685027690334)(1.94015066964286,-1.41580135288517)(1.94029017857143,-1.41475344939434)(1.9404296875,-1.41370656874187)(1.94056919642857,-1.41266071323682)(1.94070870535714,-1.41161588518621)(1.94084821428571,-1.41057208689509)(1.94098772321429,-1.40952932066651)(1.94112723214286,-1.40848758880149)(1.94126674107143,-1.40744689359905)(1.94140625,-1.4064072373562)(1.94154575892857,-1.40536862236791)(1.94168526785714,-1.40433105092713)(1.94182477678571,-1.40329452532479)(1.94196428571429,-1.40225904784976)(1.94210379464286,-1.4012246207889)(1.94224330357143,-1.400191246427)(1.9423828125,-1.39915892704681)(1.94252232142857,-1.39812766492905)(1.94266183035714,-1.39709746235234)(1.94280133928571,-1.39606832159327)(1.94294084821429,-1.39504024492634)(1.94308035714286,-1.394013234624)(1.94321986607143,-1.39298729295662)(1.943359375,-1.39196242219248)(1.94349888392857,-1.39093862459778)(1.94363839285714,-1.38991590243663)(1.94377790178571,-1.38889425797107)(1.94391741071429,-1.38787369346099)(1.94405691964286,-1.38685421116424)(1.94419642857143,-1.38583581333652)(1.9443359375,-1.38481850223143)(1.94447544642857,-1.38380228010047)(1.94461495535714,-1.38278714919301)(1.94475446428571,-1.38177311175628)(1.94489397321429,-1.38076017003541)(1.94503348214286,-1.37974832627338)(1.94517299107143,-1.37873758271103)(1.9453125,-1.37772794158708)(1.94545200892857,-1.37671940513809)(1.94559151785714,-1.37571197559846)(1.94573102678571,-1.37470565520046)(1.94587053571429,-1.37370044617417)(1.94601004464286,-1.37269635074754)(1.94614955357143,-1.37169337114633)(1.9462890625,-1.37069150959413)(1.94642857142857,-1.36969076831237)(1.94656808035714,-1.36869114952029)(1.94670758928571,-1.36769265543494)(1.94684709821429,-1.36669528827118)(1.94698660714286,-1.36569905024169)(1.94712611607143,-1.36470394355694)(1.947265625,-1.36370997042521)(1.94740513392857,-1.36271713305256)(1.94754464285714,-1.36172543364284)(1.94768415178571,-1.3607348743977)(1.94782366071429,-1.35974545751657)(1.94796316964286,-1.35875718519663)(1.94810267857143,-1.35777005963286)(1.9482421875,-1.356784083018)(1.94838169642857,-1.35579925754255)(1.94852120535714,-1.35481558539478)(1.94866071428571,-1.3538330687607)(1.94880022321429,-1.35285170982408)(1.94893973214286,-1.35187151076644)(1.94907924107143,-1.35089247376703)(1.94921875,-1.34991460100286)(1.94935825892857,-1.34893789464865)(1.94949776785714,-1.34796235687687)(1.94963727678571,-1.34698798985769)(1.94977678571429,-1.34601479575904)(1.94991629464286,-1.34504277674652)(1.95005580357143,-1.34407193498348)(1.9501953125,-1.34310227263097)(1.95033482142857,-1.34213379184774)(1.95047433035714,-1.34116649479023)(1.95061383928571,-1.34020038361261)(1.95075334821429,-1.3392354604667)(1.95089285714286,-1.33827172750203)(1.95103236607143,-1.33730918686581)(1.951171875,-1.33634784070294)(1.95131138392857,-1.33538769115597)(1.95145089285714,-1.33442874036516)(1.95159040178571,-1.33347099046839)(1.95172991071429,-1.33251444360124)(1.95186941964286,-1.33155910189692)(1.95200892857143,-1.33060496748633)(1.9521484375,-1.32965204249798)(1.95228794642857,-1.32870032905806)(1.95242745535714,-1.32774982929038)(1.95256696428571,-1.3268005453164)(1.95270647321429,-1.3258524792552)(1.95284598214286,-1.32490563322351)(1.95298549107143,-1.32396000933566)(1.953125,-1.32301560970363)(1.95326450892857,-1.322072436437)(1.95340401785714,-1.32113049164296)(1.95354352678571,-1.32018977742632)(1.95368303571429,-1.31925029588948)(1.95382254464286,-1.31831204913247)(1.95396205357143,-1.31737503925288)(1.9541015625,-1.31643926834592)(1.95424107142857,-1.31550473850437)(1.95438058035714,-1.31457145181862)(1.95452008928571,-1.31363941037661)(1.95465959821429,-1.31270861626387)(1.95479910714286,-1.3117790715635)(1.95493861607143,-1.31085077835619)(1.955078125,-1.30992373872016)(1.95521763392857,-1.30899795473121)(1.95535714285714,-1.30807342846269)(1.95549665178571,-1.3071501619855)(1.95563616071429,-1.3062281573681)(1.95577566964286,-1.30530741667648)(1.95591517857143,-1.30438794197418)(1.9560546875,-1.30346973532226)(1.95619419642857,-1.30255279877935)(1.95633370535714,-1.30163713440156)(1.95647321428571,-1.30072274424255)(1.95661272321429,-1.2998096303535)(1.95675223214286,-1.2988977947831)(1.95689174107143,-1.29798723957756)(1.95703125,-1.29707796678059)(1.95717075892857,-1.2961699784334)(1.95731026785714,-1.29526327657471)(1.95744977678571,-1.29435786324073)(1.95758928571429,-1.29345374046517)(1.95772879464286,-1.29255091027922)(1.95786830357143,-1.29164937471156)(1.9580078125,-1.29074913578835)(1.95814732142857,-1.28985019553321)(1.95828683035714,-1.28895255596727)(1.95842633928571,-1.28805621910908)(1.95856584821429,-1.28716118697469)(1.95870535714286,-1.28626746157761)(1.95884486607143,-1.28537504492878)(1.958984375,-1.28448393903662)(1.95912388392857,-1.28359414590699)(1.95926339285714,-1.28270566754319)(1.95940290178571,-1.28181850594596)(1.95954241071429,-1.28093266311349)(1.95968191964286,-1.28004814104139)(1.95982142857143,-1.2791649417227)(1.9599609375,-1.27828306714791)(1.96010044642857,-1.27740251930489)(1.96023995535714,-1.27652330017896)(1.96037946428571,-1.27564541175284)(1.96051897321429,-1.27476885600667)(1.96065848214286,-1.27389363491799)(1.96079799107143,-1.27301975046173)(1.9609375,-1.27214720461024)(1.96107700892857,-1.27127599933326)(1.96121651785714,-1.27040613659791)(1.96135602678571,-1.26953761836871)(1.96149553571429,-1.26867044660754)(1.96163504464286,-1.2678046232737)(1.96177455357143,-1.26694015032383)(1.9619140625,-1.26607702971195)(1.96205357142857,-1.26521526338945)(1.96219308035714,-1.2643548533051)(1.96233258928571,-1.26349580140501)(1.96247209821429,-1.26263810963264)(1.96261160714286,-1.26178177992883)(1.96275111607143,-1.26092681423175)(1.962890625,-1.26007321447693)(1.96303013392857,-1.25922098259721)(1.96316964285714,-1.25837012052281)(1.96330915178571,-1.25752063018125)(1.96344866071429,-1.25667251349739)(1.96358816964286,-1.25582577239344)(1.96372767857143,-1.25498040878889)(1.9638671875,-1.25413642460059)(1.96400669642857,-1.25329382174267)(1.96414620535714,-1.2524526021266)(1.96428571428571,-1.25161276766115)(1.96442522321429,-1.25077432025238)(1.96456473214286,-1.24993726180366)(1.96470424107143,-1.24910159421566)(1.96484375,-1.24826731938635)(1.96498325892857,-1.24743443921097)(1.96512276785714,-1.24660295558206)(1.96526227678571,-1.24577287038943)(1.96540178571429,-1.24494418552018)(1.96554129464286,-1.24411690285869)(1.96568080357143,-1.24329102428658)(1.9658203125,-1.24246655168278)(1.96595982142857,-1.24164348692345)(1.96609933035714,-1.24082183188202)(1.96623883928571,-1.24000158842919)(1.96637834821429,-1.23918275843288)(1.96651785714286,-1.2383653437583)(1.96665736607143,-1.23754934626789)(1.966796875,-1.2367347678213)(1.96693638392857,-1.23592161027547)(1.96707589285714,-1.23510987548454)(1.96721540178571,-1.23429956529989)(1.96735491071429,-1.23349068157013)(1.96749441964286,-1.23268322614109)(1.96763392857143,-1.23187720085583)(1.9677734375,-1.23107260755461)(1.96791294642857,-1.23026944807491)(1.96805245535714,-1.22946772425142)(1.96819196428571,-1.22866743791605)(1.96833147321429,-1.22786859089788)(1.96847098214286,-1.22707118502323)(1.96861049107143,-1.22627522211557)(1.96875,-1.2254807039956)(1.96888950892857,-1.22468763248118)(1.96902901785714,-1.22389600938738)(1.96916852678571,-1.22310583652643)(1.96930803571429,-1.22231711570774)(1.96944754464286,-1.22152984873792)(1.96958705357143,-1.2207440374207)(1.9697265625,-1.21995968355703)(1.96986607142857,-1.219176788945)(1.97000558035714,-1.21839535537984)(1.97014508928571,-1.21761538465397)(1.97028459821429,-1.21683687855695)(1.97042410714286,-1.21605983887549)(1.97056361607143,-1.21528426739343)(1.970703125,-1.21451016589179)(1.97084263392857,-1.21373753614869)(1.97098214285714,-1.2129663799394)(1.97112165178571,-1.21219669903633)(1.97126116071429,-1.21142849520901)(1.97140066964286,-1.21066177022411)(1.97154017857143,-1.20989652584539)(1.9716796875,-1.20913276383375)(1.97181919642857,-1.20837048594722)(1.97195870535714,-1.2076096939409)(1.97209821428571,-1.20685038956705)(1.97223772321429,-1.20609257457498)(1.97237723214286,-1.20533625071114)(1.97251674107143,-1.20458141971907)(1.97265625,-1.20382808333939)(1.97279575892857,-1.20307624330983)(1.97293526785714,-1.20232590136519)(1.97307477678571,-1.20157705923738)(1.97321428571429,-1.20082971865536)(1.97335379464286,-1.20008388134519)(1.97349330357143,-1.19933954902998)(1.9736328125,-1.19859672342995)(1.97377232142857,-1.19785540626235)(1.97391183035714,-1.19711559924152)(1.97405133928571,-1.19637730407883)(1.97419084821429,-1.19564052248275)(1.97433035714286,-1.19490525615876)(1.97446986607143,-1.19417150680943)(1.974609375,-1.19343927613435)(1.97474888392857,-1.19270856583017)(1.97488839285714,-1.19197937759056)(1.97502790178571,-1.19125171310627)(1.97516741071429,-1.19052557406503)(1.97530691964286,-1.18980096215164)(1.97544642857143,-1.18907787904791)(1.9755859375,-1.18835632643269)(1.97572544642857,-1.18763630598184)(1.97586495535714,-1.18691781936823)(1.97600446428571,-1.18620086826176)(1.97614397321429,-1.18548545432934)(1.97628348214286,-1.18477157923488)(1.97642299107143,-1.18405924463929)(1.9765625,-1.18334845220051)(1.97670200892857,-1.18263920357344)(1.97684151785714,-1.18193150041001)(1.97698102678571,-1.18122534435912)(1.97712053571429,-1.18052073706666)(1.97726004464286,-1.17981768017552)(1.97739955357143,-1.17911617532556)(1.9775390625,-1.17841622415362)(1.97767857142857,-1.17771782829352)(1.97781808035714,-1.17702098937607)(1.97795758928571,-1.176325709029)(1.97809709821429,-1.17563198887707)(1.97823660714286,-1.17493983054195)(1.97837611607143,-1.17424923564229)(1.978515625,-1.17356020579372)(1.97865513392857,-1.17287274260879)(1.97879464285714,-1.17218684769702)(1.97893415178571,-1.17150252266487)(1.97907366071429,-1.17081976911574)(1.97921316964286,-1.17013858864998)(1.97935267857143,-1.16945898286488)(1.9794921875,-1.16878095335466)(1.97963169642857,-1.16810450171048)(1.97977120535714,-1.16742962952041)(1.97991071428571,-1.16675633836947)(1.98005022321429,-1.16608462983958)(1.98018973214286,-1.1654145055096)(1.98032924107143,-1.1647459669553)(1.98046875,-1.16407901574935)(1.98060825892857,-1.16341365346136)(1.98074776785714,-1.16274988165781)(1.98088727678571,-1.16208770190212)(1.98102678571429,-1.16142711575459)(1.98116629464286,-1.16076812477242)(1.98130580357143,-1.16011073050972)(1.9814453125,-1.15945493451747)(1.98158482142857,-1.15880073834356)(1.98172433035714,-1.15814814353275)(1.98186383928571,-1.15749715162669)(1.98200334821429,-1.15684776416392)(1.98214285714286,-1.15619998267984)(1.98228236607143,-1.15555380870674)(1.982421875,-1.15490924377377)(1.98256138392857,-1.15426628940696)(1.98270089285714,-1.15362494712918)(1.98284040178571,-1.1529852184602)(1.98297991071429,-1.15234710491661)(1.98311941964286,-1.15171060801189)(1.98325892857143,-1.15107572925636)(1.9833984375,-1.15044247015719)(1.98353794642857,-1.14981083221839)(1.98367745535714,-1.14918081694083)(1.98381696428571,-1.14855242582221)(1.98395647321429,-1.14792566035708)(1.98409598214286,-1.14730052203682)(1.98423549107143,-1.14667701234963)(1.984375,-1.14605513278057)(1.98451450892857,-1.1454348848115)(1.98465401785714,-1.14481626992112)(1.98479352678571,-1.14419928958495)(1.98493303571429,-1.14358394527531)(1.98507254464286,-1.14297023846136)(1.98521205357143,-1.14235817060907)(1.9853515625,-1.14174774318121)(1.98549107142857,-1.14113895763736)(1.98563058035714,-1.14053181543391)(1.98577008928571,-1.13992631802404)(1.98590959821429,-1.13932246685774)(1.98604910714286,-1.13872026338179)(1.98618861607143,-1.13811970903978)(1.986328125,-1.13752080527205)(1.98646763392857,-1.13692355351577)(1.98660714285714,-1.13632795520488)(1.98674665178571,-1.1357340117701)(1.98688616071429,-1.13514172463891)(1.98702566964286,-1.13455109523561)(1.98716517857143,-1.13396212498123)(1.9873046875,-1.13337481529361)(1.98744419642857,-1.13278916758732)(1.98758370535714,-1.13220518327372)(1.98772321428571,-1.13162286376093)(1.98786272321429,-1.13104221045382)(1.98800223214286,-1.13046322475403)(1.98814174107143,-1.12988590805993)(1.98828125,-1.12931026176668)(1.98842075892857,-1.12873628726616)(1.98856026785714,-1.128163985947)(1.98869977678571,-1.12759335919458)(1.98883928571429,-1.12702440839102)(1.98897879464286,-1.12645713491517)(1.98911830357143,-1.12589154014263)(1.9892578125,-1.12532762544573)(1.98939732142857,-1.12476539219351)(1.98953683035714,-1.12420484175177)(1.98967633928571,-1.123645975483)(1.98981584821429,-1.12308879474643)(1.98995535714286,-1.12253330089803)(1.99009486607143,-1.12197949529044)(1.990234375,-1.12142737927305)(1.99037388392857,-1.12087695419196)(1.99051339285714,-1.12032822138996)(1.99065290178571,-1.11978118220656)(1.99079241071429,-1.11923583797797)(1.99093191964286,-1.1186921900371)(1.99107142857143,-1.11815023971357)(1.9912109375,-1.11760998833367)(1.99135044642857,-1.11707143722042)(1.99148995535714,-1.11653458769349)(1.99162946428571,-1.11599944106927)(1.99176897321429,-1.11546599866082)(1.99190848214286,-1.11493426177789)(1.99204799107143,-1.11440423172691)(1.9921875,-1.11387590981098)(1.99232700892857,-1.11334929732989)(1.99246651785714,-1.11282439558009)(1.99260602678571,-1.11230120585471)(1.99274553571429,-1.11177972944353)(1.99288504464286,-1.11125996763302)(1.99302455357143,-1.1107419217063)(1.9931640625,-1.11022559294315)(1.99330357142857,-1.10971098262001)(1.99344308035714,-1.10919809200997)(1.99358258928571,-1.10868692238279)(1.99372209821429,-1.10817747500486)(1.99386160714286,-1.10766975113923)(1.99400111607143,-1.10716375204558)(1.994140625,-1.10665947898027)(1.99428013392857,-1.10615693319625)(1.99441964285714,-1.10565611594316)(1.99455915178571,-1.10515702846722)(1.99469866071429,-1.10465967201134)(1.99483816964286,-1.10416404781502)(1.99497767857143,-1.10367015711441)(1.9951171875,-1.10317800114228)(1.99525669642857,-1.10268758112801)(1.99539620535714,-1.10219889829763)(1.99553571428571,-1.10171195387377)(1.99567522321429,-1.10122674907567)(1.99581473214286,-1.1007432851192)(1.99595424107143,-1.10026156321684)(1.99609375,-1.09978158457766)(1.99623325892857,-1.09930335040738)(1.99637276785714,-1.09882686190827)(1.99651227678571,-1.09835212027924)(1.99665178571429,-1.0978791267158)(1.99679129464286,-1.09740788241003)(1.99693080357143,-1.09693838855063)(1.9970703125,-1.09647064632289)(1.99720982142857,-1.09600465690868)(1.99734933035714,-1.09554042148648)(1.99748883928571,-1.09507794123132)(1.99762834821429,-1.09461721731485)(1.99776785714286,-1.09415825090529)(1.99790736607143,-1.09370104316743)(1.998046875,-1.09324559526264)(1.99818638392857,-1.09279190834888)(1.99832589285714,-1.09233998358066)(1.99846540178571,-1.09188982210909)(1.99860491071429,-1.09144142508181)(1.99874441964286,-1.09099479364305)(1.99888392857143,-1.09054992893361)(1.9990234375,-1.09010683209083)(1.99916294642857,-1.08966550424863)(1.99930245535714,-1.08922594653747)(1.99944196428571,-1.08878816008437)(1.99958147321429,-1.08835214601291)(1.99972098214286,-1.08791790544322)(1.99986049107143,-1.08748543949197)(2,-1.08705474927237) 
};
\addlegendentry{@(x) sin (10 * x) - x};

\end{axis}
\end{tikzpicture}%
  }
  \caption{$f(x) = sin(10x) - x$}

\end{figure}
Figure 1 shows that the function $f(x) = sin(10x) - x$ has 7 zeros.
\part
\lstinputlisting{./hw5_02/bisection.m}
\lstinputlisting{./hw5_02/hw5_02.m}

%%%%%%%%%%%% PROBLEM #3 %%%%%%%%%%%%%%%%%%%%%%%%%
\problem{Exercise 5.8 in Scientific Computing, page 250}
Consider the problem of finding the smallest positive root of the nonlinear equation
\(
cos(x) + 1/(1+e^{-2x}) = 0
\).

Investigate, both theoretically and epiracllly, the following iterative schemes for solving this problem using the starting point 
$x_0=3$. For each scheme, you should know that it is indeed an quivalent fixed-point problem, determine analytically
wheter it is locally convergent and its expected convergence rate, and then implement the method to confirm your results.
\begin{enumerate}
  \item $x_{k+1} = arccos(-1/(1+e^{-2x_k}))$ 
  \item $x_{k+1} = 0.5log(-1/(1+1/cos(x_k)))$
  \item Newton's method
\end{enumerate}
\solution
\part 
\begin{eqnarray*}
cos(x) &+& \frac{1}{1 + e^{-2x}} = 0 \\
cos(x) &=& \frac{-1}{1+e^{-2x}}
\end{eqnarray*}
Looking for $x_{k+1} = g(x)$.
\[
x = cos^{-1}\left(\frac{-1}{1+e^{-2x}}\right)
\]
so \[g(x) = cos^{-1}\left(\frac{-1}{1+e^{-2x}}\right)\]
\begin{eqnarray*}
g^\prime(x) &=& \frac{2e^{2x}}{(e^(2x) + 1)^2\sqrt{\frac{1}{1 - (e^{-2x} + 1)^2}}} \\
g^\prime(x) &=& 0.07
\end{eqnarray*}
This iteration is converges linearly, with $C = 0.07$.

\lstinputlisting{./hw5_03/fixed_point_iteration.m}
\lstinputlisting{./hw5_03/tex_a.m}

\part
\begin{eqnarray*}
  cos(x) &+& \frac{1}{1 + e^{-2x}} = 0 \\
  - cos(x) &=& \frac{1}{1 + e^{-2x}} \\
  1 + e^{-2x} &=& \frac{-1}{cos(x)} \\
  e^{-2x} &=& \frac{-1}{cos(x)} - 1 \\
  &=& \frac{-1 + cos(x)}{cos(x)} \\
  e^{2x} &=& \frac{-cos(x)}{1+cos(x)} \\
  &=& \frac{-1}{1 + \frac{1}{cos(x)}} \\
  2x &=& ln\left(\frac{-1}{1 + \frac{1}{cos(x)}}\right) \\
  x  &=& 0.5ln\left(\frac{-1}{1 + \frac{1}{cos(x)}}\right) 
\end{eqnarray*}
\begin{eqnarray*}
g(x)  &=& 0.5ln\left(\frac{-1}{1 + \frac{1}{cos(x)}}\right) \\
g^\prime(x)  &=& -0.5\frac{tan(x)sec(x)}{sec(x) + 1} \\
\end{eqnarray*}
Since $g^\prime(3) = 7.123$, the iteration diverges.
\lstinputlisting{./hw5_03/tex_b.m}

\part 
\[
x_{k+1} = x_{k} - \frac{f(x_k}{f^\prime(x_k)}
\]

\begin{eqnarray*}
  f(x) &=& cos(x) + \frac{1}{1+e^{-2x}} \\
  f^\prime(x) &=& -sin(x) + \frac{2e^{-2x}}{(1+e^{-2x})^2} \\
  f^{\prime \prime}(x) &=& -cos(x) - \frac{4e^{2x}\left(e^{2x} - 1\right)}{\left(e^{2x} + 1\right)^3}
\end{eqnarray*}
\begin{eqnarray*}
  g^\prime(x) &=& \frac{f(x)f^{\prime \prime}(x)}{f^\prime(x)^2} \\
  &=& 0.38
\end{eqnarray*}
This iteration converges linearly with $C = 0.38$.
%%%%%%%%%%%% PROBLEM #4 %%%%%%%%%%%%%%%%%%%%%%%%%
\problem{Exercise 5.12 in Scientific Computing, page 251}
The vertical distance $y$ that a parachutist falls before opening the parachute is given by the equation 
$y = log(cosh(t\sqrt{gk}))/k$ where t is the elapsed time in seconds, $g=9.8065m/s^2$ is the acceleration
due to gravity, and $k = 0.00341m^{-1}$ is a constant related to the air resistance. Use a zero finder to 
determine the elapased time requred to fall a distance of 1 km.
\solution
\lstinputlisting{./hw5_04/hw5_04.m}
The elapsed time required to fall a distance of $1$ km is $22.436$ seconds.
%%%%%%%%%%%% PROBLEM #5 %%%%%%%%%%%%%%%%%%%%%%%%%
\problem{Exercise 5.18 in Scientific Computing, page 252}
\begin{enumerate}
  \item Write a routine based on Newton's method to solve the system of nonlinear equations
	\begin{eqnarray*}
	  (x_1 + 3)(x_2^3 - 7) + 18 &=& 0 \\
	  sin(x_2e^{x_1} - 1) &=& 0
	\end{eqnarray*}
  \item Write a routine based on Broyden's method to solve the same system with the same starting point.
  \item Compare the convergence rates of the two methods by computing the error at each iteration, given 
	that the exact solution $x^* = [0, 1]^T$. How many iterations does each method require to attain full
	machine precision?
\end{enumerate}

\solution
\part
\begin{eqnarray*}
  f_1(x_1,x_2) &=& (x_1 + 3)(x_2-y) + 18 = 0 \\
  &=& x_1x_2^3 - 7x_1 + 3x_2^3 - 3 = 0 \\
\end{eqnarray*}
\begin{eqnarray*}
  \frac{\partial f_1}{\partial x_1} &=& x_2^3 - 7 \\
  \frac{\partial f_1}{\partial x_2} &=& 3x_1x_2^2 + 9x_2^2
\end{eqnarray*}

\[
  f_2(x_1,x_2) = sin\left(x_2e^{x_1} - 1\right) = 0
\]

\begin{eqnarray*}
  \frac{\partial f_2}{\partial x_1} &=& x_2 e^{x_1} cos\left(x_2e^{x_1} - 1\right) \\
  \frac{\partial f_2}{\partial x_2} &=& e^{x_1} cos\left(x_2e^{x_1} - 1\right)
\end{eqnarray*}
\lstinputlisting{./hw5_05/tex_a.m}

\part 
\lstinputlisting{./hw5_05/tex_b.m}
\part
\begin{figure}[H]
  \label{fighw505}
  \centering
  \scalebox{.5}{
  \setlength\figureheight{12cm}
  \setlength\figurewidth{\linewidth} 
  % This file was created by matlab2tikz v0.2.3.
% Copyright (c) 2008--2012, Nico Schlömer <nico.schloemer@gmail.com>
% All rights reserved.
% 
% The latest updates can be retrieved from
%   http://www.mathworks.com/matlabcentral/fileexchange/22022-matlab2tikz
% where you can also make suggestions and rate matlab2tikz.
% 
% 
% 
\begin{tikzpicture}

\begin{axis}[%
view={0}{90},
width=\figurewidth,
height=\figureheight,
scale only axis,
xmin=0, xmax=10,
xlabel={Iterations},
xmajorgrids,
ymin=0, ymax=20,
ylabel={Digits},
ymajorgrids,
title={Correct Significant Digits},
legend style={nodes=left,legend plot pos=right}]
\addplot [
color=blue,
solid
]
coordinates{
 (1,1.00639743044049)(2,2.53215746482295)(3,5.99532589098157)(4,12.2350888998282)(5,15.9076867327211)(6,16.2435879728467)(7,17.0203225613916)(8,17.0203225613916)(9,17.0203225613916)(10,17.0203225613916) 
};
\addlegendentry{Newton's};

\addplot [
color=red,
solid
]
coordinates{
 (1,1.00639743044049)(2,1.91782464197644)(3,2.61464073985718)(4,3.86175908436479)(5,5.9422920547507)(6,7.52863491279738)(7,9.26067477518593)(8,12.7413894831584)(9,16.6798691384422)(10,16.6798691384422) 
};
\addlegendentry{Broyden's};

\addplot [
color=green,
solid
]
coordinates{
 (1,15.653559774527)(2,15.653559774527)(3,15.653559774527)(4,15.653559774527)(5,15.653559774527)(6,15.653559774527)(7,15.653559774527)(8,15.653559774527)(9,15.653559774527)(10,15.653559774527) 
};
\addlegendentry{$\epsilon_{mach}$};

\end{axis}
\end{tikzpicture}%

  }
\end{figure}
Newton's method converges to within $\epsilon_{mach}$ of the correct solution by the 5th iteration. Broyden's method takes
9 iterations.

\end{document}

