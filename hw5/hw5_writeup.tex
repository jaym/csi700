\documentclass{jhwhw}
\usepackage{listings}
\usepackage[T1]{fontenc}
\usepackage{xcolor}
\usepackage{color}
\usepackage{moreverb}
\usepackage{paralist}
\usepackage{multicol}
\usepackage{tikz,pgfplots}
\usepackage{gnuplot-lua-tikz}
\usepackage{adjustbox}
\usepackage{float}

\definecolor{MyDarkGreen}{rgb}{0.0,0.4,0.0}
\lstloadlanguages{Matlab}%

\newfloat{Code}{H}{myc}

\definecolor{solarized@base03}{HTML}{002B36}
\definecolor{solarized@base02}{HTML}{073642}
\definecolor{solarized@base01}{HTML}{586e75}
\definecolor{solarized@base00}{HTML}{657b83}
\definecolor{solarized@base0}{HTML}{839496}
\definecolor{solarized@base1}{HTML}{93a1a1}
\definecolor{solarized@base2}{HTML}{EEE8D5}
\definecolor{solarized@base3}{HTML}{FDF6E3}
\definecolor{solarized@yellow}{HTML}{B58900}
\definecolor{solarized@orange}{HTML}{CB4B16}
\definecolor{solarized@red}{HTML}{DC322F}
\definecolor{solarized@magenta}{HTML}{D33682}
\definecolor{solarized@violet}{HTML}{6C71C4}
\definecolor{solarized@blue}{HTML}{268BD2}
\definecolor{solarized@cyan}{HTML}{2AA198}
\definecolor{solarized@green}{HTML}{859900}

\lstset{
  language=Matlab,
  columns=fixed,
  tabsize=2,
  extendedchars=true,
  breaklines=true,
  frame=single,
  numbers=left,
  numbersep=5pt,
  rulesepcolor=\color{solarized@base03},
  numberstyle=\tiny\color{solarized@base01},
  basicstyle=\footnotesize\ttfamily,
  keywordstyle=\color{solarized@green},
  stringstyle=\color{solarized@cyan}\ttfamily,
  identifierstyle=\color{solarized@blue},
  commentstyle=\color{solarized@base01},
  emphstyle=\color{solarized@red},
  showstringspaces=false,
  float=htpb
}

\relpenalty=9999
\binoppenalty=9999

\title{Homework Set 5}%replace X with the appropriate number
\author{Jay Mundrawala}

\begin{document}
%%%%%%%%%%%% PROBLEM #1 %%%%%%%%%%%%%%%%%%%%%%%%%
\problem{Exercise 5.6 in Scientific Computing, page 249}
Suppose we wish to develop an iterative method to compute the square root of a given positive number $y$, i.e., to 
solve the nonlinear equation $f(x) = x^2 -y = 0$ given the value of $y$. Each of the functions $g_1$ and $g_2$
listed next gives a fixed-point problem that is equivalent to the equation $f(x) = 0$. For each of these functions,
determine whether the corresponding fixed-point iteration scheme $x_{k+1} = g_i(x_k)$ is locally convergent to 
$\sqrt{y}$ if $y = 3$. Explain your reasoning in each case.
\begin{enumerate}
	\item $g_1(x) = y + x - x^2$.
	\item $g_2(x) = 1 + x - x^2/y$.
	\item What is the fixed-point iteration function given by Newton's method for this particular problem?
\end{enumerate}
\solution
\part
\begin{eqnarray*}
	g_1(x) &=& 3 + x - x^2 \\
	g_1^\prime (x) &=& -2x + 1
\end{eqnarray*}
The solution is known to lie between $(1,2)$. $g_1^\prime(x)$ takes the values $(-1,-3)$ for $x\epsilon(1,2)$.
This means that the iteration does not locally converge.
\part
\begin{eqnarray*}
	g_1(x) &=& 1 + x - \frac{x^2}{3} \\
	g_1^\prime (x) &=& 1 - \frac{2}{3}x
\end{eqnarray*}
The solution is known to lie between $(1,2)$. $g_2^\prime(x)$ takes the values $(-0.3,0.3)$ for $x\epsilon(1,2)$.
This means that the iteration locally converges.
\part
%\begin{eqnarray*}
	%x_{k+1} &=& x_k - \frac{f(x_k)}{f^\prime(x_k)} \\
	%f(x) - x^2 - y &=& 0 \\
	%f^\prime(x) &=& 2x \\
	%x_{k+1} &=& g(x_k) \\
%\end{eqnarray*}
\begin{eqnarray*}
	g(x_k) &=& x_k - \frac{x_k^2 - y}{2x_k} \\
           &=& x_k - \frac{1}{2}x_k + \frac{y}{2x_k} \\
	       &=& \frac{1}{2}x_k + \frac{y}{2x_k} \\
	       &=& \frac{1}{2}\left(x_k + \frac{y}{x_k}\right) 
\end{eqnarray*}

%%%%%%%%%%%% PROBLEM #2 %%%%%%%%%%%%%%%%%%%%%%%%%
\problem{Exercise 5.1 in Scientific Computing, page 250}
\begin{enumerate}
	\item How many zeroes does the function
		\(
			f(x) = sin(10x) - x
		\)
		have?
	\item Use a library routine or one of your own design to find all of the zeros of this function.
\end{enumerate}

\end{document}

