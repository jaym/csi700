\documentclass{jhwhw}
\usepackage{listings}
\usepackage[T1]{fontenc}
\usepackage{xcolor}
\usepackage{color}
\usepackage{moreverb}

\definecolor{MyDarkGreen}{rgb}{0.0,0.4,0.0}
\lstloadlanguages{Matlab}%

\definecolor{solarized@base03}{HTML}{002B36}
\definecolor{solarized@base02}{HTML}{073642}
\definecolor{solarized@base01}{HTML}{586e75}
\definecolor{solarized@base00}{HTML}{657b83}
\definecolor{solarized@base0}{HTML}{839496}
\definecolor{solarized@base1}{HTML}{93a1a1}
\definecolor{solarized@base2}{HTML}{EEE8D5}
\definecolor{solarized@base3}{HTML}{FDF6E3}
\definecolor{solarized@yellow}{HTML}{B58900}
\definecolor{solarized@orange}{HTML}{CB4B16}
\definecolor{solarized@red}{HTML}{DC322F}
\definecolor{solarized@magenta}{HTML}{D33682}
\definecolor{solarized@violet}{HTML}{6C71C4}
\definecolor{solarized@blue}{HTML}{268BD2}
\definecolor{solarized@cyan}{HTML}{2AA198}
\definecolor{solarized@green}{HTML}{859900}

\lstset{
  language=Matlab,
  columns=fixed,
  tabsize=2,
  extendedchars=true,
  breaklines=true,
  frame=single,
  numbers=left,
  numbersep=5pt,
  rulesepcolor=\color{solarized@base03},
  numberstyle=\tiny\color{solarized@base01},
  basicstyle=\footnotesize\ttfamily,
  keywordstyle=\color{solarized@green},
  stringstyle=\color{solarized@cyan}\ttfamily,
  identifierstyle=\color{solarized@blue},
  commentstyle=\color{solarized@base01},
  emphstyle=\color{solarized@red}
}

\relpenalty=9999
\binoppenalty=9999

\title{Homework Set 2}%replace X with the appropriate number
\author{Jay Mundrawala}

\begin{document}
%%%%%%%%%%%% PROBLEM #1 %%%%%%%%%%%%%%%%%%%%%%%%%
\problem{Scientific Computing 2.7, Page 97}
\[
  A = \begin{bmatrix}
	1        & 1 + \epsilon \\
	1 - \epsilon & 1
  \end{bmatrix}
  \]
  \begin{enumerate}
    \item What is the determinant of \textbf{\textit{A}}?
	\item In floating-point arithmetic, for what range of values of \boldsymbol{\epsilon} will the computed value of the determinant be zero?
	\item What is the LU factorization of \textbf{\textit{A}}?
  \end{enumerate}
\solution
  \part
  \[ |A| = 1 - (1+\epsilon)(1-\epsilon) = \epsilon^2 \]

  \part
  \[ \epsilon < \epsilon_{mach} \]

  \part
  \[
  \underbrace{\begin{bmatrix}
	1             &    0 \\
	-(1-\epsilon)  &    1
  \end{bmatrix}}_{M_1}
  \underbrace{\begin{bmatrix}
	1             &    1 + \epsilon  \\
	1-\epsilon  &    1
  \end{bmatrix}}_{A}
  =
  \underbrace{\begin{bmatrix}
	1             &    0 \\
	(1-\epsilon)  &    1
  \end{bmatrix}}_{U}
  \]
  \[
  M^{-1}MA = M^{-1}U = LU
  \]

  \part
  \textbf{\textit{U}} will be singular when \(\epsilon < \epsilon_{mach}\)

%%%%%%%%%%%% PROBLEM #2 %%%%%%%%%%%%%%%%%%%%%%%%%
\problem{Scientific Computing 2.17, Page 97}
Write out the LU factorization of the following matrix (show both \textbf{\textit{L}} and
\textbf{\textit{U}} matricies explicitly).
\solution
\[
A =
\begin{bmatrix*}[r]
  1    &   -1    &     0 \\
  -1   &    2    &    -1 \\
  0    &   -1    &     1
\end{bmatrix*}
\]

\[
\underbrace{\begin{bmatrix*}[r]
  1    &    0    &     0 \\
  1    &    1    &     0 \\
  0    &    0    &     1 \\
\end{bmatrix*}}_{M_1}
\begin{bmatrix*}[r]
  1    &   -1    &     0 \\
  -1   &    2    &    -1 \\
  0    &   -1    &     1 \\
\end{bmatrix*}
=
\begin{bmatrix*}[r]
  1    &   -1    &     0 \\
  0    &    1    &    -1 \\
  0    &   -1    &     1 \\
\end{bmatrix*}
\]

\[
\underbrace{\begin{bmatrix*}[r]
  1    &    0    &     0 \\
  0    &    1    &     0 \\
  0    &    1    &     1 \\
\end{bmatrix*}}_{M_2}
\underbrace{\begin{bmatrix*}[r]
  1    &   -1    &     0 \\
  0    &    1    &    -1 \\
  0    &   -1    &     1 \\
\end{bmatrix*}}_{M_1A}
=
\underbrace{\begin{bmatrix*}[r]
  1    &   -1    &     0 \\
  0    &    1    &    -1 \\
  0    &    0    &     0 \\
\end{bmatrix*}}_{U}
\]

%%%%%%%%%%%% PROBLEM #3 %%%%%%%%%%%%%%%%%%%%%%%%%
\problem{Scientific Computing 2.18, Page 98}
Prove that the matrix \[
A = \begin{bmatrix*}[r]
	0  &  1 \\
	1  &  0
  \end{bmatrix*}
\] 
has no LU factorization.

\solution
\[
A = \begin{bmatrix*}[r]
  a_{00}  &  a_{01} \\
  a_{10}  &  a_{11}
\end{bmatrix*}
\]
To get an upper triangular matrix \(MA = U\), we need to eliminate \(a_{10}\).
This is done by taking \(- \frac{a_{10}}{a_{00}} 
\begin{bmatrix*}[r]
a_{00} & a_{01}
\end{bmatrix*}
\) and adding it to the second row, leaving the second row equal to
\begin{bmatrix*}[r]
  0 & (a_{01} - \frac{a_{10}}{a_{00}}a_{01})
\end{bmatrix*}
. Because \(a_{00}\) is equal to zero, this set of operations is not possible since 
it would require a division by zero.

%%%%%%%%%%%% PROBLEM #4 %%%%%%%%%%%%%%%%%%%%%%%%%
\problem{Scientific Computer Problem 2.2, Page 100}
\begin{enumerate}
  \item Use a library routine for Gaussian elimination to solve the system
	\( Ax=b \), where
	\[
	A =
	\begin{bmatrix*}[r]
	  2     &    4    &    -2 \\
	  4     &    9    &    -3 \\
	  -2    &   -1    &     7 \\
	\end{bmatrix*}
	\]
	\[
	b = 
	\begin{bmatrix*}[r]
	   2  \\
	   8  \\
	  10
	\end{bmatrix*}
	\]
  \item Use the LU factorization of \textbf{\textit{A}} already computed to solve
	the system \(Ay=b\), where
	\[
	c = 
	\begin{bmatrix*}[r]
	   4  \\
	   8  \\
	  -6
	\end{bmatrix*}
	\]
  \item If the matrix \textbf{\textit{A}} changes so that \(a_{1,2}=2\), use the
	Sherman-Morrison updating technique to compute the new solution 
	\(\textbf{\textit{x}}\) without refactoring the matrix, using the original 
	right-hand-side vector \(\textbf{\texit{b}}\).
\end{enumerate}
\solution
\part
\lstinputlisting{./hw2_04/hw2_04_a.m}
\lstinputlisting{./hw2_04/hw2_04_a_out.m}
\part
\lstinputlisting{./hw2_04/hw2_04_b.m}
\lstinputlisting{./hw2_04/hw2_04_b_out.m}
\part
\lstinputlisting{./hw2_04/hw2_04_c.m}
\lstinputlisting{./hw2_04/hw2_04_c_out.m}

%%%%%%%%%%%% PROBLEM #5 %%%%%%%%%%%%%%%%%%%%%%%%%
\problem{Scientific Computer Problem 2.6, Page 101}
For \(n = 2,3,...,\) generate the Hilber matrix of order n, and also
generate the n-vector \(b = Hx\), where \(x\) is the n-vector with all of its 
     components equal to 1. Use a library routine for Gaussian elimination
	 to solve \(Hx = b\), obtaining an approximate solution \(\hat{x}\). Compute
	 the infinity norm of the residual \(r = b - H\hat{x}\) and of the error
	 \(dx =  \hat{x} - x\), where x is the vector of all ones. How large can
	 you take n before the error is \(100\%\). Also use a condition estimator to
	 obtain \(cond(H)\) for each value of n. Try to charactarize the condition 
     number as a function of n. As n varies, how does the number of correct
     digits in the components of the computed solution relate to the condition
     number of the matrix?
\solution
\lstinputlisting{./hw2_05/hw2_05_a.m}
The largest value n can take is 32 before the error is 100\%. 
The condition number grows exponentially with respect to n. After \(n=12\), 
the matrix is singular to within machine precision. Up until \(n=13\), the number 
of correct digits decreases linearly as n grows.

\problem{}
For this problem, consider the two systems of linear equations:
\(Ax=b\) and \(Ay = c\).
With
\[
A = 
	\begin{bmatrix*}[r]
	  1     &    2    &    -1   &  0 \\
	  3     &    4    &     0   &  1 \\
	  0     &    2    &     5   &  4 \\
	  1     &    2    &     3   &  4 
	\end{bmatrix*}
b = 
	\begin{bmatrix*}[r]
	    2  \\
	   15  \\
	   35  \\
	   30  \\
	\end{bmatrix*}
c = 
	\begin{bmatrix*}[r]
	   -3  \\
	   -3  \\
	    0 \\
	   -3  \\
	\end{bmatrix*}
\]
	\begin{enumerate}
	  \item Calculate the LU factorization of \(A=LU\) by hand. Then calculate the
		exact solutions x and y by solving the resulting triangular systems.
	  \item Calculate the absolute and relative errors \(\hat{x}-x\) and
		\(\hat{y}-y\) of the approximations \( \hat{x} = A \backslash b \) and \(\hat{y}=A \backslash c \),
		computed with the standard solution operation in Matlab.
	\end{enumerate}
\solution
\[
A = 
	\begin{bmatrix*}[r]
	  1     &    2    &    -1   &  0 \\
	  3     &    4    &     0   &  1 \\
	  0     &    2    &     5   &  4 \\
	  1     &    2    &     3   &  4 
	\end{bmatrix*}
\] 
\[
	\underbrace{\begin{bmatrix*}[r]
	  1     &    0    &     0   &  0 \\
	 -3     &    1    &     0   &  0 \\
	  0     &    0    &     1   &  0 \\
	 -1     &    0    &     0   &  1 
   \end{bmatrix*}}_{M_1}
   \underbrace{\begin{bmatrix*}[r]
	  1     &    2    &    -1   &  0 \\
	  3     &    4    &     0   &  1 \\
	  0     &    2    &     5   &  4 \\
	  1     &    2    &     3   &  4 
	\end{bmatrix*}}_{A}
	=
   \underbrace{\begin{bmatrix*}[r]
	  1     &    2    &    -1   &  0 \\
	  0     &   -2    &     3   &  1 \\
	  0     &    2    &     5   &  4 \\
	  0     &    0    &     4   &  4 
	\end{bmatrix*}}_{M_1A}
\]

\[
	\underbrace{\begin{bmatrix*}[r]
	  1     &    0    &     0   &  0 \\
	  3     &    1    &     0   &  0 \\
	  0     &    1    &     1   &  0 \\
	  0     &    0    &     0   &  1 
   \end{bmatrix*}}_{M_2}
   \underbrace{\begin{bmatrix*}[r]
	  1     &    2    &    -1   &  0 \\
	  0     &   -2    &     3   &  1 \\
	  0     &    2    &     5   &  4 \\
	  0     &    0    &     4   &  4 
	\end{bmatrix*}}_{M_1A}
	=
   \underbrace{\begin{bmatrix*}[r]
	  1     &    2    &    -1   &  0 \\
	  0     &   -2    &     3   &  1 \\
	  0     &    0    &     8   &  5 \\
	  0     &    0    &     4   &  4 
	\end{bmatrix*}}_{M_2M_1A}
\]

\[
	\underbrace{\begin{bmatrix*}[r]
	  1     &    0    &     0   &  0 \\
	  0     &    1    &     0   &  0 \\
	  0     &    0    &     1   &  0 \\
	  0     &    0    & -\frac{1}{2} &  1 
   \end{bmatrix*}}_{M_3}
   \underbrace{\begin{bmatrix*}[r]
	  1     &    2    &    -1   &  0 \\
	  0     &   -2    &     3   &  1 \\
	  0     &    0    &     8   &  5 \\
	  0     &    0    &     4   &  4 
	\end{bmatrix*}}_{M_2M_1A}
	=
   \underbrace{\begin{bmatrix*}[r]
	  1     &    2    &    -1   &  0 \\
	  0     &   -2    &     3   &  1 \\
	  0     &    0    &     8   &  5 \\
	  0     &    0    &     0   &  \frac{3}{2} 
	\end{bmatrix*}}_{U}
\]

\[
	L = M^{-1} = M_1^{-1} M_2^{-1} M_3^{-1}
	=
	\begin{bmatrix*}[r]
	  1     &    0    &     0   &  0 \\
	  3     &    1    &     0   &  0 \\
	  0     &    1    &     1   &  0 \\
	  1     &    0    &  \frac{1}{2}  & 1
	\end{bmatrix*}
\]
\part
\[
b = 
	\begin{bmatrix*}[r]
	    2  \\
	   15  \\
	   35  \\
	   30  
	\end{bmatrix*}
\]
\[Ax = b\] so \[LUx=b\] \[L^{-1}b = w\] \[x = U^{-1}w \]
\linebreak{}
Solve \(Lw = b\)
\[
	\begin{bmatrix*}[r]
	  1     &    0    &     0   &  0 \\
	  3     &    1    &     0   &  0 \\
	  0     &    1    &     1   &  0 \\
	  1     &    0    &  \frac{1}{2}  & 1
	\end{bmatrix*}
	\begin{bmatrix*}[r]
	   w_1  \\
	   w_2  \\
	   w_3  \\
	   w_4 
	\end{bmatrix*}
	=
	\begin{bmatrix*}[r]
	    2  \\
	   15  \\
	   35  \\
	   30  
	\end{bmatrix*}
\]
\begin{eqnarray*}
  w_1 &=& 2 \nonumber \\
  w_2 &=& 15 - 6  = 9\nonumber \\
  w_3 &=& 35 + 9  = 44\nonumber \\
  w_4 &=& 30 - 24 = 6\nonumber 
\end{eqnarray*}

Solve \(Ux = w\)
\[
   \underbrace{\begin{bmatrix*}[r]
	  1     &    2    &    -1   &  0 \\
	  0     &   -2    &     3   &  1 \\
	  0     &    0    &     8   &  5 \\
	  0     &    0    &     0   &  \frac{3}{2} 
	\end{bmatrix*}}_{U}
	\begin{bmatrix*}[r]
	   x_1  \\
	   x_2  \\
	   x_3  \\
	   x_4 
	\end{bmatrix*}
	 =
	\underbrace{\begin{bmatrix*}[r]
	    2  \\
	    9  \\
	   44  \\
	    6 
	 \end{bmatrix*}}_{w}
\]
\begin{eqnarray*}
  x_4 &=& 4 \nonumber \\
  x_3 &=& \frac{44-20}{8}  = 3\nonumber \\
  x_2 &=& \frac{9 - (3*3 + 4)}{-2}  = 2\nonumber \\
  x_1 &=& 2 - (4-3) = 1\nonumber 
\end{eqnarray*}
\[
x
	 =
	\begin{bmatrix*}[r]
	    1  \\
	    2  \\
	    3  \\
	    4 
	 \end{bmatrix*}
\]

\part
\[
c = 
	\begin{bmatrix*}[r]
	   -3  \\
	   -3  \\
	    0  \\
	   -3  
	\end{bmatrix*}
\]
\[Ax = b\] so \[LUy=c\] \[L^{-1}c = w\] \[y = U^{-1}w \]
\linebreak{}
Solve \(Lw = y\)
\[
	\begin{bmatrix*}[r]
	  1     &    0    &     0   &  0 \\
	  3     &    1    &     0   &  0 \\
	  0     &    1    &     1   &  0 \\
	  1     &    0    &  \frac{1}{2}  & 1
	\end{bmatrix*}
	\begin{bmatrix*}[r]
	   w_1  \\
	   w_2  \\
	   w_3  \\
	   w_4 
	\end{bmatrix*}
	=
	\begin{bmatrix*}[r]
	   -3  \\
	   -3  \\
	    0  \\
	   -3  
	\end{bmatrix*}
\]
\begin{eqnarray*}
  w_1 &=&  -3 \nonumber \\
  w_2 &=& -3 + 9= 6\nonumber \\
  w_3 &=& 0 + 6  = 6\nonumber \\
  w_4 &=& -3 - (-3 + 3)= -3\nonumber 
\end{eqnarray*}

Solve \(Uy = w\)
\[
   \underbrace{\begin{bmatrix*}[r]
	  1     &    2    &    -1   &  0 \\
	  0     &   -2    &     3   &  1 \\
	  0     &    0    &     8   &  5 \\
	  0     &    0    &     0   &  \frac{3}{2} 
	\end{bmatrix*}}_{U}
	\begin{bmatrix*}[r]
	   y_1  \\
	   y_2  \\
	   y_3  \\
	   y_4 
	\end{bmatrix*}
	 =
	\underbrace{\begin{bmatrix*}[r]
	   -3  \\
	    6  \\
	    6  \\
	   -3 
	 \end{bmatrix*}}_{w}
\]
\begin{eqnarray*}
  y_4 &=& -2 \nonumber \\
  y_3 &=& \frac{6+10}{8}  = 2\nonumber \\
  y_2 &=& \frac{6 - (6+2)}{-2}  = -1\nonumber \\
  y_1 &=& -3 + 2 + 2 = 1\nonumber 
\end{eqnarray*}
\[
y
	 =
	\begin{bmatrix*}[r]
	    1  \\
	   -1  \\
	    2  \\
	   -2 
	 \end{bmatrix*}
\]


\end{document}
